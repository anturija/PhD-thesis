
\documentclass[11pt]{report}
\usepackage[a4paper,margin=2cm, bindingoffset=2cm]{geometry}
\usepackage{appendix}
\usepackage{amsmath}
\usepackage{booktabs}
\usepackage{threeparttable}
\usepackage{natbib}
\usepackage[final]{changes}
\usepackage{color,soul}
\bibliographystyle{chicago}

%to stop orphan lines
\widowpenalty=10000
\clubpenalty=10000
\raggedbottom

%line spacing
\linespread{1.3}

\begin{document}
\begin{titlepage}
\begin{center}
\large
\textsc{EPFL} \\
\ \\
\textsc{CODEV and LASUR} \\
\ \\
\textsc{EDAR}\\
\ \\
\ \\
\ \\
\ \\
\ \\
\ \\
\ \\
\Huge
\textbf{Urban Development Process}
\ \\
\ \\
\large by
\ \\
\ \\
Marija Cvetinovic
\vfill
Thesis for the degree of Doctor of Philosophy \\
\ \\
\ \\
XXX 2017
\end{center}
\end{titlepage}
\begin{titlepage}
\begin{center}
\ \\
\ \\
\ \\
\ \\
\ \\
\ \\
\ \\
\ \\
\ \\
\ \\
\ \\
\ \\
\subsubsection{"Every thing possible to be believ'd is an image of truth."}

\end{center}

\begin{flushright}
― William Blake, The Marriage of Heaven and Hell
\end{flushright}

\end{titlepage}

%Roman Page Numbering
\pagenumbering{roman}
\chapter*{{Abstract}\markboth{Acknowledgements}{Acknowledgements}}
\addcontentsline{toc}{chapter}{Abstract}
xxxxxxxxxxxxxxxxxxxxxxxxxxxxxxxxxxxxxxxxxxxxxxxxxxxxxxxxxxxxxxxxxxxxxxxxxxxxxxxxxxxxxxxxxxxxxxxxxxxxxxxxxxxxxxxxxxxxxxxxxxxxxxxxxxxxxx
xxxxxxxxxxxxxxxxxxxxxxxxxxxxxxx.\\


\tableofcontents
\listoffigures
\addcontentsline{toc}{chapter}{List of Figures}
\listoftables
\addcontentsline{toc}{chapter}{List of Tabes}

\chapter*{Declaration of Authorship}
\addcontentsline{toc}{chapter}{Declaration Of Authorship}
I, xxxx, declare that the thesis entitled xxxxxxxxxxxxxxxxxxx and the work presented in the thesis are both my own, and have been generated by me as the result of my own original research. I confirm that:
\begin{itemize}
\item this work was done wholly or mainly while in candidature for a research degree at this University;
\item where any part of this thesis has previously been submitted for a degree or any other qualification at this University or any other institution, this has been clearly stated;
\item where I have consulted the published work of others, this is always clearly attributed;
\item where I have quoted from the work of others, the source is always given. With the exception of such quotations, this thesis is entirely my own work;
\item I have acknowledged all main sources of help;
\item where the thesis is based on work done by myself jointly with others, I have made clear exactly what was done by others and what I have contributed myself;
\end{itemize}
\vspace{2cm}
Signed: \dotfill
\vspace{2cm}
\newline
\noindent
Date

%Include Acknowledgements in TOC
\chapter*{Acknowledgements}
\addcontentsline{toc}{chapter}{Acknowledgements}
This work was undertaken with financial support of the 

This thesis would not have been possible without the support of many people. I would like to express my sincere gratitude to:

SNSF
SCOPES
IAUS
APEP UCD

\begin{itemize}
\item 
\item ...
\end{itemize}



%%%%%%%%%%%%%%%%%%%%%%%%%%%%%%%%%%%%%%%%%%%%%%%%%%

\chapter{Introduction}
%Arabic Page Numbering
\pagenumbering{arabic}

%%%%%%%%%%%%%%%%%%%%%%%%%%%%%%%%%%%%%%%%
Urban development is a widespread archetype for out-of-reach improvement in cities of the Global South. However, in its essence it is more a kind of constant catch up with the West and western urban paradigm than an elaborated form of intrinsic local perception, knowledge and action toward urban transitions. 
\\
In this highly competitive international arena, transitional countries experience grave consequences due to the paucity of practical experience within the dominant/ruling western ideology of the urban. They are caught in this new context of relentless rules of market economy, decentralized political and administrative powers, lack of resources, scarcity of general international investment and scant interest for dramatic shifts in all aspects of their social organization and spatial transformations. The blurred and askew morphology of post-socialist cities in transitional countries is therefore the result of continuous pressure from the negative side effects of imitating and lagging behind conventional urbanization models and accelerating globalization patterns imported or imposed by the Global North, or colloquially known as "the West".
\\
The urban transformation of Serbian cities falls into this cliché of the new post-socialist urban reality, which emerged during the "transition to markets and democracy" (Tsenkova, 2006). The dismantling of the communist system during the late 1980s represented a substantial change in all aspects of social organization, the economic model and the political system. However, Serbia is still identified as a post-socialist melting pot where representative democracy, civil society and market economy principles collide and merge with authoritarianism, vertical decision making and populism practices. In such a situation, concern about the urban has been left out and given over as a battlefield for social needs in practice and technical solutions on paper and an easy prey to the exercise of power and interest. Therefore, the practice of planning and designing Serbian cities has been narrowed down to a mere technical issue most often even without an actual or adequate realization in practice. Not to mention that very few theoretical or general methodological research studies bothered to examine alternative planning modes, techniques and instruments in transition, but continued with the manner of replication from well-known counterparts of the Global North.
\\
Among others, architects have a vital role in not only directing but also framing the path of urban formation and development in post-socialist cities. Even more so, as they are primarily focused on practice and "savoir faire" about making the built environment. Yet acquisition of land and illegal construction are spatial interventions that have marked post-socialist production of space more than any planning or theoretical activity. IN general, even though we have different drivers on the global scale and in developed countries, there is a global trend of resorting to sociological, planning or even IC approach in scientific studies on the urban. 
\\
In order for architectural intervention in space to compete for more relevancy and rigour,  architects all over the world have been gradually grown interest for scientific discourse on the context of built structures, spaces and cities in general.  In the rivalry between spatial and social basis for their interpretations, the fact that the field of architectural research is not yet standardized in terms of methodologies and techniques opens the floor for experiments and innovations \hl{(ref)}. In the circumstances of developmental bouillon or "developmental schizophrenia"(\hl{Vujosevic}) at the local level and an overall urge for architectural research framework internationally, my aim is to elaborate an architectural standpoint when applied on complex post-socialist urban reality in order to establish a methodological approach suitable, to a certain extent, for architectural scientific discourse on the matter.
\\
In my striving to contribute to post-socialist architectural research, the far-reaching aim is to capture post-socialist urban complexity and dynamics in order to skip the classical procedure of urban development based on western planning paradigm and provide its practical application on multiple levels of urban decision making. This to be achieved requires supple methodological approaches which should better correspond to post-socialist socio-spatial patterns on multiple levels (state, city, municipality, community, and neighbourhood) and explain the correlations of various urban elements. Practice-oriented, locally focused and globally tuned approach to complex urban reality of post-socialist cities envisions embracing the dynamics of urban systems and operationality of architectural performance for circumscribing visual interpretations that enable continuous conclusion drawing and up-to-date introduction of any new element that may appear in the system.
\\
This chapter immerses into the contextual, scientific and disciplinary discourse of the following research. It marks the research context, historical and scientific, and puts a spotlight on the importance of the research problem as well as of the purpose and adequacy of this thesis. Then, I outline my research drive in the mentioned scope and present in a nutshell what this research is about, how it will be performed and what are the research expectations and practical results I plan to meet. 

\section{Field of Study}
Due to growing social and physical transformations that become ever more intensified as current globalization continues to spread out profit maximization, consumption patterns and information networks (Harvey 2012), the cities have been experiencing a progressive reorganization at spatial and social levels. Even though accelerating urbanisation is a worldwide process, it still assumes different forms and meanings, depending on the prevailing local conditions (Bolay, 2007). These overall circumstances of continuous urban development influence cities to serve as the primary channel linking local realities to global social, political and economic forces (Yates and Cheng, 2002; Tsenkova, 2006).
\\
Cities are not simply market products and consumption patterns, but locally customized socio-political constructs as well (Marcuse et al. 2008). These external influences form a range of qualitatively different contextual circumstances for positive urban transition, i.e. urban development. Most settlements and cities from previous historical periods had reflected upon the various degrees of forethought and conscious design in their layout and function. This approach was referred to as a fixedly planned development, albeit many cities had tended to develop organically. Generally speaking, a range of urban disciplines (urban planning, theory, sociology, legislation and design) aim to decode and harmonize growing urban issues as a side-effect of the current globalization, urbanisation processes and spread of \hl{capitalism}. These trends are mainly affecting cities and production of urban space and bid for the expertise on managing urban development (Allmendinger, 2009; Faludi, 1973). 
\\
In practice, these disciplines are embedded in a particular social context or a territorially based system of social relations. They react to the shifts in socio-economic and political settings (Tsenkova, 2006b), but have kept privileged relationship toward Western cities, which assumed to be the sources of urban creativity, vitality and innovativeness in urban domain (\hl{Robinson, 2006:2}). Accordingly, they tend to fail substantially within the range of spatially and economically different environments that have undergone highly dramatic change in political, economic and social terms. For example, urban research and practice in transitional countries in Central and Eastern Europe (CEE) should unfold to help understand these phenomena in their immediate and wider context. The crucial is to identify patterns of the dynamic reality in these cities and be more consistent with spontaneous, everyday urban transitions. Furthermore, a corresponding change in approaching urban development can then be addressed by heterogeneous, iterative and generative process of urban space production in physical and social sense. Such approachaim to surpass the perception of cities as merely economic, social and cultural venues treating them as complex and dynamic urban systems. In these circumstances it is necessary to apply proper techniques and methodologies for urban research and analyses which encompass complexity and dynamics of cities for the improvement of their living conditions and the facilitation of social interactions in the process of urban development.
\\
However, each discipline keeps its own track and pace in approaching urban matters. My architectural background has moulded my own research interest towards gaining knowledge and understanding on management of space and built environment. Moreover, production of space is also the core concern for architects. Architecture is a discipline focused on practice and consequently it urges for parameters, categories and structure for its practice-based analyses. Hitherto history and theory of architecture have been the main fields of architectural research. On the contrary, recent trends in architectural research make a case for transforming the general body of knowledge on cities into a real-life problem-solving strategy that address human lifestyles, social relations and the concept of space \hl{Castells 2000, Dijk 2002 add more}. However, ever growing presence of such research frame of reference urges for the advancement of architecture in terms of the relevance and reliability of the knowledge herein produced (\hl{ref}).
\deleted{But the ever growing relevance of architecture for transforming the general body of knowledge on cities into a real-life problem-solving strategy that address human lifestyles, social relations and the concept of space (ref other ref than initial Castells 2000, Dijk 2002 that is addressing ICT), insists on the advancement of architecture in terms of the relevance and reliability of the knowledge herein produced}. However, missing links with the classical scientific discourse has caused a growing concern for what is research appropriate for architectural and design practice as well as for architectural stance in urban studies \hl{ref}. This concerns especially methodologies, methods, approaches, domain and credibility (\hl{ref+Savic 2016b}). Lacking the traditional scope of analysis, architectural research has been a polygon for innovations and experiments.
\\
In terms of methods, there has been a significant number of interdisciplinary, transdisciplinary and multidisciplinary endeavours in applied research with an architectural focus in urbanism (ref). What is more, applied fields of research acknowledge the use of methodological hybrids (Datta, 1994, De Lisle 2011). This has opened doors for applied social sciences to investigate new methodological opportunities when confronted with complex and multiplex social phenomena (De Lisle 2011). Even more so as methodological and epistemological rigidity leads to ignoring the realities of the practical and cause catastrophic scientific failures of practice-oriented research (Rogers 2008, De Lisle 2011).
\\
What I have recognized as a crucial change in the methodological paradigm of an applicable urban research from the architectural standpoint can then be condensed/boiled down to the rise of the global concept from static to iterative and dynamic. Commonly speaking, a static world is one in which all transitions are according to a known law and which do not give rise to uncertainty. When defining the evolution of analyzing and simulating an urban phenomenon or process, it is fundamental to state that the existence of a problem depends on the future being different from the past, while the paradigmatic possibility of finding the solution to the problem depends on the future being like the past. Therefore, a transition in some sense is a condition of the existence of any problem. The complex empirical realities of urban transitions collide therein with the powerful and dominant policy of continuous comprehensive production of knowledge. A scientific approach towards formulating the dynamics of urban transitions have to count on uncertainty as one of its fundamental facts and in this way accept and deal with an open-ended future and the limits of human knowledge about it.
\\
Gaining knowledge has come to be a strategic activity rather than a search for truth (Kirby 2013). So science becomes incapable of controlling society and the rationalized reality appears false and irrelevant (Alfasi et al., 2004). Given these conditions, the growing gap between the formal structure and the dynamics that takes place in cities triggers an internal and independent process by which the system tends to spontaneously self-organize (Portugali 2011). Therefore, a city should be conceived as an organism, not a mechanism \hl{Charles Laundry, The Creativity City}. In these terms, the city is interpreted as a living system which is constantly mutating and emitting new elements, a container for processes of coming to be, breaking up and falling out, fragmenting and recomposing \hl{ref}. Contemporary cities tend to be concentrations of multiple socio-spatial circuits, diverse cultural hybrids, and sources of economic dynamism - a venue where the past and the present converge upon one another. The city tells a story of one society and its attempts to move towards a positive vision of the future, through complex ranges of processes that flow together to construct a single consistent, coherent, albeit uncertain, interactive and multifaceted time-space system (Graham, 1998). These ceaseless processes are the core of spontaneous, everyday urban development. Grasping the scope of urban development occurs as a major challenge for modern science about cities.
\\
My intention is not to produce another pattern applicable to certain cities to a certain extent, but rather to apprehend a process that embodies the complexity and dynamics of the mentioned relations in a transparent way.  This framework of research enables to ponder upon means of generating a vibrant and fluid context open to permanent transformations and, most importantly, to grasp the idea of an adjusted and balanced model, adaptive to changing views and situations of accelerating urban development. (Portugali Complexity cognition and the city). This to be achieved requires supple approaches which should aim at explaining the correlations of various urban elements and to better correspond to the socio-spatial patterns of the range of urban environments. In such a plenitude of factors, I have chosen case study as an adequate research method and a neighbourhood as a relevant level of analysis.
\\
Dynamic urban context is a complex phenomenon with a plenitude of data. Case study research method enables close, in-depth and holistic examination of a great deal of data, but requires a bounded environment in order to accurately describe and illustrate such a context and to use it for broader interpretations and demystification of modern cities. Specifying physical limits is not in itself enough for circumscribing the identified complexity of urban transitions, the issue of scale is also at stake. In urban terms, different spatial and social elements are intensified or muted at different levels (global, national, regional, local). In order to acquire active follow-up, interpretation and assessment of urban issues, it is important to define a representative environment, a robust source of prominent urban "processes". Thereupon, I argue for a neighbourhood level of analysis, because it may become a paradigm for complexity and dynamics of modern urban context. It serves as an urban micro environment, which eventually increases the body of knowledge on cities concerning the methodologies used to deal with urban development and corresponding urban transitions.
\\
\textbf{"The contemporary city is a variegated and multiplex entity - a juxtaposition of contradictions and diversities, the theatre of life itself" (Amin and Graham, 1997).}

\subsection{Background}
At the beginning of the 21st century, the world experienced a progressive reorganization at an economic, political and social level: profit maximization, globalization of urban processes and the devastating history of deindustrialization and dematerialization of the world (Harvey 2012). Nowadays, while about 50 percent of the world population lives in urban environments (United Nations 2008), the question of techniques and methodologies for urban development research and analyses should undoubtedly address these major shifts in urban life and contemporary cities (Healey 1997).
\\
Cities are rather primary venues, power poles and capacity builders of economic, social and cultural development at stake in modern societies (Castells, 1998). Conversely, cities are dynamic and diverse urban entities that are given to shaping their autonomous and innovative future on the basis of human resources and creative human potential (Knight and Gappert, 1989; Yigitcanlar, 2008). The prosperity of cities depends on how competitive they are on a global economic scale, how flexible they are in terms of adjusting to current trends and needs, and how fertile they are for the development of knowledge and the application of innovation. These major uncertainties of contemporary life, created mainly but not exclusively by the current method of production and management, are acutely symbolized by concerns about urban development (Healey 1997).
\\
Urban development is widely accepted even though also contested category usually associated with urbanisation processes in "so called" developing countries. Lots of professionals in urban research and practice use the term, not to mention the great number of people around the world affected by their work. Nevertheless, the notion of the word "development" itself means different things to most of them.
However, traditional and widespread interpretation comply with the western paradigm of development: modernisation and economic growth \hl{ref}. Interpreted in this way, the notion of urban development actually promotes the leading hierarchies and categorization of cities in the world, based on the impact of globalization, new transnational  economic  progress  and  networking  of  cities \hl{ref}. Both of these approaches impose the hierarchical relation among them, as Jennifer Robinson (2006) bluntly puts it, "while some are exemplars and others are imitators". Besides, the chronological paradigm of western urban planning dilutes when it is spatially translated to these qualitatively different environments (Robinson 2002), causing them to lose their substance as an urban phenomenon through the ill-decoded application of western patterns (Bolay 2004). In addition, modern urban thought could be stuck in this rut, inducing negative background effects on the whole gamut of urban activities (Amin et al. 1997), causing urban conflicts to thrive on the basis of inequitable power relationships, and cultural differences, as they develop from an individual level towards a socio-urban dimension (UN Habitat 2009).
\\
In  this  respect,  Jennifer  Robinson  summarizes  that  categorization  and  differentiation  of  cities, according to Modernity and Development, are a pure product of colonial past. This actually means that in the scope of widely praised universal image of "cityness" as the final goal of the ambition of cities, successful examples of cities are included inside these categories. \underline{World cities} are defined in relation to their regional, national and international influence inside a global economy where the country categorization is transferred into this world city categorization \hl{ref}. Conversely, \underline{global cities} are categorized according to their industrial and communication potential of transnational management and control \hl{ref}. They both focus on the characteristics of cities and potentials in the scope of global economy, its flows and networks. This approach has proved to be insufficient, exclusive and restrictive for cities in less developed countries, if we keep up with the same terminology at the national level. On the other hand, cities which are outside these categorizations but with a same ambition and vision of "cityness" are regarded as \underline{third World} or \underline{developing cities}. Consequently, there are even more categorizations such as those of western, wealthy, third world, developed and developing cities. However, with all of them together, there is still a vast number of cities which are left out and with barely any possibility to ever fit in any of the categories (Robinson, 2006). 
\\
\textbf{"Ordinary cities also emerge from a post-colonial critique of urban studies and signal a new era for urban studies research characterised by a more cosmopolitan approach to uderstanding cityness and city futures. This can underpin a field of study that encompasses all cities and that distributes the difference amongst cities as diversity rather than as hierarchical categories. It is the ordinary city, then, that comes into view within a postcolonialised urban studies" (Robinson, 2006)}.
\\
This brilliant insight put forward by Jennifer Robinson in her book "Ordinary cities between modernity and development" questions the geopolitics of urban theory and urban development (Fraser 2006). Taken from this standpoint, each and every city is an indicator of what an urbanized society is and what course of urban development it may take. My research scope in this thesis perseveres with post-colonial critique of urban studies and the notion of ordinary cities, introduced by Amin and Graham (1997) and further developed by Robinson (2002). This concept approaches the knowledge of diversity and complexity that exists within the world and "distributes the differences amongst cities as diversity rather than as hierarchical category" Robinson (2002). 
\\
Ordinary cities approach provides unique assemblage of internally different, distinctive and \added{context-based} urban transitions as well as overlapping space-, time- and relation- networks across cities. In other words, it is not only necessary to examine the ways in which countries/cities interface with the global economy, but also social, cultural and historical legacies that each country/city carries into the era of globalization. Within such domain for explanations, this thesis revolves around the interpretation of urban development as an answer to the question "how can cities facilitate urban transitions while also maintaining the culture and values of the community itself?" \hl{(ref article: Does Placemaking Cause Gentrification? It’s Complicated.)} The idea of indicating what encompasses urban development of an ordinary city lead to identifying its internal and external influences \added{that constitute the core of maintenance, transformation or change processes in an urban system of a city}, when treated equally within the global hierarchy of cities (Robinson 2002). 
\\
This approach makes a worldwide, broad, general and mutable process of urban development actually connected to place - making an actual urban setting a vital factor for case specific uncertainties and a polygon for transformation of global aspects to meet local specifications. My aim is to move away from the general theoretical research into an on-site practice-based investigation. Consequently, this research project attempts to show how the real-life focus on Savamala neighbourhood in Belgrade eventually increases the body of knowledge on post-socialist urban environment and the methods used to deal with complex and dynamic urban context. Complying with "ordinary cities" approach, I would like to elaborate that post-socialist cities in transitional countries meet extraordinary difficulties when copying urban models from the West.  The cause is found in the lack of the institutional infrastructure and cultural patterns essential for the functional unity present in western cities (Petrovic 2009). Furthermore, fundamentality and intensity of economic and political change in Balkan post-socialist countries may be a historic exemplary of social transition hard to find in a "typical" capitalist city (Sykora 1994). Its internal environment is in a state of flux, with the rapid adjustment of the physical, economic, social, and political structures of the city itself (Sykola, 1999).
\\
Included in this range of spatially and economically turbulent surroundings, post-socialist cities in transitional countries have undergone highly dramatic change in political, economic and social terms. The mayor consequences of such transition introduce, on one hand, the disastrous effects of increasing social polarization (inequity), deinstitutionalization of socio-spatial practices (informality) and unfair wealth redistribution (poverty). On the other, the huge socio-cultural base inherited from the socialist period with centralised and authoritarian practices dominate post-scoialist urban governance. This has had a profound influence on the spatial adaptation and social repositioning of post-socialist cities.
\\
Turbulent social times, such as the disintegration of Yugoslavia’s political system and the introduction of new context of market economy, decentralized administrative powers and a lack of investment and resources are reflected in chaotic urban development pattern. Urban systems of post-socialist cities are highly susceptible to tense on-going transformations, diverse but reciprocal in their nature: economic transformations (transformation of production and consumption in relation to space, income polarization and poverty), political transformations (urban governance, political voluntarism, participation and decentralization), spatial transformations (demographic trend and distribution of functions) and social transformations (social exclusion-inclusion, social activism and informality). In other words, what proceeded after the end of the socialist era is a neoliberal model of urban planning with the supremacy of market-oriented solutions for urban problems (Sager 2011). Conversely, with the huge socio-cultural base inherited from the socialist period, cities in transitional countries have continued to be centres of economic growth with a variety of services, expansion, technological innovation and cultural diversity. While some trends and directions within these transformations are clear and defined, uncertainty dominates decision making and implementation in the turbulent environment of post-socialist cities (Nedovic-Budic 2001). Therefore, the post-socialist period in these cities contains prevailing characteristics of the disintegration of the preceding system rather than a coherent vision of what should follow. 
\\
In practice these conditions ended by having the strategic plan as an advisory long-term urban vision, but leaving the real actions and decision making to political and market forces. Thenceforth, urban development of post-socialist cities most often has exceeded and diluted the common strategic framework defined from top-down: to establish clear links between the process of strategy development, its institutional framework, the hierarchical structure of long-term and short-term objectives of all actors involved, and the real-time changes happening simultaneously in an urban environment. The major characteristics of post-socialist urban development are: a multitude of actors, various economic, social and political interests, social aspects and fragmentation of urban spaces. Consequently, post-socialist cities lack complex, operational logistics \hl{(check!!!Repetti et al. 2010)} to link top-down changes to bottom-up interventions in urban systems. There exists an growing discrepancy between the national and global levels, on one side, and city and neighbourhood levels, on the other. 
\\
The  conceptual  framework  explained  herein  pinpoints  the  blurred  and  askew  morphology of  post-socialist  cities which  requires  dynamic  solutions  in  order  to  skip  the  classical, western procedure  of  urban  formation  and development. Consequently, this particular context shows the increased need for proper techniques that are spatially and temporally adjusted to current issues. The far-reaching goal actually is to transform the negative side effects of imitating and lagging behind the western urbanization model and those of the accelerating globalization into a development impetus suited to these environments.
\\
Urban development of post-socialist cities is perceived as a dynamic concept, a multi-dimensional integrated system composed of qualitatively different and semi-autonomous processes, with the inclining tendency to improve the economic, social, demographic, political and technological state of an urban environment. In view of all this, we need an overarching theory of urban development that can encompass all discrepant decision making forces: future-oriented urban projections (urban planning strategies), in situ transformation forces and potentials (urban transformations), and follow the creative paths of urban dwellers (participatory urban design activities) for imagining new urban futures. The question of facilitating and localizing urban transitions rests with overlapping urban scenarios from dissonant levels of decision making, tracking cultural identities, requirements and needs of all urban actors, and, in general, indicating contextual processes of maintenance, transformation and change of an urban system. 

\subsection{Problem statement}
The focus of this thesis is urban complexity and dynamics of post-socialist cities. The issue is not addressed as a problem to solve, but rather as a moving target for an exploratory observation of the way how cities function and how various urban transitions condition urban development of post-socialist cities.
\\
Post-socialist cities are treated herein as a range of qualitatively distinctive cities that "deal differently with their difference" \hl{ref}. In their incompleteness, plurality and informality, post-socialist cities in transitional countries represent dynamic and diverse arenas of contemporary urban life, experience and theory. Included in this range of spatially and economically developing surroundings, transitional countries in Central and Eastern Europe (CEE) have undergone highly dramatic change in political, economic and social terms. The disintegration of Yugoslavia’s socialist system led to the destabilization of the institutions and the social value system in Serbia. Such confusing political and social circumstances have deprived an average citizen of sufficient information about the possibilities and tools to take an active part in the development of their city. These factors provoked a legal void susceptible to shady deals and questionable public-private partnerships (illegality); a lack of strategically proactive urban governance, which resulted in tolerance to illegal building practices (informality); the increasing social polarization (inequity); and poverty in this region, the number of poor people had reached 100 million in CEE by 2001 (Tsenkova 2006a). This rather organic path of urban development leads to the classifying of post-socialist cities in transitional countries as unregulated capitalist cities (investment-led) with third world urban development elements (substantial illegal activities and informal markets) (Petrovic 2009).
\\
Conditioned by the geographic location of Serbia (CEE), murky circumstances of transition (towards liberal market, private property, profit motive and consumer sovereignty) are followed by a set of decentralization and democratization protocols for joining EU, availability of European research and civil sector funds, as well as the promotion of participation and engagement from the ground up \hl{ref}. Having said that, the lack of successful urban planning models and actions make possible that the rising economy of social exchange and local capacity building could contribute to an improvement of life and functionality of urban structures and systems, and effectively address the tensions between top-down and bottom-up urban planning in a post-socialist city. Tracing institutional articulation of post-socialist context involves structural analysis of administrative procedures and content analysis of policy agendas. It serves to systematically deconstruct local urban governance in terms of political, economic and cultural aspects of transition with a multitude of actors, variety of interests, conflicted strategies and fragmented implementation. In the long run, the identification of relations and influences on post-socialist urban governance examines how urban actors, space and regulatory framework rely on planning and decision support systems as means to forecast and orchestrate any movement of the system. In this manner, any element of urban systems, human or not, is attributed agency.
\\
Conversely, under the hood of scientific neutrality, urban development concept is critically approached, broken down and recomposed as a process of urban transitions, not as an indicator or the final product in urban practice. Urban development of post-socialist cities is seen as a complex, multifaceted network of urban transitions that evidence: 
\begin{enumerate}
\item the level of urbanity - qualitative processes of maintenance, transformation and changing processes of an urban system;
\item legitimacy of different layers of urban decision making top-down urban planning strategies, tactical urban transformations, and bottom-up participatory activities;
\item urban key agents - constitutive elements for the morphology of urban decision making;
\item numerous urban conflicts, social practices and contextual resources resulted from the incompleteness, plurality and informality of post-socialist cities.
\end{enumerate}  

\section{Thesis Aims and Scope}
\textbf{research scope:} grasp the actual urban development process in cities
\\
\textbf{research impact:} the quality of an urban system generates a vibrant and fluid context open to permanent transitions gives rise to potential to originate diverse opportunities for new rounds of exchanges among research, innovation, action and development (Bolay et al. 2011).

\subsection{Research Objectives}
\textbf{Overall objective:} encompassing complexity and dynamics of urban transitions as an urban development indicator at the local level in a rather transparent way
\\
\deleted{identify variables in objectives}

\subsubsection{RO1}
re-formulate urban development concept in terms of urban transitions to fit the idea of dynamic state of an ordinary city in its full complexity
\begin{itemize}
\item RO1a: identify what urban system complexity is \deleted{and sort out active urban key agents and contextual resources and map their interconnections and networks} 
\item RO1b: map \deleted{how socio-spatial patterns of an ordinary city are constituted in} urban networks
\item RO1c: trace the morphology of decision making 
\item RO1d: define dynamic state of a complex urban system in an ordinary city - description of empirical reality of urban networks and processes
\end{itemize}

\subsubsection{RO2}
gain an in-depth understanding of the level of urbanity in an ordinary city as an indicator of \deleted{urban development/urban transitions/}urban dynamics
\begin{itemize}
\item RO2a: elaborate the level of urbanity \deleted{connect the level of urbanity to urban dynamics socio-spatial patterns in an ordinary city}
\item RO2b: \deleted{elaborate and encode socio-spatial patterns (spatial, social and technical differences and specificity) of post-socialist context - local urban conflicts, social practices and contextual resources} \added{connect the level of urbanity to urban dynamics}
\item \deleted{RO2c: connect urban transition processes to socio-spatial patterns}
\item RO2d: contextualize the level of urbanity categories \deleted{urban transitions} in post-socialist cities 
\end{itemize}

\subsubsection{RO3}
conceptualize a methodological hybrid for tracing urban complexity and dynamics
\begin{itemize}
\item RO3a: specify a neighbourhood level of analysis 
\item RO3b: describe urban complexity - networks - to indicate the morphology of urban decision making
\item RO3c: trace the level of urbanity - urban processes - to indicate urban dynamic
\item RO3d: proceduralize urban transitions for circumscribing urban development process
\end{itemize}

\subsection{Overall Research Question}

\textbf{Overall research question:} HOW To investigate \deleted{socio-spatial patterns of} post-socialist cities in order to reinvent a more inclusive and flexible approach to understanding Urban development dynamics engaging the complexity of an urban context? 

\subsection{Main Concepts}
The city is regarded as a geographically condensed, highly structured economic, and the most complex social phenomenon (Mumford 1961). "Time" and "social interactions", in the modern qualitative sense of the term, are now the leading determinant for the way urban systems function \hl{(ref SNF1)}. Urban structures interact in an environment that is constantly undergoing transitions, as they themselves are not permanent and unchangeable. As a result, this constantly influences and changes our point of view, influencing our way of solving problems that exist in our environment, as we and all of our surroundings are in a constant state of flux (Harvey 2003). This sort of relativism, where the interactions of as many elements as they emerge determine the context in which they are placed, should be a formative factor in addressing urban complexity and dynamics in terms of urban development prospects and circumstances. The theoretical stronghold of this thesis is the interpretation of urban development, namely going away from qualitative notion of the term and indicate its operational equivalence with more neutral and relativistic idea of urban transitions. Urban transitions encompass complexity and dynamics of an urban environment within the combination of urbanity (description of the state of an urban system and the agency of its transitions) and urban decision making (sorting urban elements and transitions according to the layers of interventions - planning, investment-based transformations, and participatory activities).
\\ 
\textbf{Urban development} is rather a generic term for circumscribing the progress of and in cities addressed in the \added{blurred} field of practice-oriented research \hl{World Bank ref}. Nowadays, when cities are primary venues, power poles and capacity builders (Castells, 1998), the theorem that the growth of cities expand opportunities seems to hold up. Moreover, urban development concept has been easily mixed up with urbanization and economic growth and more often ruled out by the appealing righteousness of sustainable development trends \hl{ref}.  In this sense, urban development has been either patterned or predicted referring to whether it is the part of a model or a project for a city or an urban environment. However, in both cases it implies change. The programmed change is usually assumed positive in its intention or marked as developmental if it has positive economic or, less often, social outcomes \hl{ref}. In reality urban change is most often the consequence of power struggle and has conflictive outcomes on different stakeholder groups \hl{Fainstein 2010 and else ref}. Yet it has been bounded only spatially - referring to a city or a part of the city. Not to mention that today's solution may be the conflict of tomorrow \hl{Holden 2015, ref}.
Therefore, this thesis approaches urban development concept in relativistic terms\hl{(explain time-space concept in a footnote}. In this sense, urban development is applied as an overarching codifier for urban complexity bounded rather as a comprehensive overlay for urban dynamics, not as its qualitative, prognostic nor delineative indicator. In other words,  urban development is circumscribed herein by a set of premises as follows: 
\begin{itemize}
\item Urban development is treated as a process of urban transitions over time;
\item Urban transitions indicate every socio-spatial reference that affects an urban system;
\item Urban transition is the consequence of urban decision making;
\item Urban transition affect the range of human and non-human elements of an urban environment;
\end {itemize}
This interpretation of urban development not only emphasizes its processual nature, but also moves away from its project- or model-based feature by incorporating locally contingent socio-spatial patterns \hl{(Guy and Henneberry 2000)} and non-human basis of urban agency\hl{(Healey 1991 add others)}. The units of analysis are temporarily and spatially bounded urban systems, either whole cities or its \added{conventional parts} \hl{(ref)}. \textbf{Socio-spatial patterns of urban transitions} is a provisory term  that contributes to develop an understanding of development processes beyond mere strategic economic and social framing of needs and events and taking into account sporadic and spontaneous agencies of urban systems. The sensitivity to this range of needs, events and agencies means that whatever happens refer to the state of an urban system - the processes of maintenance, transformation and/or change which we define as urban transitions \hl{ref}. Accordingly, the complexity of an urban system, which involves the unpredictable and uncertain in its structure, is bridged by emphasizing the reference to its state and corresponding urban dynamics. This approach indicates political aspect of urban processes not that of urban structures. Moreover, it coincides with the political view of urban planning \hl{ref}, though it takes a more inclusive turn with all the agents of interventions, relations and events taken into account, not withstanding their nature, function or purpose. In other words, urban development becomes reconfigured to a fine-grained urban dynamics, adding up elements to the battlefield of urban decision making, while it enables labeling the complexity of urban systems.
\\
In general, the important research challenge of this thesis is testing the legitimacy of urban decision making for urban development. The issue at stake is to encompass planning, power struggle, economic interests, design and participation in an overarching urban decision making procedure. Namely, the source of urban transitions are decisions made through these various top down, bottom up and interest-based interventions, relations and events. Political and governance practices are open and susceptible to choice, through contestation and struggle, and accident, historical or natural, "but decisions become locked in" and instigate urban transitions - maintenance, transformation or change of the current state of an urban system (\hl{Hudson and Leftwich 2014}). \textbf{The morphology or urban decision making} therefore comprises and reconciles all its different layers that spread urban transitions through and across an urban system and engages certain level of forethought. These layers are: top-down urban planning strategies, tactical urban transformations, and bottom-up participatory activities recognized on site \hl{ref}. They serve to enclose the historical continuum of global urban trends and patterns in a local socio-spatial framework and translate them into an internal, on-going interaction of individuals or constituted groups.
\\
Identified overarching decision making procedure acknowledges human agency. Through these interactions, urban actors initiate the process of their integration into the environment through an appropriation and transformation of space. In this sense, we could refer to the classical vision upon cities as a setting that consists of: venues (their spatial and built environment) for social interactions (economic, political and cultural), social practices (policies and processes) and reproduction of social order of all urban actors (Firmino et al. 2008). The way cities function shapes the expectations and actions of all the urban actors involved, who also influence the constitution of the city itself. The network of these internal and external influences between human and non-human elements engaged in urban transitions introduces urban agency as a property of all urban key elements. Henceforth, people (urban actors and stakeholders), objects (built environment), territories (space), institutions (regulatory framework), infrastructure and social  aspects (political, economic and cultural circumstances) are all correlated through the morphology of urban decision making. They are also granted agency in urban transition where they figure as \textbf{urban key agents} \hl{(Firmino et al., 2008) and ref}. This multitude and diversity of elements is an urban system and, while emboding its dynamic state, it is rather blackboxing the agency of urban dynamics than decoding it.
\\
\textbf{Urbanity} is another rather blurry concept, applied often in architectural research and practice with the potential for decoding urban dynamics. In general terms, it relies on urban complexity as an active attribute of the overall state of an urban environment (Canuto et al. 2012). Cities are simultaneously the source of both problems and solutions of contemporary life. Cities are the polygon of contemporary decision making. Socio-spatial patterns of urban transitions bend the way how decision making layers address urbanity as its constitutive reality and its ultimate positive goal \hl{ref}. The conceptual framework of urbanity examines the urban key agents, numerous urban conflicts, social practices and contextual resources and how - in their incompleteness, plurality and informality - they form urban transitions.
\\
Moreover, this thesis argues that an overarching definition of urbanity concept improve scientific capacity for grasping urban dynamics. It elaborates how the level of urbanity figures as an indicator for contextual processes of maintenance, transformation and change of an urban system, incorporating simultaneously its state and the transitions. The relationship between the physicality of urban form and the social components of urban life generates the level of urbanity - the quality of continuous harmonization of the variety of structural elements, social factors and vested interests existing in an urban environment (Holanda 2002, Canuto et al. 2012). Moreover, all these urban key elements are assumed to be equal agents in the continuous process of urban development that has been marked by maintenance, transformation and change of the urban system in order to improve its living conditions and facilitate social interactions.

\subsection{Methodological justification}

Following contemporary relativist trends for rethinking space, time, globalization and cities, future research challenge can be defined as "visualizing cities as unformed, unorganized, non-stratified, always in the process of formation and deformation, eluding fixed categories, transient nomad space-time that does not dissect the city into either segments and ‘things’ or structures and processes" (Smith 2003:574). Accordingly, a corresponding change in approaching urban development can then be addressed by heterogeneous and iterative approach that has surpassed the perception of cities as merely economic, social and cultural venues treating them as complex and dynamic urban systems. In these circumstances it is necessary to apply proper techniques and methodologies for urban research and analyses which encompass complexity and dynamics of cities for the improvement of their living conditions and the facilitation of social interactions in the process of urban development. 
\\
Bearing in mind the complexity of such relativist approach to the urban and the necessity of practice-oriented knowledge. This thesis proposes a mixed-method case-study approach \hl{(check Flyvbjerg et al, 2012)}. According to Kuhn's paradigm shift (1962) science about the city is constantly swinging as a pendulum between scientific and hermeneutics approach - quantitative analysis vs. descriptive study \hl{(Portugali Complexity Cognition and the City)}. Mixed research method in this case provides complementary information and in-depth knowledge of the problem. However, it has been solely moulded according to qualitative data sets. The research is influenced by the choice of an innovative methodological approach,  but  the  set of qualitative techniques  and  their  sequence  are  guided by the requirements of the research problem \hl{check ref(Flyvbjerg, 2004; Aitken, 2010)}.
\\
The choice of the methodologies is justified by the \hl{process-driven}, correlational research design and the exploratory character of the research itself. This thesis suggests the potential of the combination of multi-agent system (MAS) and actor-network theory (ANT) methodologies. ANT has been extensively applied in sociology for the analysis of cities and the urban \hl{ref}, while MAS itself is more mathematical-computational method for agent-based modellings \hl{ref}. MAS-ANT hybrid methodology  herein serves to capture local urban dynamics and reframe complexity of permanent urban transitions for urban development. This argument is built on the usefulness of ANT for describing urban reality. ANT approach provides a potential capacity to afford openness and flexibility necessary for founding logical argumentation before tracing urban dynamics \hl{ref}. It will be then demonstrated that MAS adds the framework of action when applied over ANT. Finally, its application is presented on the case study of a post-socialist neighbourhood in Belgrade. In this case, the researcher had the opportunity to be educated in Belgrade and to work in the architectural production in the Serbian capital. Therefore, the  researcher (me) is  to  some  extent  familiar  with  the  local  context  and has possibilities to access certain data.
\\
This thesis adopted MAS-ANT methodology in order to:
\begin{enumerate}
\item describe complex urban reality \deleted{socio-spatial patterns} (urban agency, decision making) \deleted{local conditions and track local urban knowledge)} in a post-socialist city (ANT);
\item understand \added{how the level of urbanity path serves for tracking socio-spatial patterns of transition} \deleted{understand the processes of urban transitions} in Belgrade, Serbia (MAS)
\item indicate \added{the processes of urban transitions} \deleted{the level of urbanity path for tracking urban dynamics} (ANT+MAS) 
\end{enumerate}
DIAGRAM

\section{Contribution}

Relate Contribution to Conclusions

The idea is to create visual interpretations that can be easily computerize (html5), and then easily changed. This enables the continuous generalizations and conclusions drawing and the introduction and description of new elements.

A by-product would be this new definition of urbanity and urban development.

This proposal aims to define a method of solving concrete problems through a process of understanding and dealing with current difficulties as they emerge and evolve.

elaborate inappropriateness of urban development concept for describing and guiding urban processes in cities outside the Global North. Especially not for tracing and directing urban transitions in the way that brings social and environmentally sustainable benefits to the inhabitants and the urban environment.

However, it is a widespread rarely criticised and unbeatable concept, especially in practice and practice-based research. Without delving into hidden motives and circumstances (economic, political, colonial), as this is rather an architectural approach to the urban, I use redefine the core of the concept, but keep its scope and aim to produce an operational methodological framework for practical investigations of the pallette of different cities around the world.

\section{Thesis Structure}

The study is structured in seven chapters.
\\
This chapter sets the path for reaching the research objectives, its crucial role is to provide the basic understanding and scientific justification of what forms and conditions urban complexity and dynamics and how the problem is approached within the limits of this research. The next \textbf{CHAPTER 2} contains an extensive literature review concerning the applicable concepts and the chosen methodologies. These concepts form the essence for categorizations with the chosen methods.
The conceptual and methodological parts build the theoretical framework for this thesis. 
\\
\textbf{CHAPTER 3} relies on the primary statements from this introductory chapter, builds on the range of indicators identified within the theoretical framework and further elaborates the methodological approach and the scientific argument of the research.
\\
In order to substantiate proposed hypothesis,  presented  theoretical framework will be tested on an elucidated case study. In \textbf{CHAPTER 4} the choice of Savamala neighbourhood in Belgrade is clarified and data collection procedures are summarized in the form of a linear and chronological case study report.
\\
The following chapter moves forward to hypothesis testing and consecutive application of the chosen research methodologies. Data analysis with Actor-network theory is the core of \textbf{CHAPTER 5}.
\\
\textbf{CHAPTER 6} deals with system building according to the postulates of Multi-Agent system.
\\
\textbf{CHAPTER 7} presents the actual hybridization of two methods.
\\
In \textbf{CHAPTER 8} the performed research is pulled together. The resulting discussion is drawn upon the outlined background information on the theoretical framework, the deconstructed MAS-ANT methodological hybrid, and collected and analysed data on the cases study from the previous chapters. Based on these results, this thesis is concluded on two separate levels, regarding research and theoretical framework.


%%%%%%%%%%%%%%%%%%%%%%%%%%%%%%%%%%%%%%%%%%%%%%%%%%

\chapter{Literature Review}

%%%%%%%%%%%%%%%%%%%%%%%%%%%%%%%%%%%%%%%%%%%%%%%%%%

This chapter outlines xxxx. ...

\section{Conceptual Framework}

xxxxxxxxxxxxxx

\subsection{New conceptualization of urban development}

At the beginning of the 21st century, the world experienced a progressive reorganization at an economic, political and social level: profit maximization, globalization of urban processes and the devastating history of deindustrialization (Harvey 2012) and dematerialization of the world. The question of techniques and methodologies for urban development research and analyses should undoubtedly address these major shifts in urban life and contemporary cities (Healey 1997). 
\\
Cities are no longer perceived as geographical entities with their distinct identities. The urban now is rather concentration of multiple socio-spatial circuits, diverse cultural hybrids, sources of economic dynamism and complex range of multiple processes that flow together to construct a consistent, coherent, albeit multifaceted time-space system (Graham 1998). The city is perceived as a complex set where past, present and future converge upon one another; a dynamic entity which embodies the social narrative and the attempts to govern its social interactions and spatial distribution i.e. urban development. In political terms, urban development is anything what may happen to a city in terms of maintenance, transformation and change of its original state (Friedmann 1987). Such a context implies that physical spaces are constantly intermingling with social constructions of these spaces (Firmino et al. 2008), annihilating the idea that a place is a single material object and transforming it into a “space of flows” (Castells 1998). The “city” concept is thereafter redirected from spatially bounded, people-centred phenomenon to dynamic and complex urban systems, which in their incompleteness and indeterminacy, are stages where all urban elements participate in their "making", changing and transforming. In other words, a city is perceived as a nexus that balances relational proximity in a fast-moving world with ‘time-space extensibility’ and all human actors and material objects engaged in networks extended beyond the immediate corporeal environment (Graham and Marvin 2001). 
\\
Thus, it is necessary to shift the deterministic concept of urban development to a more comprehensive vision that considers complex networks and their dynamic interfaces that generate better understanding and strategizing of urban development (Huang 2012). Apart from the confusing mix of global and local influences, the complexity of such stand-alone artefacts is encumbered with layers of infrastructure that progressively interweave and infiltrate urban systems, life and culture in cities (Graham 2001, Portugali 2011). The powers of such networking support a complex restructuring of urban elements, along with a capacity for recombining economic, political, cultural, technical or natural factors (Murdoch 1998). Such urban heterogeneity consists of operationalization, interrelation and interaction of socio-technical assemblies within a city (Graham and Marvin 2001) . These become extended over the times and spaces of urban life  (Mitchell 1996) and offer us an opportunity to construct dynamic, sophisticated and synthesised approaches to contemporary urban development . Consequently, cities nowadays are in a constant state of flux, with the rapid adjustment of its physical, economic, social, and political structures (Sykola 1999) to the information flows and infrastructural scapes, so that the urban present is not any more attributed only to spatial forms, economic units and cultural formations but also to integral and complex socio-material and sociotechnical systems in cities (Farias and Bender 2011). 
\\
The concept of progress is central to modern society and it is orientated towards a positive vision of the future. In an urban scope, this concept corresponds to that of risk, where the control of all future events is calculable and predictable in probabilistic terms. This new concept of urban planning is based on the notion of an open-ended future, which implies that uncertainty must be accepted and managed, authorities and actual urban actors should be ready for new requirements and renewability as conditions change, and professionals are to increase their knowledge of risk and vulnerability in urban environments. In this sense, the planning of discourse relates to a master narrative of modernity, including ideas of rationality, objectivity, scientific evidence, values and possible control through normativity.
\\
There are different and numerous interpretations of what is and should be urban development, such as \hl{(Evropski regionalizam 1, World Bank and find other references)}:
\begin{itemize}
\item the course of a culture
\item meeting the needs of human and natural worlds
\item economic growth
\item "right to development" between the developed and developing
\item modernization
\item emphasize raising the living standards by addressing issues of health and safety, inclusion and equity
\end{itemize}  

\hl{Mornings after Nedovic Budic}
system maintenance, transformation, change - system of political order (ref. Friedman 1987)
Figure 1 (p4 pdf): Planning and law in the public domain - system maintenance, transformation and change
Planning as a future-oriented activity for managing urban development and change
According to Stark (1992) “transition” is the period, stage, process or policy that leads from one period or situation to the next. The theory of transition is rooted in the democratization theory that views transition as primarily a political process (Offe, 1997). The study of transition espouses (1) a comparative approach; (2) an emphasis on democratization (civil society, political society, rule of law and constitutionalism, state apparatus, economic society with an institutionalized market); (3) categorization of the pre-transition situation such as authoritarian, totalitarian, post-totalitarian or sultanism and (4) the deterministic influence of the past on the path of transition (i.e. path dependency). The theory also includes a “moment of discontinuity” defined as a period, where the structure and function of a country or city does not correspond to the external environment with which it has to interact (Thomas, 1998).
During the transformation periods, institutional and organizational structures are under reconstruction, property markets are affected, and urban development may be disturbed (Tas¸an-Kok, 2004)
According to Stark (1992) “transition” is the period, stage, process or policy that leads from one period or situation to the next. The theory of transition is rooted in the democratization theory that views transition as primarily a political process (Offe, 1997). The study of transition espouses (1) a comparative approach; (2) an emphasis on democratization (civil society, political society, rule of law and constitutionalism, state apparatus, economic society with an institutionalized market); (3) categorization of the pre-transition situation such as authoritarian, totalitarian, post-totalitarian or sultanism and (4) the deterministic influence of the past on the path of transition (i.e. path dependency). The theory also includes a “moment of discontinuity” defined as a period, where the structure and function of a country or city does not correspond to the external environment with which it has to interact (Thomas, 1998).
    the extent of change that would warrant qualification as transitional vs transformational or evolutionary (e.g. substantial vs minor; abrupt or revolutionary vs gradual, etc.);
    the direction of change (i.e. could retrogressing be considered as a transitional state as well);
    determination of the start and end points
    the length of the “moment” of discontinuity.
    As an alternative, would transformation then be characterized with continuity and incremental change?
    would the evolutionary perspective recognize the constant presence of change and the complex influences that espouse the change to take a particular direction.
    the changes happen through invention (and
    re-invention), innovation, borrowing and imposition.

Robinson (2006) clarifies   her  cosmopolitan  tactics  for  surpassing  hierarchical categorization of cities in the world, which in terms modernity and development of kept them apart. Her main point is to avoid the hegemony of western urban theory, as  well as the growing  strength of a discourse of development, which from 1970s onward has been emphasizing the differences between cities in the west and elsewhere, by: 

\deleted{\subsubsection{An Urban Development Process in an Ordinary City}}
\subsubsection{An Ordinary City in a Constant State of Change}

\begin{enumerate}
\item Dislocating accounts of urban modernity from the big cities of the west which claims to be its 
originator
\item Tracking  and  gather  differences  within  the  world  of  cities  in  order  to  justify  that  people  in different  places  have  invented  new  ways  of  urban  life  and  their  particular  production  and circulation of novelty, innovation and new fashions
\item Enriching  all  cities  with  better  future  perspective  depending  on  their  distinctiveness  and creative potential, without any hierarchical order among ordinary cities; they are all equal as 
sites of the production and circulation of modernity.
\end{enumerate}
In many scientific studies, interest lies in ordinary city constant state of change: dynamics and the structure of complex systems.

acknowledge the role of agency, decision making and urbanity

Max Weber, poses the city as a place of civility, civics and other formations of urban culture, and  the  non-urban  as  disordered,  chaotic  and violent.

\hl{waves of planning 2006}
case study choice - post-socialist choice:   most  unique  and  innovative  policies  and  processes  take shape under difficult circumstances [Harris 2001] and at times of change [Watson 1998]
the value of transitional and hybrid situations and attempting to identify characteristic  examples  for  his  typology  of  diffusion,  Ward  [2000]  found  that  most  variety  and subtleties are displayed in countries that are ‘neither the major Western world powers nor their colonies’
The evolution is presented as a series of cycles or ‘waves’, each resulting from internal and external influences that can substantially change the ways in which local planning systems operate.


\subsubsection{The Constitution of Urban Key Agents/Urban Agency}

\hl{ref (Volic et al, 2012)}:
"Bianchini introduces the concept of ‘cultural planning’ (Bianchini, 1999) which represents ‘a strategic and integral use of cultural  resources  for  urban  and  community development’ (Mercer, 2010). "
"Cultural  planning considers cultural resources as urban resources that play a strategic role in planning and in the
new economy (ibid)."
"In that way ‘a diversity of offer and mix of functions within the architectural context would be achieved, which could  lead  to  the  sustainable  city  renewal, together with a strong national or local support and  with  participation  of  local  population’ (Vaništa Lazarević and Đukić, 2006). "
focus on "soft infrastructure (daily  life, work and recreation, local rituals, ambience and atmosphere, a sense of belonging)."

\hl{Getimis XX, Peric 2016}:
Regulatory framework = Planning System
planning culture -  steering styles, norms, values, belief systems, visions and frames of the actors involved in the planning process
planning system: institutional, legal and regulatory framework

\hl{Mornings After Nedovic Budic 2011}
"establishing a planning system requires careful balancing of the roles of government and markets (Nedovic´-Budic´, 2001) and the roles of national and sub-national levels, especially given the potentially still important tasks retained at the state level as a developer and investor in capital projects, strategic planner, or only a “guarantor” (Djordjevic´, 2004).
\hl{Adjustments of Planning practice Nedovic budic 2001}

• define the planning context and roles from the legal, economic, and political perspectives (e.g., enabling legislation and planning mandates;
urban land value, ownership, and finance; and
planning process and activities);
• identify the new forms and patterns of urban development;
• evaluate the change in the plan-making functions and methodologies by focusing on the gaps and/or mismatch between the processes and tools used
and those required in the current urban planning context.



\hl{ref Peric 2016}:
glorification of the neoliberal principles without taking into account the public interest demand:
    Systemic factors:
nominal decentralization of spatial planning power
in reality: tight cooperation between the city and the national government - 
glorification of the neoliberal principles without taking into account the public interest demand
planning was transformed into an instrument of the ruling political party
    Cultural factors:
authoritarian role and personified leader - strategic decisions are fully in hands of the prime minister (questionnaire)
elementary ignorance of democratic decision making
deviation from path dependency manner in planners behaviour -  before planners acted in concert with the authorities, now planners are completely side-lined
planners in Serbia are not updating their competences over time (navigating and steering different interests and interest groups)
    Local factors:
experts are focused on the technicalities and technical knowledge, not on strategic decision making (path dependency - socialist traditions)

\hl{Vujosevic 2015 Regionalizam u Srbiji 2}
osnovna pitanja (pitanja od znacaja) za transformaciju "post-socialistickog planskog diskursa" u Srbiji (p56):

    "pravna drzava" u prostorno planskim okvirima
    upravljanje zemljistem i mehanizmi (od politickog ka planskom and trzisnom i vice versa)
    regulisanje evaluacije gradjevinskog zemljista
    selektivna decentralizacija teritorijalnog upravljanja

INDIKATORI UPRAVLJANJA (Kaufmann, Kraay, Mastruzzi)

6 dimenzija -> "level of governance perception"

    pravo glasa na izborima and odgovornost
    politicka stabilnost and odsustvo terorizma
    efikasnost vlade
    kvalitet propisa
    zakonska pravila
    kontrola korupcije
    
\hl{Cities in Transition 2013}
urban economy transition:
    national macro-economy policy reforms
    promotion of privatization
    structural economic change
    
\hl{Mornings after Nedovic Budic}
The idea of transition also implies an identifiable starting point—perhaps a ubiquitous “socialist city” (or planning) as described by French and Hamilton (1979)—and an end point—a “capitalist city” (or planning)

\hl{waves of planning 2006}
The  theory  of  transition  is  rooted  in  the democratization  theory  [26].  Among  other components,  this  theory  advances  the  outcomes  of  transition  as  being  significantly  influenced by the past (or pre-transition) and being ‘path-dependent

\subsubsection{The Morphology of Urban Decision Making}
On the contrary, according to one of the leading urban theories of David Harvey and Manuel Castells, urban planning cannot be seen as an autonomous process of spatial development, but rather it is situated in its political economic context and constantly overlaps current economic and social changes  (Taylor, 2006). In other words, urban planning in practice is intrinsically connected to the property market (which in turn involves a particular political ideology) and this tends to maintain current social order (Dear and Scott, 1981; Taylor, 2006), both of which are grounded in the development and expansion of industrial capitalism, neo-liberalism and consumerism (Ellin, 1999; Harvey, 1989). In other words, urban areas are, and have always been, the spatial and symbolic manifestations of broader social forces (Giddens, 1992).
maintenace, transformation and change of the system - from Friedman 1987

\hl{Vujosevic 2015 Regionalizam u Srbiji 2}
strukturisanje and uredjenje odnosa u interakciji and komunikaciji:
    drzava - javni interes
    trziste - privatni interes
    zajednica - zajednicki interes
    firme - korporacijski interes
    udruzenja - kolektivni interes
institucionalni kapital:
    politicke and administrativne jurisdikcije
    neformalne institucije (privatni, javn, volonterski sektor)
vrste kapitala:
    intelektualni
    socijalni
    materijalni
    politicki
    
\hl{Zekovic et al. 2015}
mayor characteristic of urban planning in transition are collisions and mixtures of:
    comprehensive vs incremental planning
    centralized vs decentralized decision making
    top-down vs bottom-up approach
    interventionist vs entrepreneurial - market driven urban management
    
\hl{Lazarevic Bajec 2009}
A significant change that can be perceived in the contemporary planning is the change from the ethically based decision making (in the public interest) towards decision making based on the dominant economic criteria.

\subsection{How to frame urbanity to grasp system dynamics}

Side by side with the historical continuum of global development patterns, socio-political framework at the neighbourhood level is shaped by human integration into the local environment. Appropriations, adaptations and modifications of space are the main agencies of physical interventions, which are followed by continual adjustments of its political, economic, and cultural structures (Sykora 1999). This process captures the pace of change and the multi-layered nature of transformation, with the focus on transitions in local economy, society, system of governance and the spaces of production and consumption. A systematic approach to such dynamics should integrate different modi operandi, transcend multiple scales and recognize temporality of information, actions and intentions that are followed up by satisfactory results (Tardin 2014). In urban theory the concept of urbanity is used as a parameter for the quality of urban spaces (ref), and in this way, we argue, it grasps and operationalizes urban dynamics.
\\
In this chapter, we will track the variety of definitions for “urbanity”, argue for its latest update in the course of Actor-network explanation and update it to better indicate the parameters of urban dynamics through the contextual processes of maintenance, transformation and change. We can distinguish two different standpoints on the matter – sociological and architectural. Sociological approach in urban studies uses urbanity as an association to the city referring to its original definition from Oxford dictionary. This definition dates back to the 16th century French interpretation of latin word “urbanitas”. In this way urbanity is closely related to civility  and indicates “cultural dimension and symbolic infrastructure of cities” (Zijderveld 2011). Conversely, urbanity concept for architectural research is generally the indicator of urban quality. For architects it sounds widespread and familiar at least in its normative sense as an articulated, “zero-friction” vision of urban development (Hajer 1999, Wüst 2005). However, the review of the literature on this matter shows that this concept is very often taken for granted and in architectural research used without explanations. On the one hand, several authors use the term for their analysis without clearly stating their understanding of it (ref). On the other, we deal with slight difference in individual clarifications from a number of authors.
\\
From the historical viewpoint, the term “urbanity”  is used as a qualitative indicator for cities and urban life. It has been a more common reference for German tradition of urban studies, policy and planning. In this scope, urbanity most often denote the urban way of life and properties of typical urban structures which historically pertain to the tradition of a “good European city” (Prigge 1996, Wust 2005, Lossau 2008). In this battle for dominance of either social or spatial reference, the first invokes socio-cultural dimension of cities (Durth 1987, Haussermann and Siebel 1997, Christiaanse 2000) while the second turns to their architectural and design qualities (Neuffer 1976). However, all of them agree to a certain point that acknowledging difference and heterogeneity as well as embracing fragmentation and contradictions in social and spatial sense are the prerogatives for accessible diversity and therefore quality of the urban (Durth 1986: 1838; Herterich, 1988: 273; Krämer-Badoni 1996: 75, Wust 2005, Marcus 2007). 
\\
The condensed meaning of urbanity as a structural continuity of spaces in cities has been adequate for application in urban design practice. This viewpoint was backed up by theories from Camillo Sité, Jane Jacobs, Kevin Lynch and Christopher Alexander (ref). This line of research has led to the formulation of parametric vision of urbanity as an architectural category for spatial configuration of urban spaces (ref). The idea of breaking urban space down into components is bounded up in space syntax set of theories and techniques. Nowadays, this is the major field for the practical application of the concept of urbanity. The most common definition of urbanity in space syntax domain explains it as “the generic need for people and societies to access differences as a means for social, cultural and economic development (Marcus 2007:10)”. In this respect, the operational definition of urbanity stemms from its integration in urban morphology  and refers to it as accessible diversity and efficient integration to locally capture the spatial capital (Marcus 2007). Such narrow-minded interpretation politicizes the term, artificially tames urban complexity, and debunks its relevance to interpret urban dynamics  and to deal with diversity, the unexpected and the non-planned in cities (Groth and Corijn 2005, Wüst 2005). Therefore, we would like to contest this reductionist view on urbanity as nothing more than an instrumentalized, aestheticizing perception filter (Münkler 1989, Wüst 2005).
\\
Without close scrutiny of the adequacy of these polarized explanations, we address the potential of the classical notion of urbanity to stand for not a condition but a process (Hortmann 1990). It may incorporate contradictions of value and identities that make the urban system nurtured locally and open to constant, flexible, spontaneous transformation or change (Groth and Corijn 2005). Even in the scope of space syntax, there has been a recent tendency to explain urbanity as an experience that incorporates all human and non-human actors and to analyse it with ANT (Rheintantz 2012). Urban reality in this way was amplified towards a heterogeneous, dynamic set of flows (deAguiar 2013).
\\
If seen as a processual property, urbanity concept brings us back to the sociological interpretations of urban way of life and urban culture (Farias and Bender introduction). In the light of assemblage and ANT theory,  urbanity is an immanent property that emerges within socio-spatial networks at multiple scales (Kamalipour and Peimani 2015). This approach emphasizes the idea that a city is just a fixed actualization of the urbanity in space-time (Farias and Bender p297). The biggest challenge for urban analysis, addressed by the practitioners of ANT, is the definition of “rational urbanity” offered by Spanish anthropologist Delgrade (Farias and Bender p211): “immanent instability and effervescence of forms to minimal sociality based on constant movement, ambiguity and transitivity and on the principles of anonymity and indifference”.
\\
This rather contested and open-ended articulation of contemporary urbanity aims at balancing top-down, market and civic-based governance roles, responsibilities and outcomes within the interrelations of biophysical and sociocultural urban elements – people and spaces, regulatory framework and urban structures (Groth and Corijn 2005, Tardin 2014, Holden et al 2015). The multifaceted character of man-nature and nature-culture interactions constitute urban dynamics over time (Tardin 2014). In these circumstances, identification of harmonized social practices defines the character of space and the corresponding agenda for social encounters. 
\\
Our line of research aims at combining parametric nature of urbanity with ANT description. In this way, the test of the urbanity level justifies opportunities for continuations, options, transitions and turnovers to reach the collective demand towards the common good in the public sphere (Holden et al. 2015:4). It has been widely accepted that spatial, social and human capital has been created from the accessible difference/discrepancy/change (Becker 1964; Coleman 1988) . Contextual resources are formed by making use of available social, human and spatial capital (Grönlund 2007). These resources reside in real and symbolic reconstructions and restructurings of everyday urban spaces and practices and could therefore be addressed as spatial capacities and social potentials on the local level (Swyngedouw and Kaika 2003). Benefitting from social practices maintain the urban system (Tardin 2014) and harnessing contextual resources indicate the possibility for transformations. Conversely, favouring dynamic, spontaneous nature of urban process propose positive vision of urban conflicts.
\\ 
When defined as such, urbanity may serve as a valid scope for setting up an urban agenda of development that offers categories for restricting complexity towards a structural unity of urban key elements. The categorized urban reality enables tracking urban dynamics through the identification of social practices, contextual resources, and urban conflicts that reflect maintenance, transformation and change processes in urban systems. Bearing in mind this complex vision of urban reality, it is the micro level, in our case the neighbourhood, where the test of urbanity could find its expression to the fullest in terms of the societal challenges and the production of urban spaces (Blotevogel et al. 2008).
\\
Network of All Active Agents and Contextual Resources

the relation between urban life and urban form creates potential (Marcus)
level of urbanity broadens the opportunity for change (Marcus)

\section{Epistemological Framework}

Production of space is the core aim of architecture, a discipline focused on practice. Consequently, architectural approach urges for parameters, categories and structure for its practice-based analyses. In terms of methods, there has been a significant number of interdisciplinary, transdisciplinary and multidisciplinary endeavours in applied research with an architectural focus in urbanism (ref). What is more, applied fields of research acknowledge the use of methodological hybrids (Datta, 1994, De Lisle 2011). This has open doors for applied social sciences to investigate new methodological opportunities when confronted with complex and multiplex social phenomena (De Lisle 2011). Even more so as methodological and epistemological rigidity leads to ignoring the realities of the practical and cause catastrophic scientif
\\
A very important issue of architectural approach for urban analysis is the added layer of societal challenges on the core of space production. We have recognized this socio-spatial mixture as “urbanity”. Following contemporary trends for labelling the complexity of urban dynamics, all human and material, social and technical elements are assumed to contribute together to continuous transitions. The approach with a potential capacity to afford such openness and flexibility is actor-network theory (ANT). Therefore, ANT has already been applied as the explanatory construct that studies associations and symmetrical relationality of all active elements of an urban environment (Farías and Bender 2011). The contribution of ANT for architectural analyses lies in: (1) the role of non-humans, (2) approaching the environment as a relational process, and (3) mapping the transitions through horizontal links and associations among actors (Latour 2005, ANT article). Although ANT enables exhaustive systematic description of urban dynamics, in concrete case studies it rarely brings up something new in terms of facts, analysis, conclusions. It provides a detailed description of confined urban environments, but meets its limits when confronted with complex real-life urban processes. Bearing in mind that this methodological approach hesitates to offer explanations and to analyse individual behaviours, it tend to fail on the operational level. In a nutshell, ANT “urges for methodological revisions, adaptations or complements  in order to facilitate a wider understanding of the undercover processes and mechanisms” (ANT article).
\\
The practical field of architectural research requires not only reliability and credibility of data, but also generalizability of the results. We have to consider that no single method is without its limitations, though to keep in mind not to limit the research to the shortcomings of only one method. Being aware of the advantages and shortcomings of ANT, we take into consideration mixed method approach in order to provide adequate scientific discourse and an operational framework for the research question. 

A General overview of Methodologies for 
Methodologies for understanding urban development complexity and its dynamics

\subsection{ANT in analysis of urban development}

In recent urban studies, the grasped complexity and dynamics of networked urban system has been extensively reinterpreted by Latour’s Actor-network theory (ANT), with all human, social and technical elements which are symmetrically treated within a system. All these entities contribute together to a dynamic perpetual networking, where understanding of all phenomena, including the social ones, lies in the associations among them  (Latour 2005). Differently put, it brings up the reproduction of inherent complexity and incompleteness of urban development in 3 gradual steps:  (A) labelling all active elements of an urban system (B) identification of their roles, and (C) focusing on the associations among them (Table 1). The contribution of ANT lays in: (1) instating socio-material topology of urban networks, (2) navigating the interpretative dualism of urban theory (nature/society, local/global, action/structure), (3-3) elaborating the supremacy of associations that configure the relational understanding of the city, (3-4) overcoming spatial hegemony in complex urban reality, and above all (3-5) radicalisation of  symmetry principle for human  actions  and non-human  materials  that  allows tracing the consistency and extensibility of urban phenomena beyond its spatio-temporal manifestation  (Latour 1993, Murdoch 1998, Farias and Bender 2011) (Table 1).
\\
Even though human is still the essential, inseparable urban element, this blending establishes new interpretation of cities as composite entity where all objects (physical spaces and structures, tools, technologies, data, formulae and regulations, institutions and, of course, humans) are mutually constituting through enactment, interaction and translation of different elements (Farías and Bender 2011)In  Latour’s (2005:71) words - "any thing that does modify a state of affairs by making a difference is an actor” – actor is granted activity by others, can be subject or object of an activity  (Latour 1996). As such, the heterogeneous body of associations and symmetrical treatment of humans and non-humans contribute to place action outside the actors where ‘[a]n “actor”... is not the source of an action but the moving target of a vast array of entities swarming toward it’ (Latour 2005:46). The figuration of a relation is what counts, not its nature, function or purpose; the network is established when arrangements between actors produce stable patterns of performance and practice (Smith 2003).
\\
ANT methodology redraws principal urban theory concepts in actor-network terminology – naming only a few: social order; scale, power, decision making, governance, urban development . The wide field of ANT application in urban research and practice addresses urban core by encompassing not only analytical views on theory and research (Boelens 2010), but also planning methodologies, policy and practice recommendations, and development prospects (Healey 2013). All these works adhere to the basic ANT principles: (1) treatment of material objects and representations through actor-networks; (2) reduction of well-known dualities and general concepts to in situ actors and networks; (3) the nature and process of networking in terms of associations and translations (Table 1).
\\
Being anchored in science and technology studies (STS), an early applications dealt with the nature of human/non-human exchange in mapping land cover projects and GIS allowing reconciling data with different ontologies and addressing "nodes, links and type of links" terminology оf actor-networks (Comber et al. 2003). The analytical lenses in architectural, housing and planning studies have focused on materiality/artefacts/objects and up-to-date fruitful application of ANT: (1) for identifying non-human actors which happen to be missing, silenced, or even rendered invisible in practice of housing system, markets and policy (Gabriel and Jacobs 2008); (2) as an interpretative tool for processes and mechanisms under review distinguishes active mediators and passive intermediaries (Cowan et al. 2009); (3) as a theory of action for interpreting complex associations of people and things in architecture (Fallan 2011); (4) for demystifying the complexity of stabilizing/destabilizing object enactment mechanisms as a way to readdress the position of ‘plan’, ’implementation’ and ‘design’ in governance and planning process (Duineveld et al. 2013); (5) for assessing relational aspect of assemblages as a way of explaining the influence of innovative tools for spreading explicit and tacit knowledge in planning and building sustainable cities (Georg 2015) (Table 1).
\\
Furthermore, network related ANT framework has been stretched to analytical research tendencies toward urban practices. While Doak and Karadimitriou rely on Callon’s four steps in actor-network translations  (Callon 1986) to map complex redevelopment processes once reduced to a set of associations in social relations and material objects and stabilized by intermediaries (Doak and Karadimitriou 2007); Holifield  advocates for the version articulated by Latour (1996) and political usefulness of ANT suggest using  intermediary/mediators role of risk assessment changes as a tool for "tracing the (contested) assembling without taking the existence of social relations" like capitalism and class "for granted" (2009:647). Similar stance has been taken by Boelens (2010) to promote relational view on spatial planning and how it interacts with behavioural urban regime in a way that ANT serves to identify actors and see how they organize from the ground up, and not being identified from above through an objective, vision or plan. ANT seems to have been recently gaining attention as part of the wider poststructuralist approach to cities  (Smith 2013), therein further emphasizing its role in the process of production and acceptance of associations in terms of evaluating the positionality of researcher agency in human geography (Ruming 2009), and  reflecting the process of production and acceptance of associations in urban enclaves (Wissink 2013).
\\
Even though this post-structuralist ANT tenet mainly holds on flattened, network-oriented interpretation of system dynamics, it has been recently argued that the role of material objects must also be acknowledged in all its vigour and heterogeneity. Tracing back non-human elements from Latour to Foucault, it is obvious that material objects can be everything but passive and have been playing various social roles such as: (2-1) reflecting and maintaining social order, (2-2) facilitating social relations, (2-3) moral and political signposts, and (2-4) intermediaries of the social across space and time (Sayes 2014; Van Assche et al. 2014) (Table 1). Henceforth, non-humans, when granted agency, become intermediaries/mediators and actors and their active engagement in urban development refers back to various levels of urban decision making: (X) upholding legitimacy of urban planning, (XX) underpinning multiple realities of real-estate interest, and (XXX) personalizing participatory urban transformations through actor-network perspective (Latour, 1996; Rydin, 2010; Latour et al., 2012; Van Assche et al. 2014) (Table 1).
\\
ANT seems to continue to provide a conceptual framework for interpreting and guiding various ways of examining networks and has demonstrated a substantial coherence as “a pragmatic approach to study actual practice in concrete sites and situations” (Farías and Mützel 2015:526), which affords focusing on description of performativity of the black-boxed social world through: (1) active role of non-humans, (2) sociology of translations (3) free associations, (4) inseparable actor-networks, (5) urban assemblages (Latour 1996 ). The concept of assemblages is aptly after capturing the complex relationality of dynamic urban system, though it fails to go beyond ‘follow the actors’ technique of examining human-human-nonhuman interactions (Cowan et al. 2009) and to facilitate wider understanding of their normative and transformative nature (Gabriel and Jacobs 2008).
\\
The rudimentary yet hyper dynamic circumstances of transitional societies offer an insight from within the network on how the body of norms, projections and structures of urban development unfold and upon the network of how the associations and translations of basic elements are formed and developed. In Serbia, urban planning framework withstands complex and elaborated institutional legacy yet holding rather a symbolic meaning (Nedović-Budić 2001), fragmented and uncontrolled spatial transformations is governed by the constellation of different, often illegitimate, interests (Petrovic 2009), and, on site, the spectrum of active but powerless urban actors struggle to develop flexible social patterns and networks (Cvetinovic et al. 2013). Therefore, the case study of a post-socialist neighbourhood in the capital of Serbia is a good illustration for observing the relationships between top-down urban planning, interest-based urban transformations and bottom-up urban design activities. Moreover, very few methodological research studies bothered to examine urban  development modalities  in  transition,  apart  from  replications  of  the  approaches  taken  by  neo-liberal  or institutional economies (Tsenkova 2007). In this respect, we aim to examine utility of ANT analysis for understanding developmental reality of Savamala neighbourhood in Belgrade.

\subsection{Multi-Agent System}

Very important for developing hybrid methodologies is the correspondence of the individual epistemological framework (ref). In this respect, we have realized that the dynamics of urban reality interpreted by ANT matches the concepts of agency, communication, cooperation and coordination of actions, where all elements influence each other simultaneously (Ferber 1999). This interpretation corresponds to the Multi-agent system (MAS) approach for complex computing systems.
\\
This approach has already been applied in urbanism as a simplified problem solving strategy primarily used for the urban decision making process. The multiple urban actors and stakeholders are converted into agents and used for simulating social organisations in which these agents are embedded (Bousquet et al. 2004). A multi-agent paradigm is actually very useful as structuring method that gradually builds the capacity and flexibility of systems. Its potential lies in analysing  operationality, functionality, usability and extensibility of decision making mechanisms on urban land use and land cover (D. Brown et al. 2005), housing market dynamics (Diappi and Bolchi 2008) and Planning Support Systems (PSS) (Saarloos et al. 2008). MAS methodology is, in fact, a process generation tactic based on the principles of the ecosystem management (levelling, fluctuating, evolving). It applies the technique of categorizing the process infrastructure with apparatuses (set of fields of influences and major forces) and procedures (set of operational agencies) (Bousquet and Le Page 2004).
\\
In general terms, MAS go along with ANT as it also aims to explore and understand the system, not to predict the future. However, the radical difference that contributes to its operationality is the focus on spatio-temporal dynamics. MAS tests the impact of interactions and structures that emerge from these interactions (Crooks, Patel, and Wise 2014). It can therefore serve to complement actor-networks (ANT) with a systematic framework where MAS analysis of agent behaviours provides fine tuning for qualitative discrepancies in the system. 
\\
The characteristics of MAS which are very useful for an operational update of ANT in exploring urban dynamics are:
1.	Profiling elements as agents - the complex system is divided into subcategories, all of which are identified as independent subunit (agents) and then the activity among these subunits is coordinated. This allows for agent typology, “an object-oriented approach and, as such, to distinguish actors, activities, flows, investments, facilities, regulations, rights, issues, forces, opportunities and constraints” (Hopkins 1999; Saarloos et al. 2008). Distinguishing active-passive roles of the agents (proactivity, sensibility, capacity for interaction) may be crucial for representing real forces in an urban environment. 
2.	Describing the impact of procedures/agencies by categorizing the agents accordingly (Arsanjani et al. 2013). These agencies in our interpretations may be maintenance, transformation and change patterns.
3.	Exploring a generative bottom-up typology of the system by defining rules that govern urban dynamics (Bretagnolle and Pumain 2010). Identification of the rules facilitates bridging the gap between top-down (evaluation of global trends) and bottom-up agent behaviours (local decisions which lead to emerging landscape patterns over time) (Bone et al. 2011). 
4.	Analysing complex systems through the agent-based view on urban decision making (links among agents' perceptions, representations and actions), control (hierarchical relations among agents) and communication (the syntax of the interaction between human decision-makers and biophysical changes) (Bousquet and Le Page 2004; D. G. Brown et al. 2008).
5.	A multi-agent model for simulating urban dynamics - The aim is to understanding and exploring the system. The model is able to describe the emergent phenomenon and the dynamic behaviour of the system,  and to draw some consequences on the environment and agent behaviour (system dynamics) (Diappi and Bolchi 2008). The primary modules of such model are borrowed from its application for programming the systems in computer sciences and they envision: (1) E – environment; (2)  A - assembly of agents; (3) O - set of objects; (4) R - assembly of relations, the agent’s interaction with the environment (agent behaviour); (5) Op - assembly of operations making it possible for the agents (A) to perceive, produce, consume, transform and manipulate objects (O) through their relational behaviour (R); (6) U - laws of the universe, the reaction  of  the  environment  to  this  attempt  of  modification (Brown et al. 2005).
\\ 
As elaborated in the previous chapter, the level of urbanity reflects multilateral, multichannel  nature of cities that incites not only the constellation of social practices and harnessing contextual resources, but also evidences the production and the challenge of urban conflictual issues. A constant change of urban actors and urban structures also accelerates flows of social practices (policies, actions and processes) that together induce the complexity and diversity of city life, build urban experiences and urban capacity (Robinson 2006). Urban conflicts thrive on discriminatory power dynamics, clashes of cultural differences and a series of confrontations of opposing viewpoints within a city and they tend to progress from a personal level to a socio-urban dimension. Contextual resources are transformed agents enabling horizons of possibilities in spatial and social sense.
\\
This is the theoretical ground on which our hybrid methodology identifies the concepts for its categorical convergence. For our case study, the combination of MAS and ANT methodological approaches involves taking into account all active agents regardless of their sort (ANT), their interdependencies and interconnections (ANT and MAS), and map their contributions (MAS) to continuations, transitions and turnovers of the urban development on the neighbourhood level.

\section{Local framework}

\subsection{Socio-spatial patterns of transition \deleted{a post-socialist city}}

<hl{Thomas 1998}
macro integration factors of political systems in pre-transitional period in CEE, which initiated the transition (one of them each time):
    cultural identity
    political regime
    mode of production
Transition theory:
    emphasis on democratization through 5 arenas
        civil society
        political society
        the rule of law
        state apparatus
        economic society with an institutionalized market
    categorization of pre-transition period
        non-democratic political systems
    path dependency - determine the influence of the past on the path of transition
    international influence is crucial

\hl{Mornings after Nedovic Budic}
The idea of transition also implies an identifiable starting point—perhaps a ubiquitous “socialist city” (or planning) as described by French and Hamilton (1979)—and an end point—a “capitalist city” (or planning)

\hl{Adjustments of Planning practice Nedovic budic 2001}
eastern and central european context - a laboratory of changes for the west (Maier 1994)

Sjoberg - theory on post-socialist

In such a situation, urban planning was not a priority (Sykola, 1999), and it was not considered effective for managing local urban issues (Maier, 1998; M. Vujošević and Nedović-Budić, 2006). Therefore, planning was narrowed down to just one technical issue and very few theoretical or general methodological research studies bothered to examine alternative planning modes in transition, apart from replications of the approaches taken by neo-liberal or institutional economies (Begovic, 1995).
\\
Thus, the crucial failures of post-socialist urban planning have come about through the lack of consensus on priority goals, action-oriented programs of implementation and coordination of different levels, sectors and areas. 
\\
In transitional countries, the course of merging socialist and neoliberal socio-economic condition, regulatory practices and organizational solutions led to inefficiently operationalized and inconsistently formalized institutional reforms rather known as "growth without development" \hl{Vujosevic}.
The post-socialist urban governance fails substantially through the lack of consensus on priority goals, action-oriented implementation and horizontal and vertical coordination.
\\
\added{Due to these circumstances, the urban development of post-socialist cities is perceived as a multi-dimensional integrated system composed of qualitatively different and semi-autonomous processes, with the inclining tendency to improve the economic, social, demographic, political and technological state of an urban environment.}

\section{Theoretical Framework}

general - social sciences

urban studies

urban development

decision making

urbanity

summary of the chapter, visualization of the framework and conclusions

%%%%%%%%%%%%%%%%%%%%%%%%%%%%%%%%%%%%%%%%%%%%%%%%%%

\chapter{Methodology}

%%%%%%%%%%%%%%%%%%%%%%%%%%%%%%%%%%%%%%%%%%%%%%%%%%
Before delving into the data \added{sampling} and outcomes of this research, it is crucial to delineate the research process and procedures. Within the scope of this thesis, the research process involves the development of an organized body of knowledge on urban development processes in post-socialist cities. The aim of this chapter is to justify the choices made about what and how to research and the means to collect and analyze the data.
\\
The chapter starts with a presentation of a larger framework where the research objectives presented in the introduction are conducted into the context-specific research questions and working hypotheses. In the following, an explanation for the choice of case study method, the criteria for the case study selection, as well as mixed method methodological approach are listed, along with a brief overview of the methods and techniques used.  

\section{Research Framework}

This thesis starts from the trendy term of urban development in order to scrutinize urban complexity and dynamics in a more operational, procedural manner. The following layers of this research, reflect its challenging nature:

\begin{enumerate}
\item trace and propose a value-neutral definition of urban development and identify the corresponding concepts that comply with it;
\item elaborate the validity of a post-socialist neighbourhood as a case study that blends and reveals the complexity and dynamics of a modern urban context;
\item apply Actor-network theory framework for the descriptive analysis of a post-socialist neighbourhood;
\item construct a MAS-ANT visual hermeneutic set, an engine for agent-based representations of urban dynamics.    
\end{enumerate}

The logistical construction of the inquiry involves an exploratory journey through facts, phenomena and theories of a conceptual framework within urban studies using the proposed methodological hybrid of Multi-agent system and Actor-network theory. The fundamental question stays the same: it is crucial to understand what is going on in cities under the hood of urban development and especially how it is occuring.
The current body of knowledge on this matter gives us an input on how to transform and adapt the general concepts mentioned into indicators of the complexity and dynamics of urban development processes. Theoretical framework has provided the foundation of facts, phenomena and theories in this direction, by acknowledging the conversion of general concepts into indicators as follows:
\begin{enumerate}
\item \textbf{Concepts into indicators}
\begin{itemize}
\item urban development - dynamics of urban processes;
\item urban agency - urban key agents and urban networks;
\item urban decision making - the morphology of urban decision making layers: top-down, real estate, bottom up;
\end{itemize}
\item \textbf{Indicators into dependent variables}
\begin{itemize}
\item dynamics of urban processes: urbanity and the morphology of urban decision making;
\item urbanity: socio-spatial patterns of urban transitions - urban transitions and socio-spatial patterns;
\item layers of urban decision making: urban key agents and urban networks;
\item urban transitions (maintenance, transformation and change processes): urban networks and socio-spatial patterns;
\end{itemize}
\item \textbf{Independent variables}
\begin{itemize}
\item human and non-human agents;
\item urban networks;
\item socio-spatial patterns (social practices, urban conflicts, contextual resources);
\end{itemize}
\end{enumerate}

Bearing in mind this extensive re-categorization and structuralization of urban development, MAS-ANT methodological hybrid proposes the road map for an inclusive and flexible approach for exploratory research - describing, tracing and representing dynamics of urban processes. Actor-network theory illustrate urban agency and decision making concepts while Multi-agent system operationalizes urbanity concept at an qualitative level and brings up the logics of the whole MAS-ANT procedure. Such provisional statements shed new light on the overall research questions proposed in the introduction and makes this thesis a methodological exploration.

Desired outcomes are dependent on the success of cross-pollination of concepts through MAS-ANT mixed research method. They are intended to influence both theoretical and practical domain. The research is guided in the way that it:
\begin{itemize}
\item questions the concepts of urban development, urbanity in general and urban decision making in post-socialist cities;
\item proposes the terminology of transition which connects the processes of maintenance, transformation and change to urban conflicts, social practices and contextual resources at the local level;
\item invents visual interpretations for practical uses.
\end{itemize}

Consequently, the research is built on 3 hypotheses. Each hypothesis addresses a theoretical and a methodological issue and they are drawn in an consecutive order. Hypotheses justification is built gradually on describing, exploring and proceduralizing in order to master complexity and dynamics of urban development processes. 

\subsection{Context-specific Research Questions}

\textbf{Overall research question:} How To investigate socio-spatial patterns of post-socialist cities in order to reinvent a more inclusive and flexible approach to understanding Urban development processes engaging the complexity of an urban context? 

\subsubsection{RQ1}
What constitutes an inclusive approach (complexity and dynamics) to urban development?
\begin{itemize}
\item RQ1a: (indicator: figuration of human and non-human elements as urban key agents) What constitutes spatial and social differences and specificity in an ordinary city? 
\item RQ1b: (indicator: urban networks of all human and non-human elements) How do cities as specific socio-spatial phenomena are manifested through urban dynamics?
\item RQ1c: (indicator: morphology of urban decision making) Why does the morphology of urban decision-making determine pathways for urban development (urban transitions)?
\item RQ1d: (indicator: urban transitions) How do urban transitions redefine/describe urban complexity and dynamics in terms of ordinary cities \hl{doctrine}?
\end {itemize}

\subsubsection{RQ2}
Why do the level of urbanity traces determine pathways for urban development dynamics (urban transitions)? 
\begin{itemize}
\item RQ2a: (indicator: socio-spatial patterns in terms of local urban conflicts, social practices and contextual resources) What are the conditions for specifying the level of urbanity in an ordinary city?
\item RQ2b: (indicator: \deleted{contextual processes of} urban transitions) How to frame contextual processes to embody the dynamics of socio-spatial patterns in post-socialist cities?
\item RQ2c: (indicator: urban dynamics) How does the level of urbanity systematically approach urban transitions?
\end {itemize}

\subsubsection{RQ3}
How to frame urban development process to embody complexity of urban systems and dynamics of urban transitions?
\begin{itemize}
\item RQ3a: Why does tracing the level of urbanity within the morphology of urban decision making embody the dynamics of urban transitions?
\item RQ3b: How to frame the morphology of influences among different decision-making levels to describe/interpret the complexity of urban networks? \deleted{a dynamic urban development model of an ordinary city in a constant state of change} \hl{H3a: decoding urban dynamics}
\item RQ3c: How to design the framework for action in order to operationalize urban development concept?
\end {itemize}

\subsection{Research Hypotheses}
\textbf{Central hypothesis:} Urban development process, interpreted through MAS-ANT methodological approach, embodies networks of urban key agents initialized by the morphology of urban decision making and determines its level of urbanity.
Such relational \added{system} \deleted{object/structure} is a transparent engine for capturing the complexity and dynamics of \hl{urban transitions (urban development processes)}. 

\subsubsection{H1}

H1: Actor-network theory (ANT) gives an exhaustive image of the complexity of an urban environment (neighbourhood) by providing openness and flexibility for describing urban processes: figuration of human/non-human agency, blurred and askew morphology of urban decision  making and networks of all active urban key agents.
Breaking  down  the morphology of influences among different decision-making layers (top-down urban planning, real estate transformations, bottom-up participatory activities) through mapping networks of interactions and interconnections among urban key agents (urban actors, built environment, space, regulatory framework, infrastructure, 
social practices) clarifies the agency of urban transitions - urban development processes - in a post-socialist city.

\subsubsection{H2}

H2:  Urban development processes  is  determined  by  upgrading the  level  of  urbanity.  In value neutral sense, an overarching definition of the level of urbanity improves scientific capacity for grasping urban dynamics. The level of urbanity analysed through Multi-agent system (MAS) methodological approach indicates opportunities for urban transitions (maintenance, transformation, change) within socio-spatial patterns of an urban environment. 

\subsubsection{H3}

H3: Complex urban development processes set as an iterative procedure of tracing the level of urbanity within the urban agency map reinterprets a multi-layered morphology of urban decision making in terms of system flexibility and dynamics (transformation, maintenance, and/or change). A methodological hybrid that combines Multi-agent system (MAS) and Actor-network theory (ANT) offers a framework for capturing urban dynamics and reframing urban complexity at the neighbourhood level.

\section{Research Design}

The aim of this section is to present the research reasoning and the adopted methodology, namely the logical sequence that connects empirical data to the research questions, hyptheses and their conclusions. In designing the research process, the defined goals are assumed as exploratory in its nature and methodological in terms of urban studies. The study is gradually built from specific observations of the literature towards an in-depth analysis. An exploratory standpoint is chosen with regard to theoretical and practical strivings of the research. This division is crucial for establishing the research methodology. The first (theoretical) relies on secondary data and inputs theoretical constructs, while the second (practical) provides primary data and empirical evidence from the study field.
\\
The theoretical summary of urban development processes and the critical overview of the corresponding general concepts from urban theory (urbanity, urban decision making) is performed in the literature review in Chapter 2. It works as the structural catalyst for the chosen methodologies, as a kind of cross-pollination of concepts within the MAS-ANT methodological scope. On the other hand, MAS-ANT methodological approach is practically tested through the case study method. The application of this methodological hybrid in an hierarchical order (first ANT than MAS) for the analyses on the selected case study enables practice-oriented understanding of the situation in post-socialist neighbourhoods. Data display at the end gives an outline of the final blending of MAS and ANT methodologies and how they re-order and re-interpret the field data. This synthesis aims at transferring tacit into explicit knowledge about urban development processes in post-socialist neighbourhood in Belgrade.
\\
The so-called cross-pollination procedure justifies proposed indicators (operational definitions of the concepts used) and enables connections among independent and dependent variables constructed within the research hypotheses. This is the core logical construction of the research inquiry. The research further follows an inductive method of reasoning within the case study. The point of departure was the case study. Interpretative and participatory action research methods are used for the data collection. These qualitative methods are overlapping case study in order to support proposed theoretical categories (indicators and variables). Principal data sources were documentaries, open-ended interviews, workshops, and questionnaires, which contributed to the structuralized description of post-socialist empirical analysis performed with Actor-network theory (ANT). Multi-agent system (MAS) further made use of qualitative evidence to elaborate urban networks and the involvement of the key agents in urban affairs. Finally, MAS-ANT diagram displays the research results and facilitate interpretations of maintenance, transformation and change processes in an urban environment.
\\
The main study focus is to invent a looping procedure which examines the relations among the variety of urban elements, explores the "specificities and globalities" of the particular context, and catalyzes the framework of action on the neighbourhood level. The reach of this research is incremental, open-ended procedure-building based on pragmatic approach through iterative and collaborative techniques towards:
\begin{enumerate}
\item understanding the phenomenon,
\item creation of an overall framework,
\item identifying the pattern of dynamic reality in terms of urban transitions. 
\end{enumerate} 
The entry point for this methodological exploration is a case study.

\subsection{Case study}

This research adopted an in-dept case study inquiry as the adequate method for collection and framing of empirical data. Case study serves as a data collection engine, catalyser and boundary framework. Of particular importance is exploratory and descriptive character of the case study method. In general, the first captures the process, the second prepares and illustrates the incidence/prevalence of the phenomena \added{(Yin 1994)}. These features provide us with a comprehensive framework for describing contemporary phenomena with extensive types and sources of data \added{(Feagin, Orum and Sjoberg 19919)}. The goal is the holistic description of urban development and understanding the processes at stake over time \added{(Wanborn 2010)}. In this manner, the case study takes embedded approach with multiple units of analysis \added{(Scholz and Tietje, 2002; Yin 2009)}: urban key agents, the morphology of urban decision making, urbanity and urban transitions. These units of analysis define the scope of investigation - which elements are studied in detail and which processes are to be excluded \added{(Harrison 2002)}.
\\
Therefore, case study offers discovering complexity of urban development processes and recounts their dynamics by adding the time dimension to the analysis \added{(Feagin et al. 1991)}. However, the set of well-known components for \textbf{designing a case study} triggered its application in this research, such as \added{(Yin 2009)}:
\begin{enumerate}
\item focus on HOW and WHY questions about the researched phenomenon;
\item units of analysis, information relevant for the case construction, depend on the definition of research questions;
\item exploratory nature of research hypotheses, as each proposition is built on something relevant within the scope of the study and for one or more units of analysis;
\item linking findings to the hypotheses, units of analysis and theoretical background, i.e. "pattern matching" \added{Campbell 1975};
\item data collection focus for the case study, while testing methodologies and existing theories provide rich theoretical framework therefor.
\end{enumerate}
Case study is commonly but not exclusively applied in sociology. In general, it is used for grounding observations and concepts about social phenomena in their natural setting. Consequently, it has been increasingly put to use in other disciplines including urban studies and architecture \added{(Feagin et al. 1991)}. Even though major critics are directed towards single case research focus and doubts about the scientific generalizations based on an individual case, this research builds on Flyvbjerg's elaboration \added{(2006)} that careful and strategic choice of cases and thereafter the units of analysis contribute to the collective process of knowledge accumulation. Advocating the case study scientific relevance, Flyvbjerg (2006) distinguishes several selection strategies: random, extreme, multiple, critical and paradigmatic cases. Information oriented selection, based on the expectations about the information content is the most proper for the scope of testing methodologies. For example, extreme case circumstance enable close examination on the embedded units of analysis.
\\
Flyvbjerg (2006) also states that the descriptive manner chosen herein puts forward the path for scientific innovation, which in this thesis is hybridization of methods for urban data anlysis. Henceforth, the most important here is this opportunity for application of multiple methods \added{(Yin)} and consequently methodological hybrids, MAS-ANT. Data obtained from the case study aim to contribute to objectives of the research by providing the local layer with real-life data. Accordingly, case study enables testing such methodological approach by systematization and validation of case study data analysed by the involved methods, Actor-network theory and Multi-agent system. In these circumstances, case study is referred to as a sort of data sampling strategy, used to select, manipulate and prepare a representative subset of data points for the analyses by the chosen methods. It delivers patterns, trends and structures in the larger data-set afterwards.
\\
Case study research process is broadly divided in three parts: designing, conduction and reporting. For \textbf{conducting case studies}, the most important is to ensure variety but also convergence of data. It is essential to have  sampled  sufficient  points  of  view  to  develop  a balanced picture \added{(Harrison 2002)}, but also to provide converging lines of inquiry within the multiple sources of evidence \added{(Yin 2009)}. Case study usually involves  variety of data sources, both human (interviews, workshops) and non-human (documentation, archival records, direct observations and physical artefacts). With this plenitude of data, the phenomena and processes become supported by multiple data sources and ensure constructing validity through triangulation \added{(Denzin 1987b}. In this research, triangulation is applied on two levels \added{(Patton qualitative evaluation and research methods)}:
\begin{itemize}
\item data triangulation,
\item methodological triangulation. 
\end{itemize}

Therefore, one of the main reasons for choosing case study is structured and bounded data plan and its incorporated units of analysis, cross-referencing methodological procedures and the resulting evidence triangulation with the mixed method \added{(patton 1987, Yin)}.
\\
As of this thesis, case study is the part of larger multi-method study and \textbf{reporting} is reduced to the general structuring tactic for 
the descriptive data about the selected case. However, documenting relationality between the research problem and the case and constructing validity is elaborated within the reasoning for case study selection. This whole research design is linear-analytic in its structure: its starts with the issue of problem and the literature review, then present the logic of research design and  chosen methods, findings from data collection and analysis, conclusions and implications. The data collection process through case study method will retain the same linear-analytic manner in its descriptions and implications in a broader scope and take the chronological course according to the causal sequences within case history. In order to maintain the chain of evidence, case study presentation must \added{(Yin)}:
\begin{enumerate}
\item explicate and justify the boundaries of the case;
\item design the research according to the known constraints;
\item indicate exhaustive data collection process;
\item consider alternative perspectives and different points of view;
\item display sufficient evidence
\end{enumerate}

The case study should illustrate, in great depth and clarity, the embedded units of analysis, which are being researched through the MAS-ANT methodological hybrid. Such research design encircle hypotheses testing by logical argumentation for building the methodological framework and simulation of the framework application on the case study. The choice of case study method for data collection is justified by its feasibility for structuring the chain of evidence and confirmability of data by triangulation. On the other hand, reliability of the case study method is determined by its ability to document the methodological procedure with data and its external vality by the transferability of the procedure in other contexts and cases.
\\
Limiting the case study method to the data collection reduces the risks of common deficiencies of the method. Unreliability of soft data is dealt with ANT flatten reality approach, while researcher subjectivity in interpretations and selections is prevented with methodological rigidity in classification and interconnection of data. Finally, in multi-method research there is no need for explanations and internal validation of the case study logic. Moreover, generalizations are reduced to the analytic ones on the methodological level, in terms of categories and networks. In this research, systematization of collected data are used for further analyses and case study is rather a narrative of urban development as an contemporary social process within its real-life context. Therefore, the selected paradigmatic case should be a valid representation of a setting suitable for extensive application of the proposed methodological hybrid within data analysis and data display procedures.
\\
\textbf{"The case study can enable a researcher to examine the ebb and the flow of social life over time and to display the patterns of everyday life as they change." (Feagin et al. 1991)}

\subsection{Case study selection}

As it has been stated multiple times, this research is established on the basis of mixed method and its adequacy for research on complex and dynamic urban phenomena. Verifying such methodological hybrid in practice means that proposed categories and mechanisms within the methodological procedure should be tested by further research activities. These activities include observing real-life context, putting forward the coined phenomena and relations and postulating the correspondence of proposed methodological structures to the reality, which, if exist, answer the research question. In a nutshell, this methodological framework act as an a prior logic to explore particular instances, but it still must account for their various deviations, and aim at few conclusions that contribute to the general, scientific body of knowledge on the urban. The case study point of view herein is deduced from general statements and signifies as a derived, localized, contextualized form of researched phenomenon, in this case urban development process. It gives overview of relations, factors and influences in a holistic manner, providing understanding of a phenomenon (unit of analysis) within its operating context \added{(Harrison 2002)}.
\\
The case study is used in the first stage of the research process, whereas other methods (Actor-network theory and Multi-agent system) are suited for hypotheses testing and conclusions drawing. However, strategic case study selection is crucial for this research in order to maximize the utility of information from a single case and small samples for the units of analysis \added{(Flyvbjerg 2006)}. Therefore, the case is selected on the basis of expectations about the correspondence of its data content to the proposed methodological hypotheses. The elaboration of paradigmatic case study for this research is based on \added{Yin's (XXXX)} criteria: (1) exhaustive data collection process with sufficient interpretative and artefactual evidence, (2) multiplicity and variety of data sources, especially human, but with (3) explicit case boundaries and (4) precise data constraints.
\\
(1)(2) Case study database is built upon investigator's report (narrative, notes, tabular material, diagrams etc.) and the quality of reporting depend on an extensive evidentiary base. In order to provide exhaustive evidence, case study choice should rely on heterogeneous data sources \added{(Yin)}:
\begin{itemize}
\item exact documentation, archival and qualitative data and records documentation (service records, maps, charts, lists, survey data, personal records);
\item physical artifacts (tools, instruments, works of art)- insights into cultural features and technical operations
\item interviews targeted focused insightful depend on the construction of questions
\item participant-observation - workshops -  as direct observations, insights into motives
\item direct observation - visiting the site. cover changes in Savamala over time -  covers events in real time contextual cover context of events weakness: selectivity, reflexivity
\end{itemize}

Bearing these in mind, the best circumstances for a comprehensive data collection is knowledge of the local language, previous general knowledge of the context, professional connections and extensive site visits. Even though foreknowlodge can impact the neutrality of the researcher; within this research project, quick and systematic understanding of the local context of urban development process facilitated data collection and improved flat classification, a well-known feature of Actor-network theory. 
\\
(3) In his argumentation on case study method in management research, \added{Harrison (2002)} argues for its maximal benefits in the circumstances "where the theory base is weak and the environment under study is messy". This also contribute to rejecting general misconception stating that theoretical (context-independent) knowledge is more valuable than concrete, practical knowledge. Moreover, following \added{Flyvbjerg (2006)}, case study can be extremely useful for transferring tacit (context dependent knowledge) into explicit, general knowledge. Explicating the domain of the practical knowledge -
any historical background to the research problem, its time-space transitions and the immediate political, economic and cultural circumstances where it emerges and evolve should be taken into consideration as a chronological sequence. Fine tuning of these various factors and processes present on site and  - if properly described - provide an adequate capacity to explain correlational links among identified urban key agents of urban development.
\\
(4) Finally, by placing high priority to abundance and calibration of data, it become less likely that an overall scrutiny of relations, behaviours and processes could be possible in a wider context. Regions, metropolitan areas and cities could be difficult to handle through the embedded units of analysis. Therefore, neighbourhood level of analysis is already fixed by the hybrid method.
\\
Consequently, the case study choice retained neighbourhood level of analysis, but the one that bounds up all recognized indicators of urban development process. On the other hand, the best option is that the  researcher  is  to  some  extent  familiar  with  the  local  context and  is capable of accessing certain data.  My native country of Serbia with its turbulent  burgeois and socialist past and transition of today is taken as an exemplary case of intensive congregation of factors, trends and power struggles in one place. The adopted case study field is Savamala neighbourhood in Belgrade, a historical but deteriorating city quarter in Belgrade, where a set of bottom-up urban transformations and participatory spatial interventions are colliding with top-down imposition of master planing and swift, investor-based developments in the area. This multitude of influences with different sources and extensive but limited time-span give an opportunity for a holistic study of complex social networks and processes. 

\textbf{"Case study research is flexible and can be adapted to many areas of knowledge creation. And the researcher is continuously confronted with the question ‘does this make sense?’" (Harrison 2002)}

\subsection{Local Context of the Savamala Neighbourhood Case Study}

The choice of case study method for data collection is most suitable when
the  contextual  conditions  are  believed  to  be highly relevant for the phenomenon being explored \added{(Robson, 1993; Yin, 1994)}. The hypotheses of this research were examined within the real-life context of Savamala neighbourhood in Belgrade as an exploratory basis for building the methodological framework of analysis for urban development processes. 
\\
Selected Savamala neighbourhood case study should should feed MAS-ANT analytical framework with site-specific data on \added{(Harrison 2002)}:
\begin{enumerate}
\item context - global and local, outer and inner in reference of time and space;
\item content - urban key agents and urban decision making layers that put forward urban development processes;
\item income and outcome variable - link the process of transitions to urban elements and networks.
\end{enumerate}

The boundaries of the research are spatial, though Savamala neighbourhood is not an administrative unit nor it has its own local authorities. It is rather a place on the mental map of Belgrade and important landmarks of the city, than an official unit and the exact spatial boundaries are drawn according to the survey conducted among professionals and citizens.
\\
This neighbourhood is a scaled example of pre-socialist material legacy, socialist cultural and societal matric, a transitional reality and a condensed case of its multi-faceted circumstances of post-socialist urban development. These also frame the epistemological constraints of the case study research in this case.
\\
The units of analysis comprise a knowledge-based chain of decision making and a dynamic, interactive process of interdependences and interconnections among all active urban key agents and contextual resources identified in Savamala through the qualitative inquiry exclusively.
\\
To sum up, urban agency and urban networks are not spatially bounded phenomena, they develop as the products of interaction between human and non-human elements in  particular localities that contribute to an understanding of the broader urban systems and enhance theoretical frameworks. \added{(Giddens, 1984; Grubovic XXXX)}

\section{Adopted Methodology}

This thesis adopted in-depth case study research design with hybrid methodology approach. The reasoning behind the selection of methods applied is elaborated in relation to the objectives, questions and hypotheses of the research (Figure 1).  
\\
For this purpose, we perform this research on two stages: methodological and case study level. We relied on qualitative data, collected from extensive literature review, expert interviews and participatory workshops. The application of these data sources depends on the stage of the research process: case study data collection, ANT and MAS data analysis and MAS-ANT data display. The case study is limited to the collection of qualitative data through the range of techniques: (a) extensive review of written sources, (b) interviews, (c) workshops, and (d) questionnaires. Conversely, data analysis presents the combination of: (1) the level of urbanity as an indicator of urban dynamics; (2) theoretical stances for exhaustive description from ANT; (3) operational categories used in MAS. Finally, data display give an overview of urban transitions through MAS-ANT methodological cross-pollination. 
\\
The initial stages of this research started off with qualitative inquiry. Data are collected in human and non-human chunks of analysis. Content analysis of data sources (urban planning policy agendas, urban planning documentation, archival and media sources etc.) provided an insight into the local context of urban planning institutions and land development practice. Interpretative research by  direct observations through semi-structured interviews and participatory-action research directed data analysis and further stratified the categories within the mixed-method approach. Online surveys validate the findings based on the working assumption that the soft data give a valuable insight into the complexity and dynamics of urban development circumstances in the local context. This forms the basis for "MAS on ANT" analysis of the obtained data and the final systematization within urban development system of urban transitions.

\subsection{Savamala Case study - Data collection} \label{sec:predis}

The case study research design is adopted as the most comprehensive one, to a certain extent as a research strategy, for an overview of possible categorizations and linkages in terms of complexity and dynamics of urban development processes \added{(Meredith 1989, Harrison 2002)}. In short, case study research design seeks patterns of the available and myrad data in the bounded space-time of the selected case \added{(Denzin et al. XXXX)}. A systematic approach for this empirical research is founded on the verification of the hybrid methodology and the data collection process has been recurrent, iterative consultations and interpretations of qualitative data \added{(Figure XXX based on interpretative research explain how it is deviated in my case figure 9.3 and figure 9.4. for the systematic approach for empirical research Harrison 2002 Flynn 1990)}. Thus, participant observations, interviews,  and surveys are all eligible methods which can be deployed in these circumstances.
\\
The interpretative itinerary directs how the soft data are collected and built into the artificial reconstruction of the developmental reality through MAS-ANT methodological approach \added{figure xxx}. The implementation of the case study (case study protocol) is executed in circles:
\begin{enumerate}
\item preliminary identification of the morphology of urban decision making, urban key agents and urban networks at the local level from the scientific literature, official documentation and records, and media coverage;
\item recognition of  urban key agent and urban networks structuralization through human perception of the objective reality with participatory action research and semi-structured interviews;
\item verification of urban key agents, socio-spatial patterns and urban transitions through on-line surveys for different professional cliques;
\item triangulating key observations and data sources;
\item examination  alternative interpretations and assertions of generalization for urban development (urban transitions) elements and networks through interviews with  key-informants (members of different interest, knowledge and action groups).
\end{enumerate}

Initial chain of evidence is presented in the longitudinal distribution of case study data, with time-frame (chronological) and linear-analytic (causal) references. Therefore, the case study report is structured according to the articulation of three layers of the morphology of urban decision making in Savamala: top-down urban planning, real-estate transformations and participatory bottom-up activities. The morphology of urban decision making is the bounding factor for all phenomena, themes, issues built into the case study. This manner of the systematization of collected data corresponds to all case study protocol topics (which are also the research indicators/variables), it also re-orders the protocol procedure differently so that it forms the basis for methodological analysis with Actor-network theory and Multi-agent system. Therefore, case study account seeks to "tame" urban development processes in Savamala by building MAS-ANT patterns into the empirical data within the case study narrative. Thereafter, the outcome variables must be clearly identified and interpreted with MAS-ANT data display. 

\subsubsection{Qualitative inquiry}

\textbf{"Science is not achieved  by distancing oneself from  the world; as  generations of scientists  know,  the greatest conceptual and methodological challenges come from  engagement with the world" Whyte}
\\
Qualitative inquiry is applied as a research instrument which enables scientific processing of soft data - meanings, experiences and descriptions \added{(Yin, 1994)}. The researcher is faced with the challenge of coping with large amounts of incommensurate data \added{ref}. In this thesis, it serves to incorporate the socially constructed knowledge of urban phenomena into the MAS-ANT modelling \added{(Mertens 1998, Flick 2002, Grubovic)}. In order to ensure replicability of the hybrid method, it is crucial to collect data in a coherent way and condense the complex spectrum of issues into a logical unity familiar not to the researcher only. High research priority herein is to cover the wide panoptikum of humanly moulded data. The proposed reporting scope of the morphology of urban decision making put forward the data structure through: top down management of urban issues, legitimacy of real-estate interests and the dynamism of bottom-up urban agency. To do so, qualitative inquire follow the case study protocol proposed in the previous section ((1) documentary analysis, (2) preliminary interviews, (3) workshops, (4) surveys, (5) in-dept interviews; while the range of data sources within these separate inquires should coincide with the concept of ‘supporters, opponents and doubters’ for any recognized data point of importance \added{(Pettigrew XXXX, Harrison 2002)}. 
\\
A common criticism revolves around internal and external validity of qualitative data \added{ref}. Within this methodological research, the validity issue is not particularly at stake as the perceptions and interpretations of urban actors (human factor), whatever they may be, influence urban systems and networks in their raw format, the same in which they appear in interviews/discussions/surveys. The major threat has been either researcher bias (directing the interactive data collection processes  towards confirming the researcher's preconceived notion) or reflexivity of interviewee's interests rather than statement of their perceptions or opinions. However, triangulation of qualitative data as well as iterative case study conduct should reduce these negative effects.
\\
Iterative nature of the case study protocol also helps in continual evaluation and update of data sources and circular data collection for MAS-ANT data analysis. Evidence and circumstances under which the data are collected are summarized within the following data collection procedures: 

\paragraph{(1) Documentary analysis} xxxxxxxxxxxxxxxxxxxxxxxxxxxxxxxxxxxxxxxxxxxxxxx
\\
Documentary analysis is rather qualitative research technique for identifying and interpreting documentary evidence in order to support and validate facts and incorporate them in a scientific research. In this thesis. documentary data are  used directly for data collection within the initial research phase. By addressing the first research question (\textit{RQ1: "What constitutes an inclusive approach (complexity and dynamics) to urban development?"}), the documentary research method provide an insight into the first round of independent variables. Not only that interviews and surveys may not be appropriate and cost-effective in this phase \added{(Mogalakwe 2006)}, but also
improve the preconditions for interviews by introducing basic issues and concepts, indicating the potential interviewees and setting the path for open-response dialogues \added{(Robson, 1993; GRUBOVIC)}. Documents are naturally occurring objects, independent and beyond particular scientific production within a research project; through their concrete and semi-permanent existence beyond the produce and the context of its production, they indirectly narrate the circumstance of the social world as well as the actors and circumstances of their production at a specific time and space \added{((Jary and Jary 1991; Payne and Payne 2004; Mogalakwe 2006)}.
\\
It must also be recognised that documentary narratives may be inaccurate, fragmented and subjective \added{(Forster, 1994)}. An early data validation is performed following \added{Scott's (2006)} criteria for assessing the documentary sources and data: authencity (genuine, original and reliable material), credibility (fatefull explanations and accuracy), representativeness (reliability for the research), meaning (whether the documents are clear and comprehensible). In this respect, reinforcing the robustness and rigour of this research in the first place is enabled by a preliminary investigation of documentary data sources, which is followed by examination, categorization and the identification of their limitations accordingly \added{Scoot 2004} \hl{Figure XX}. Moreover, the documentary data are continually adapted and validated in progress within data analysis through the process of triangulation of data obtained from other qualitative research techniques (interviews, workshops and surveys) \added{(Cochrane, 1998;  Yin, 1994)}.
\\
Generally known documentary sources encompass: public records, the media, newspapers, official gazettes, minutes of meetings, reports and blueprints, visual documentation etc; and they are broadly categorized according to their proprietary rights in: personal, public, and private sources \added{(Payne and Payne, 2004)}. According to ANT approach, the focus has been put on public and publicly available private sources providing the adequate context for data analysis. In the course of the case study, variety of formal and informal documents and their figuration and impact on the current developmental reality in Savamala (2010-2016) were examined, naming but the few:
\begin{itemize}
\item post-socialist urban literature;
\item legal documents: laws, by-laws, strategies, official gazettes etc.;
\item technical documents: spatial and urban plans and projects;
\item internal contracts, reports, projects, meeting minutes etc.;
\item media coverage. 
\end{itemize}

As  a  part  of  the  preparation  for  interviews,  documents  were  reviewed  in  order  to provide  a provisional overview of independent research variables, further developed through ANT and MAS data analysis and MAS-ANT data display. Documentary evidence was also used later on to supplement detail and to expand upon and support or challenge points raised during interviews, and was also utilised to generate additional questions or themes for investigation. In this thesis, documentary evidence have been the data core that set up iterative and continuous nature of the research process. 

\paragraph{(2) Preliminary interviews} xxxxxxxxxxxxxxxxxxxxxxxxxxxxxxxxxxxxxxxxxxxxxxx
\\
The ethnographic nature of Actor-network theory requires the usage of soft data and the application of qualitative data collection techniques, as diverse as possible. Consequently, the preliminary interviews are held in a rather unstructured manner, characterized by an open-response style. Open-ended interviews are focused on the certain topic, but are usually directed in a rather non-predefined way. Even though the questions can be scripted, the interviewer usually cannot predict what the contents of the response would be. Open-ended interviews may be: informal, semi-restrictive and structured \hl{ref}.
\\
The preliminary interviews set out the framework for constructing dependent variables of this research in the context of the second (\textit{RQ2: "Why do the level of urbanity traces determine pathways for urban development dynamics (urban transitions)?"}) and third (\textit{RQ3: "How to frame urban development process to embody complexity of urban systems and dynamics of urban transitions?"}) research questions. In this case, these interviews are targeting local experts in urban research and practice, local and city authorities staff as well as activists operation the ground of Savamala. Several interviews are done in iterations. At times, interviews also take a snowballing course, when the informants were identified by the interviewees from the previous round. The interviews were carried out in circles and the level of dialogue restriction varies from the occasion and interview iteration. The focus is usually on interviewee's thoughts, experiences, knowledge, skills, preferences ideas. All the interviews are undertaken face-to-face.
\\
These interviews are carried out during an extensive period of time, in the overlap with other data collection processes and even during the first stages of data analysis. The timing was also influence by the on-going processes in Savamala as well as the formal and informal channels for establishing the contacts for these interviews. In several cases, the interviews were directed in the least restrictive, informal way, in a form of a dialog, without actual preparation, but instead asking the questions spontaneously in the course of a wider context of the research topic. Subsequently, no two informal interviews are alike. The expert and activists interviews and conducted in a semi-restrictive manner. These interviews follow a general outline of issues of interest, but some questions are generated spontaneously or go to other topics when interviewee's answers prompt for more, connected, information \added{Payne and Payne 2004}. The research questions were grouped into X major themes \hl{Table XXX}. Only few interviews were recorded as these interviews were used for the further identification of major concepts within research variables, and clarity and understanding of interview content was sufficient. 

\paragraph{(3) Workshops} xxxxxxxxxxxxxxxxxxxxxxxxxxxxxxxxxxxxxxxxxxxxxxxxxxxxxxxxxx
\\
Topic-oriented workshops are qualitative research techniques moulded for this research in combination of participatory action research (PAR) and simple participant observations. They are applied for re-problematization of research questions in the light of critical reflection and dialogue between and among participating actors \added{ref mcintyre PAR}. The aim of this technique is to transfer tacit knowledge concerning the urban development reality in post-socialist context of Belgrade into explicit indicators in order to overcome the single-discipline limitation and lead to rethinking the definitions and conceptualizations of the independent variable and dependent variable and their interconnections within this research project \added{ref whyte PAR ref}. 
\\
Context-specificity of participatory action research provides multiple opportunities for practitioners and scientists to construct knowledge and integrate theory and practice within a "theory of possibility" \added{ref mcintyre PAR}, in this case on urban development prospects of the Serbian capital of Belgrade. Combined with participant observation approach, these workshops are explored and valued how the experience of the selected participants influence their collective  realities. In this context, gathered information and constructed explanations from participants are used for testing the systematization of MAS-ANT approach in terms of the morphology of urban decision making, urbanity and urban transitions. For construction an adequate circumstances for such workshops, it is essential to believe that the participants are capable to create and influence the urban condition \added{ref mcintyre PAR}, while the actual data were evaluated afterwards in terms of their
confidentiality, trustworthiness, and credibility.
\\
Within the scope of this research, three workshops have been organized: expert, PhD and student workshops. Two of them (expert and student workshop) were on a precise, research oriented topic and one (PhD student workshop) with a broader context in mind. Table XXX gives an overview of the organized workshops and evaluates their role in building a relationship between theory and practice according to the above mentioned criteria. All the workshops were moderated by local experts and the participants (more during PhD and student workshop) were engaged in using urban research methods for  constructing  knowledge on the topic. Even though the active role of participants is crucial for the success of the workshop, uncertainty of their choice is also very important \added{ref mcintyre PAR}. In these terms, participants control the conduct of the workshop, they choose to react, interact or stay passive. Workshop data also test the the credibility of data collected with two previous qualitative research techniques (documentary analysis and preliminary interviews).

\paragraph{(4) On-line Surveys} xxxxxxxxxxxxxxxxxxxxxxxxxxxxxxxxxxxxxxxxxxxxxxxxxxxxx
\\
On-line survey is a type of questionnaire. Questionnaires  are  better  suited  to  collecting  ‘factual’ information \added{Payne and Payne 2004}. However, they can be used for provisory verification of subtle and complex data sets, if the questions could be formulated in simple, non-technical and unambiguous, and use the language easily understandable to all participants \hl{ref}.
\\
The on-line surveys are conducted simultaneously with workshops, but over an extensive period of time (before and after the workshop events). The structure of the online surveys slightly varies and correspond to the workshop; they are adapted to each of workshop audiences: experts, phd students and young professionals, and future professionals (students). As workshop follow the initial documentary analyses and interveiws, surveys are supposed to encircle most of the previously results and test their manifestations in MAS-ANT methodological scope. The survey results are separately evaluated and partially quantified, if possible on several topics. Anything ambiguous, biased or likely to arouse anxiety is tried to be avoided by substituting indirect questions \added{Payne and Payne 2004}.
\\
The surveys are prepared with Survey Monkey online service. The survey content is built up as a combination of open-ended and closed questions, categorized according to the prevailing topic. They are structuralized in a linear, complexity-growing manner - from general to more specific questions. The order of questions has an important influence on the answers. At several points, leading questions, where certain answers are expected, are used. At some points, filter questions are used, where respondents are only required to answer certain questions if they have answered a previous question in a particular way. The on-line, internet polling nature of the surveys was specially beneficial for this self-completion parts of the questionnaires, where the question sequence depend on respondent answer. 
\\
An important constraint was the dictate of brevity, especially for the surveys targeting experts. Namely, respondents’ attention spans are usually short. 
However, the wording and clear instructions are recognized as vital for the success of the surveys, as well as the possibility for respondents to easily navigate through questions and their meanings as well as to without special effort answer in an simple, clear way. To a certain extent, these constraints reduced the impact of these online surveys on data analysis.

\paragraph{(5) In-dept Interviews} xxxxxxxxxxxxxxxxxxxxxxxxxxxxxxxxxxxxxxxxxxxxxxxxxxx
\\
In-depth interviews pertain to the final stage of data collection process. They are carried out on specific topics within the research framework in order to provide an in-dept account and validation of research hypotheses. All the interviews are face-to-face encounters, but not all of them are recorded. Recorded or notes-based conduct was left as an option for interviewees to decide. Few recorded interviews are summarized and cross-referenced with the notes taken during the non-recorded ones to ensure clarity of interview content and verification of research data.
\\
The collection of data herein was predominantly by a semi-structured interviews. However, the sequence of addressed topics is more rigid and correspond to that of the research framework and the structuralization of research questions and hypotheses (RH1, RH2, RH3). The question grid is restrictive, and the exact same questions on specific topics are prepared for each interview, even though the slight variations occurred in response to rather open-response style of the dialogues. The careful wording of in advance prepared questions contribute to avoiding ambiguity or specific undesired connotations, but the semi-structured nature of interviews enable the researcher to develop a positive rapport with the interviewee and vice versa \hl{ref}. Even though semi-structured nature of interviews is more adequate where the range of interviewee accounts and overviews about the research topic are not well known in advance \added{King 1994}, in this case they are useful to obtain the factual  information on the matter is and to facilitate the data collection from interlocutors from different professional and disciplinary backgrounds. 
\\
The choice of interviewees is such that it covers the systematic categories of informants identified with previous collection techniques and correspond to the proposed MAS-ANT categorization. \hl{Table X} presents the interview grid and qualitative evaluation of the capacity of the herein collected data to be used for MAS-ANT data analysis in terms of their reliability and credibility for this research project.

\subsection{ANT and MAS approaches - Data analysis}

The hybrid field of overlapping MAS and ANT methodological approaches proposes an innovative concept to define causal relationships among urban key elements in order to elaborate the process of urban system evolution (urban dynamics) in terms of maintenance, transformation and change prospects. In  addition,  some  information  was  gathered  from  newspapers  and  magazines. However,  although  the  development  of  various  media  made  research  on  the  elite easier, in Serbia the volume of writings on illegal building, especially on the elite’s activities, was small due to a lack of freedom of speech and fear of journalists and reporters. As  a  part  of  the  preparation  for  interviews,  documents  were  reviewed  in  order  to provide  a  context  for  the  case  studies. Documentary evidence was also used later on to supplement detail and to expand upon and support or challenge points raised during interviews, and was also 
utilised  to  generate  additional  questions  or  themes  for  investigation,  thereby contributing to the iterative and continuous nature of the research process.  

\subsubsection{ANT Discourse analysis}

According to our interpretation, ANT is addressed here neither as a network in technical sense, nor a theory in social sense (Latour 1996), but as a methodological approach which prioritizes  “relations over their characteristics” ( Cerulo 2009: 536) and “action over mind” (Ibid., 543). We will explore these relational and operational elements that mould urban development circumstances in Savamala. We focus on an actual post-socialist urban setting and generation of maintenance, transformation and/or change of the current state of the affairs when global aspects are transformed to meet local specifications and vice versa. In terms of post-socialist cities, copying urban models from the West meets extraordinary difficulties because these cities lack the institutional infrastructure and cultural patterns essential for the functional unity present in western cities (Petrovic 2009). Furthermore, fundamentality and intensity of economic and political change in Balkan post-socialist countries may be a historic exemplary of social transition hard to find in a “typical” capitalist city (Sykora 1994).
\\
In Savamala, we were confronted with a dynamic reality, a battlefield of different influences, interests and interpretations which determines the future of the urban system itself. In our research project, we turn to qualitative data collection, overlapping case study with interpretative and participatory action research. These methods provide the data for ANT analysis enforced with correlational study, while engineering approach of logical argumentation and schematic interpretation are used for the dissemination of data. Principal data sources are human and non-human based  collected through the range of collection methods [(1) extensive review of written sources, (2) interviews, (3) workshops, and (4) questionnaires], provided from the key informants [(A) experts, (B) young professionals, (C) participatory activities, (D) Belgrade Waterfront Project] (Table 2). These key informant categories are identified from the aggregated human and non-human bearers of action and meaning (Latour 2005) among the traces of relevant influences, interests and interpretations in Savamala.
\\
On the level of data analysis, diagnosing urban development circumstances could be determined through a transposition of the current state of this neighbourhood into the elements which could denote the urban flux. These elements, when gotten together into functional networks, form a unique set that indicate factors of maintenance, transformation and/or change of the system, what is in our case a neighbourhood of Savamala. The successful application of ANT for these purposes involves transposing the terminology of ANT into urban development factors and an exploratory analysis for identifying these factors within a real-life context. In order to apply the identified theoretical principles for the on-site analysis of a dynamic urban reality on the neighbourhood level, we propose to reformulate them in the step-by step methodology, which will be the following: (1) identify human/ non-human entities; (2) flat reality of intermediaries figurations and translations between mediators; (3) traceable associations among those constituted as actors; (4) track stability/instability of agency among actors; (5) simplify and functionalize relations in urban assemblages based on the established roles and nature of links among them (Table 3). As a part of a broader study on post-socialist urban development model, we are examining actor-networks in Savamala rendered from a composition of different layers of decision making that, through coordination and predominance, bring up urban dynamics. The level of analysis is neighbourhood. 
\\
Central methodological issues for translating ANT terminology onto an urban environment indicate: 
1.	All human and non-human actors: From ANT viewpoint, the source of an action accounts equally for humans and non-humans, and only action/relation counts (Latour 1996). Animals, object, texts, symbols, events, even mental concepts may be actors depending on their activities and/or relations (Cerulo 2009).
In our case study analysis, we distinguish figuration of all particular human and non-human entities that are subjects of translations on the neighbourhood level of Savamala. Our argument is grounded in the local context of planning procedures and practice concerning Serbian urban system and post-socialist neighbourhood level, as well as bottom-up activities in Savamala. In this manner, we ponder the complexity of our case study neighbourhood to be made up of human - people (urban actors and stakeholders), and non-human entities - urban structures and territories (natural and urban space), institutions and policy agendas, urban and communication infrastructures  (Mitchell 1999, Firmino et al., 2008) and social aspects (economic, political and cultural) (Table 3). These operational categories of urban key agents are traced through the extensive content analysis within theoretical scope of urban studies. All case-specific entities are identified within content analysis of various sources on the morphology of decision making in Savamala (post-socialist urban theory, planning legislation and documentation, media sources) and from on-site examinations.
2.	Intermediaries and mediators: Following Latour’s definition (2005), these human and non-human entities become "the means to produce the social" (Ibid.,38) only when their role in the system enact them as intermediaries or mediators (Ibid.). In his words, intermediaries are simple bearers of meaning  and mediators actually change actions/relations they are engaged in .
Based on the content analysis of research, legislative, operational and media data, we have recognized that certain elements through only through certain figurations in networks take the actor role (Table 3). We distinguish 4 element types (entity, human, artefact, and event). For example, all 4 matter differently whether they are taken individually or in a set/group and for artefacts it is crucial to consider if they are of strategic, technical or repository type. Visually, the shape of the nodes depends on what figuration of an element makes it an actor.
3.	Free associations: One of mayor achievements of ANT is its attempt to redefine sociology as tracing of associations and thereafter designating social not as a quality of an element-entity, but “a type of connection between things that are not themselves social” (Latour 2005:5).
Aforementioned urban key actors (urban actors, spatial forms, regulatory framework, and social aspects), after being denoted as mediators, have an active role in networks, and in ANT terminology it is referred to as the performance of subject (human entities) and the enactment of objects (non-human) (Farias 2011, Callon 1986). As part of ANT data analysis, we juxtapose the recognized entities and convert them into actors. The established actors are those who associate and form networks (Table 3). The reason to reinterpret classical categories of scale, structure and the social in network terms is grounded in qualitative inquire from experts (Table 2). These categories are not taken for granted but applied only when they influence actors’ relationality.
4.	Stabilizing and destabilizing agencies: When applying ANT for urban analysis, the importance lies in avoiding pre-established social science categories (Farias 2011). It is essential to refer to agency as a relation that connects multiple actors, and distributes causality and explanations across networks in stabilizing or destabilizing manner (Ibid.). 
Based on expert insights and data from PhD workshop and documentation on local, regional and national level, we examine complexity and interactions among the actors on the neighbourhood level, how they cooperate/coordinate/negotiate/collide and organize into networks according to their roles (Table 3). In graphical terms, node colours correspond to the agency of actors and active, but standardized networks they are engaged in. The difference between association and agency in our interpretation lays in their dynamics – these networks, though standardized have the bipolar potential – ability to influence actor-networking.
5.	Urban assemblages: Urban assemblage is a trendy term for aggregating, not identity altering, relations of heterogeneous urban actors (Muniesa et al. 2007), “relations of exteriority” based on actor-networks (Farías 2011:15).
According to ANT social and structural descriptions of urban dynamics, data validated in workshops with researchers, professionals, activists, young professionals and citizens are channelled visually through actor-network diagram. Body of actor-networks are comprehended without any preconceptions of society, social realm, social context and/or social ties and visualized by the size of nodes (actors) and the colour of links between them (networks) (Table 3). The size of the node equals the centrality of an actor and its influence. Actor’s influence is assigned approximately according to its presence in time, number of its relations, and types of the relations. Conversely, colour of the links relate to the nature of links in which these actors engage and produce specific social effects.
\\
This 5-step ANT framework aims to illustrate urban development of a post-socialist neighbourhood in Belgrade – Savamala. For logical argumentation on network formation and development, we accept as basic rules the major ANT assumptions: (1) everything that matters is an actor and therefore engaged in a network(s); (2) there is no context or any non-associated element in the system. In this respect, we visualize Savamala urban development circumstances (all context-related, history-related, on-site and documentation-related data) in terms of actors (human and non-human) and the nature of links they are engaged in relative to their activities, priorities and relationships.

\subsubsection{Structural analysis}

According to MAS, system dynamics relates to individual elements and their communication, free will, belief, competition, consensus and discord etc. This all must be taken into account, as it actually is in its application for computer programming. The system of analysis therefore consists of:
•	E - environment: static, defined by the level of analysis - post-socialist neighbourhood;
•	A - assembly of agencies: static state of urban key agents;
•	O - set of objects: passive contextual elements – spatial capacities and social potentials;
•	R - assembly of relations: active elements – social practices and urban conflicts;
•	Op - assembly of operations: active morphology of urban decision making;
•	U - reaction of the context: static, research goal – urban development in terms of maintenance, transformation and change of the system.

In more technical terms, MAS-ANT cross-pollination happens on the level of agent profiles:

AGENT PROFILE = AGENT STRUCTURE + AGENT PREFERENCES + AGENT BEHAVIOUR

We readdress actors as agents and come up with agent profiles configured from the combination of agent structure, agent preferences and agent behaviour. Each agent profile element refers back to both ANT and MAS categories:
•	AGENT STRUCTURE addresses ANT categories of (a) the nature of actors; (b) structure and networks of influence; (c) secondary and socially functional networks; and the sum of interpretations translate actors to agents and constitute an assembly of agencies A from MAS. 
•	AGENT PREFERENCES regard from ANT (d) social artefacts (political, economic and cultural) which they are involved in the operationalization of objects O (contextual resources) and relations Op (social practices and urban conflicts) from MAS.
•	AGENT BEHAVIOUR are ANT (e) networks of translations and MAS explanation of HOW (pro-activity, sensitivity, interaction) these agents reference back to the system development U (maintenance, transformation and change).

Sorting all the data about an agent in these categories provide us with full description of how urban system works. ANT analysis furnishes exhaustive categorization of elements and networks - the detailed image of agent structure and the field of their influence (networks). Agent preferences are represented through object and relation categories from MAS and correspond to the level of urbanity theoretical stances . In practice, it signifies that all urban conflicts and social practices could be identified as directed relations (R). Conversely, spatial capacities and social potentials are referred to a set of objects (O) to be activated. The agent behaviour are the products of multi-criteria MAS analysis (Arsanjani et al. 2013). In practice, urban dynamics is defined as how the agents A behave to perceive, produce, consume, transform and manipulate objects O and engage in relations R in order to enable maintenance, transformation or change of the system.

An urban development model is based on:

Measuring the efficiency of urban planning

Testing the legitimacy of urban transformation interests

Recognizing the opportunities of bottom-up urban design initiatives


\subsection{System Building - Data display}

Reporting the findings is done in a visual manner, where all the data are categorized and built in the visualized system of urban transitions. In this manner, data visualization techniques are used for data reduction and operational display of data.

\subsubsection{Setting a procedure}

The research is executed in hierarchical order. ANT serves for the identification of all actors (human/non-human, material/non-material) and flattens the social into a panoptic internalized ontology. MAS traces the character of their appearance in networks and their internal relations and connections. In this way, “actors” (ANT) are transferred into “agents” (MAS). Finally, theoretical layer represents a generative body of concepts suitable for tracing urban dynamics. Such triangulation is carried out in three steps (Figure 1):

1.	Interpreting agent structure:
Identify all context-related, history-related, on-site, literature-related and empirical data according to ANT principles - that everything that matters is an actor and therefore engaged in a network(s) and that there is no context or any non-associated element in the system (ANT article). 
2.	Qualifying social practices, urban conflicts and contextual resources Transfer topology of ANT into topography of MAS:
Monitor spatial and social data categories in order to add agent preferences to the profiles. 
3.	Evaluate development-based scenarios (maintenance, transformation, change)
Referring back to ANT assemblage networks and MAS analysis of agent behaviours in networks

The data collection and data analysis processes are carried out iteratively from: (1) collecting context-based information and knowledge (2) ANT classification of the data on the local context; (3) MAS analysis of agents’ behaviours (ANT article). The key informant categories are identified after the traces of relevant influences, interests and interpretations in Savamala. The descriptive nature of ANT premises enables data structuring in terms of set of human and non-human agents and urban assemblages on the neighbourhood level. Henceforth, the behaviours of agents are identified by qualitative surveys.
From the above presented approach, tracing urban dynamics comprises structuring an urban environment according to clear categories as well as simulating autonomous actions and interactions in order to study blurred processes of constant system evolution. The set of networks involve the heterogeneous distribution of urban key elements acting at sites (human and non-human) and causes and consequences of actions and forces. All these urban key elements are assumed to be equal agents in the reproduction of social practices, operating contextual resources and dealing with urban conflicts. These continuous processes of maintenance, transformation and change reflect agent behaviours and contribute to the system dynamics.

\section{Methodological Framework}
summary of the chapter, visualization of the framework and conclusions

%%%%%%%%%%%%%%%%%%%%%%%%%%%%%%%%%%%%%%%%%%%%%%%%%%

\chapter{Case study: Belgrade - Savamala}

%%%%%%%%%%%%%%%%%%%%%%%%%%%%%%%%%%%%%%%%%%%%%%%%%%
As it has been already pointed out, urban development is a general and vague concept as long as it is detached from an actual urban setting. In a local context, urban development importance is overthrown by the concern with how global aspects meet local specifications and how on-site forces and uncertainties govern the concrete dynamics of the urban system. In order to move away from the abstractions and generalizations bounded in methodological terms, all the theoretical stances that address urban complexity and dynamics are traced in a real-life context.
\\
Any historical background to this research problem, its transformation over time and the immediate surroundings where it emerges and changes, should be considered as a chronological sequence. This, if described properly, will provide a suitable means for explaining causal links among the identified factors and elements of urban development. Given the implications of positive theory, it will be possible to predict the future relationships and behaviour of the elements in question.
\\
This chapter shows an overview of socio-spatial circumstances in their autochthonous discourse, their direct manifestations in the capital city of Belgrade and final repercussions in the Savamala neighbourhood. The context is being attenuated throughout the chapter from the nation-state aspect, a citywide dimension to the neighbourhood level. A linear, factorial analysis is performed on the neighbourhood level in order to distinguish the layers of its urban decision making.
These organizational levels (national, city, neighbourhood) and directional layers (top-down, real estate, civic engagement) interlace the space-time boundary of the context for tracing its intrinsic complexity and dynamics.
This contextual narrative is conducted in two directions: chronological discourse on urbanity and causal links within the morphology of urban decision making. 

\section{The State of the Society and the Urban}

\deleted{table diagram on different periods}
\deleted{A. intro + topics: population info, political and social, urban culture, urban form, urban planning}

Investigating autochthonous urbanity in the chosen context implies attributing socio-spatial circumstances to the continuous urban transitions. A versatile geographical landscape of Balkan Peninsula harboured an amalgam of cultures and religions and produced a condensed history of social, political and economic turmoils that compose urban structures and processes \hl{(waves of planning 2006)}.
The key factor is understanding how a structural unity of urban key elements has been deployed in different space-time frameworks and how they reference back to the social and urban processes of maintenance, transformation and change (urban transitions).
\subsection{The Evolution of the State Affairs}
Historically speaking, the first permanent human settlements in this region were established during the  Neolithic  period \hl{(Krstic and Bojovic 1972)}. However, the chronological discourse on urbanity in this research is built upon the following periods:
\begin{enumerate}
\item the Ottoman dominance;
\item the Serbian state (1804-1914); \footnotemark
\item Yugoslav self-managed socialism, the union of Southern Slav nations;
\item the post-Yugoslav, post-communism transition.
\end{enumerate}
\footnotetext{Even though Serbia officially gained independence from the Ottoman empire in 1878. and proclaimed the Kingdom of Serbia in 1882, the continual change in urban and social issues had been gradually changing from the first Serbian uprising in 1804 and holding the capital city between 1807. and 1813.}

A structural unity of urban key elements is bounded in the broad, factual and chronicled identity of the pertaining urbanites \hl{ref}. The choice of the periods does not strictly follow the historical continuum \footnotemark, but relies on the socio-spatial constituents of the contemporary urbanites. The urbanity in this area is reflected through the fluctuating relations with European cultural and geopolitical realm \hl{(ref)}, strongly spatialized identity of the local population \hl{(ref Savic 2014)} and a continuous revolution through the transitions of the mentioned periods. Henceforth, the chosen periods indicate the key points of the alterations and development in terms of: the political sphere, a socio-cultural realm, formation of urban actors, urban culture and urban profession and the distribution of urban forms.
\footnotetext{Exclusion of antique and medieval historical heritage based on the incoherence with contemporary urban key elements crucial for this analysis. In the same terms (urban key agents), the excluded period of the Kingdom of Serbs, Croats and Slovenes (The first Yugoslavia) shows  continuity with either its precedent (the Kingdom of Serbia) or the successor (SFRY).}
\deleted{B. Periods}
\subsubsection{Ottomans}
Ottomans were dominating the central part of the Balkan peninsula (today’s Bosnia-Herzegovina,  Macedonia  and  Serbia,  Kosovo  in  particular) for around 500 years.\footnotemark Since the Ottoman empire was an Islamic theocratic state, decision making was conducted through the complex centralized administrative structure under the supremacy of the Sultan. Local christian majority was marginalized, rural population ruled over by the constant threat of extinction. Urban nexuses were populated mainly by ruling social class of Ottomans and local converts. As the Ottoman rule of this region could be characterized as an authoritarian imposition, any type of development was slowed down and reduced to a minimum \hl{waves of planning 2006}. It was an archaic agrarian society. The territory of contemporary Central Serbia was the border zone between the Ottoman and Austrian empire, the Muslims and the Christians, the East and the West, The Europe and the Other \hl{ref}. Drina and Sava rivers were natural borders of two civilization patterns. Few, predominantly small cities in the area were susceptible to the often war destruction, either in Turco-Austrian wars or during population revolts. The fortress of Belgrade was the borderland vulnerable to frequent conquests and invasions of the two empires. The cities developed organically, under the influence of the influence of Islamic planning and building traditions \hl{ref from waves of planning 2006}. However small, those cities were centers of social and cultural life of the said state. Unhygienic, haphazard construction and urbanization patterns are typical for Ottoman cities in the Balkans \hl{Kadijevic XXXX}. 
\\ 
\textit{"The settlements (called kasaba  if  small;  varos  if  large)  had  a  distinct  structure  including:  central section (carsija) for public functions like baths (amam), schools (metresa), coffee houses and entertainment  places  (kafana),  worshiping  buildings  (dzamija),  crafts  and  trading  posts (bazar), and travel inns (han); and a residential section (mahala) separated into two parts - upper for Muslim residents and lower for the Christian population. Residences were built around yards (avlija) surrounded by high walls used to protect the privacy of the extended family." \hl{waves of planning 2006}}. The christian population either was excluded from social structures and any form of decision making or gathered around christian neighbouring states from the West and participated in their warfare against the Ottomans. Consequently, local religious powers (christian - catholic and orthodox) mastered the societal context of the subdued population.
\\
The Serbs cherish negative collective memory of this period, which resulted in the extended dissolution of Ottoman cultural and urban heritage and disregard and subsequent dilapidation of the Ottoman architecture, while retaining unconsciously certain social, urban and decision making practices \hl{(Blagojevic 2009)}
\footnotetext{The year 1459. is marked as the year when Serbian despotate is officially overthrown by the Ottomans, and in 1804. started the Serbian revolution against the Ottoman rule. However, the very first Ottoman penetration into the area of Balkan medieval  states was after the Battle of Kosovo Polje 1389. Serbia officially gained its independence again during the Congress of Berlin in 1878. \hl{(ref Corovic)}}
\subsubsection{The Serbian State}
The liberation of Serbs slowly but surely started with the uprising in 1804. From 1882. onward, Serbia is recognized as a constitutional monarchy with capitalist social order. Even during the constitutional years before the official recognition of the country, the population in Serbia raised from 678,192 in 1834. to 1,669,337 in 1884. and 2,922,058 in 1910. \hl{ref old statistics}. The total number of 2,922,258 was reached in 1910. 
In parallel with wiping out the surface signs of Ottoman legacy, the mentality of deprived, rural majority of indigenous population under the theocratic regime had survived and contributed to the patriarchal, paternalistic and authoritarian political model \footnotemark  at stake more often than not in the Kingdom of Serbia \hl{(Vukmirovic et al 2013)}. Moreover, the state was poor, dominated by a weak and vulnerable, autarchic economy based on either trade or foreign investments and closely related to the state and to privileged groups \hl{(Vukmirovic et al 2013, Samardzic in Doytchinov 2015, add ref for foreign investments)}.
\footnotetext{with the huge rural hinterland and its traditional notions and influences}
\\
Dubravka Stojanovic (2010), a famous historian who investigates this period of Serbian state constitution, explains that the state affairs, civil sector and society circles have developed without any real interconnections from the very beginning of the Serbian modern nation state. She further develops the thesis that the state is usually taken for granted as a society itself, the supreme driver of development and modernization, and the most important source of influence/authority/status/wealth for an individual. In other words, the political ruled out and dominated the economic and the social during the rise of modern Serbian state \hl{(Stojanovic, 2010)}.\footnotemark However, the Kingdom of Serbia was slowly gaining its status in the European realm, forming close ties with European centers either by educating professional staff there or by importing models, systems and structures from the West \hl{ref}. This phase had been also crucial for the foundation of Serbian elite and a modest emergence of local "quasi" bourgeosie and aristocracy based principally in the capital. Those circles were filled with the soldiers, clergy, scarce intelligentsia and highly obedient bureaucrats, while the  first  signs  of  Civil  Society (traders, scholars, officers, civil servants) also emerged, along with new challenges and  opportunities  for  development \hl{Vukmirovic et al 2013}.
\footnotetext{Stojanovic (2010) gives an example of how the state authority was dominating the political discourse during 1882-1914 period: the reasoning behind the issue of significant liberty of press in the kingdom of Serbia at the beginning of the 20th century was such that the press did not have the force of public, so that the authorities did not pay any attention on the press releases and the topics therein covered.}
the increasing number of engineers educated in the West (mostly in 
Austria), orientation towards western models and disregard and subsequent dilapidation of the Ottoman architecture hl{Blagojevic 2009}
\\
Nonetheless, the urban population in the Kingdom of Serbia was dispersed in small cities and counted around 13 percent of total population \hl{(ref before WWI)}. This period is marked by radical changes in all domains of urban and social life. Still even the capital city of Belgrade during those times was characterized as rather big and insufficiently regulated dorp. Samardzic \hl{(in Doytchinov 2015)} suggests that the most important benefit for the Serbian national revolution, 1804-1830, was gaining authority over the city of Belgrade. The task of building the new state through building, organizing and modernizing its capital city posed a challenge to backward peasant society, as Serbia was at those times\hl{(Samardzic in Doytchinov 2015)}. Namely, cities with their pluralistic, diverse cosmopolitan character were identified as the enemies of the nation and the church. They tended to keep the traditional, rural flavour with the permanent problem of the over-population of the existing housing stock, as well as illegal and non-quality constructions in the suburbs \hl{Roter Blagojevic in Doytchinov 2015}.
\\  
Urban morphology of Serbian cities in the 19th and beginning of 20th century was obviously strongly affected by Western European ideas originating in France, Germany and Great Britain \hl{(ref other than waves of planning 2006)}. This selective, undiluted borrowing of European building principles, methods, and techniques transformed them into modern towns \hl{(waves of planning 2006, Kadijevic XXXX)}. Architectural and city planning practice was gaining momentum with an emphasis on urban growth with a new orthogonal street network with central places (piazza, square. place) and within a modular grid of plots, blocks and streets \hl{Kadijevic XXXX}. An additional functional layer of traffic, sanitation and utility infrastructure contribute to the modernization of urban structures and harmonization of urban systems and networks \hl{(ref)}.

\paragraph{Urban Agendas, Institutions, Practice}
The pioneers of this new wave of construction and activity were individuals educated abroad (mostly in Vienna and Budapest, later in Paris and Germany) and foreign experts under the auspices of the state \hl{(Branko Maksimovic)}. They usually received the state scholarships to attend elite european universities and were given the opportunity to engage in practice straight afterwards, even without any significant professional experience \hl{(Mladjenovic projekat Rastko)}.
However, town planning was often used as a controllable tool at hand to consolidate political power by high authorities (Governments and even rules). \footnotemark The Ministry of Construction/Civil Engineering, backed up later with specialized Architecture section, was the leading actor in architecture and planning of the time. The young state administration controlled by centralized political will, directing the development of architecture, adapting it to the needs of newly established, young Serbian state \hl{(Maksimovic,waves of planning 2006)}. However, in such petite, young state, human resource with high expertise are often scarce, so that, when the Faculty of Architecture was open in Belgrade (1897), the same professionals working for the Ministry obtained the teaching positions - pursuing in this manner parallel careers \hl{(ref)}. In these circumstances, private architectural practice was highly underdeveloped \hl{(Mladjenovic projekat Rastko)}. Such circumstances of centralized power, overlapping of jurisdiction and competence and biased relations between political and economic strata opened the floor for corruption, privately driven initiatives and land speculations by powerful and rich individuals \hl{(ref)}.
\footnotetext{There is an anecdote that Milos Obrenovic, the first ruler of not yet independent state, who happened to be illiterate, controlled planning and building documentation and visited construction sites with the negineers in order to control them, intervene and implement his own spatial visions \hl{(ref)}}
\\
This institutional framework and building practices were followed by the constitution of regulatory and legal framework of urban development, planning and construction. The first adopted document was a sort of regulation plan - Regulation  Line  for  Construction  of  Private  Buildings (1864), followed by the adoptation of laws - Law on Construction of Public Buildings (1865) and and the Law on Expropriation of Private and Real Estate Property for Public Use (1866) \hl{(waves of planning 2006)}.
\\
The planning and construction activity during this period had a decisive influence on the Europeanisation of Serbian development path \hl{(Kadijevic XXXX, ref)}. Nevertheless, representing south-eastern European periphery, with strong nationalist political and social discourse and colonial imperialism in culture and arts also hindered the urban emancipation and institutionalization \hl{(Vukmirovic et al 2013)}.\footnotemark 
\footnotetext{The period of The Kingdom of Serbs, Croats and Slovenes (1918-1929) known as the Kingdom of Yugoslavia (popular also as "the first Yugoslavia") from 1929. till the German occupation in Second World War is excluded as it follows the trends set forth during the previous period (1804-1914) with small insignificant shifts and  certain emancipatory and modernization efforts and successes. On the contrary, the country comprised almost the same the territory as its descendant SFRY with the population of 11,984,911 inhabitants in 1921. The most important document is the Building Act (1931), up-to-date urban regulatory document of the time. It features the Regulatory Plan as the main instrument of urban development, regulates technical building details and prescribes the format of planning documents and the planning procedure \hl{(waves of planning 2006)}}.

\hl{Blagojevic 2009}: 
egocentric-megalomanic attitude (1912. and 1922.): in crucial historical moments, reflect the state strive to ascertain the changed condition of sovereignty; intervention in historical time, new history re-established as tabula rasa.

\subsubsection{SFRY}
During the Second World War (WWII), Yugoslavia lost significant amount of people and numbered only 15,772,098 inhabitants in 1948. After the Second World War, a new political order was installed, the country was a federation of six socialist republics: Slovenia, Croatia, Bosnia and Herzegovina, Serbia, Montenegro and Macedonia. In the post-war geopolitical divide, the Socialist Federal Republic of Yugoslavia (SFRY) expanded beyond the borders of its predecessor (the Kingdom of Yugoslavia).
\\
The socialist period in Yugoslavia is known as self-managed socialism\footnotemark. Political regime at stake was decentralized yet authoritarian with both capitalist and socialist elements. It was single party system with a president for life (Josip Broz Tito). However, decision making was partly shared between central and supreme authority of the government and republics, municipalities and even several public enterprises. Economic reforms put to work were quasi-market and quasi-liberal with self-management of the most of enterprises, societal ownership over large industrial enterprises societal, and a number of small, private businesses (services and crafts) \hl{(Estrin 1991)}. The economic system survived on non-market mechanisms and administrative decisions \hl{(Estrin 1991)}. Its most important trait is "social  ownership" , i. e.  workers' self-management and control of  enterprises \hl{(Estrin 1991)}. Finally, cultural exchange was significant either with the west or the east \hl{(Hirt 2009, Vujosevic and Maricic 2012)}. The country experienced successful economic reforms, cultural revival, labor productivity, urban and infrastructural development and construction in its first years (until the late 1970s). Urban milieu of socialist Yugoslavia was significantly egalitarian and diverse with high standard of living. These social circumstances indicated low level of under-urbanisation resulting in less marginality and less class disparities socially; and consequently less autonomous and less heterogenous urban forms \hl{(Vujovic and Petrovic 2007, Stanek 2014)}. The dysfunctional combination of factors and the unfortunate course of events generated by deep economic crisis (unemployment, inflation), the decline of legislative power and ever-increasing regional disparities its final years \hl{(Estrin 1991, Stambolieva 2013)}.

\footnotetext{Self-managed socialism or market socialism are popular terms for the type of socialism applied in SFRY and is known as pure Yugoslav brand \hl{(Estrin 1991)}}.
\\
Social, political and economic circumstance also reflected onto cities and urban life. The overall concepts of urban development and planning in SFRY were based on CIAM principles and proclaimed various urban standardization, center-periphery urban dichotomy and local community concept in decision making \hl{(Fisher 1962, waves of planning 2006)}. Urban institutions, regulations and profession in general functioned on quite mature, comprehensive, and multi-disciplinary base. Well conceptualized, they were failing in the implementation of guidelines and control measures of various urban processes; as Nedovic Budic \hl{(Mornings after XXXX)} states "it was well conceptualized but failing in implementation just as the state socialism itself". Urbanity of Yugoslav cities result in being strongly tied to the underlying political organization and socio-economic order of the time.
Consequently, the transitions in urban actor matrix, pertaining cultural patterns and urban forms, as well as professional approach are linked to the most influential political and economic shifts actually based on the legislative reforms (1945, 1953, 1963, 1974 and post 1989) \hl{(Pajovic 2005, Mornings after Nedovic Budic, waves of planning 2006)}:
\begin{enumerate}
\item  post-war reconstruction (1945-1953);
During the first years, SFRY was under strong influence of soviet political ideology. These were also the years of reforms and deployment of the ideology in socio-political, economic, and cultural terms of the newly established state. However, with slow but significant separation from the Soviet model, Yugoslav administration put to work the principle of workers' self-management of the enterprises \hl{(ref)}. The principle was that employees  had  a  key  role  in  the  decision-making structures  of  their enterprises, while in practice decisions  were  guided  by  management,  with  workers particularly  involved over socio-economic issues\footnotemark \hl{(Estrin 1983; Lydall,  1984)}. Such structure proved unsatisfactory and inefficient in creating effective capital  and  labor  markets from the very beginning.
While spatial resources have been expropriated, the state retained considerable control over the country's development by allocating investment  funds centrally and by putting urban planning in service of the regime \hl{(Estrin 1991)}. The structure of urban institutions was strongly centralized (central-command planning) and planning instruments are directed to support socio-economic development plans \hl{(Borovnica 1980, Pajovic 2005, Mornings after Nedovic Budic, Peric 2016)}. 
\footnotetext{questions of welfare, employment and pay \hl{(Estrin,  1983; Lydall,  1984)}}.
\item  institutional decentralization(1953-1963)
\\
The period was initiated by the constitutional change in 1953. \hl{ref}. The reform of social instead of state ownership was introduced in 1952. and came to the fore only in correlation with the new legal framework \hl{ref}. Such "social ownership" and nominal workers' control over the surplus was effectively in practice a form of non-ownership, with the plurality of self-management interests, delegate structure, accumulation decision making, but no legal individual rights over the assets \hl{(Estrin 1991, Zec 2012)}. In parallel, the country's external affairs were marked by the politics of neutrality and initiating the foundation of Non-aligned movement (NAM). The policy of institutional decentralization put forward the first generation of urban planning laws in the same manner \hl{(ref)}. As this new legal framework was enacted, a set of decentralization practices emerge in professional urban planning, especially from 1959. to 1970. In this light, new professional organizations spread around urban centres. Planning profession profited through organizational division of urban planning structures and the dispersion of roles among newly established entities (1954-1959) \hl{(Mornings after Nedovic Budic)}. 
permitted private ownership of small and medium business enterprises \hl{Hadzic 2002}
\item  strengthening of the republican level legislation (1963-1973);
\\
In 1963. the Constitution of the Socialist Federal Republic of Yugoslavia  and the Constitution of Serbia were adopted. This phase was the emancipatory phase of Yugoslav self-managed socialism marked by a strong economic factor in terms of the decentralization from state investment  funds  to  socially  owned  banks (1965). The reform aimed at the transition to market liberalism. It was grounded upon the belief that, under the liberal market conditions, enterprises based on social ownership would behave like those privately owned \hl{(Estrin 1991, Zec 2012)}. Consequently, these principles influenced urban development of Yugoslav cities and contributed to the golden age of Yugoslav planning profession under the second generation of urban laws adopted during this period. Urban planning discourse at the time was highly comprehensive, integrated, fully decentralized process closely coupled with economic and social spheres, with high level of public participatory concerning physical development \hl{(Mornings after Nedovic Budic)}.
\item dissolution, deficit and tensions (1974-1989);
\\
Declining labor productivity, inefficient enterprises  and  the  absence  of  effective  financial  discipline brought the emancipatory model of self-management to the verge of collapse \hl{(Lydall,  1984, 1989, Estrin 1991)}. The state of crisis exerted the tensions among the republics and further state decentralization through  agreement  by  all  interested parties was welcomed \hl{(Zec XXX)}. It brought about blossom of ethnic nationalism and later on was perceived as a means to guide  the  federal  political  structure  into  the post-Tito era \hl{(Estrin 1991, Post soc tranzicija Vujosevic)}. Conversely, urban planning organizations in these times of crisis were confronted with new economic conditions and the market. The attempts of adjustment brought in the third generation laws with significant proliferation on the republic level with hyper-production of urban statutes and regulation \hl{(Borovnica 1980, Pajovic 2005, Mornings after Nedovic Budic, Peric 2016)}.
\item unfortunate split and warfare (1989-1992);
\\
These final years at first indicate that, most of all, the peculiarities of the Yugoslav economic order led to the decision of high political (party) structures for Yugoslavia to abandon this unique socio-political system and  move towards western version of capitalism \hl{(Estrin 1991)}. On the contrary, economic  questions  have  increasingly  been  overshadowed by  ethnic  tensions, and in these circumstances, extensive and ungrounded decentralization after the Tito's death weaken the federal government against the republican decision making bodies and bring the federal country to the critical breaking point \hl{(Estrin 1991, Post soc tranzicija Vujosevic)}.
\end{enumerate}

\hl{Samardzic in Doytchinov 2015} 
political seen as a religion - political religion of Marxism

\paragraph{Urban Agendas, Institutions, Practice}
communist institutional and ideological framework
\hl{(Vujosevic and Nedovic 2006)}
1950s -  The planning model was locally adapted, Soviet-based, centralized  administrative  planning,  with  hierarchical  control  mechanisms  and  substantial  legal  power  exercised  through  5-year  plans.
Spatial  planning  and  controlled  urbanization  were  used  to  pursue  economic  growth  objectives  and  to ensure  rational  use  of  resources  through  “top-down”  allocation  (Dawson  1987; Papiü 1988).
physical aspects of urban growth
1965 on -  Planning  legislation  designated local communities (or communes) as the main planning and implementation authorities and was updated in 1984 to improve the coordination and integration of plans and policies. T
 	\hl{waves of planning 2006}
 	SFRY
while the five-year plans, investment plans and town
plans outlined a framework for harmonized development of settlements as manifestations of the new social organization [Dobrovic 1946], planners continued to apply the articles from the 1931 Building Act.
In 1949 the Master Urban Planning Regulation was passed.  Despite the inevitable linkage with the Eastern European/Soviet political ideology, this Regulation was drafted following extensive consultation of the German, English, Swedish, Dutch, American and French planning legislation.
the legal frameworks ensuing after the Second World War were based mostly on the Western models combined with widely spread Yugoslav model of self-management.
Achievements:
(a) national,  republic,  provincial,  and  local  agencies,  bureaux  and  institutes  were established [BGD: JUGINUS 1957, ZUKD 1962, IAUS 1958];
(b) professional associations were founded [Bakic 1988];
(c) experts were educated locally and abroad, mostly in Western Europe and North America;
(d) publications and professional conferences and symposia became regular;
(e) integrated interdisciplinary character of planning profession emerged (Vriser 1978);
(f) planning became a socially accepted practice (Piha 1986).
The period from the 1960s to the late 1980s could be characterized as the golden era of spatial (physical) and town planning in the Socialist Federative Republic of Yugoslavia:
    numerous  plans  covering  diverse  spatial  and  administrative  units
    interdisciplinary planning  teams
     increased public  participation 
The major innovation of this period was ‘integrated” or ‘integral’ planning, introduced as the ‘Basic Policy on Urbanism and Spatial Ordering’ and passed by the State Parliament in 1971.

\hl{Mornings after Nedovic Budic}
Former Yugoslavia departed from the Soviet centralized planning model soon after the WWII to develop a participatory system of integral (i.e. comprehensive or integrated) planning.
Haussemann (1996):  the diversity of planning systems that emerged under the broad umbrella of communist regimes and socialist planning.
Urban planning activities, therefore, range from the initial role in supporting the economic sector planning during the
post-WWII recovery, to an integrated fully decentralized participatory process concerned with physical development and closely coupled with economic and social spheres; and to post-1989 strife for re-legitimatization and re-establishment of a planning system that would address the changing societal circumstances and needs.
Planning context:
    1980s - golden age: SFRY among most decentralized systems of planning and policy; the principle of cross-acceptance practiced a decade before in developed western countries;  an integration of the social, economic, environmental and spatial/urban aspects into documents at all levels, except federal; massive number of detailed plans
        The 1985 Law on Planning and Arrangement of Space
        self-management planning, decentralized, bottom-up consensus building, cross acceptance;
        key authority: municipalities,
        actors: citizens, workers, socio-political organizations (communist party and para-political organizations)
        public land ownership

\hl{adjustment of planning practice nedovic budic 2001}
    1985:
        socio-economic development and harmonization of interests
        Spatial plans: republic, municipality, settlement, regional,special usses
        urban plans: general, regulation, detailed
        transparency: extensive public participation and professional control
        implementation: mid-term plans
        sustainability: environmental protection
        land ownership:
            land: societal, municila
            buildings, societal, enterprise, municipal, cooperative, private

\hl{Vukmirovic et al 2013.}
after WWII: the process of establishment and implementation of the general planning can be presented as development of single  paradigm.
Planning  in  Socialist  Yugoslavia  was  the  dominant  type  of  regulation  and  control  of  modern  society, economy and urban space.
rationalist model of planning is practiced as an 'eminently modernist project' (Vujosevic, 2004). all decisions were taken in the political centre.
all the results of the experts have been subject to permanent supervision of 
the state leadership that gave opinions and suggestions for further work and development. 
After Tito's death in 1980, the level of political differentiation increased.  
after 1990:
this  situation  created  a  fertile  ground  for different malversations,  while urban planning represented an uncoordinated set of fragmented interventions, ‘justified’ by the lack of funds, 
regulations, tools for implementation and, above all, well-defined, comprehensive development concepts.

hl{ref Peric 2016}:
the communist period (until 1989)
communist regime
centralised planned economy
the master urban Planning Regulation brought (1950) based on the western european planning legislation (Nedovic-Budic and Cavric, “Waves of planning”, 412.)
----- 1970s - semi-market economic system: western models in combination with a yugoslav self-governance model
decentralization, participation, strenghten the role of planning experise - paving the way for integrated and comprehensive planning
constructing new complexes, mass provision of affordable housing (social apporach)
"the self-managed neighbourhood with all relevant facilities, which was the core of new housing areas, was borrowed from Western experiences" - NBG example
---- 1970s and 1980s:
 integrated planning
interdisciplnarity
inclusion of not only experts of all kinds but also local politics and civil sector
however hierarchical political structure was supreme and no decision could be taken without the consent from the communist party
""ome authors even note that the principle of ‘cross-acceptance’ was used in Yugoslavia before it was implemented in Western countries"  (Vujosevic Collapse of stratgic thinking and Nedovic Budic and Cavic Waves of planning)

\subsubsection{Post-communism}
\hl{add poverty statistics where poor people are mentioned, from Grozdanic new authors find}
\hl{from Grozdanic find references}
Lack of consistency and completeness in the economic environment, followed 
by frequent changes of institutional regulations concerning business,  
Lack of stability in economic conditions – exchange rate and prices;

The crisis on multiple levels brought self-managed socialist system and SFRY to its halt. The dissolution of Yugoslavia coincide with the break down of other communist regimes in Eastern Europe. Conflicts and warfare led to the disintegration of Yugoslavia into several sovereign states. Only two of the federal republics, Serbia and Montenegro, stuck to SFRY state legacy and formed Federal Republic of Yugoslavia (FRY). Later on, it was transformed into a confederation - the State Union of Serbia and Montenegro (SCG) until the state referendum in Montenegro in 2006 followed by its separation and independence. During these geopolitical restructuring, the term "former Yugoslavia" has been commonly used retrospectively to designate the communist state 1945-1992.
\\
These political transformations were broadly present in all ex-communist countries. They are denoted as a policy of openness towards economic, political and cultural influences from the West, and consequently a transition towards markets and democracy \hl{(Vujosevic and Maricic)}. In contrast to other East European countries, the years that followed showed significant stagnation and regress in most of ex-Yugoslav republics. In Serbia, post-communist transition distinguishes two phases \hl{Mornings after Nedovic Budic} 
\begin{itemize}
\item 1990s with strong re-centralization, multi-paty system, sharp economic decline and international isolation;
\item 2000s with regime change, re-decentralization and progress towards democracy, wild proto-capitalism and unprotected public interest
\end{itemize}       
     
The urban transformation of Serbian cities falls into the cliché of the new post-socialist urban reality, which emerged during the "transition to market-driven economy and democracy" (Tsenkova, 2006). The dismantling of the communist system during the late 1980s represented a substantial change in all aspects of the economic model, the political system and social organization.
During the nineties, Europe was economically and culturally flourishing with peace, prosperity and order installed across its territory. Nonetheless, Serbia missed the train to join vibrant and healthy European realm and submerged into isolation and wars. \footnotemark This initial transitional period in Serbia is known as "blocked transformation" \hl{(Lazic and Vuletic 2009, Vujosevic and Maricic 2012)}. Economy was isolated, grey and semi-martial, stagnating with disastrous effects of hyper-inflation, pauperization and vanishing production. Social enterprises became private through murky privatizations and they were building their success rather by political and business connections than professionalism \hl{(ref)}. Being surrounded by wars and participating in them clandestinely, politically powerful actors in Serbia set up centralized decision making system for all political, economic and cultural topics of high interest \hl{Mornings after Nedovic Budic}. The political scene was dominated by one figure, Slobodan Milosevic. He ruled over Serbian and partly FRY realm through a type of nationalist dictatorship followed by political unrests from dissatisfied population and organized opposition supported by the West. This top-down government principle was supported by newly established economic elites, a powerful interest group, who supported authoritarian style of governance \hl{Vujovic and Petrovic 2007}. The situation of halted political decentralization caused institutions to collapse under the centralized authority, while political and economic pressures also deepened societal regressions \hl{Mornings after Nedovic Budic}. 
\footnotetext{Serbia was under the full UN sanctions since 1992 to 1995.} 
\hl{Mornings after Nedovic Budic}
\\
Serbian social structure had been crumbled accordingly. 1991. census document that Serbia with Kosovo numbered 9.791.475 inhabitants (7.836.728 without Kosovo) \hl{(jugoslovenska evidencija)}, while in 2002. the number of inhabitants in Serbia without Kosovo declined to 7.498.001 and it further plundered to 7.186.862 inhabitants \hl{(ref)}.
Without mentioning the rise in crime, drug trafficking and corruption, vast number of refugees migrating from war-zones, numerous local soldiers coming back from war with social disorder, immense emigration of young and educated population, the blossom of unprofessional, private local media influenced social relations in cities \hl{Vanista Lazarevic in Doytchinov 2015}. Uncontrolled immigration, self-destructive emigration,  and ethnic cleansing made the cities fertile ground for social and ideological hatred. The result was acculturation and political, ideological and social terror from the backward and paternalistic political actors in conjuction with nouveau-riche economic ones and further dissolution of intrinsic urban quality reached in previous periods \hl{Samardzic in Doytchinov 2015}. The same trend was dominating urban planning. 'Deplannification' and 'de-professionalization' were legitimized through top-down decision making resulting in urban deterioration, degradation, space occupation and illegal construction \hl{(Vujovic and Petrovic 2007, Vukmirovic et al 2013)}.
\\
\deleted{economic}
\hl{(Vujosevic and Nedovic 2006)}
Both domestic and foreign investments were only sporadic, while the informal sector of the economy expanded under the circumstances of the countries international isolation. 

The dismantlement of the socialist regime in 1990s radically altered the function of public space by making it a venue for the struggle for human rights, freedom of speech, movement and actions. Expecting to gain more freedom, people fought to replace socialism by a new neo-liberal order. Unfortunately, this new transitional episode, which is still in progress, has brought subtly controlled and carefully restricted freedom \hl{tech4tev online platform paper}. With the end of Milosevic's regime in the year 2000, the country entered fairly dynamic, but insufficient economic recovery. These transformation targeted primarily banking sectors and privatization of public companies enterprises. Transitional processes in the early 2000s are designated by marketization, privatization, and deregulation - the instruments of wild capitalism. Proto-capitalism accumulation and dominance of the tycoons, who were made rich in 1990s, were main catalysts of grand redistribution of social wealth and allocation of assets and incomes after the destruction of the former economic system \hl{(Vujosevic et al. 2010)}. Nearly destroyed industrial production, high percentage of unemployment, debt crisis, and income gap suggest unstable socio-economic sitution and long unsustainable prospects of development \hl{(Vujosevic et al. 2010, Mornings after Nedovic Budic)}. \hl{Vujosevic et al. 2010}. Especially the case of growing financial debts and strong influence of international political, economic and financial actors has been making the country rather a semi-colony than a prosperous environment on development tracks. Serbia has become a kind of new, inner periphery of Europe as \hl{Göler (2004)} puts it \hl{(Vujosevic et al. 2012)}.
\\     
\deleted{political}
This colonial attitude also reflects the ideological background of post-communist/post-socialist transition.
Serbia is a poor country with regressive rather than progressive tendency in development, stuck on the verge of social, political and economic crisis. In the last 50 years, its geopolitical profile changed multiple times. \deleted{box maybe?} In 2003, Yugoslav federation (3rd Yugoslavia) was replaced with Serbia and Montenegro confederation. With the independence of Montenegro, Serbia also got its country status in 2006. The Kosovo declaration of independence was adopted in 2008. In 2009 Serbia applied for the candidacy status for EU, while in 2012 it had it approved. \deleted{box end}.
Being on the European path, but not yet a EU member, puts Serbia in a position to reconsider its position in the global geopolitics and economy and adopts priorities and strategies for development accordingly. Its economic dependence on international capital and non-critical and non-strategic attitude towards political, social and cultural spheres of European integration suggest that alternative options are neither framed nor researched nor taken into account at all \hl{Vujosevic et al 2012}. With cumbersome institutional structure and non-competent administration inherited from socialism/communism, the principles of governance were not high-end priorities in Serbian political culture \hl{(trkulja 2012)}. When the capacity for research, strategic thinking and governance is reduced to zero, the country becomes a blind follower of global forces and a polygon for power and interest struggles \hl{Vujosevic and Maricic 2012}.
Thus the majority of reforms were exclusive, \hl{Vujosevic et al 2012}, revealing extreme asymmetry of power over their creation and implementation in practice \hl{(Vujosevic 2015 Regionalizam 2)}.
In the course of history, manipulation, paternalism and clientelism have been carefully and gradually braided features expressed to their fullest in  "systematic and organized mobilization of interests and bias" of Serbian political domain nowadays \hl{Vujosevic et al 2012}. In sum, in the lack of broad political and societal dialogue, transitional reforms were imposed by political and economic elites, while corruptive channels of the same elites dominate decision making process and everyday practices \hl{(Vujosevic and Maricic 2012, Vujosevic et al 2012}. 
\\
The post-Milosevic reconstruction of Serbian society contains prevailing characteristics of the disintegration of the preceding system rather than a coherent vision of what should follow (Stanilov 2007). However, the ideal changes should have considered radical shifts from (Petrovic 2009):
\begin{itemize}
\item Totalitarian to democratic political system
\item Planned to market-based economy
\item Public to private property ownership
\item Supply to demand driven economy
\item Industrial to service based society
\item Isolated to integrated position in the world economy.
\end{itemize}

\deleted{social}
Even with overall mantra of re-decentralization and democratization, top-down and centralized approach is manifested at all levels: national, regional, and municipal \hl{Vukmirovic et al 2013}. Centralization may be linked to the most powerful regional deindustrialization in Europe happening Serbia and significant territorial disparities among regions and between the inland and the capital, or the "Serbian spatial banana"\hl{\footnotemark} \hl{(Zekovic 2009 Regionalizatija, Vukmirovic et al 2013)}. The main loser of the transition are ordinary citizens, disempowered and impoverished during extensive periods of conflicts, crises and turmoils \hl{Mornings after Nedovic Budic, Vujosevic et al. 2010}. Difficult life conditions caused prolonged demographic recession, which already started in 1990s (brain drain, aging population, refugees) \hl{ref}. In pursuit of better life conditions, significant amounts of population migration toward economic centers and in this case mainly the Capital\hl{Vukmirovic et al 2013}.
\\
\deleted{urban actors and urban culture}
The social structure nowadays looks more like that of a Third-World country than of either European welfare state or its middle-class predecessor SFRY. A tiny layer of wealthy people, weakened and decimated middle class and rising number of poor population are urban actors of Serbian cities. Parallel to the still present nationalist discourse, mass consumerism and globalization values has entered Serbian society and cities after 2000. The loss of certain traditional values may not be a problem per se, as of exchanging the affinity to authoritarianism and cultural and ethnic isolation for cultural diversity, active communication, collaboration and participation in urban affairs and decision making etc. However, several authors argue that installation of contemporary neoliberal democracy and globalization values contributes to the economic and cultural erosion of the middle class society values inherited from socialism, threatens the sense of solidarity and empathy and dissolves the sense of overall public good and community bonds 
\hl{Cvejic 2010, Vukmirovic et al 2013, Samardzic in Doytchinov 2015}.
\\
\paragraph{Urban Agendas, Institutions, Practice}
\deleted{urban planning}
In the same manner political elites has also been indifferent towards urban policy leaving it at the mercy of wild capitalism principles dominating Serbian post 2000 discourse \hl{Vujovic and Petrovic 2007}. The dominance of the political over other spheres of urban life and the deficiency of systematic decentralization supported from political elites make spatial development not a priority, but rather spatial transformation has become a side-effect of political and economic decisions (Stojkov, 2009). Lack of policies, instruments and institutional measures neither contribute to horizontal, vertical and cross coordination nor increase a body of knowledge and data on the urban, resulting in reinvigorated illegal construction practices and illegitimate real estate transformations \hl{(Trkulja 2012, Cities in Transition 2013)}. Within the regulatory framework, the most disastrous effects are affecting urban land. Urban land is territorial capital neglected during socialism, where there was no competitive land market and restricted private property ownership. Consequently, the deficiency of property policy, laws and institutions as well as the lack of substantial urban policy, urban land management approaches and measures and urban land use strategies and rules are the major pit-holes of current urban planning, regulatory framework and practice \hl{(Stojkov, 2009, Zekovic et al. 2015)}
\deleted{define horizontal, vertical and cross coordination (coordination among individual decisions and public policies may be)}
\\ 
These factors provoked a legal void susceptible to shady deals and questionable public-private partnership (illegality), a lack of strategically proactive urban governance which resulted in tolerance to illegal building practices (informality), and the increasing social polarization (inequity) and poverty in this region {the number of poor people had reached 100 million in CEE by 2001 (Tsenkova 2006a)}. Such circumstances have had a profound influence on the spatial adaptation and repositioning of post-socialist cities in terms of (Stanilov 2007):
\begin{enumerate}
\item Urban management (illegality loosens official strategic action planning and implementation)
\item Urban patterns (informality reduces the spatial scale and spatial formalism of urban structures)
\item Urban impact, urban social practices (inequity leads to social and spatial stratification of urban structures).
\end{enumerate}

\hl{Mornings after Nedovic Budic}
    1990s:  shrinkage of planning activity
        The 1995 Law on Planning and Arrangement of Space and Settlements formalized this centralized control of planning and urban development;
        centralized power - republic
        state land ownership
        key authority: republic
        actors: citizens

\hl{adjustment of planning practice nedovic budic 2001}
    1995:
        rational use of space
        spatial plans: serbia, region, special uses, infrastrucutre
        urban plans: general city, general landscape, general infrastructure, regional, urban plan, urban project, subdivision plan
        transparency: professional control, public review
        implementation: urban permis and agreement
        land ownership
            land: state, municipal
            buildings: state, municipal, private

\hl{ref Peric 2016}:
post-communist phase (until 2000)
cold war ended in the world -> market-based economy everywhere
Serbia: political, social and spatial degradation
authoritarian political regime, strong decisive role of the national government, marginalized position of planners
high unemployment, the lowest GNP ever
turning housing into private property
slow administrative procedures
growing number of refugees
illegal construction
recentralization
top-down planning approach
exclusion of citizens and land development exclusively driven by private investments

\hl{(Vujosevic and Nedovic 2006)}
The context and issues of planning practice in the 1990s in post-socialist Serbia can  be  summarized  by  the  following  points  (Vujoševiü  2003;  Vujoševiü  et  al. 2000): 1) the lost legitimacy of planning; 2) a nonexistent or vague notion of public interest; 3) centralization of government and planning institutions; 4) lack of planning  expertise,  support,  and  administrative  capacity  at  the  local  communal level; 5) lack of regional governance and planning (Stojkov et al. 1998); 6) lack of 
strategic  planning;  7)  inadequate  or  ineffective  implementation mechanisms;  8) prevailing “physicalism” in planning; 9) lack of participation, openness and transparency in the planning process; 10) confusion regarding local ownership patterns and revenues, which hindered foreign direct investment (FDI) Ĉorÿeviü (2005); 11) persisting distortions in land markets; 12) undeveloped national land-use policy; and 13) political instability, manipulations, corruption and clientism. Thus by the new century, planning in Serbia laboured under a range of contextual difficulties.

\subsubsection{Urbanity of Serbian cities}

hl{ref Peric 2016}:
Serbian version of history - next periods running down the inventions from the previous on (public participation from the communist)
brief historical overview of the planning evolution in Serbia:
analytic coherence in terms of :
    context
    planning practice
    planning process

This rather organic path of urban development led to the classifying of post-socialist cities in transitional countries as unregulated capitalist cities (investment-led) with third world urban development elements (substantial illegal activities and informal markets) (Petrovic 2009).
\\ 
These historical processes of urban transitions affect Serbia's competitiveness on global scale by problematizing and reducing its local structural qualities and territorial capital \hl{Vujosevic and Maricic 2012}. Local experts describe current state of Serbian society as a whole as growth without development or even developmental schizophrenia \hl{(Vujosevic and Maricic 2012, Vujosevic ?)}.
\deleted{political-economic}
Its source may be traced back to the first years of the Serbian state (the kingdom of Serbia), when political overruled any other social domain \hl{(see XXX)}. This disposition of factors is still at play. In light of recent party pluralism in Serbia, multiplication of political actors do not bring democratization of the political sphere and constructive dialogue. On the contrary, political discord in Serbia is performed as a a struggle between enemies but their objective is to dislodge others in order to occupy their place. In reality, all political parties together work on preserving the dominant hegemony and stagnant power relations \hl{(Mouffe 2002)}.
\\
\deleted{urban form}
Instead of showing rigour and determination in finding creative, strategic locally adapted solutions, the path of development in Serbia is marked by western imports, stale dogmas and overrepresented international influence. However unlikely, this attitude also dates back to socio-urban practices of the previous periods and regimes. During most of the times (Ottoman, Serbian Kingdom, the communist regimes), imposition was the rule of the thumb of urban transitions \hl{(waves of planning 2006)}. Even though importing ,at least on the conceptual level, has had been continuously at play; at certain points in history locally grounded contemporary systems\footnotemark had emerged \hl{(waves of planning 2006)}.
local professionals  
\deleted{urban profession}
\footnotetext{For example, the local administrative commitment to planning in the Kingdom of Srbia; locally adapted and intrinsically unique world trends by distinguished local professionals (1914-1940); and interdisciplinary and participatory approaches in integrated planning process during the socialist era \hl{(waves of planning 2006)}.}
\\
\deleted{social}
\deleted{urban culture}
In this light, social and urban revitalization from the wars of the 1990s is still encumbered with inherited local tradition of nationalism, hierarchy, authority and parochialism coupled with consumerist and neoliberal global forces \hl{(Stupar 2004)}. The urban quality of Serbian cities still suffer from constant waves of economic migrants from rural areas who have difficulty to identify with a modern capital, like it was during the first years of the Serbian capital \hl{(see XXX)} \hl{Doytchinov 2015}. Not to mention strong nationalistic, spatialized identity \hl{(Savic 2014)} that threaten multicultural and multi-ethnic urban fusion cherished during the socialist period \hl{(Stupar 2004)}.
\\
These historically bounded socio-spatial patterns embody actual urbanity of Serbian cities. As already mentioned before, Serbia is under strong influence of social, spatial and, in general, an overall policy of centralization. In this respect, the capital city of Belgrade is condensed and depicted example of urban transitions present in Serbian cities.
As of the imagination of a famous contemporary Serbian author, \textit{"Belgrade is a mill for producing the urbanite "psychological amalgam" out of the autochthnous peasant Serb from the mountains with the civilization washing over him from the northern plains"} \hl{(Velmar-Jankovic)}. Not only a capital or economic center, Belgrade is the model and the microcosm of the nation \hl{(Zivkovic XXXX)}

\hl{Vujosevic 2015 Regionalizam u Srbiji 2}
tehnicko retusiranje sistema planiranja
1992.- uspostavljeni okruzi, prvi put posle SFRJ kao primer blage i selektivne decentralizacije - 29 okruga - samo teritorijalne ispostave ministarstva (puka transmisija centralnih vlasti i kadrovska politika)

\hl{(Mornings after)}
1995 Law on Planning and Arrangement of Space and Settlements of the Republic of Serbia

\textbf{"Za Srbiju moze vaziti jedno opste pravilo: cija vlada toga i drzava,cija vlast toga i sloboda" (in Serbia there is a rule at stake: They who form the government, they lead the country and rule over freedom) Dubravka Stojanovic in the book "XXX" citing an anonymous MP during the period of the Kingdom of Serbia}

\subsection{Belgrade - a City in Constant Transition}

\deleted{Figure Belgrade 19-20century, Roter Blagojevic in Doytchinov}
\deleted{1.geopolitical data}
The city of Belgrade has the fixed location in the northwestern part of the Balkan Peninsula and Southeastern Europe for centuries. It  has  been  built  at  the borderline of two large geographical areas, where the Pannonian Plain meets the Balkans. It overlooks the confluence of the Danube and the Sava rivers and spreads along their banks to the north, south, east and west. First traces of its existence date back to around 5000 BC, from Vinca culture developed nearby, Celtic settlement of Singhidunum, Belograd mentioned in the Papa Iohannes letters from the 9th century \deleted{ref  "LIBI, t. II (1960) (2_151.jpg)" http://www.promacedonia.org/libi/2/gal/2_151.html}, Alba  Graeca,  Alba  Bulgarica,  Nandor  Alba,  Griechisch Weissenburg, Castelbianco, variety of names used until 19th century depending on its current rule, finally to the modern Serbian capital Belgrade. From the early period of the Roman empire split, and even more so during the Ottoman rule, the city was the crossroads between the East and the West. Comparably, its contemporary location is the junction of two pan-European transport corridors (Corridor VII from Romania to Germany, and Corridor X from Greece to Austria and Germany). As a node of regional importance, Belgrade also belongs to the category MEGA4 of the European areas of growth and development.
\\
Its extraordinary location has made Belgrade suffer from continuous invasions and consequent waves of deconstruction and rebuilt. In the course of its history, Belgrade has been heavily destructed and rebuilt forty times, bearing in mind that it was the only European capital to be bombed at the end of 20th century \hl{(Samardzic in DOytchinov 2015)}. Therefore, it is hard to speak about the city of Belgrade in historical terms as there are several  dozen  consecutive  Belgrades fading and reappearing over time \hl{(Grozdanic 2008)}.
\\
Belgrade is the historical capital of Serbia since the constitution of the Serbian nation state in the 19th century (1867). Its geographical position has always been close to the national border, a location easy to reach and occupy \hl{Doytchinov 2015}.
Nonetheless, it also had been the capital of Yugoslavia, throughout its overall period of existence - Kingdom of Serbs, Croats and Slovens (1918-1947), Socialist Federal Republic of Yugoslavia (1945-1992), Federal Republic of Yugoslavia (1992-2003); and the capital of the State Union of Serbia and Montenegro (2003-2006), before the final split and the re-establishment of the Republic of Serbia (2006-). After the national revolution period, abrupt shifts in its status and field of influence\footnotemark complicate its ideological, political and cultural articulation in local terms and towards national and global discourses. Equally, in historical terms, there had been only three periods of peaceful and prosperous urban transitions:
\begin{itemize}
\item when Ottoman garrisons left the city and before the outbreak of WWI (1867-1914)
\item the period between the two World Wars (1918-1941)
\item  the period of Yugoslav self-managed socialism (1941-1991)
\end{itemize}
\footnotetext{From 19th century onward, Belgrade was at first the capital of a small nation-state, then the capital of a big federation and afterwards the capital of an ever declining territory, finishing finally as the nation-state capital once again \hl{(HIrt 2009)}}.
\\
\deleted{3. influence of Serbian society on urbanity + urban culture}
\deleted{2. population - urban actors}
\deleted{population rise diagram}
Even during the periods of conflicts, the number of Belgrade's inhabitants has been constantly rising. When Serbian principality gained control over the city, it was rather an Ottoman dorp (kasaba) with 27,000 inhabitants. Before WWI, the population rose to 90,000 (1910.) and reached more than 300,000 at the brink of WWII. During the Yugoslav period the population almost tripled (from 397,911 in 1948. to 1,133,146 in 1991.).
\\
Its metropolitan area has been expanding accordingly, but with different pace and rate from the population growth. Namely, low urbanization has been caused by agrarian society basis and consequently poor industrialization in the first years as the capital of the nation-state. Low-rise, low quality, often illegally built residential areas are at the core of Belgrade's urban expansion at those times. Belgrade has been densifying, expanding and settling down the layers of spiritual and material heritage, with the various human factors directing its urban transitions. Urban culture 
and city form are an expression of social and spatial continuity as well as destruction and discontinuity \hl{(Grozdanic 2008)}. In the case of Belgrade, Roman fortress and initial grid grid street structure are still recognizable in the central area of the city. The foundation of Belgrade's landmark Kalemegdan Fortress dates back to roman period, but it also keeps traces of a medieval town within the walls during the Serbian medieval state. However, in this research the historical development of Belgrade is described within the extended discourse of urbanity
\begin{enumerate}
\item Ottoman Belgrade (1521-1804)
\item The capital of the kingdom state (1804-1941)
\item The socialist city (1945-1991)
\item Post-communist urban path (after 1991)
\end{enumerate}

\paragraph{Ottoman Belgrade}
During the Ottoman rule of the Balkan peninsula, Belgrade pay a toll of its borderline position during the ceaseless Austro-Ottoman wars. The city frequently passed from Ottoman to Habsburg rule, yet it kept an overall flavour of the Ottoman border town with an urban structure of the Oriental city \hl{(Hirt 2009)}. According to the statistics, the city population did not cross 25,000 during this period.
\\
Even though Habsburgs occupied Belgrade several times
(1688-1690, 1717-1739, 1789-1791), it received some basics of European urban patterns only during their longest rule between 1718 and 1739 \hl{Samardzic in Doytchinov 2015}. Strict Austro-Hungarian administration also imposed urban regulations in terms of equal plot sizes, street patterns and building shapes with necessary facilities and numerous ornaments in currently popular european architectural styles \hl{(check styles ref Kadijevic, Mladjenovic XXXX)}. Habsburgs also introduced focal public spaces into the settlement structure with important civic and community buildings built around it \hl{Kadijevic, Mladjenovic XXXX}.

The city has been continuously rebuilt under Islamic principles, with central street (carsija) and organic street network, mosques (dzamija), market places (bazar), every time it passed back under the Ottoman jurisdiction \hl{(ref)}. Yet, Brush’s map from 1789 reveals great mixture of both Ottoman and Habsburg urban matrix - an overlapping of narrow, curved streets with parts with straight street grid and the Great Oriental Market located located in the central square of the city \hl{(Roter Blagojevic in Doytchinov 2015)}. at the end of the 18th and the beginning of the 19th century, most sources confirm that the city was rather a small ruined and negligent Ottoman fortification with civilian neighbourhoods suffocating under uneven, congested and unhygienic urban structure, while the rare, sporadic Christian neighbourhoods were scattered around and immersed in the suburban landscape \hl{Roter Blagojevic in Doytchinov 2015}. Finally, the city that Ottomans left behind after the symbolic "city key delivery" in 1867 was actually a hybrid Oriental-Occidental city \hl{(Blagojevic 2009)}.

\paragraph{Belgrade (1804-1940)}
The first record of Belgrade’s population is from 1838. There are 8483 Christians, 2700 Muslims, 1500 Jews and 250 foreigners, in total - 2963 people. This record dates back to the times between the enactment of the Turkish Law (Hatisherif) in 1830, which institutionalize Belgrade as a seat to both the Serbian and Turkish administration and the official establishment of Belgrade as the capital of the Ottoman vassal state of Serbia (1841).\footnotemark In the time span before the city evolved into the capital,the spatial concept and the building construction patterns still followed vernacular Ottoman traditions \hl{Roter Blagojevic in Doytchinov 2015}.
\footnotetext{In 1841. Prince Mihailo Obrenovic moved the capital from the city of Kragujevac to Belgrade.(ref "History (Important Years Through City History)". Official website.}
\\
When Serbs gained limited authority over the city (rather a town at that time), they addressed their construction efforts more toward European-like architectural design than the planning. The town was divided into three parts: the town encircled by the Moat with predominantly Muslim population and minorities (Jews), the Fortress with the Turkish garrison, and village-suburbs outside the Moat populated by Christians \hl{Blagojevic 2009}. The Turkish Plan of Belgrade made in 1863. actually legitimize the religion-based population division \hl{ref +Roter Blagojevic in Doytchinov 2015}.
Growing interference of the Serbian side in the city administration and management made these individual, chiefly civic, buildings, newly built and decorated in various historicist styles, stand for strengthening Serbian national identity and its obvious striving to join the broader context of European civilization \hl{(Hirt 2009)}. In this respect, the yawning gap between the Oriental and Serbian (more European-like)\footnotemark part of the town in terms of urban matrix, structure, and even urban culture has became obvious even before Belgrade had become the capital of the Serbian principality.
\footnotetext{The very first urban planning efforts were oriented towards outer city development of Christian/Serbian neighbourhoods, for example on the river port and outside the Moat. \hl{(Blagojevic 2009)}}
\\
Therefore, even before the official takeover of the city from Ottomans, the Serbian new capital had asserted itself as the supreme national administrative, economic and cultural center. The period of Serbian state construction was also the period of shaping the European identity of its capital city and gradual application of European planning ideals therein.\footnotemark The year of the official departure of Ottoman administration and military from Belgrade (1867) coincide with the date of the very first try with General Urban plan of the city.
\footnotetext{Introduction of squares and plazas in the urban matrix, building ornamental fountains and placing sculptures glorifying national heroes on horseback were among the indices of European urban design trends to date \hl{(Hirt 2009)}.}
\\
The necessity of having an urban plan that regulates the development of the capital city originate from Prince Mihailo Obrenovic's vision of putting Serbia in the cultural and social league of other Central and West European countries. A detailed geodetic survey of Belgrade's soil was undertaken by Emilian Josimovic himself in 1864. A mathematician by education and an engineer by profession, the plan he prepared in 1867. was labeled as the regularisation plan. Very technical in its apporach and with significant practice-oriented data, the mentioned plan correspond to what today is a combination of a general and master plan, a plan of implementation or feasybuility study.\footnotemark Josimovic's plan envisioned re-unification of the two parts (free Serbian city and Ottoman fortress) clearly separated within the previous Ottoman plan from 1963 (see above). The plan could be considered in line with the essentials of the European planning paradigm of the period (ring zone). 
\footnotetext{The plan contained many numeric data and comparative tables; proposals  for  implementation with procedures and responsible institutions, dynamics of works, principles for calculations of value of land and cost of regularisation \hl{(Blagojevic 2009)}}.
\\
An eminent local architecture theorist and urban historian \hl{Ljiljana Blagojevic (2009)}, however states that with the proposition of an idiosyncratic urban structure, Josimovic's model goes beyond simply reproducing European models. He insists on a network of open and free green areas (parks and town wreaths) created for the sole purpose of circulation, recreation and leisure to all its citizens. \hl{Blagojevic (2009)} argues that by this emphasizing these new public and social space, Josimovic was actually praising the liberated 19th century Serbia.
IN opposition, it is generally accepted within Serbian urban planning discourse that Josimovic's plan rids the city of its former identity \hl{(ref)}.
In fact, Josimovic pioneered a long-lasting paradigm of de-Ottomanization (or de-Orientalization) and Europeanization that was first imposed upon Belgrade, then on other cities in Serbia and finally on society as a whole. Establishing an official policy of destroying Ottoman urban legacy and traditional urban structures originating from the 16th century, the General Urban Plan of 1867 stripped Belgrade of segments of its collective memory and left it susceptible to tabula rasa approaches when solving urban conflicts - this may be the harshest criticism made by its opponents \hl{(Roter Blagojevic in Doytchinov 2015)}.
However, both cluster of authors agree that Josimovic instituted a 
paradigm of uncompromising radicalism which has nurtured the generations of urban professionals in Serbia \hl{(Blagojevic, Roter and other)}.
\\
In practice, General Urban Plan from 1867 was coupled with Law on Regulation of the Town of Belgrade. However, both the plan and the law was rejected in Parliament \hl{check where} under the politically biased circumstances.\footnotemark This was also the relief for Belgrade Municipality authorities and administration as they were lacking technically skilled stuff to accomplished demanding reconstructions proposed in the plan. It appears that multiple urban stakeholders of the time hailed the prolongation of the status quo, where any further urban regulations did not have an appropriate legal basis \hl{Roter Blagojevic in Doytchinov 2015}. It was therefore no surprise that the town was still greatly  expanded  beyond  control,  but mostly in terms of  low quality illegal  construction  in the suburban areas \hl{Roter Blagojevic in Doytchinov 2015}.
\footnotetext{Land and property owners at those times were influential enough to lobby against the law as they were reluctant to accept any changes that could threaten their premises.}
\\
By the end of the century, two more urban plans were proposed - Stevan Zaric's(1878) and Jovan Beslic's (1893) plans - and Construction law for the city of Belgrade was adopted in 1896. These urban plans tend to build on Joksimovic's plan and continue with European modernization trends. However, by gradual abandonment of the core innovation of the Joksimovic's plan - idiosyncratic idea for the network of urban parks  and town wreaths - these plans show more of copying western models than fine tuning of traditional and modern within the boundaries of newly established country and its rising capital \hl{Blagojevic 2009}. On the contrary, the adopted law (1896) was not at all modernizing, but served for the legitimization of the system and interests at play. It was a populist compromise to leave the state of affairs as is, with no sanctions for not abiding the law and where the corrupted judiciary failed to apply it in practice \hl{(Dubravka Stojanovic konf BGDH2O)}. 
\\
The regime change\footnotemark and the extended period of peace (from 1804. onward) gave rise to social prosperity, cultural upswing and stabilized institutional framework at the beginning of the 20th century. Consequently, it brought a multiplicity and heterogeneity of urban forms, renovation of public spaces and large scale projects. Coupled with the military success in the Balkan wars, the overall circumstances made the very last years before the WWI the peak of Serbian social and cultural revival of the time. Instated financial mechanisms, formalized bureaucratic procedures, trained administration and improved public service facilities brought success to several projects taken up during these years (1905-1912) \hl{(example and ref)}. Finally, Master Plan of  Belgrade 1912 was prepared by a young Parisian engineer Alban Chambon in the typical manner of the European academic tradition of the 19th century. The plan was a symbol of the rising social potential and the manifest of the majestic ambition of the ruling class to be the part of Europe and nothing but the Europe. 
\hl{(Blagojevic 2009, Roter Blagojevic in Doytchinov 2015)}. Nothwithstanding the authorities, local experts asked for preservation of the inherited urban pattern and local expertise. They opposed "haussmannization" of the town and the demolition of the heritage. \hl{(Roter Blagojevic in Doytchinov 2015)}.
\footnotetext{In 1903. after the military coop when both king Aleksandar Obrenovic and queen Draga Masin were executed, Karadjordjevic dinasty was set on the Serbian throne.}
\\
The pre-war Master Plan from 1914. settled down the tensions by locally framing the Monumental City design tradition\footnotemark \hl{(Perovic iskustva proslosti)}. This planning trend was influential during the first years of the new, larger state - The Kingdom of Serbs, Croats and Slovenes. The  General  Plan  of  Belgrade  from  1923 was the first urban plan to be officially adopted. It was of the same nature as the Master Plan from 1914, but even more megalomaniac and with an egocentric attitude boosted by the significance of the new larger state. The expansion of the state also made the city expand territoriality toward the north to include the Zemun area, which had a largely Slavic population yet had been ruled by the Hapsburgs. The plan targeted radical after war reconstruction reconstruction with dense urban fabric with medium-scale residential and mixed-use buildings. However, the plan was prepared by the team of exclusively foreign architects which may explain why the proposed interventions actually negate Belgrade’s  topography  and  its  urban  character \hl{(Grozdanic 2008,Blagojevic 2009)}.
\footnotetext{Monumental City design refers to orthogonal street system, distribution of urban parks, multiple long diagonal vista and spectacular public plazas at the intersections \hl{(Hirt 2009)}}
\\
The General Regulation Plan 1939.
a paragraph in which state Belgrade was before WWII
\\
\textbf{conclusion on the period}
In the course of the 19th century, Belgrade paved its way as the national capital facing Europe out of remnants of an Oriental border-town. However, swift transformations and abrupt changes of its urban system were rather supplemented by sluggish adaptations, imported innovations and generally maintenance of uninstitutionalized practices and of the variety of nepotistic relations within decision making structures. Urban development of Belgrade as the capital of the first Serbian nation state therefore seems like a bouillon of doings and not-doings in the city that nevertheless eventually produce results. For example (based on \hl{(Stojanovic 2015)}):
\begin{itemize}
\item urban transformations - sluggish adaptations
\\
\begin{itemize}
\item Partial decision making: hyper-production of solutions with the total lack of strategy and systematic approach results in ungrounded and nonfunctional urban projects and consequently the doubt and reluctance to complete them
\item 
\end{itemize}
\item urban change - imported innovation
\begin{itemize}
\item Apotheosis of the western models
\item Intervention initiatives left to private individuals or funds 
\item Economic and political actors and interests braced together
\item disregard for the opinion of local experts
\item importing grandious, inadequate, self-glorifying ideas: discourse of smallness
\end{itemize}
\item urban maintenance - maintenance of stagnant and backward practices
\\
\begin{itemize}
\item In the lack of political will for implementation, decision makers constantly stick to "temporary solutions" while the costs of implementations and adjustments are rising.
\item Incompatibility of political ideology and the economic model at play: the politics of urban growth thwarted by the inefficient economic model (no budgetary allocation for the capital city, tax system was not adapted to urban environment prerogatives,\footnotemark no incentives on urbanization and construction)
\item translation of ottoman nepotism tradition into the party state: decision making in the multi-party nation state reduced to party level
\item party interest holds supremacy over any other interest: strong liason between the political and the urban in the party state
\item disregard for the public interest - the policy of obstruction and destruction as a party campaign
\item reproduction and expansion of corruption mechanisms: the culture of populist measures for the party sake 
\end{itemize}
\end{itemize}
\footnotetext{Tax system did not stimulate the construction}
the major question was the relationship between the city and the policy, the authorities and the professionals

\textbf{"It happened that in the capital the opposition held the power very often, obstructed by the state authorities and breakdown of the decision making system. When the party in power was changed, it adopts new standards and forced to dismantle all the structures brought up by the previous regime." \hl{(Stojanovic 2015)})}

\paragraph{Yugoslavia}
In SFR Yugoslavia Belgrade became an important multinational and  multifunctional  metropolis. During these 50 years, the image of a dorp between the East and the West was transformed into a modernist city with a cultural scene spreading its tentacles both toward the East and the West. Belgrade was and is the symbol of Yugoslav self-managed socialism. The system reflected itself in the social layer and spatial structures of its capital city.
\\
Consequently, all political and planning decisions could be traced also within urban transitions happening in Belgrade of those times.

\begin{enumerate}
\item urban planning in service of the regime (1945-1953)
\\
Initial post-war goals were simple and state-forward: to rebuild the war-damaged urban fabric . As of the Soviet model, the first post-war local plans were to strictly follow the orders provided in the five-year national economic plans. 
\\
The Design of the General plan for the city of Belgrade in 1948 was the first post-war urban planning document prepared by Nikola Dobrovic, the director of Serbian Urban planning institute. The plan primarely dealt with   transport system modelling a new transportation network more suitable for the expected population growth. The design of the plan was preceded by Several transportation studies (for all types of transport). The presented design seemed radical and unrealistic and not only that it resulted in rejection, but the chief architect Professor Nikola Dobrovic was transferred to work at the Faculty of Architecture.
\deleted{(effective, comparing to the legacy of bombardment in 1999)}
\item professionalization of planning (1953-1963)
\\
Having this reconstruction task accomplished in a decade, the multidisciplinary teams (comprising planners, architects, engineers etc.) set to work on the construction of massive industrial complexes in order to cater for the exploding population growth \hl{(Hirt 2009)}. The result was the construction of the New Belgrade, an urgent, gigantic mass housing estate, built according to CIAM principles and in complience with Athens charter.
\\
It may also be said that this phase actually started with The  General  Urban  Plan  of  Belgrade of 1950 by Milos Somborski. The plan endorsed urbanization of New Belgrade (Novi Beograd). However, the plan not only dealt with the expansion of the city to the lecft bank of the Sava river and with projects for Novi Beograd, but also proposed the reconstructin of the central zone of Belgrade \hl{(Grozdanic)}.
\item comprehensive, integrated planning process at work (1963-1974)
\\
The ideas and ideals of comprehensive and integrative urban planning found its actualization in The  General  Urban  Plan  of  Belgrade  adopted in 1972. The authors, Aleksandar Djordjevic and Milutin Glavicki, by honoring the values of the past, called for historic preservation and architectural contextualisation of Belgrade's urban fabric. Moreover, they advocated more  rational  use  of  land  and integration of city functions by also making the final decision on Ada Ciganlija zone by attributing it leisure and recreation functions exclusively. They also took into account that some relatively large industries are located in attractive parts of the city (e.g., in Novi Beograd) as a result of the communist policy of prioritizing industry over other land uses and thereupon proposed more equitable destribution of new infrastructure projects and related  facilities and better  transport connections  among  the  parts  of  the  city  \hl{(Hirt 2009)}.
\item the pioneers of urban planning decentralization(1974-1989)
\\
Yugoslavia’s continuing political decentralization and democratization in the 70s was mainly visible through the local level of decision making, successful participatory initiatives and the multiplication of projects and implementations. In Belgrade’s fabric from the 1970s, this is reflected in the break with severe Modernism principles and timid introduction of new building styles \hl{(Hirt 2009)}.

The Modifications  and  Supplements  to  the  General  Urban  Plan  of  Belgrade  up  to 2000 were adopted in 1985. The author was Konstantin Kostic. The Plan did not differ much from its predecessor, but its purpose was to propose and implement more realistic solutions \hl{Grozdanic}
\end{enumerate}
 
\textbf{Yugoslav urban discourse}
Urban transitions during the self-managed socialist era were in fact interventions in the urban fabric of Belgrade that a clear break with pre-WWII spatial and built patterns \hl{(Hirt 2009)}. Superior architectural desing quality and progressive trends in urban planning made those districts in Belgrade built in SFRY globally recognized and attributed to Yugoslav socio-spatial discourse. To a certain extent, it may be stated that the modern stylistic ideas for modern  public  buildings  and  large  housing  estates were adapted to the local conditions in Yugoslavia and Belgrade \hl{(ref)}. Moreover, these Modernist ideals of industrial efficiency and progress influenced the corresponding social practices and produced a new wave of urban culture in Belgrade’s Modernist districts.
\\
It may be argued that the essence of Belgrade's contemporary urbanity is based on the management of conflicts and resources and the production of urban practices during the SFRY period. The city had undergone constant transformations while the corresponding system of planning was evolving - from the initial phase of selective borrowing, through system transformations by its internal adjustments to the final synthetic innovation represented in its own model of integrated, participatory planning \hl{(Nedovic budic waves of planning)}.

\paragraph{Post-communist}
The post-communist period influenced Belgrade the same as the rest of the country, if not even more intensely. Urban Belgrade suffered a certain decline with the post-communist transition, a sharp one at first and a questionable recovery with several periods of prosperity later on. This periodization goes along with that indicated in the country/region:
\begin{itemize}
\item The capital of the 3rd Yugoslavia in post-communist circumstances (1990s)
\item The post-communist capital in transition to markets and democracy (2000s)
\end{itemize}

\subparagraph{The isolated metropolis of the 1990s}
Political and social circumstance of the Yugoslav "break-up" changed the climate in the capital of the crumbling state and of new, shrunk Yugoslavia-to-be.  Under the umbrella of protecting the sovereignty, territorial integrity,  and functional unity of the truncated state, the country had been re-centralized. As a consequence, the constitutional role and the authority of the local, regional and city levels were weakened and reduced to the minimum \cite{Vujosevic 2015 Regionalizam u Srbiji 2}. 
\\
In times of raging civil war in the region, Serbian and ex-Yugoslav capital was caught in a stagnant and even backward position.
The major population outburst happened during the 90s when the refugees of Serbian origin came to Serbia from war-affected regions in Slovinia, Croatia and Bosnia and Herzegovina. Most of them settled down in Belgrade as in the re-centralized country in the state of economic crisis, only the capital may have offered any possibility for people evicted from their homes and deprived of their possession to start a new life.
The antithesis of the fascinating development of communist Belgrade was its ideological  degradation  based  on regional militarization, nationalization, ruralization, pauperization and, in general, the state of corruption, crime and chaos \cite{ref}. 

In times when previous political ideology were falling apart, the necessity to produce an appealing "creo" for the imploding populous had retrograde social effects. In the 90s, the ideology of self-managed socialism in the social realm was replaced by previous forms of social relations, namely by traditional models, mythicizations of the nation-state and exceptionalist discourses of heroism and smallness \hl{(explain in footnote)} \cite{(Savic 2014, Samardzic in Doytchinov 2015)}. Surprisingly, the actors and regulatory frameworks involved in both ideological concepts were the same. The middle class, already deprived of its economic assets and the acquainted cultural matrix, became confused and apathetic.
The influx of refugees migrating from war-zones paired with extensive braindrain was also complicating the social structure of the city \cite{Vanista Lazarevic in Doytchinov 2015}.
\\
Excluded from the map of global cities during this nationalist regime , Belgrade experienced the blossom of illegal construction, a naked vandalism of overbuilding and inappropriate public space occupation, and the flourish of informal business practices, the product of crony economy at high levels \cite{ref}. At the city level, the results were the rise in crime, drug trafficking, and corruption, an overall state of moral decay in local communities in general \cite{(Prodanovic Stariji i lepsi Beograd)}.
\\
During a decate of continuous crisis, the physical structures in the city began to deteriorate. The refugee crisis, lack of any official construction projects and the explosion in the number of illegally built dwellings\footnotemark signified an extensive state of shock for the city. However, NATO bombing in 1999 was actually the peak of the crisis. The bombs brought the real war over the rooftops of Belgrade and collateral civilian damage in the city. A number of buildings still bear signs of the damage, naming but only the most significant - The Yugoslav Ministry of the Defence building in the central urban area of Belgrade.\footnotemark 
\footnotetext{The majority of illegally built dwellings in Belgrade were homes for the upper classes and Milosevic's elites constructed and ornamented in rather lavish and kitschy styles that best represents the state of values and qualities of these "nouveau riche" profiteers of transition \cite{(Hirt 2009)}}.
\footnotetext{The building was designed to symbolize the decisive WWII battle when partisans defeated Hitler's forces in the canyon of Sutjeska. The form of the two buildings represent the canyon itself. The designer was famous Serbian architect Nikola Dobrovic.}
\\
Taking all this into account, the picture of urban actors and urban culture of the 90s was gloomy, poor, silenced image of stagnation. While the planning profession was suffering from a major legitimacy crisis at the state level, local planning regime in Belgrade was also in the state of collapse. The institutional framework and planning practice in Belgrade had rather a symbolic and superficial role, or even worked as a means at hand of the politicians (Vujosevic and Nedovic-Budic, 2006). As such, the urban transitions the city was undergoing were happening either spontaneously or they are directed outside the corresponding regulatory framework.

\subsubsection{Belgrade Now}
%1. conclusions from the previous historical part in intro in 2000s
Belgrade stepped into the 21st century in the state of the prolonged emergency.
The city went through the state of emergency under the NATO bombs in 1999. Almost did so again in October 2000 when the opposition took over the power after the citizen revolt and mass demonstrations. And then once more in 2003 the official state of emergency was lifted after the assassination of the Serbian prime minister Zoran Djindjic. The year 2003, 2006 and 2008\footnotemark also figure the points of discontinuation in Belgrade's history as the state capital.
\footnotetext{2003 was also the year when Yugoslavia was officially replaced by the state union of Serbia and Montenegro. In 2006 Montenegro became an independent state. In 2008 Kosovo proclaimed independence}.
To be brutal, it actually means that Belgrade lost its Europe-wide significant role as a metropolis where the West and the East meet and break the iron curtain \cite{(Grozdanic 2008)}.
\\
%the role of the capital
Respectively, its role as a capital of the unstable geopolitical realm puts the city time and again in the position to rule over the minorities of different nationalities and different lifestyles in the remote areas of the country (1st, 2nd and 3rd Yugoslavia). If not otherwise, these circumstance strongly endanger the acceptance of the Belgrade as the symbol of the state. Not to mention, the clash of interests if reaching the broad societal accord for the privileges it, as a capital, deserves.
\\
In general, such circumstances imply ethnic diversity and multicultural urban fusion.
Unfortunately, Belgrade had dominated these state realms with negligent imposition and pressurized  assimilation in an overall poor  rural  environment, under  the obsession of the nationalist agendas, and in forms  that  fostered ethnic misapprehensions  and  conflicts \hl{(other ref Samardzic in Doytchinov 2015)}.
Therefore, over the course of different states, Belgrade has always been perceived as predominately the centre of ‘Serbian-dom’ \hl{(Heppner in Doytchinov 2015)}.
Coming off as a retaliation, Belgrade has been often governed in such a way that its own ruling class jeopardize its development by a current of alienated, estranged  decisions \hl{Samardzic in Doytchinov 2015}.
\\
Its vulnerable geo-strategic location, turbulent history and unsettling sociatal framework make the city struggle to set up its modern identity and accordingly to defend the landscape and public interest of all its inhabitants. As a consequence, Belgrade has always been combination of the rural in physical and organizational terms and strong tendencies to rise cultural and ethnic cosmopolitanism. The respect for urban memory of the city is still present through the centrality of its diversified cultural-historical matrix, from the center (Knez Mihajlova street), New Belgrade to Zemun. Its unique and stronlgy bonded unity of historic and architectural heritage and its vibrant civic life lead to having Belgrade voted ‘Southern European City of the Future’ in 2006-2007 \cite{(Hirt 2009)}.
On the other hand, global real estate market forces introduced in Belgrade with the liberalization and democratization present from the political changes in the year 2000 press for de-industrialization and the aesthetics of globalization in the capital city \cite{Grozdanic}.
\\
%urban form - conflicts and resources
Regardless of relative de-industrialization, Belgrade  still is employs the largest amount of the country’s industrial labor-force (20 percent) \cite{(Hirt 2009)}. Not to mention, ever increasing commercialization of the urban fabric started in 1989. While in 1990s dominant were small and local retails chiefly around the city center, in 2000s  malls and hyper-markets sponsored by a combination of Western and Serbian capital thrive in greenfield areas at the fringes of the central urban zone (New Belgrade) \cite{(Hirt 2009)}.
As a sign of cultural and real estate revival of the new political regime (2000-2004), these commercial zones as well as several interesting location in the historical cores of Belgrade and Zemun were topics of numerous  architectural  competitions\footnotemark \cite{ref(Stupar 2004 isocarp)}.
The goal was to put together or available creative and expert forces to work on the renewal of urban culture, improving living conditions in the city and preserve the oasis of nature in the urban fabric (riverbanks, parks within city blocks, grecn areas, urban forestry) \cite{Grozdanic}.
However, the issue of inefficient transportation system and traffic congestion has been systematically neglected.
In the rush to build more and quicker, the new structures reveal subtle eclecticism of styles and scales, sometimes even tricking the regulatory rules, or built illegally or informally.
\footnotetext{\hl{add years for each} architectural competitions were open for: the Belgrade marina, multifunctional business center 'Usce", the historical core of Zemun, sport complex Tasmajdan, numerous central squares, pedestrian streets, new office blocks, affordable housing etc.)}
These uncontrolled and impulsive actions and faulty procedures result in urban expansion (overconsumption of agricultural land) and rampant sprawl and the loss of public space. Coupled with unresolved transition of property from the socialist period, Belgrade has been slowing growing into an spatially inefficient city  on the road of suburbanization \cite{Zekovic et al. 2015}.
\\
%population & urban actors
\deleted{population rise diagram}
In 2002, Belgrade covered 3.6 percent of Serbian territory and 17.3 percent of Serbian population is living in its metropolitain area \cite{Cities in Transition 2013}.
Belgrade metropolitan area also accommodates the highest share of highly educated population in Serbia (13.76 percent) \cite{Vukmirovic et al 2013}.
According to the 2011. census, the city has a population of 1,166,763.
The population of the metropolitan area\footnotemark stands at 1,659,440 people and stands for 17 municipalities. 10 of them are classified as "urban" and 7 are "suburban" municipalities, whose centres are smaller towns \hl{(ref "Urban Municipalities")}.
In fact, the  city  is additionally burdened by joining the predominantly rural  settlements  and  conglomerates  of Barajevo, Grocka,  Lazarevac,  Mladenovac,  Sopot,  Surčin  and Obrenovac \cite{Samardzic in Doytchinov 2015}.
Not to mention the devastating legacy of the 1990s that threatened the  urban  culture generally and eroded the civic order and the value systems instated during the communist era.
Notable socio-spatial stratification with the formation of very expensive districts (the historic core and several traditionally wealthy neighborhoods) and very poor ones (near the large industries and in the far-out outskirts) enlarge social divisions and collisions of interests and life styles   \cite{(Hirt 2009)}.
\footnotetext{the administrative area of the City of Belgrade}                 
\\
%urban culture - conflicts & resources
With the at least nominal revival of democratic diversity and rising influx of global trends, Belgrade once again became a place of striking extremes and contrasts. 
Spontaneous revival of civic, cultural, artistic activities in Belgrade happened with support and out of interest from international organizations and initiatives of the same nature. Moreover, the presence of market-oriented value system and capital prompted the lively lifestyle of the Serbian capital to transform into a famous and infamous European destination for mischievous, casual and exciting nightlife \cite{Vanista Lazarevic in Doytchinov 2015}.
However, the rise of the civilian values has been continuously threatened by the non-urban tendencies,  disordered,  chaotic  and violent practices mushrooming around as the result of weak, biased or completely absent institutions and regulatory framework. Current urban identity of Belgrade is the combination of the city's position and politics, urban culture and traditional values full of gaps and deficiencies which will decide its urban future.

%urban development
Tracing  urban  development  of  Belgrade  reveals  challenges  and  traumas both from the long history of deconstruction and reconstruction with its explicit repercussions in the very recent events and the mentality of its urbanites (urban actors).
Unrealistic perceptions of scarce and unorganized elites that the clash of Eastern and Western life modes and urban forms happening in Belgrade as the biggest city along their border have been hindering regularization, institutionalization and articulation of urban forms and practices \cite{other ref Samardzic in Doytchinov 2015}. Evidently, Belgrade still is an European periphery and has a marginal role within the European urban network \cite{Vujovic and Petrovic 2007}. 
As a matter of fact, \textit{"more  than  a  "global  city",  Belgrade  is  a  rural  or  post-rural  conglomerate characterized by visual, emotional, ideological and material traumas of wars, holocaust, poverty, lack of efficient institutions and rule of law, a micro-culture of  individual  irresponsibility  and  incompetent  development  solutions."} \cite{Samardzic in Doytchinov 2015} 

Accordingly, although the case of Belgrade presented a high degree of urban planning strategies and its practical implementation during the previous socialist regime, urban planning was continuously hindered by political instability, convergent socio-economic forces and inconsistent planning systems during the transitional period of the 1990s and the early 21st century. 
However, Serbia still finds itself in a post-socialist proto-democracy without the developed institutions of a representative democracy, civil society and market economy (Vujosevic et al. 2010): 
\begin{itemize}
\item Urban planning has not been a priority (Sykola, 1999) and planning documentation has already been turned into symbolic documents \cite{Nedovic-Budic 2001}
\item Urban transformations mainly concerned land use and property ownership changes overwhelmed with powerful economic actors who take advantage of the undefined environment in order to protect and promote their activity and extend their property ownership 
\item The topology of powerless urban actors (ordinary citizens and the civic sector) who have almost no prospects for meaningful social participation and who are not defending their rights therein \cite{Vujovic et al. 2007} 
\item Fragmented spatial development dominated by informality and confused eclecticism that shows the characteristics of urban design bricolage rather than the purposeful stratification of socialist and post-socialist layers upon the urban fabric \cite{(Hirt 2008)}. 
\end{itemize}
These circumstances imply that decision making in urban terms is performed through negotiations between investors and local governments, where local authorities and civic sector, even though they possess legal empowerment, lack adequate and operational instruments for exerting their power and acting as equals in the negotiation process (Bajec 2009). In addition, public interest in local authority services is a result of the direct influence of political programs of those who are involved in local authorities and are active protagonists at global and national political scene at the same time (Djokic et al. 2007). The pervasiveness of such uncontrolled and even illegal development leads to the deconstruction of urbanity (Vujović et al. 2007).
In such environment, even though the western planning paradigm involves corrective factors for urban failures inherited from the free market (Nedović-Budić 2001), the path dependency tradition of urban development in Belgrade traces a different urban planning framework that was not considered sufficient and effective for managing local urban issues. Complex institutional legacies influence the behaviour of all urban actors, prevent the development of flexible social patterns and networks and fall short of providing overall legitimacy for the constellation of different interests in the post-socialist context of Belgrade (Petrovic 2009). Furthermore, very few theoretical or general methodological research studies bothered to examine alternative modes for urban development in transition, apart from replications of the approaches taken by neo-liberal or institutional economies (Tsenkova 2007).
   
\textbf{Alberto Moravia: Belgrade is a rare city at the confluence of two big rivers which also represent a synthesis of several world metropolis}

\subsection{Savamala}

Savamala is a neighbourhood in Belgrade situated on the southern bank of the Sava River in the old part of Belgrade (Figure 2). The neighbourhood of Savamala is rather a place on the mental map of Belgrade and important landmarks of the city, than an official administrative unit (Figure 1).  Its name means “Sava neighbourhood”, and it is derived from the Turkish word for neighbourhood “mahala”, combined with the name of the river whose bank it is situated on. The first official mentioning of Savamala was around 200 years ago after the resolution of city authorities to spread the urban structures to the river in order to set forward its urban development. During all these years Savamala has been a venue with a plausible collision (traditional/modern; past/present) rich in tradition, history and heritage. Its physical layout can be described as: (1) an appealing location in almost a geometrical centre of the physical layout of the present city of Belgrade, (2) an attractive but deteriorating neighbourhood with irrevocable potential for renovations and refurbishments, (3) an area within the walking distance from the city centre but still aloof from its ever-growing hustle and bustle. In a nutshell, this neighbourhood is a scaled example of pre-socialist material legacy, socialist cultural and societal matric, a transitional reality and a condensed case of its multi-faceted circumstances of post-socialist urban development (Table 4):
•	Pre-socialist past marks its presence in Savamala through architectural and cultural heritage  (Figure 3).
•	Cultural and behavioural patterns from Yugoslavian socialist regime . 
•	Post socialist backtracking 
•	Transition prospects : From recently established economic constellation, Savamala has a potential to become an attractive urban area for investments.

Savamala is a typical East-European neighbourhood caught in post-socialist processes of economic and political change in Balkan transitional countries. In these circumstances, such cityscapes cannot resist copying urban models from the West, but meet extraordinary difficulties in doing so, because these cities lack the institutional and cultural infrastructure essential for the functional unity present in western cities (Petrovic 2009). Serbian society as a whole experience the period of radical shifts from: totalitarian to democratic political system, planned to market-based economy, public to private property ownership, supply to demand driven economy, industrial to service based society, and isolated to integrated position in the world economy (Petrovic 2009). Therefore, Savamala, with its even more intensive top-down and bottom-up pressures, is a representative testing environment. 

Savamala neighbourhood is situated in the central urban zone of the city of Belgrade, on the southern bank of the Sava River. It is a unique area in Serbia with such plausible collision between traditional and modern and past and present, rich in tradition, history and heritage. But World wars, authoritarian rule and the current economic crisis have left their marks. Savamala is now a traffic bottleneck with intense pollution and urban noise. For decades its existing spatial conflicts and socially disadvantaged population have been neglected by both the authorities and professionals (“Urban Incubator Belgrade” 2013). Before the spin-off of cultural organizations, activities, and conversions of old neglected houses to trendy cafés and restaurants, in the neighbourhood, Savamala had a reputation as a home to outcasts, prostitution and criminality.
Even though Savamala was known as a Christian neighbourhood in Ottoman Belgrade, its rich cultural and architectural heritage dates back to the period of Serbian monarchy from the 19th and early 20th century. Furthermore, during the communist rule Savamala was disregarded as the legacy of the capitalist era and was turned into a transit roadway surrounded by corresponding building stock (warehouses and manufactures). After the major political shift in the year 2000, the attractive location of this neighbourhood put it at risk to become a polygon of exercise of interest for corrupted public authorities and powerful private developers working together under the hood of urban development and economic prosperity. Despite the ownership change, Savamala was saved for a while from this newest development trend, mostly because of its long-term decay that made it a complicated case for the limited investments with short-term turnovers that were dominating in Serbia. However, it has recently changed as powerful international investors found a counterpart in Serbian authorities on various levels to jointly use their economic and political dominance for gaining control over a highly profitable waterfront area of the capital city (Zeković et al., 2016; Cvetinovic et al., 2016b).

Savamala is a typical East-European neighbourhood caught in post-socialist processes of economic and political change in Balkan transitional countries. In these circumstances, such cityscapes cannot resist copying urban models from the West, but meet extraordinary difficulties in doing so (Petrovic 2009). Serbian society as a whole experience the period of radical shifts from: totalitarian to democratic political system, planned to market-based economy, public to private property ownership, supply to demand driven economy, industrial to service based society, and isolated to integrated position in the world economy (Petrovic 2009). Therefore, Savamala, with its even more intensive top-down and bottom-up pressures, is a representative testing environment. 

hl{Blagojevic 2009}
Hadzi NIkola Zivkovic desing the CUstom House (Djumurkana 1835)
Janke’s planning was also part of the outer city development, such as that of the suburb Savamala (Abadzijska carsija), 

Ottoman period
	\hl{Roter Blagojevic in Doytchinov 2015}
Serbian settled down after the 1804 uprising near the church on Kosancicev venac and established trade and commerce area on Sava slopes
A new mercantile district develops on the Sava riverbank, around the port. That is the only connection of Belgrade with the city of Zemun and the European neighbors. The suburb Savamala develops fast, rising above the marshy terrain called the Venice Pond.
Ananstas Jovanovic view on Savamala
The first straight streets with new  houses  are  Savamalska  and  Abažijska,  form  the  new  city’s  commercial center,  where  all  the  Serbian  merchants  and  craftsmen  are  supposed  to  be settled, after leaving the Old Town’s Main Street. These streets are representing Princ Miloš’s urban visions and they show for the first time the emergence of
long, straight wide traffic corridors in Belgrade.

\subsection{Conclusion - Urbanity}
\hl{(Grozdanic 2008)}
aspire to affirm memory and to establish continuity in the process 
of    urban  (town)  planning. 

\section{Stimulants and deterrents of urban decision-making tradition in Savamala and Belgrade}

The purpose of the case study used in this research is to build a social phenomenon to represent a modern urban environment in the economy of a transitional country and to accurately describe and illustrate it in order to establish a context for a new methodological approach (Yin, 2009). 
In post-socialist cities, urban planning should link the top-down changes (linked to national and global level) to the bottom-up changes in the urban systems of the city, by emphasizing diversity and reciprocity in the nature of the on-going transformations: economic transformations (transformation of production and consumption in relation to space, income polarization and poverty), political transformations (urban governance, participation and decentralization), spatial transformations (demographic trend and distribution of functions) and social transformations (social inclusion, social activism and informality)

While some trends and directions within these transformations are clear and defined, uncertainty dominates decision-making and implementation in the turbulent environment of post-socialist cities (Nedović-Budić, 2001). The internal environment is in a state of flux, with the rapid adjustment of the physical, economic, social, and political structures of the city itself (Sykola, 1999). This captures the pace of change and the multi-layered nature of transformation, with the focus on the process of change in the city’s economy, society, system of governance and the spaces of production and consumption.

\subsection{Top-down management of urban issues}
\hl{Stupar 2004}
Belgrade levels of networking:
    EUropean - danubian - integration in  Regional  spatial  plan  of  Belgrade  administrative  area  (2004)
    macro-regional
    metropolitan
    
\hl{Vujosevic and Nedovic Budic 2006}
The  Act  on  Self-governance (2002)
The  Planning  and  Construction  Act (2003) - radical in its return to private ownership  of  denationalized  urban  land -  many of the premises of the 2003 planning law are built on unrealistic expectations and inaccurate assessments of the local circumstances and context (Kresic 2004). 
hierarchical system of plans
Regional  spatial  plans/regionalni  prostorni  planovi  are  being prepared for the  autonomous  provinces  —  territorial  entities  at  the  NUTS2  and 
NUTS3  levels,  as  well  as  for  Belgrade’s  administrative  area. Municipal  spatial plans are at the NUTS 4 level. 
The  Self-governance  Act  establishes  the  role  of  the  Chief  Architect  and he  Communal  Manager
planning activities are embedded in a particular societal context or territorially based system of socio-economic relations: the political system and practices,  the bureaucracy  and governance, and the economic system (in this case, the market).

\hl{Grozdanic}
The  General  Plan  of  Belgrade  2021,  adopted  by  Belgrade  City 
Assembly in 2003. It was drawn up under the leadership of Professor Vladimir Macura PhD, and Miodrag Ferenčak MSc - planning as a process;  introduction  of  a  market  economy  and  the  process  of  privatization
area is the same as the previous plan (1985), 2.5 times more inhabitants
The General Plan of Belgrade 2021, is the strategic base of its spatial 
development.
the approach is desirable, this Plan doesn’t consist of all elements which are, according to the Law, necessary for direct implementation, such as regulation of streets,  parcelization,  systems  of    infrastructure  etc.  By  fragment  implementation  which doesn’t  provide public interests, instead of regulated lines,  we usually have on the city map,
many unconnected points which lead us to deregulation and haos. 

\hl{Stojkov and Dobricic 2012 02}
Spatial planning system reflects political system (political culture and entity) (Arbter 2001)
Spatial and Development Plan (of the Republic) - binding for spatial and sectorial strategies
Vertical coordination – ministries, Governement, prime minister, Parliament (spatial plans)

 	\hl{waves of planning 2006}
after 2000
despite this pronounced draw on contemporary international experiences, the new Planning and Construction Law adopted in 2003, used the 1931 law as its main source

\hl{adjustment of planning practice nedovic budic 2001}
current planning and urban development issues in Eastern and Central Europe:
    legitimacy of planning
    "borrowing" methods
    inertia
    site-specific urban project,  prim. aut. -> no wider space-time concerns
    need for better integration of physical planning, economic factors and market mechanisms
    not adequate planning documentation
    ineffective administration
    need for more public participation
    inadequate attention to environmental problems

    2000s: In most of the local planning activities, private interest and investment rationale prevail. The political corruption is at the core of the suspected causes of the unjustified decisions.
        The 2003 Law on Planning and
        Construction deal with illegal construction; modelled after Fench planning law (engineering flavour and EU common terminology) - This Law incorporated three previous laws—Law on Planning and Arrangement of Space (Zakon o planiranju i uredjenju prostora), Law on Construction (Zakon o izgradnji) and Law on Construction Land (Zakon o gradjevinskom zemljistu).  recession to the functionalism and physicalism and avoidance of more
        complex issues of social, economic, political and ethical implications of planning interventions.
        The Strategy for Development of Serbia- never formally adopted
        Spatial plan of the Republic 2010
        City of Belgrade Development strategy - raise the City of Belgrade to the level of large European cities
        private land ownership
        key authority:municipality
        actors: public, land developers, investors, engineers

\hl{adjustment of planning practice nedovic budic 2001}
    2003:
        balanced, coordinated development, harmonization with EU norms
        Spatial plans: Strategy for spatial development of Serbia, special uses, regional, municiipal
        urban plans: general, regulation (general and detailed)
        transparency: as in 1990s
        implementation: urban conditions, building permits; directorate for urban development (direkcija za izgradnju gradova), chief architect
        sustainability. explication of sustainable development premises
        land ownership:
            land: public (national, local), private
            buildings: public (national, local), private        
              
\hl{Vujosevic and Maricic 2012}
institutional and organisational arrangements in planning:
    old institutional zombies (from socialims)
    new management agencies and similar organizations (incapable)
planning model (mixture of different models):
    planning as crisis management
    planning as a mechanism for support of uncontrolled privatisation and wild marketisation (domination of big projects instead of strategic programs)
    planning with an institutional framework accommodated to the needed social, economic, cultural and environmental transformation of the society
        
Serbian state (1804-1914)
In 1896 the Belgrade Building Law is adopted (with amendments in 1898 and 1901) and from 1897 on, the Building Code for the Town of Belgrade 69  regulates all the issues related to the construction in the separate parts of the town. 

\hl{ref Peric 2016}:
contemporary period (until present)
re-decantrilazion of political and administrative power
neo-liberal paradigm without institutional capacity to support it
strong relations between politicians and private investors (built on the 90s) - domestic tycoons
Privatization Law 2004
Law on Conversion 2009
Government elected in 2012 (SNS) stoped the practice of the collaboration with tycoons and turned to foreign investors
mega-projects, stopped partly by the economic crisis

2003 master plan Belgrade - proposed urban changes \hl{ref(Stupar 2004 isocarp)}:
urban changes:
privatization  of  housing  and  business  spaces, 
adequatecommercialization  of  urban  historical core,
completion  of  residential  areas,
decreasing  ofresidential/commercial  sub-urbanization  and  socio-spatial  segregation. 
  
planners lacking the knowledge from humanities, relying only on technical and engineering skills, lost monopolistic position, became passive observers
From the end of the Second World War up to the 1960s, urban planning in the former Yugoslavia almost exclusively reflected the top-down tradition of the communist institutional and ideological framework (Dawson, 1987; Papić, 1988). Surprisingly, in the 1960s the political and administrative system was decentralized politically and liberalized economically, and for a time Yugoslavia was known for having one of the most decentralized decision-making systems, which applied equally to social, economic, environmental, and spatial (urban) planning and policy (Simmie, 1989; Miodrag Vujošević, 2003). This “bottom up” participatory approach based on the “cross-acceptance” principle was introduced, at least nominally, more than a decade before it was contemplated or practiced in certain developed Western countries (Cullingworth, 1997). By the end of the 1980s, both the Yugoslav urban system and planning practice had become dysfunctional despite their innovative features, because of the hypertrophied and bureaucratized social and political system (Očić, 1998).
In the light of these circumstances, and in order to make the planning and policy process more effective and efficient, Serbia is searching for a new planning and policy model that not only meet the general principles of a pro-active, transparent, adaptive, inclusive, evaluation-and-feedback-based planning procedure, but also one that can be supportive and compatible with the development of its civil society, based on the authority of law and traditional social rules and in harmony with the rules of the European Union.
Bearing this in mind, the case of Serbia is a good illustration of how a planning system can adapt to changing political and socio-economic circumstances, because the extreme variations in the planning practice and the response to the societal circumstances since the late 1980s offer rich opportunities to observe the relationships between planning and its broader social and spatial context (Thomas, 1998).

Having the process of urban development broken down into three strategic approaches for addressing urban conflicts, we can state that urban planning procedures aspire to generate an action plan for development that achieves common viewpoints, goals and priorities within an ordinary city, as well as a set of strategies optimised over time within the institutional mechanisms for their implementation, monitoring and evaluation (Fisher, 2001).
Therefore, urban planning theory has always complied with the prevailing theoretical framework of social studies (Portugali 2011). On the contrary, in order for planning activities to be effective, they should be embedded in a particular social context, and they react to the shifts in socio-economic and political settings. Having the operational framework of urban planning defined as such, it becomes conspicuous how its practice has always complied with an overall planning paradigm, being simultaneously intrinsically connected to the property market and tending to maintain current social order (Taylor 2006). Scientific research in the field operated within this framework and during the same period comprised, first, a normative planning model based on a top-down decision making process, and then, collaborative and communicative planning when the diversity of values, meanings, and interests have emerged more vigorously so that the role of the urban planner changed from being a technical expert to a mere facilitator (Taylor 2006).
Accordingly, general urban planning in former Yugoslavia incorporated the process of paradigm change in Kuhn’s sense of meaning and set a specific pace of progress dependent on the current state of socio-economic and political affairs at national and city level (Bajic-Brkovic 2002). The discrepancy between theory and practice initiated the abandon of the previous planning model with one fixed future vision of urban environment in the 1980s, but real dissolution of the planning paradigm started in the 1990s due to the disintegration of Yugoslavia’s socialist system and the destabilization of the institutions which brought about the lack of legitimacy in urban planning in post-socialist cities of the newly established state (Vujosevic 2010).
The major characteristics of a post-socialist chaotic pattern of urban development are: a multitude of actors, various economic, social and political interests, social aspects and fragmented spatial development. In other words, what proceeded after the end of the socialist era is a neoliberal model of urban planning with the supremacy of market-oriented solutions for urban problems (Sager 2011). Thus, the crucial failures of post-socialist urban planning have come about through the lack of consensus on priority goals, action-oriented programs of implementation and coordination of different levels, sectors and areas. In practice these conditions ended by having the strategic plan as an advisory long-term urban vision, but leaving the real actions and decision making to political and market forces. Thenceforth, urban development of post-socialist cities most often has exceeded and diluted the common strategic framework defined from top-down: to establish clear links between the process of strategy development, its institutional framework, the hierarchical structure of long-term and short-term objectives of all actors involved, and the real-time changes happening simultaneously in an urban environment.

\hl{Vujosevic 2012}
centralized planning system, re-centralized in 90s, like the country
suffer from democratic deficit syndrome and inefficiency
lack of political will (no societal consensus) for introducing strategic development-oriented planning - hyperproduction of documents without concrete, operational, and sustainable action plans
chaotic decision making: combination of crisis-management, planning-supporting-privatization-marketization, project-led, poor coordination of governance
no critical overview of contextual and methodological shifts in planning, orientation toward technical issues
preaching sustainability from international organizations and questionable experts - marketing issue
important  is to integrate traditional spatial, urban and environmental planning with strategic planning policy
planning culture in Serbia: manipulation, clientelism and paternalism - strategy of persuasion (Sager 1994)
existing system nurture multiple institutional zombies from previous times and most institutions act like management agencies rather than taking strategic approaches
EU documents do not fully correspond to the situation in Serbia and should be properly translated
chaotic situation, unreflecting the impact of contextual factors, and importance of stakeholder collaboration and strategic governance
"There has been a lack of planners and other experts experienced and knowledgeable in practical planning under the new circumstances of political pluralism and radically
changed structure of stakeholders and concomitant institutional arrangements."
" Planning practices suffer from a generally inadequate information and research support,"

\hl{Vujosevic 2015 Regionalizam u Srbiji 2}
	    >2010
regioni - nisu stvarne plansko-razvojne jedinice 

\hl{Vujovic and Petrovic 2007}
2002 - Law on Local Governance
City Mayor, City Manager, City Architect
the distribution of power in the City (Belgrade) between the center and the districts
2003 - Law on Planning and Construction
slow socio-economic transformation and low rate of foreign investments

Belgrade:
the Master Plan of Belgrade 2021  made  by  the  Urban  Planning  Institute  of  Belgrade  in  2003 \hl{ref „Official Gazette of the City of Belgrade“ no. 27/03.} 
the  Regional  Plan  for  the  Administrative  Territory  of  the  City of Belgrade adopted in 2004 (urbel link) - the protection, spatial planning and development of the metropolitan region
The City of Belgrade Development Strategy (March 2011)
\hl{Vukmirovic in Doytchinov et al 2015}
4 actualisations  since  2003  reflect  the  steps  of  maintenance  of  the  urban plans  under  the  Law  on  Planning  and  Construction  and  the  changing  needs of the Republic of Serbia and the city of Belgrade (Master Plan of Belgrade 2021).
The City of Belgrade Development Strategy defines the basic recommendations for strengthening Belgrade’s identity as a criterion for the economic development and especially for the tourism, as well as the affiliation of the citizens to the territory they live on. 
competitiveness by  the activation of dormant resources as well 
as the mobilization of the local capital, inactive so far.

\hl{Stupar 2004}
Master plan of Belgrade (2003):
    reinvent and renew some forgotten and neglected  segments  (ex-industrial  areas,  traffic  nodes,  riverfronts,  suburban  and  rural neighborhoods)
    global  social-cultural-economical integration, inter-urban cooperation and creation of recognizable Belgrade image
    urban infrastructure
    long-term integration of natural recourses and built environment,
    renewable energy
urban changes:
    privatization  of  housing  and  business  spaces,  adequatecommercialization  of  urban  historical core,
    completion  of  residential  areas,
    decreasing  ofresidential/commercial  sub-urbanization  and  socio-spatial  segregation. 
2 main phases:
    until 2006
    2006-2021
activities of the city government and its institutions (2000-2004):
    -  reconstruction and transformation of built environment; 
    -  introduction  of  new,  technologically  advanced  and  efficient  modes  of  urban
    government/management; 
    -  new ideas. 
project initiators:
    numerous  architectural  competitions  dealing  with interesting  zones  of  Belgrade  area (the Belgrade marina, multifunctional business center 'Usce"(New Belgrade), the historical core of Zemun, spot complex Tasmajdan, numerous central squares, pedestrian streets, new office blocks, affordable housing etc.)
    the  city  government,  in  cooperation  with  Agency  for Building  Land  and  Construction,
    professional  associations,
    foreign  embassies
    international  and  national  companies,

\hl{Maksic 2012}
Ministry of urbanism:
    spatial and urban planning
    housing and building
    infrastructure
    inspection and supervision (associated with each ministry)
In this respect, the Ministry of Civil Engeneering, Transport and Infrastructure is the key public actor at national level in the domain which (1) conduct administration tasks, (2) govern strategic construction, site-development and infrastructure equipment works, (3) carry out survey jobs, and (4) perform inspection and supervision actions in the field (RealCorp 2015 paper)
Agency for Spatial Planning
    spatial plan - republic and regional
    technical assistance for plan preparations
    spatial planning training
Commisions for expert control of plans (?)
No regional level planning institutions (everything associated with the agency)
local-municial/city - > government/service for local plans
Private sector: EU: policy formulations (conferences, workshops, forums)
Serbia: involved individually in building process (investor urbanism: building codes and regulations defined by investor interests)

\hl{(Hirt 2009)}
refugee problem
In 2002, Serbia adopted a National Strategy for Resolving the Problems of Refugees and Internally Displaced People. 
resolving the status of urban land
\hl{(Hirt 2009)}
the Master Plan (City of Belgrade, 2003), the Regional Physical Plan (City of Belgrade, 2004), and the Development Strategy (City of Belgrade, 2008). These
documents set a number of goals and strategies around the themes of improving environmental sustainability, economic competitiveness, social cohesion and territorial polycentrism, and strengthening cultural identity.

\subsection{Legitimacy of Interests in a Post-socialist City}

Tactical Urban Transformations 

\hl{Lazarevic Bajec 2009}
in the transition process, powerful economic actors take the advantage of the non-defined environment in order to protect and promote their activity and extend their property ownership.

\hl{Samardzic in Doytchinov 2015}
The underdevelopment of the democratic institutions and the absence of rule of  law  enabled  the  oligarchy  in  the  modern  history  of  Belgrade  to  involve occasionally into war adventures and social engineering. The evolutional “sin” of Belgrade’s citizenry thereby was the complicity with the ruling elite, which attempted to enforce “national” territorial policies, while promoting the social egalitarianism.

As we have mentioned urban development relies on much more than strategic urban planning, in spite of the propensity of the scientific community to control and govern it to the greatest extent possible. Every urban issue relies directly on the economy and the mode of production and consumption in modern global cities. Namely, the capitalist economy needs urbanization to absorb the surplus products, so that the deregulation of land use and property markets is the precondition for capitalist accumulation and thereafter proceeding to economic growth (Harvey 2012). Following Harvey’s line of thought, the power extracted from the exclusive control over property or land is the source of capital produced by its locational, infrastructural, social or cultural capacity. In other words, the contextual resources of an urban environment in a developing country make it appealing for incoherent distribution of resources and responsibilities (Bolay et al. 2005). Furthermore, within a range of different urban actors, influential economic and political actors tend to abuse their powers and appropriate urban space, when the regulatory framework is blurred and biased as it is in post-socialist cities (Djokic et al. 2007). 
Therefore, urban governance in post-socialist cities is more reactive to the interests of capital investments, as well as being tolerant of illegal practices than it is strategically proactive, which leads to organic rather than comprehensive entrepreneurial city development (Petrovic 2009), a laisser-faire economy and a global consumer culture which dissolves the democratic capacity of countries in transition (Ellin 1999). The main characteristics of urban transformations in post-socialist cities are marked by:
•	investor urbanism stemming from loose regulatory framework and vulgar economy patterns (Vujosevic 2010)
•	pluralist political life and political voluntarism which dominates the implementation of laws (Djokic et al. 2007)
•	citizen resignation and political passivity holding back the expansion of participation (Vujovic et al. 2007).
These circumstances lead to the multitude of interests, initiatives and small-scale projects with no effective and binding policies and institutionalized regulatory means for synchronization and coordination among them. Moreover, within such blurred framework urban actors with no political or economic power become marginalized and deprived of their rights to be actively involved in designing their urban environment (Bolay et al. 2005).

\hl{ETHZ 2012 Belgrade Formal Informal}
negotiations with the state (informal campaign of 90s become secularized), pressure and tactics = pressure, money and connections
after 2000 legal form of informality

\textbf{BELGRADE WATERFRONT PROJECT}
covers two  spatial,  geomorphological  and  administrative units divided by the River
the  international  competition in the 1980s for the centre of New Belgrade, while in the 1990s  it  was  performed  in  the  competition  for  the 
Sava Amphitheatre and the project Europolis 
	\hl{Vukmirovic in Doytchinov et al 2015}
SPIN
the  area  is  listed  as  a development  area  in  all  three  strategic-planning documents mentioned above: as Sava Amphitheatre and part of New Belgrade’s centre in the Spatial Plan for Belgrade 2021 and the development of the Sava waterfront in the Belgrade City Development Strategy. 
Based  on  these  documents  the  Urban  Institute  of Belgrade  started  the  preparation  of  an  urban  study that was supposed to precede the announcement of an international competition for the Sava Amphitheatre. 
By  noting  that  this  project couldn’t be financed only by public funds and loans
a  Belgrade  Waterfront Master Plan was presented in Dubai in March 2014 
by  Mohamed  Alabbar,  the  director  of  the  newly established  company  and  potential  investor  “Eagle Hills”.
citisizm: investor urbanism and adopting legislation
the project seriously is endangering the public interest. 
landscape strategy is missing, administrative positive, public interest negative (as Beko factory)
	\hl{Vanista Lazarevic in Doytchinov 2015}
situated on Sava amphitheatre
90ha of land
an  investment  fund  from  Dubai  the  flagship mega project
About 1.5 million m² are to be built up within the framework of the project.
a loan from the Kuwait Fund amounting to € 25 million for the relocation of the railway station
The first phase consists of building a huge shopping mall and a tower. Mohamed Alabbar,  the  investor  famous  for  the  skyscraper Burj Khalifa in Dubai, has estimated the investment volume on € 2.5 to 2.8 billion. 
the project proposed by foreign architects
The purpose  of  the  development  project  is  to  create  a multi-functional  complex  with  luxury  hotels,  trade and  business  premises  and  exclusive  apartments overlooking  the  Sava  River. 
	\hl{Maruna 2015, Peric 2016}:
usurpation of both the formal planning procedure and the professional expertise in the creation of planning solutions
brownfield regeneration project
regional and national importance (historical site vicinity and the confluence)
in reality: tight cooperation between the city and the national government - the prime minister and the city mayor are form the same political party (decisions made on national level imposed on the local/city level)

\textbf{The Beko Master Plan}
PROPAGANDA
	\hl{Vukmirovic in Doytchinov et al 2015}
"a multifunctional complex replacing the former “Beko” textile factory in the immediate vicinity of the Belgrade Fortress. It is located about 500 meters away from the city centre and on top of a cultural axis that connects some of Belgrade’s most important destinations. Designed by Zaha Hadid Architects.
property sold to Lambda development (Greece)
CEP working on the plan requested by the Department of urban development of the Municipality (2008)
public informed in 2009
he draft plan was put on a public display in the summer 2011 and was 
adopted in March 2012. The former master plan for the Beko area from 1969, 
which defined a recreational area on the disputed site, ceased to be valid.
Display on Belgrade Design Week 2012: covering of 94,000 square meters and includes edge residential spaces, galleries, offices, a five-star hotel, 
a  state-of-art  congress  centre,  retail  spaces  and  a department store."
In Master plan 2021, existing detailed plan for the area marked as not mandatory and in BGD Development Strategy the area is marked as brownfield.
Conclusion from public and experts: " The general conclusion of this event was that the problems were not in contempt of procedures and norms, but in 
their deregulation and relativism, as well as the lack of transparency behind the certain decisions."
landscape strategy is missing

\textbf{The City on Water - Belgrade Port}
		\hl{Vukmirovic in Doytchinov et al 2015}
architects Daniel Libeskind and Jan Gehl
The Spatial Plan has provided a public professional survey for the location carried out by the Urban Institute of Belgrade in the form of an invited competition for five public professional institutions:  the  Faculty  of  Architecture  of  the  University  of  Belgrade,  the Association  of  Urban  Planners  of  Belgrade,  the  Architectural  Association  of Belgrade,  the  Academy  of  Architecture  and  the  Architect’s  Club.  The  teams 
delivered in December 2006 five visions for the future transformation of the 
470 hectares large territory including the area of the Port of Belgrade
The  Belgrade  Port  Company,  which  since  has  been  privatized,  started  in 2008 own activities for finding the best solution for the area.
The  vision  for  the  Danube  waterfront for upgrading the quality of the public spaces in Belgrade of Gehl Architects was presented in the Belgrade Chamber of Commerce in February 2009, while Daniel Libeskind held a lecture titled The Vision of Belgrade Waterfront 58  at the University of Belgrade in March 2009.
A final presentation of the project was arranged by the Belgrade Land Development Public Agency at the Real Estate and Investment Fair in Cannes in 2009.
The  plan  envisages  the  construction of  residential  and  commercial  buildings,  objects for cultural facilities, a congress centre, a school, a 
nursery and a hotel. The main landmark of the area would be a 250 meters high skyscraper, symbolising the meeting of the two rivers. 
The  City  Parliament  adopted  a  Detailed  Regulation Plan  for  an  area  of  470  hectares  in  Ada  Huja  in accordance with the Spatial Program for the territory in  December  2012.
despite the  high  quality  approach  which  was  in  line  with  the contemporary town planning principle, the main goal of the owner and the project was to  offer big investment possibilities to everyone interested in investing in the location
the controversial privatization of the Port of Belgrade the focus of public attention
an unresolved situation with the ownership of land and the significantly reduced interests of the new government elected in May 2014 currently put aside the project

\textbf{Beton Hala Waterfront}
PORPAGANDA
			\hl{Vukmirovic in Doytchinov et al 2015}
the  International  architectural  competition  to  design 
the Beton Hall Waterfront Centre in Belgrade could be found in the Belgrade 
Spatial Plan for 2021 and the City of Belgrade Development Strategy in the 
domain of Belgrade’s brownfields, the revitalisation of Kosancicev Venac and 
the rehabilitation of Savamala  (January - June 2011)
 two equal first prizes, two equal third prizes and one honourable mention.
The project  Cloud  designed  by  Sou  Fujimoto  Architects (innovative)
The work on the detailed plan of Beton Hall and the connection of the riverbank 
with  Kalemengdan  started  in  2012  in  accordance  with  the  competition 
proposal.
landscape strategy is missing, symbolism wiht the iconic shape is strong

\subsection{Network of civic engagement}

Following Arnstein’s ladder of participation idea, each society is left to mix and match the participatory processes that meet its needs and influence power relations (Fisher 2001). Accordingly, participatory planning aims at achieving certain end-results, contributing to the efficiency of society as a whole in a process for accumulating social capital, and creating institutionalizing networks of civic engagement (Putnam et al. 1993). It means that every society may be able to produce its own space with a strong impact of its ideology and cultural spheres, and thereby controls its urban development (Lefebvre 1974). The identity of an ordinary city constantly in flux is then defined as the process of self-understanding, self-creation and self-representation of an operating urban environment by its urban actors, all of whom are mobilized to intervene responsibly and who willingly integrate their customs and needs into this process (Bolay 2004).
Participatory planning is a process that is usually designed to address urban conflicts with the aim of resolving or exploiting it successfully (Fisher 2001). Namely, from the introduction of communicative planning approach in 1980s onward, public participation has been introduced as a tool for exploiting democratic capacities of modern society in order to locally mobilise all available human resources to transform a crisis of aggregated urban conflicts into an opportunity for urban development (Healey 1996). Efficient public participation measures, calculates and includes local complexity in tracing urban development according to local, social and up-to-date characteristics (UN Habitat 2009). What is more, successful participation firmly relies on the accessibility, transparency, responsiveness, and accountability of all institutional processes. This citizen power places these individuals or constituted groups into political and economic processes, and it deliberately includes them in decision making for the future of their society.
However, practical application has shown that public participation in Arnstein’s terms lacks popular sovereignty in order to place all urban actors and stakeholders equally within the decision making process, and has been particularly aggravated by thriving neo-liberal market policies (Mouffe 2002). The influence of this trotting up and down the ladder of participation is especially accentuated in ex-authoritarian states. In this sense, the trends of commercialization and free market policies in transitional countries led to the decline of public realm, the deconstruction of urbanity and the abuse of public space (Hirt 2008), which urge a different approach to bring forth an adequate societal realm, induce livelihood and mould the positive urban future.
The alternative vision was recently set in practice with the paradigm of tactical urbanism whose main goal is to set forth economic, political, cultural and spatial transformation in global cities by instigating creative interventions that guide their change, giving them unique identities (Lydon 2012). The conceptual core of such an approach circumvents involvement of the least powerful urban actors in decision making, encourages them to creatively trace their cultural identity through adequate professional supervision and bring positive changes, develop social capital and organisational capacity that involves shaping a physical and a social component of cities (Bolay 1996). In regard to participatory urban design, it therefore becomes important to have a critical society; a populace which is being trained to know, show and actively express their needs and directly apply them in urbanism (Ostrom 1995). Moreover, their needs must also be modified to what urbanism can actually offer; they need to act or interact with the world around them, which is in flux (Harvey 2003).
This way of bottom-up, step-by-step urban development for surpassing current profit orientated neo-liberal trends and benefitting from spatio-social contradictions could be especially effective in cities that are going through traumatic post-socialist transition. Bearing in mind that such activities have already been operating in practice in Savamala neighbourhood in Belgrade (“Urban Incubator Belgrade” 2013), it is essential to incorporate its transformative capacity of building local knowledge on current urban conflicts and contextual resources, congregating ideas and setting a comprehensive overlay of urban scenarios for interventions within an exhaustive model of urban development.

Moreover, the way cities function shapes the expectations and actions of all the urban actors involved, who also influence the constitution of the city itself.

The identity of a city in flux is defined as the process of self-understanding, self-creation and self-representation of an operating urban environment by its urban actors, all of whom are mobilized to intervene responsibly and who willingly integrate their customs and needs into this process (Bolay Jean-Claude 2004). On the other hand, these individuals or constituted groups are the actual “makers of the city”. They determine space as a social product of their values, the logic that pilots them, the relationships and representations that influence them and the aspirations that motivate them (Lefebvre, 1974).
Public participation in terms of the bottom-up, action-orientated and socially inclusive engagement of all individuals or constituted groups measures, calculates and includes local complexity in tracing urban development according to local, social and up-to-date characteristics and in marking potential clashing spatio-social points (UN Habitat, 2009). What is more, successful participation firmly relies on the accessibility, transparency, responsiveness, and accountability of all institutional processes. This citizen power places these individuals or constituted groups into political and economic processes, and it deliberately includes them in the future of their society (Arnstein 1969; Fisher 2001).
Following Arnstein’s ladder of participation idea, which denotes the feasibility of subtle nuances of participation, information exchange and decision-making distribution between different social actors, each society is left to mix and match the participatory processes that meet its needs and influence power relations (Fisher 2001; Arnstein 1969). The influence of this trotting up and down the ladder of participation is especially accentuated in ex-authoritarian states. Accordingly, participatory planning aims at achieving certain end-results, contributing to the efficiency of society as a whole in a process for accumulating social capital, and creating and institutionalizing networks of civic engagement (Putnam, Leonardi, and Nanetti 1993). It means that every society (with an emphasis on its human capacity) may be able to produce its own space with a strong impact of its ideology and cultural spheres, and thereby controls its urban development (Lefebvre, 1974).
Participatory planning is a process that is usually designed to address urban conflicts with the aim of resolving or exploiting it successfully (Fisher 2001). It is therefore important to have a critical society; the populace must be trained to know, show and actively express their needs to directly apply them in urbanism, and their needs must also be modified to what urbanism can actually offer and they need to act or interact with the world around them, which is in constant change (Harvey, 2003).  They must be made to understand their multidimensional environment and how they live in it; they must believe that they are not “points” on a Euclidian plain, but rather that they are in constant movement relative to everything around them. Such practices emphasize the equality between the role of all urban actors, stakeholders, authorities and professionals in this process.

Participatory Urban Design Operations

\hl{Vanista Lazarevic in Doytchinov 2015}
	The  individual  and cultural  initiatives,  such  as  the  Cultural  Centre Grad, the Mikser Festival, or the Design Incubator Nova  Iskra  were  the  real  forerunners  on  this  new path.  Throughout  their  unconventional  ‘step  by step’  acting  and  the  renunciation  of  any  plans  or public  financial  support,  the  individual  activities succeeded in resisting and overcoming all obstacles. Certain  individuals,  like  the  architects  Nemanja Petrović  and  Nina  Mitranic  from  the  Savski  Venac community,  to  which  a  part  of  Savamala  belongs, helped supporting the process.

This article attempts to analyse the often blurred and twisted structure and puzzling development prospects of neglected urban neighbourhoods in post-socialist cities. In the case of Savamala neighbourhood in Belgrade, the study is grounded upon recent boom of bottom-up spatial interventions and small-scale cultural projects. These activities have gradually grown into a kind of informal platform for active participation and management of urban conflicts. Therefore, they put forth an alternative strategy to overcome the rigid administrative procedure of urban development and to transform the negative side effects of imitating and lagging behind the conventional urbanisation model and unsuitable urban patterns, as well as those of the accelerating globalisation into a development impetus suited to these societies.  

In the meantime, taking advantage of the long gap in development, a number of local and international organizations and cultural entrepreneurs have focused their actions on Savamala. Their initiatives to transform abandoned places and to reactivate them through participatory, cultural, artistic and educational activities were mainly supported by the local municipality Savski Venac and international cultural institutions and programmes. What at first seemed as a sum of ephemeral local activities has become a driving force for a possible urban future of Savamala, at least the future preferred by most of local urban actors, who have taken an active role in it. According to the organizers, it has gradually grown into a testing ground for the implementation of an innovative tool for “action and learning by doing” (Lydon, 2012) in the urban planning of transitional post-socialist cities (Cvetinovic et al., 2013). However, this new picture of trendy and rather safe Savamala renders the same threats of expulsion of local population and hidden gentrification (Krusche and Klaus, 2015). Participatory activities are mainly rounded up in an urban transformation programme named Savamala Civic District, in the Urban Incubator Belgrade project activities and their successors and cultural entrepreneurial collectives.
All these circumstances bring to light that Savamala has kept its shape , but different social conditions have influenced its development. Namely, four crucial political periods have left their mark on Savamala: pre-socialist, socialist, post-socialist and transitional . All of cultural and architectural heritage dates back to the pre-socialist period when Savamala was promoted as major trade and artisanal area and communication hub with bus and train station in its proximity, while noise and pollution have been caused by its role as a passageway for heavy transit introduced during the socialism. Therefore, we could summarize its life-cycle as follows:
•	Pre-socialist period: amorphous urban form of the neighbourhood, recognizable cultural and architectural identity; 
•	Socialist period: disintegration of tradition and heritage, middle-class society and marginalized groups living in the area
•	Post-socialist period: lack of data on social structure, deteriorating industrial area and abandoned buildings, and leasehold of empty plots to private investors without transparent bidding procedures;
•	Transitional period: market led economy, dominance of private ownership, vivid night life, creative cluster and limited citizen participation governed by non-governmental sector, and start of the huge redevelopment project initiated by a foreign  investor.
However, several important characteristics have been continually developed during the different periods such as (Figure 1): (1) restricted and ideologically-framed civil rights, (2) state control over capital areas, resources and infrastructure, (3) top-down approach to spatial and social development and renovation and revitalization, (4) public ownership of land and building stock, (5) hybrid market circumstances, and  (6) societal self-management planning (Vujović and Petrović, 2007; Petrovic, 2009; Vujošević et al., 2010; Simmie, 1989) . These characteristics have made Savamala a scaled example of “pre-socialist material legacy, socialist cultural and societal matric, a transitional reality and a condensed case of multi-faceted circumstances of post-socialist urban development” (Cvetinovic et al., 2016a).

Data are collected from context-based information and knowledge and also traced from relevant influences, interests and interpretations on Savamala.  In this respect, the actors of bottom-up participatory activities in Savamala taken into account are (Figure 2):  (1) Cultural centre “Kulturni Centar Grad” (KC Grad) cultural center, (2) Old depository in Kraljevića Marka Street (MKM), (3) Mikser multidisciplinary platform, (4) Nova Iskra design incubator, (5) Urban Incubator Belgrade project (UIB), (6) Ministry of space collective, (7) Ne da(vi)mo Beograd initiative (NDVBGD), (8) My piece of Savamala – participatory urban design workshop, (9) The game of Savamala - participatory urban planning workshop, (10) Savamala, a place for making participatory project, (11) Streets for cyclists NGO, (12) Common space in Kraljevića Marka 8 street (KM8).

\section{Case study Framework}

\hl{(Grozdanic)}
Researching experiences and messages from the  past, through the various cultural and  temporal layers, we aspire to affirm memory and to establish continuity in the process of    urban  (town)  planning.  Respecting  their  imortance  and  meaning,  it  is  necessary    to  recognize, decompose  and  study the elements of the origin and  evolution of settlements   in order to estimate conditions and needs of today and to plan future development.The basic message  with  reference  to  urban  planning  is  the  necessity    of    complex  and  contextual approach. 

%%%%%%%%%%%%%%%%%%%%%%%%%%%%%%%%%%%%%%%%%%%%%%%%%%

\chapter{ANT Data Analysis}

Bearing in mind that actor-network explanations give real results only in strongly defined situations (Farías and Bender 2011), we apply flattening composition of all heterogeneous human/non-human actors (ANT)  in Savamala, identified from qualitative data collected on 5 different levels (Table 3), to visually describe urban reality of a post-socialist neighbourhood.

We applied ANT for  interpreting the state of the local context. Most prominent characteristic of ANT is flattening the social by symmetrical treatment of all human, social and technical elements (Latour, 2005). Therefore ANT serves for structuring the data on human and non-human agents and urban assemblage networks on the neighbourhood level. Actor’s existence is its status in a connection or connections. According to ANT, actors do not exist if their networks aren’t labelled. In this way they become agents.

\section{A forward-thinking overview of building an urban development model for Savamala}

Following the circumstances found through in-depth case study research design, actor-network map is illustrated in node-link form and the visualization of the empirical and theoretical data on actors and networks is structured in the following way:
1.	All human and non-human actors – case study description: 
In ANT terms, Savamala neighbourhood is represented as a venue (territory/space natural or urban) with material constitutional elements (built environment – urban structures), wherein a variety of urban actors and stakeholders (individuals and groups) – interrelated to these social factors and within a specific regulatory framework (policy agendas and institutional relations) – engage in actions. Since we have argued that the rapid flow of people and information in the modern globalised world has profoundly transformed the perception of space and time, lifestyles and our sense of community and self (Ellin 1999), we must then state that the vital cohesive force of the modern city incorporates also technical solutions (urban infrastructures) and technologies (communication and media).
A multitude of  these human and non-human actors shape top-down, interest-based and bottom-up developmental action and influence multi-layered decision making structure in terms of decisions  for maintenance, transformation and/or change of the system. 
•	Top-down urban planning actors – institutions and policy agendas
Institutional organization of urban regulatory framework in Serbia corresponds to the administrative organization of the Republic . The Ministry of Construction, Urbanism and Infrastructure is the key public actor at national level in the domain which: (1) conducts administration tasks, (2) govern strategic construction, site-development and infrastructure equipment works, (3) carry out survey jobs, and (4) perform inspection and supervision actions in the field (Maksic 2012). Conversely, cities and municipalities have legal means and rights to make their own strategies, plans, and programs, as well as local regulations and rules in terms of urban development . 
National and city authorities, planning bodies and policy agendas are subjected to continuous pressure to solve an old issue of Belgrade’s peak waterfront area. These initiatives date back to 1920s . The exact area of intervention in these planning phases varies from Gazela Bridge to the far end after Dorcol marina, but their common denominator is the relocation of bus and railway station.
•	Interest-based transformations  - influential business stakeholders and corporate bodies
Powerful investors use its economic and political dominance to gain a good bargain for buying highly profitable waterfront area of the Serbia capital and to ensure that its future development serve their needs. The most important ones hitherto active in the extended area of Savamala;
o	Lamda Development investment for Beko factory renovation ;
o	“City on water” project by Luka Belgrade ;
o	Eagle Hills and Belgrade Waterfront Project (BWP) .
•	Bottom-up participatory and urban design activities – Artists and cultural workers, National and international cultural institutions and NGOs (Figure 4)
Having identified the transitional capital of Savamala in the local context, from 2008 onward a number of small-scale public initiatives and creative services have found their place in Savamala (Cvetinovic et al. 2013) . In absence of an overall urban development strategy, independent cultural entrepreneurs , supported by the municipality Savski Venac, have started the transformation of unused warehouses and craft shops into spaces open for public participation and social production  . These associations and private initiatives have finally introduced the opportunity for an alternative strategic path to influence urban development of the neighbourhood (“Mikser Festival” 2012) and made it famous on a global scale as one of creative clusters in European metropolis.
Agency and relationships of the above identified human/non-human actors in Savamala are tracked by their associations within different levels of decision making (top-down urban planning, interest-based transformations and bottom-up participatory and urban design activities) in a visual manner. 
2.	Intermediaries and mediators (Figure 5)
Starting with ANT, its open approach to comprise whatever may be an element of a complex urban system and its loose definition of actor’s relationality, we faced rather straight-forward indicators of actors’ influence on site and have realized that their human/material nature should be acknowledged as it unmistakably designates their roles in processes of urban development. Therefore, we have shrunk this wide conceptual field to the category of actor nature, which tells us if human/non-human element serves as intermediary or mediator. In this respect, we differentiate its human, entity, artefact, and event figuration, and we indicate if it is an individual or group element. So to speak, the nature of an element defines if it bears or changes meaning – in one manifestation they do, in the other not  (Figure 5).
3.	Free associations (Figure 6)
From our qualitative data (expert questionnaire, workshop, interviews and documentation), we have realized that classical urban categories (social, structure and scale) cannot be fully undermined, though they are used not as explanations, but as associations of performativity and enactment (network of influence, social function categories and social artefacts) (Figure 6). Thus pertaining artefacts are converted into actors. In other words, these association-based actors actually operationalize urban concepts and categorize actual forces and actions.
4.	Stabilizing and destabilizing agencies (Figure 7)
While tracing interactions and interconnections among actors collected through participatory action research methods (Table 3), we have revealed that various social manifestations of these actors work (de)stabilizing. We spot the actors according to their social function/action, elaborating that these agencies and bring an additional layer of urban reality explanations (Figure 7). In fact, the differentiation of functional and supportive networks indicate possibility that the actors change their roles by alter their network engagement. The notion of supportive/secondary networks is laid out more as a significant subset of actors already figuring in any of socially functional networks explicating their bipolar character .  
5.	Urban assemblages (Figure 8)
After having illustrated Savamala urban environment through actors, their figuration and agency, we interpret complexity and dynamics of its urban development through node-link reality. Taking into account post-socialist context, significant pressure from private investors and articulation of civic initiatives and participation in Savamala, we have identified the network of translations which refer to different layers of decision making. These translations consider centrality of actors and nature of links among them and represent in this sense an “assemblage” process of agency dissemination. These overarching urban assemblage networks of management, verification, consulting, administration, planning, construction, regulation, control, finances, implementation and social aspect networks (Figure 4) encompass significant number of humans, nonhuman, action, agencies or forces that have a figuration in Savamala, allowing us to outline and trace among them the distribution of any political, economic and cultural repercussions on its urban development (Table 5 and 6). 
At the final stage, we translate what we have perceived through 5-step ANT framework of Savamala urban development analysis data sources onto visual ANT relational map (Figure 9). 

\hl{Lazarevic Bajec 2009}
the social framework in the four key fields: state, governmental
power  -  executive  and  legislative;  civil  society  –  social  power  (physical persons,  individuals,  households,  civil  associations);  corporative  economy –  economic  power,  corporations  and  financial  institutions  (legal  entities); and political community, political power – social movements and political organizations (Friedmann, 1992, 27).

\hl{Vujosevic 2004 Belgrade Metropolitan Area Governance}:
strategic behaviours and political power relations of key groups of actors:
    spatial and urban planners (architects, town planners, engineers, public administrators)
    economic actors (enterprises, services, corporate business, landlords, banks, TNCs)
    political actors (political parties, movements and leaders, lobbysts)
    citizens (status, position, power, age, education, lifestyle)
    NGO - "third sector"
    
\hl{waves of planning 2006}
several factors or determinants of the system evolution, including: internal political process and regime; ongoing international relations; economic forces; level of centralization  of  government;  professional  culture;  and  source  of  educational  expertise.

\hl{Urbani razvoj u Srbiji Ministry of Space 2014}
documents on urban development serve to define public interest in cities.
In  practice nowadays, public interest is defined by the most powerful social class - in the brace between political and economic elites.
urbanization is rather spontaneous, public interest defined ad hoc, while illegal construction and occupation of space is tolerated.
urban decision making favours political interest over civil and public ones, while political interests are often tied with individual interests of economically powerful actors, which led to inadequate construction practices, overload of the infrastructural systems and lowering urban life conditions, worsen by non-transparency of the process and pertaining corruption.
urban actors:
    political structure, authorities, political bodies: defines and protects public interest
    investors: provide capital and pursue their particular interests
    citizens
    professionals. operationalization and balance of interests
practice of identifying public interests with economic ones, in tese circumstances professionals serve only as a technical body/staff to pursue investor's wishes
Institutional framework in Serbia adopted to the european administrative framework (Official gazette RS 09/2014 )
protection of public interest happens through non-institutional, non-governmental organizations. for these activities significant resources are required (time, knowledge, human capital, networking) as well as the notion with whom they confront.
local authorities emphasize that lack of financial means and resources contribute to the poor public participation
for citizens it is important that:
    are informed adequately and on time
    understand the process, actions and consequences
    are involved in all planning phases
Planning process simulation:
    initiated by public institution or private sector (through the enterprise certified for urban planning practice) - draft version of the plan is based on the investor's interests and guidelines
    the initiative aim at local authorities (Secretariat, Institute, Department)
    the commission for plans takes the decision if it is in accordance with the documents of higher authority  and with the urban legislation, the commission checks the feasibility study of the plan
    plan design with the supervision of the commission
    local authorities organize public inspection during 30 days with filing objections process
    objections are discussed in an open-to-public session organized by the commission for plans
    the report over the public inspection and final decision is made in a closed session by the commission for plans
    City Assembly rectifies the plan
    the decision is officially valid when published in the local gazette
Procedures in question:
    when citizens are informed
    how they are informed - the adequacy of the document for a wider public (clear, simple), to which extent the document is available to public
    to which extent
    when, where and how the public sessions are organized
    how the initiatives and objections are addressed and answered
    transparency of all decision making procedures and accessibility of all the following documentation
Council of Europe - recommendations
important is the dialogue between: investor, citizen, local authorities, planning body
    
\section{Urban assemblage map: urban key agents and contextual resources}

The contextual analysis of social circumstances in Savamala has shown that the contextual capital which was identified therein has been gradually attracting a number of small-scale public initiatives and creative services to settle in Savamala (Cvetinovic et al., 2013). The very first bottom-up activity in Savamala was the constitution of MKM cultural space in 2007. However, the intensive aggregation of participatory activities started when KC Grad gained abandoned building in Braće Krsmanovića street for their cultural activities in 2009, though the peak was from the end of 2013 and this condensed interaction between urban spaces and civic life lasted for 2 years (Figure 2).

When we analysed the structure of agents, we addressed their basic characteristics already identified within the key categories from ANT methodology (Figure 3). These ANT categories indicate the figuration of the chosen agents in the concrete environment. They are adapted according to our interpretation of ANT methodological approach. Agent structures are circumscribed based on roles these agents play in Savamala, as follows: (1) agent nature – its operational manifestation, (2) level of influence – the boundaries of activities and target groups, (3) structural networks – agent’s primary activity, (4) socially functional networks – social function, and (5) secondary networks – subordinate function(s) (Cvetinovic et al., 2016b, Cvetinovic et al., 2016a). In terms of agent nature, chosen bottom-up agents figure as sets of horizontal entities of events/projects/activities. In our case, the strict focus on bottom-up activities has limited the scope of secondary networks characteristics mainly to either those focused on urban or NGO sector or small-scale services. Moreover, socially functional networks formed from ground up are mainly formal/informal collectives with non-transparent or unclear internal organization or foundation procedures. Consequently, various structural networks clarify agents’ roles and indicate the paths of their behaviours and networking capacities at the local level.
For further analyses we have chosen the most influential public and private organizations. However, several of these agents have unclear and transparent funding structure – while they receive some public funding, they are also partly profit-oriented (KC Grad, Mikser). Nova Iskra is the only explicit privately-based organization.  Agent’s social function is strongly connected to their level of influence in this case. All these bottom-up actors are active at local, less often city level, and international level, though their international visibility is also more in the domain of funding – several are recipients of international financial support (foreign embassies and foundations, European cultural and art organizations and programmes) or under direct supervision of international entities (Urban Incubator Belgrade was the initiative of Goethe Institute). However, there are others with transparent financial scheme (Ne da(vi)mo Beograde initiative ). KC Grad and Mikser, for example, even though some of their activities are publicly funded, they also incorporate profitable services (café-bars, shopping areas, concerts, exhibitions and other lucrative events/activities).
Furthermore, following the nature of these agents, we apprehend that the cultural and artistic activities in Savamala do not belong to the institutionalized art and culture. In this respect, most of them relate to NGO sector or they acquire or occupy publicly owned spaces which they use for these activities. MKM and KM8 are municipal spaces shared with different NGOs and offered for multiple projects/activities/events from different actors. Finally, the majority of these agents aspire for a consulting role on the wide field of urban issues, culture, art and education or to implement a range of ideas/solutions/interventions on urban or social level. In Serbian context, they aim to provide an alternative body for catalysing available human resources and translating global knowledge into the local context of Savamala and Belgrade.

\hl{Samardzic in Doytchinov 2015} 
The  urbicide  in  Belgrade  is  fed  by  the  mentalities and  the  logic  of  incompleteness:  unfulfilled  urban development  plans,  vane  political  promises  and abandoned  projects.
In  Belgrade  the  authorities  are  everywhere,  while  lawlessness  is  pervasive as well. The administrative buildings, the ministries and the remaining state administration are scattered by 200 points in the center of Belgrade and Novi Beograd. The terror of the officials armed with official cars, sirens and police escort is part of a daily scenario, including occasional dignitaries from abroad. 
The absence of a “middle class” in the transition process and at the top of the political decision-making correlate probably with the weakness or absence of institutions, including the institutions of culture: the temporary closed National museum and Museum of contemporary art, no Opera house, the bankrupted cinemas etc. The insufficient economic potential of the middle class particularly affects the stability and effectiveness of the institutions. 

\hl{Lazarevic Bajec}
non-existance and inefficiency of institutions: therefore the decisions are made in negotiations between investors and local governments, while investors have clear and rationally-defined criteria of their interests, the local community, public or civic sector,even though they posses legal empowerments, but they lack adequate and operational instruments for performing their power and being equal in the negotiation process

\hl{Savic 2014}
strongly spatialized identity through the exceptionalist discourses of smallness.

\hl{(Urbani razvoj u Srbiji Ministry of Space 2014)}
Authorities approach the economic revival by focusing their capacities and attention on investors and adapting the regulatiory framework to serve their needs. Citizens are excluded from the decision making. Even though
public participation is a strategic goal in urban regulations, its importance is often denigrated and it is only nominally existent in practice. 

\hl{(Hirt 2009)}
The entry of global capital, including Western development firms, is underlying another residential trend, which may become more prominent in the future: the trend toward building large, flashy and often gated communities, targeting expatriates, employees of foreign firms and embassies, and Belgrade’s top business echelon. 

\chapter{MAS System Building}

%%%%%%%%%%%%%%%%%%%%%%%%%%%%%%%%%%%%%%%%%%%%%%%%%%
Within an urban system, all agents are interdependent. The agent, while being influenced by the others, also influences them simultaneously (Bousquet and Le Page, 2004). MAS traces agent profiles and the character of their inter-relations and inter-connections. Agent profile is a combination of agent structure, agent preferences, and agent behaviour. The behaviours of agents are identified by qualitative surveys and analysed using multi-criteria MAS analysis of their profiles and references to system development (maintenance, transformation or change). We defined it here through the categories of: social practices, urban conflicts and contextual resources (spatial capacities and social potential) which are continuously produced within this initial networking.

We have already mentioned in the introduction that the level or urbanity and the morphology of urban decision making circumscribe urban dynamics. Herein, we will not elaborate the logics and the influences of urban decision making on agent behaviour and urban dynamics of contextual processes. However, we apply the morphology of urban decision making for the initial distribution of the collected data. Urban decision making in Savamala is conducted in continuous negotiations among 3 layers: top down urban planning, real estate transformations and bottom-up participatory activities. Even though there is a multitude of active actors that cross their influences in Savamala, the limits of this paper allow us to interpret in detail only the major ones in order to present the logics MAS-ANT cross-pollination method. Savamala has been marked by all mayor transformations of Serbian society over time and, following the trends, it has also fallen under the massive, but rather disputable waterfront mega-project, that aims to remodel Belgrade landscape according to modern high-rise metropolis patterns (Figure 2).The identification of urban key elements in Savamala is based on the analysis of Serbian urban planning regulatory framework, current urban and architectural projects on site, as well as insights from experts in urban planning and urban research and urban activists collected in an online survey.

\hl{Adjustments of Planning practice Nedovic budic 2001}
Differentiation between the universal and regime-based, society-specific, and culturally unique features of planning is necessary in preparing for the international exchange of planning ideas and practices. Identification of planning mechanisms that have emerged in reaction to the new political context is a first step toward developing an effective new system.
understand:
1. forces and trends - identify gaps in planning
2. uniqueness of the circumstances - contextually sensitive solutions
3. applicability of experiences and tools - exchange experience, knowledge, ideas

\hl{Vujosevic and Maricic 2012}
it is important to revise institutional and organizational framework and patterns in Serbia in order to make the way to implement major strategic aims already defined in regulatory and strategic documents in Serbia
contextual resources, conflicts and practices for example

\hl{ref (Volic et al, 2012)}:
cultural policy and law on culture in Belgrade and Serbia, relationship city-state authorities; reduction of the cultural policy to city marketing - pointing out indicators in current Belgrade waterfront activities;
Cultural policy and consequently cultural actions/interventions are  politically biased - rely on the consensus of experts identified within the institutionalized cultural framework.
The role of the Secretariat of culture
Objectives of cultural policy since 2001 (p3pdf) 
Master Plan Belgrade in 2021 phase gives importance to culture: attractiveness of urban areas with planning and organizational solutions - for cultural ventures?

\section{Dynamism of urban agency in Savamala}

As discussed above, a kaleidoscope of collected data on Savamala neighbourhood have revealed sets of relationships between the identified actors, the level of urbanity and urban dynamics. To gain a fuller appreciation of the different types of contextual processes that instigate this dynamics, we rely on the qualitative data obtained in 4 rounds of consecutive data collection: literature review, questionnaires, workshops, interviews. 

compare architectural competitions of the first years (200-2004) \cite{Stupar} with the lack of the same with BGDH2O

\hl{Grozdanic}
transport system should be renewed -  prim.aut. and reconceptualized

\cite{(Hirt 2009)}
Belgrade on the road of suburbanization
an exception to the general suburbanization trend in Eastern Europe  because of the outstanding appeal of its center because attractive areas in the vicinity of the center continue to have land available for residential development. (and because of the non-functional old-fashioned traffic infrastructure prim.aut. connected to Belgrade Waterfront as well prim.aut.)

\section{Profiling the agents}

For the analysis of Savamala bottom-up urban transformation, we relied on the defined agent profiles. First of all, we investigated agent structure and preferences. We treated these as dynamic features of urban agency in Savamala. Then we tracked behaviour of these agents and their influence on the state of urban environment in Savamala. Finally, this allowed us to sum up their capacities and limits to influence urban transformations and changes.

These elements are initially sorted according to the pertaining layer of decision making and chosen to express the balance of the different networks of influence (national, city, local). In this respect, top-down urban planning actors taken into account are:  (1) The Ministry of Construction, Urbanism and Infrastructure (MCUI), (2) Belgrade General Urban Plan 2021 (GUP BGD 2021), (3) Spatial Plan for Special Uses for “Belgrade waterfront project” (SPSU-BWP), (4) City architect (CA), and (5) Urban planning institute (UPI). As follows, the actors of investor-based real estate transformations incorporated in our analysis are:  (6) Lambda development (LD) and (7) Belgrade Waterfront Project (BWP). Moreover, Savamala has been recently established as a hotbed of creativity and participation in Serbian context, and the main protagonist of bottom-up participatory activities taken into account herein are: (8) KC Grad cultural center, (9) Mikser multidisciplinary platform, (9) Urban Incubator project (UI), (10) “Ne davimo Beograd” initiative.

On the level of data analysis, the figurations of these agent structures were disembodied through ANT in terms of: (a) the nature of actors; (b) structure and networks of influence; (c) secondary and socially functional networks; (d) social artefacts (political, economic and cultural); (e) networks of translations (ANT article). Figure 2 summarizes the structure of chosen urban key agents based on the data provided from the key informants: (A) experts, (B) participatory activities, (D) Belgrade Waterfront Project and corresponding regulatory framework. Agent structure is the product of thorough ANT analysis. Different layers of decision making reflect the difference in agent nature. Top-down agents are activated in the form of documents (SPSU-BWP and GUP BGD 2021), institutions (MCUI and UPI) or the assigned roles in the public domain (CA). Various bottom-up agents figure as a set of horizontal entities of events, projects or activities or all of them together. Structural networks interpret the functions of these agents at the local level. The dual position of the Ministry of Construction, Urbanism and Infrastructure as both normative and executive planning body after the ambiguous discontinuation of Republic Agency for Spatial Planning opens the floor for twisted institutional practice reflected further on in agent behaviour. The choice of policy agendas is based on their problematic engagement in the enactment and changes of regulatory mechanisms for the wide area of Belgrade waterfront. SPSU-BWP and GUP BGD 2021 received harsh professional criticism. The first relied on the multiple misinterpretations of Law on Construction and Planning, while the modifications of the second enable construction of profitable high-rise commercial and residential buildings in the coastal area. The drafting of the documents was delegated to UPI. Furthermore, in the process of SPSU-BWP adaptation, the city architect and its cabinet  exert power and influence in expert control and public inspection process (Iniciativa ne davimo beograd). The key inspirator for these actions has been the private, project-oriented project for the Belgrade waterfront. Similar ambiguities have happened to the amendments to the City Regulation Plan (CRP) dealing concerning Beko factory location nearby. Foreign investor figurated by LD (6) financed the drafting of CRP and directly influence the makers of the plan by employing them afterwards. Conversely, the initiative NDBGD (10) is the key entity performing control of urban governance outside the top-town regulatory framework. Through its actions (events, media and publishing) NDBGD has been constantly advocating for more transparency and participation in Serbian urban planning framework. KC Grad (8) was the first informal collective to settle down in Savamala, while Mikser (9) was the local stakeholder that took an active part in UI projects, and both of them are the only non-formal organizations to be included in new plans for Savamala (SPSU-BWP, GUP 2021, Master Plan BWP).
The second level of data analysis is MAS analysis of agent contents, fields of interest and influences. In doing so, agent preferences are revealed in terms of objects (contextual resources) and relations (social practices and urban conflicts). The visualization of agent preferences confirms that the spatial and locational capital of Savamala is the main attraction for human and institutional actors on various levels. What is more, the distribution of urban conflicts between top-down and real-estate agents proves that centralized state is the central power pole of urban development in Savamala. When complemented with information from expert and participant interviews, the conclusion is such that urban regulatory framework has been purposely distorted to accommodate private investors’ interests. In so doing, city authorities and city planning departments are the pillars of such biased governance mechanisms, while failing to provide adequate expert control of the planning processes, development plans and implementations and to enable smooth transformation of post-socialist urban systems. While social potentials are nominally addressed by policy agendas, the leader of gradual, small-scale, participatory urban changes and possible brownfield regeneration has been cultural initiatives and NGOs recently settled in Savamala. Finally, SPSU-BWP is a flagship among regulatory documents that legitimizes blurred, non-transparent, un-feasible, interest-based urban planning procedures. In addition, BWP in itself manifests an official disregard for expert opinion and induces citizen revolt and disbelieve in public authorities. On the city scale, investment-based initiatives like “Belgrade waterfront project” reduce the potential of democratization, decentralization of urban processes.


Based on our qualitative research on Savamala, the most prominent aspects in direct correlation with agent functioning at the local level are: political (participation, transparency, and institutionalization of culture), economic (public funding), and cultural (global flows of ideas, trends, information and knowledge). Consequently, we have recognized the following clusters of resources, conflicts and practices (Figure 4):
•	Spatial capacities (SpC):  (1) accessibility; (2) central position in the city; (3) brownfield area; (4) architectural diversity; (5) proximity of the river; (6) deteriorating area; (7) green area; (8) waterfront area; (9) recreation area;
•	Social potentials (SoP): (1) lack of private investment in the area before 2012; (2) architectural and cultural heritage; (3) social diversity; (4) aroused interest for this neighbourhood from cultural and artistic groups, individuals and organisations; (5) trade and artisanal area – cultural heritage and traditional crafts; (6) creative cluster; (7) participative and self-organisational initiatives in the area (KC Grad, Mikser, etc.); (8) small commercial area; (9) underdeveloped area; (10) diversity of interests and power poles in the area;
•	Urban conflicts (UC): (1) disintegration of heritage; (2) lack of systematic investments in constructing industry (debt crisis 2008-2012); (3) lack of data about the state of physical structures; (4) lack of data on social structure in the neighbourhood, (5) attractive location for private investments, (6) poor population, squatters and marginalized groups in the area;
•	Social practices (SP): (1) support of urban related activities (urban design and public participation); (2) support design activities (interior, fashion, graphic), art, culture, education on city level; (3) translation of global trends into local and regional practices; (4) design, communication and creative industry activities in Belgrade; (5) local and global economy trends in the area; (6) develop waterfront recreation area and sustainable transport (cycling).

The data in Table 1 show how different agents opt for mentioned contextual resources, urban conflicts and social practices in Savamala and what is the relation between their nature and these preferences. Accordingly, we may conclude that contextual resources, either spatial or social, are the attraction factors that make Savamala a neighbourhood saturated with different actors and interests. On the one hand, all bottom-up agents that have active approach to urban environment through projects, activities and events, also direct their initiatives toward solving urban conflicts. While, on the other hand, those that include profit converge rather to social practices that maintain current urban order. Consequently, these agents refer to their contextual preferences, organize and engage in networks on local or superior levels and in this way influence the state of urban environment in Savamala.

\hl{Expert workshop - Bata Stojkov prez}
top-down urban planning:

    legalization of illegal construction
    investor dictate
    political party power over planning, development, regulations
    participation as a farce
    public hearings as pure formality, disregarding negative comments
    actual key actors:
        the state
        privileged foreign and domestic developers

    state government

    prompt law changes (lex specialis)
    capital infrastructure sacrified
    cultural patrimony set aside
    land offered for free
    protected assets presented to private developer with no role of public institution: Spatial Plan of Spatial Uses

    city planners:

    low capacity
    succumbed under political pressures

institutions (group of individuals):

    afraid to defend public interest
    public pea

    local government

    weak
    follow orders from national level
    realize projects without objection
    weak planning commission, following political orders (new - transitional reality)

foreign  investors:
    free from any legal duty (Law on Applying Agreement on Cooperation Between Serbia and Emirates):
        public procurement
        expropriation
        laws
        urban regulation
    violating Constitution
    private project as public interest (lex specialis)
    no feasibliity study

Bata Stojkov questions/video check:

    the traditional model - is it applied or not?
    feasibility study in urban legislation? (for special uses)
    
\hl{Expert workshop - Ksenija Petovar prez}
4500 inhabitants (2002 Census)
poor citizens
Savski venac (3 wealthiest municipalities in Belgrade)
non-institutionalized cultural and other activitiee 
    municipality
    foreign actors: states, cultural centres
    cultural, artistic and educational activities
    not among local citizens
civil sector:
    alternative status of Savamala is limitation
    not accepted in Serbia  as key factor for diversification of social power structure and cultural development
    dependent on visible / invisible power structure and group interests in authoritarian society with power concentrated in parties and state nomenclature - private, State, City officials (land use, property structure)
    powerful interest groups linked to authorities
    neither socio-political power nor public support
    short lasting, no certain future
    
REAL ESTATE TRANSFORMATIONS
\hl{Vanista Lazarevic in Doytchinov 2015}
there are still many unresolved issues of property ownership due to  the  inefficient  local  courts  (one  example  being denationalisation).  They  have purchased property at extremely low prices and are now waiting for the future gentrification of the entire area, to make profit. The economists say that the limited public resources, the urgent needed investments in the infrastructure and  the  current  financial  crisis  have  contributed  to  the  collapse  of  the  real estate market in Serbia as well as to the sudden halt of investments after April 2010. Still, Serbia’s tycoons have already managed to secure successfully great investments in this area.

\hl{Vujovic and Petrovic 2007}
	Political Actors
enterpreneurial mode
convert political capital into economic one - in the blurred transition of old institutions (socialist and  new liberal ones)
political voluntarism and corruption, "buddy" and brotherhood networks - enormous concentration of wealth and power to political and economic actors.
concentrated on the image of the Serbian capital in the Balkans
	Foreign Investor Council
main problem is the "governmental control over the supply of larger pieces of land" (political actors became powerful economic actors prim.aut.)
	Urban planners
private and group interests rule the actions of planners
planners are deprived of dignity and professional authority
private investors dominate Belgrade's urban development
underdeveloped legal framework allows politicians to hold more power
"pervasiveness of uncontrolled urban development in Belgrade - deconstruction of urbanity and the abuse of its public spaces"
Local government un-transparent and semi-legal system
    
\hl{Vujosevic and Nedovic Budic 2006}
Key issues of its legitimacy, role, mission, political background, contents, and procedures were ignored.
planning as techical activity - socialist tradition
explore newly generated distortions in the triangle power – knowledge – action (Friedmann 1987)
New planning should be grounded in the authority of law and traditional social rules. 
General principles of planning would include the following qualifiers: pro-active, flexible, indicative, adaptive, inclusive, monitored, evaluation-and-feedback-based - how to achieve it in a society with different social values

\chapter{MAS-ANT data display}

Historical, contextual and on-site data are filtered in these rounds of data collection going from an abstract document-based version to group analysis in professional and educative workshops and elaborated interpretations from local experts. Such classification enables pointing the key actors of maintenance, transformation and change processes. ANT analysis of these results highlights individual features of these actors and constitutes agent structures, while MAS elaboration explains their behaviours and the involvement in urban affairs.

They come into existence from already identified agents and have a crucial role for tracing urban development process: (a) resources instigate transformations, (b) practices identify system maintenance, and (c) conflicts boost potential changes. Moreover, these interactions also define what is possible (through contextual resources) and what is happening (urban conflicts and social practices). The analysis of the identified bottom-up agents according to these principles gives us the opportunity to determine their influence on the system evolution, their capacity to intervene and their biases that cause eventual negative effects.

\section{Body of urban relations: Urban Development dynamics}
Data display

Post-socialist urban development induced radical political, economic and cultural shifts in neighbourhoods in Belgrade. Savamala is therefore a representative case for intensive collision of top-down and bottom-up pressures. Endowed with a prime location in Serbian capital, Savamala has been directly or indirectly targeted by most of Master General Urban Plans since the beginning of the 20th century (GUP 1923, GUP 1950, GUP 1972, GP 2003 (revised in 2005, 2007, 2009, 2014), and GUP 2016MUP 1923, MUP 1972, MP 2021) as well as having been capturing the attention of national and international capital through glorious architectural projects (“Town on the Water”, CIP Europolis, Beko Masterplan, Belgrade Waterfront Project etc.). In Savamala, a complement of post-socialist urban development is found in small-scale cultural practices, crowdsourcing activities, creative industries, urban manufactories, and cooperative economies which, slowly but surely, spread from the upper Savamala to the riverbanks (Cvetinovic et al., 2013). The hybrid field of overlapping MAS (multi-agent system) and ANT (actor-network theory) methodological approaches proposes an innovative concept to define causal relationships among all different urban elements and developmental prospects of their interrelations and interconnections.

Agency and relationships of the above identified human/non-human actors in Savamala is the cornerstone of bottom-up networks constituted on the local level. As they primarily depend on the contextual preferences which the agents attribute to their activity and relations, tracking these associations is also a crucial factor of urban transformation or change, if there is any. Therefore, with MAS-ANT method we aim to estimate if this bottom-up management of social exchange and urban transformation contributes to an improvement of life and functionality of urban systems.
The analysis of agents’ structure and preferences and qualitative data on Savamala neighbourhood indicate urban assemblage networks formed and contributed from ground up. Namely, implementation and management of participatory activities is the focal point of urban interventions in Savamala and these networks involve the range of local and city NGOs as well as several IOs, initiatives and collectives. In few cases (Urban Incubator Belgrade, Mikser festival etc.) municipal authorities provide support in these managerial networks. However, local, municipal and city authorities as well as international funding organizations (embassies, foreign institutions) take part in financial networks (funding instruments) and for several projects in the implementation networks (Savamala, a place for making; The game of Savamala; Camenzind, NextSavamala and Savamala design studio projects within UIB). In the case of UIB, the activities in Savamala also comply with the campaign of Goethe Institute to focus part of their activities in their branches worldwide on “Cities and Urban space”. Speaking of these participatory projects, they are pillars of bottom-up research and education networks and in this manner they cooperate with Serbian and European universities . On the limited level few agents (My piece of Savamala, KC Grad, Mikser) engage in consulting networks with municipal and city authorities (City mayor, City architect, Municipality of Savski venac) and real-estate actors (Eagel Hills, Belgrade Waterfront Project). Finally, the sole interest of Ministry of Space collective and Ne da(vi)mo Beograde initiative, which are overlapping in the course of individual participants and actions as NDVBGD is the initiative led by this collective, is activation in control and verification networks for all urban questions, problems and solutions and they refer back to city and national urban and political authorities and experts.
Figure 4 visualises MAS-ANT analysis of these bottom-up networks in reference to agent structure, preferences and behaviour. In this diagram the relations between contextual preferences, addressed social aspects and these agents with their explicit structure (level, nature, functions) are explicitly represented. In this respect, we can acknowledge urban system references - the indicators of maintenance, transformation and/or change.

Based on agent profiles and networks, we have identified multiple maintenance, transformation and change actions that influence the state of urban environment in Savamala. First of all, the settlement of civil organizations has supported service and commerce activities and recreation zones already present there.  Several traditional craft shops have been in Savamala for decades and now, following the hype of this neighbourhood low-profile bars and restaurants, as well as art and culture initiatives fostering cooperation, globalization and modern business trends position there. Their significance not only on city but also on an international level promote Savamala among architects, artists and all young creative workers for the region and, in general, Europe.
Visible spatial transformation are: activation of waterfront area (for a while activities and events were organized on the abandoned ships on the Savamala coast before they were removed), preservation and improvement of cycling paths (initiative of Streets for cyclists NGO); preservation of skills and traditional crafts (Savamala a place for making), fostering the sense of community and sharing (UIB was the pioneer in participation, followed by Goethe guerrilla collective who organizes and supports civil, participatory and design activities and operate in KM8 community space), as well as informing and educating public (The game of Savamala, My piece of Savamala etc.) are the major social transformations which have been directly induced by this pioneer bottom-up agency. Moreover, local population emphasizes that these participatory programmes, in reference to their organizational preferences and capacities, take into account the needs of the locals, youngsters (UIB) (Müller-Wieferig and Herzen, 2013) and marginalized groups (Ministry of space and NDVBG) (Mitić and Miladinović, 2016). Conversely, development of Savamala creative cluster and small-scale hype brownfield regeneration and public place design are mayor smooth transformations that have made Savamala visible on the international scale.
Finally, urban change induced by these bottom-up activities is limited in its scope, but it shows significant potential if these activities encounter understanding and support from city authorities. Savamala civic district, participatory urban upgrade, and brownfield and urban heritage regeneration are their ultimate goals. It is also important to mention that the combination of Savamala spatial capacity (its central urban and the proximity of bus and train terminal) and the primary activity of these bottom-up agents (inclined to boost knowledge and vision building as well as experience sharing potentials) led to prompt and adequate reactions to current refugee crisis that has hit Europe and Belgrade. These activities for helping refugees/migrants are coordinated by Mikser and financially supported by many national and international organisations – United Nations High Commissioner for Refugees (UNHCR), CARE International (Cooperative for Assistance and Relief Everywhere), Red Cross etc., as well as by supplies and care from Belgrade citizens.  These efficient actions also speak of competence and alertness of bottom-up agents to respond to the dynamics of modern urban context.

The final stage of the analysis is MAS-ANT cross-pollination on the level of agent behaviour. The broad  domains  of  the  agent  profile  answer  the  question  of  who,  what  and  how  acts  in  the  network  of complicate relations among human and non-human urban elements. The behaviour of chosen key agents in Savamala enables us to trace their interrelations in networks {urban assemblage networks from ANT (Ant article)} and connections to political, economic and cultural factors of post-socialist urbanity (MAS). In this way, urban key agents personify the link between recognized social artefacts and contextual resources, urban conflicts and social practices. In Savamala, we have realized the correlation between spatial capacities and urban conflicts, which are dominant preferential elements of the chosen urban key agents. Table 1 explains how political, economic and cultural tenets of post-socialist urbanity are distributed among urban key agents and in which way they influence Savamala.
The level of post-socialist urbanity directly depends on the balance between how the society/local community as a whole profit from contextual resources and reduce the negative effects of urban conflicts. Our analysis reveals that centralized decision making and provisory rules, among other political aspects, are the front runners of biased exploitation of the spatial capital. What is more, the private investors, who easily satisfy their interest regarding Serbian spatial capital, do not have any concern or interest for social benefits of the society. Instead of enabling the social potential of current local cultural and civil initiatives, they contribute to their expulsion. Not to mention that disjunction between expert and practical knowledge and regulatory and implementation actions contribute to the deterioration of the quality of urban life.

Another layer of ANT-MAS cross-pollination is the illustration of agent behaviours in terms of active-passive relations. Namely, ANT assemblage networks altogether describe Savamala urban dynamics, while MAS interpretation adds active-passive roles to urban key agents. Table 2 therefore shows the distortion of post-socialist institutional framework where policy agendas are passive elements despite their essence as a regulatory framework for strategies and actions. What is more, personalized institutional relations, figuring in particracy dominance over political power, make public authorities more vulnerable to the variety of political and economic interests. Leaving the decisions on public interest in hands of individuals eases the abuse of power and weakness of the institutional framework leads to the lack of control and implementation of sanctions . 
 
These are only several conclusions that could be formulated from the illustrated agent profiles for urban key agents in Savamala. Knowing that in Serbian institutional discourse there are no mechanisms serving for translating destructive conflicts into constructive and productive elements (Vujosevicet al. 2015), an elaborated indication of the weakest links as well as unleashed socially harmful relations may be a driver for interventions from either top-down (public expert institutions) or ground up (civil organizations, NGOs, independent professionals, artists and cultural workers). Even though these conclusions may sound familiar and obvious, the lack of methodological and evidence-based explanations leads to dissolutions and manipulation of the information. Therefore, MAS-ANT methodological approach, modest in initial conclusions but reach in simple illustrations and clarifications, builds a framework for actions outside biased expert and institutional dimension. It describes urban dynamics, positions the level of urbanity of the chosen environment and indicates the field of action, but without clarifying the single steps in that direction. The next logical step is upgrading the level of post-socialist urbanity that should therefore enable the complex planning logistics and link the top-down changes to the bottom-up changes in the urban systems. In order for these actions to better correspond to current post-socialist urban reality, diversity and reciprocity in the nature of the on-going transformations must be acknowledged and taken into account. Economic (transformation of production and consumption in relation to space, income polarization and poverty), political (urban governance, participation and decentralization), spatial (demographic trend and distribution of functions) and social (social exclusion, social activism and informality) transformations all have their signifiers in behaviours of chosen urban key agents and could be traced accordingly.

hl{Blagojevic 2009}: 
the notion of erasure and the condition of  tabula rasa, theorists associate it with Emilijan Josimovic (for example Branko Maksimovic, Perovic): that " we (Serbians) are a nation which broke off totally with the old obscure Asian customs and prejudices, and that all which is progressive, beautiful and good now adheres to us."

\hl{Cities in Transition 2013}
insititutional and legislative reforms 2013
    ad-hoc interventions  (helped by rushed-in decentralization and problematic horizontal coordination)
    urban economy completely dependent on external investments and influences (EU)
    deteriorating housing areas and low affordability
    mobility -> cars - > congestions
    mainstream world culture but disintegrating cultural heritage
    emergence of regional urban systems
    large urban areas hold the bulk of country's intellectual and educational human capital
    short-term strategies and governance interventions (with the aim to pass easier and quicker through transition)
    domination of free market ideology for generating urbanization modalities and spatial patterns and configurations
    
\hl{HIrt 2009}
land use changes reflect the process of de-industrialization and territorialization of urban economy;
from 1989. commercialization of urban fabric

\hl{Vujosevic et al. 2010}
poor legitimacy of transition reforms
unsustainable development pattern
lack of political dialogue (consensus) on broader social issues (goals, content and modalities)
domination of anti-planning and anti-development concept

\hl{Zekovic et al. 2015}
global economic crisis is deeper in SEE (South Eastern Europe):
    low development status
    low economic growth
    high unemployment
    poverty
    informal economy
    informal building
    uncertainties about:
        impact of globalization
        inappropriate institutional framework
        poor technical infrastructure
        high public debt
    prolonged regulatory gap in terms of
        economy
        investments
        urban development
        urban economy
main problems in Serbian urban land market:
    missing or inefficient regulatory mechanisms and institutions
    ways/instruments for financing urban development
- zoning regulations and corresponding taxation are not harmonized with strategic spatial and urban development, predominantly administrative zoning regulations are in place
THE LACK OF POLITICAL WILL (Vujovic and Petrovic, 2007) - administrative approach to urban-land management and top-down decision making - top-down structures and individuals in these positions tend to keep power in their hands
Issues:
    Conversion of agricultural and forest land to urban land
    Privatization of urban land and the conversion of leasehold on urban land in public ownership into property right
The regulatory framework supports and administrative approach rather than market approach
- with market rules at work in practice, post-socialist urban planning becomes inefficient

	\hl{Vukmirovic in Doytchinov et al 2015}
Belgrade stepping into the neoliberal trends, disregarding the relevant planning documents and causing them to change, lacking the long term vision - contradictive and inconsistent manner of  city branding

\hl{Samardzic in Doytchinov 2015} 
The  urban  development  of  Belgrade  is  revealing  challenges  and  traumas both from the recent history and, as well, from the long term conjecture and mentality: poverty, sharp social, cultural and ideological differences, inheritance and influence of nationalism, socialism and political religion, undeveloped or inappropriately developed infrastructure.
All the shortcomings of the contemporary Belgrade, the inadequate solutions or the lack of them in the urban development, the maintenance and building of infrastructure, the cultural and social policies framework, can be discerned from both the historical perspective and the analyses of the current condition. 
The residents of the city, whether the natives or the newcomers, destroy their own habitat. At the same time Belgrade has remained the last major urban haven in its part of the world. 

\hl{Finansiranje komunalnog opremanja gradj zemljista 2013 SKGO}\\
gradjevinsko zemljiste je najvredniji deo teritorijalnog kapitala i kljucni faktor konkurentnosti lokalnih samouprava
politicki faktor krucijalan u procesu donosenja odluka koje se komunallnih usluga and komunalne infrastrukture
polje za intervencije: ekonomski faktori  - izvori finansiranja komunalnog opremanja zemljista, validne ekonomske procene ulaganja u opremanje zemljista
ekonomska regulacija
    odvojenost operativne od regulatorne uloge
    stabilnost regulacije
    transparentni mehanizmi
    
\hl{Maksic 2012}
    loose horizontal and vertical coordination - lack of public sector engagement on the regional level (regional level is crucial for enabling coordination)
    investors engage individually
    institutional engagement should bring together different interest groups
    decentralization is needed (governmental and administrative)
    
\hl{Thomas 1998}
defensive policies of socialist states in early transition:
(1) market  forces  cannot  be  eliminated  but  can  be  regulated.
(2) regulation coordinated  with  other  countries.
(3) the growth  of public  spending curbed.
(4) the welfare  state can be defended  but  not  extended.
(5)  That  privatization  is  unavoidable
(6)  That  equality tempered  by  the  preserve incentives  and  competition.
(7) the  power  of  international  finance  may  be  contained
(8) European  integration  provides  opportunities  which  no longer  exist  at the national  level.

\hl{Stojkov and Dobricic 2012 05}
modeli implementacije:
definicija:
    definicija modela - uzorak za izradu, priblizan opis pojave/objekta u stvarnom svetu
    definicija planiranja - odluke o akcijama u buducnosti za postizanje ciljeva
    jedinstveni i kontinuirani proces
    sistem planiranja - logicki, funkcionalan and vremenski koherentan
    zavisi od tipova and metoda planiranja
tipovi:
    strateski projekti
    razvojni and konkrenti projekti
    strategije and politike prostornog razvoja -> opste: nacionalni and regionalni nivo
    zastita prostora -> prostorni plan podrucja posebne namene
    planska resenja tehnicke prirode -> realizacija u prostorni (uglavnom lokalni nivo i infrastrukturni projekti)
    pravila koriscenja, uredjenja and izgradnje -> lokalni nivo
krucijalni elementi uspesnje implementacije:
    interna konzistentnost
    izbegavanje preterane kompleksnosti i detaljnosti
    izbegavanje fragmentisanosti and usmeravanje na celinu (+ integralno planiranje)
    prednost resivim problemima
    povezanost sa merama, instrumentima and sredstvima: plansko-programski, finansijski, normativno-pravni, organizacioni
    strukturirani planski ciljevi - koherentni skup opstih, posebnih i detaljnih ciljeva (Boisier, 1981 - Planning a Regional system): opsta opredeljenja, relativno konkretiovane ciljne propozicije, konkretizovani iskazi o sadrzaju, vremenu i prostoru
    neophodnost dijagonalne koordinacije (ne hijerarhizovana implementacija, formalna, institucionalizovana podela and klasicna administrativna pravila) - interaktivni model implementacije (Alexander and Faludi , 1989
    fleksibilnost and aposteriorno zakljucivanje
model implementacije kao precica ka zeljenom sistemu planiranja dok ne zazivi u praksi

\hl{Vujosevic et al. 2010}
no consensus what public interest now is
    poor political legitimacy of the  reforms
    no clear political will
    no societal consensus on the most important issues

\section{MAS-ANT diagram: scope and operationality}

Discussion

\hl{Vujosevic and Maricic 2012}
Good points of Yugoslav self-management abandoned
Bad points kept: paternalism, manipulation, clientelism

 	\hl{waves of planning 2006}
planning systems evolve over time by going through cycles or waves that are characterized  by  their  own  dynamics  of  innovation,  imposition,  borrowing  and  adjustment,  and changed through transitional processes. The evolution is affected by internal and external context and results in a certain quality, style and system of settlements and planning, which
are envisioned to mature over time.

\hl{Vujosevic 2015 Regionalizam u Srbiji 2}
centralni problem: iscrpna i precizna identifikacija interesa
formiranje odnosa medju stakeholderima na osnovu (relacije):
    znanje
    akcija
    moc - direktna and indirektna
        kulturna
        ekonomska
        vojna
        politicka
upravna centralizacija and koncepcija vlasti u drzavnoj oligarhiji iz 90-ih jos uvek uticajna
TERITORIJANA INTEGRISANOST POLITIKA and PROJEKATA (nacionalni i subnacionalni nivo) - ref to BGD na H20
    horizontalna (sektori, usluge, tela)
    vertikalna - upravljanje
    teritorijalna - sve na datoj teritoriji
    vremenska - vremenski horizonti
PROBLEMI SRBIJE (show historical overview):
    neadekvatna regionalizacija  - region
    nepostojanje mehanizama za artikulaciju zajednickog interesa na regionalnom nivou - dominantna partijsko-politicka interesna afilijacija
    razmere raspodele budzeta - republika and opstina

\hl{Vujosevic 2012}
territorial capital - development patterns endangered  (p9PDF)
    unfavourable demographic structure and demographic recession - aging and braind rain
    number of refugees
    regional development differences
    spatial chaos (low construction and urbanization, illegal construction)
    
urban processes are long duration processes (Fernand Braudel)

In Savamala, we have identified a dynamic, interactive actor-networks articulated through decision making mechanisms of top down planning, interest-based real estate transformation and co-design and creative participation actions. In our case, global and local social factors (economic, political and cultural), placed in the particular spatially and socially constrained context (Serbia, Belgrade, Savamala), are the main forces of urban development and they constitute social artefacts (actors) and social aspect networks (urban assemblages) (Table 3). The detailed mapping and visualization of these actor-networks also accounts for contextual, post-socialist and transitional circumstances, avoiding explanations coming from the reproduction of social order, power and class. In other words, the collision of these grand narratives is present in the current Serbian context through: (1) crisis of common social values and civic society standards, (2) lack of healthy investment interest and fair competition, (3) absence of public interest and public good concerns, (4) a battlefield of significant power pressure and interference of interests from authorities, business actors and civil actions.
Based on the performed ANT analysis, on Savamala urban development prospects we may specify the following: (1) lack of elaborated, strategic policies in urban development and investment; (2) cumbersome institutional structure; (3) distribution of publicly owned empty plots and spaces in Savamala to private investors/owners; (4) vertical clientelism in institutional framework (Vujovic and Petrović 2007); (5) up-to-date legal documents and policy agendas which do not correspond to urban reality; (6) overpowered and personalized Nation State as a key actor on citywide scale (BWP example); (7) semi-legal institutionalizations become official practice and a pool of opportunities for future exploitations; (8) provision of instruments for powerful actors to realize their interests through controversial institutionalizations; (9) unregulated economic incentives and measures; (10) economic aspects strongly influence political aspects and actors in post-socialist context; (11) institutionalization of private interests of powerful economic actors and marginalization of civic initiatives and public interest; (12) “growth without development” (Vujošević and Maricic 2012) roots in top-down approach to regulatory, managerial and financial networks; (13) privileged foreign and domestic developers in Waterfront/Sava amphitheatre/Marina Dorcol Redevelpment (Djordjevic and Dabovic 2009, Stojkov 2015); (14) political actors in Serbia have support for the replication of Thatcher-Regan model (Vujosevic 2015); (15) housing and commercial purpose of 80 percent of BW spaces (Zekovic et al. 2016);  (16) spatial fragmentation and unequal distribution of resources in Savamala: Waterfront and Upper (Urban) Savamala; (17) no adequate educational framework; (18) lack of participatory and communication culture; (19) biased role of media in advertising urban projects (BWP); (20) apathy of population concerning semi-legal, anti-constitutional, neglected public interest issues in BWP; (21) Lack of strategic development goals for cultural institutions and agendas - activities and initiatives (such as those in Savamala) are short lasting with no certain future (Vujosevic 2015); (Petovar 2015); (22) civil initiatives in Savamala have neither socio-political power, nor sufficient public support and funding (Petovar 2015) (Table 5,6). 

\hl{ref (Volic et al, 2012)}:
what institutionalization of culture really means:
    sustainability of capital investments
    all levels of community involved in local consultations
    investment in culture for global competitiveness and local consumtion
    connect people and communities
    assessment and measurement of investments in culture
    bring together experts from different fields
    bring together different levels of management
    provide a sense of identity and belonging for the community members
"In the current context  of  the  Serbian  polity,  ‘binding’  of  all stakeholders  and  moving  towards  a  common interest  might  seem  as  a  difficult  endeavor."

\hl{Doytchinov 2015}
While most Western capitals only gradually assumed national symbolism, those new and emerging capitals in eastern and South Eastern Europe entered this process almost immediately.
\hl{Roter Blagojevic in Doytchinov 2015}
Belgrade a constant struggle between the traditionalism and modernism, the conservative and the progressive

\hl{Vanista Lazarevic in Doytchinov 2015}
The stakeholders in the urban regeneration process:
A ministry or at least an agency responsible for urban regeneration must be established in order to create and regulate procedures, 
incorporation of the gentrification and regeneration into legislation,
A master plan is needed for all public regeneration projects,
All the regeneration projects must be the subject of public, national or international design competitions,
A National Urban Design Framework must be established with key design principles integrated into the planning guidelines,
local architects communicate the ideas with the public,
“active protection of the cultural heritage” - fit it in with the life of the city, not doing it produces the neglected areas.
Belgrade’s city branding happened as a result of luck or spontenaity and not within the framework of rules and regulations  and  certainly  without  a  strategy.

\hl{Vujovic and Petrovic 2007}
political voluntarism dominates the implementation of laws, therefore the quality of laws become irrelevant
vertical clientelism make its ways up and down through inefficient and corrupt government institutions 

\hl{Zekovic et al. 2015}
approaches, methods, instruments, institutional and organizational arrangements:
    urban land consumption
    urban land market in compliance with urban development
    land use and urban land use management
    tax system 

\hl{Vukmirovic et al 2013}
The inconsistency of expert opinions with those of the public is seen 
as  a  main  problem  for  implementing  planning  documents,  which  leads  towards  something  which  can  be named “spontaneous occurrence”

\hl{Mornings after Nedovic Budic}
the loci of political power seem to exert substantial influence over the planning system and practice
The professional cultures involved in planning—planners, architects, engineers, economists, sociologists, geographers—are also prominent factors - the formation of the Serbian planning doctrine and planning system, occasionally pulling in different directions.
is currently needed for Serbia is to settle on an operational and effective practice that would exert some order and care in the process of urban development. Institutional improvements and a firm action in curbing corruption and opportunism in urban transactions and land development would be important first steps.
currently in Serbia:  the day-to-day administration of urban
development

nearly destroyed industrial production \hl{Vujosevic et al. 2010}
- instruments to fight it (but the absence of measures of political transition to implement them are lacking) \hl{(Vujosevic et al. 2010)}:
        tax on extra profit
        reform of the security sector
        reform of tribunal
        reform of prosecution
        restitution
        denationalisation  

\hl{Grozdanic}
The rational use of land as the rare and limited recourse, is one of the aims of planning. But, in practice those aims are  not  fully  implemented, because  of  the  real  imperfections  of  urban  legislation  and regulation and personal weaknesses some of the participants in the process of planning and building, which sometimes, unfortunatelly, cause bad urban design and functions.

%%%%%%%%%%%%%%%%%%%%%%%%%%%%%%%%%%%%%%%%%%%%%%%%%%

\chapter{Conclusions}

%%%%%%%%%%%%%%%%%%%%%%%%%%%%%%%%%%%%%%%%%%%%%%%%%%

\section{Conclusions related to the research framework}

Moja ideja je bila da ovaj rad najvise odgovara na pitanje c) kako se metode ANT i MAS dopunjuju - ANT za opis dinamickog stanja, a MAS za opis dinamike promena.

Medjutim, ucinilo mi se vazno da predstavim svoju interpretaciju ovih metoda, zato sam uvela ovu ideju o arhitektonskom istrazivanju. Sto se toga tice, cini mi se da je trend pretvaranja u socioloski, planerski ili informaticki pristup globalan, ali nisam sigurna koliko imam vremena da detaljno istrazujem literaturu o tome. Jer ANT dosta koriste sociolozi za analize gradova i urbanog, a sama MAS je vise matematicka-kompjuterska metoda. Takodje sam se zbula jer mi se ucinilo da moram ovu kombinaciju metoda da ispratim nekim teoretskim okvirom, pa sam uvela pojam urbaniteta, jer mi nivo urbaniteta obuhvata i stanje i promenu. A u celom doktoratu sam mislila da obradim kako onda kombinacijom urbaniteta (opis stanja i agenata promena) i odlucivanja (sortiranje agenata i promena prema slojevima: planiranje, investicione transformacije, participativne aktivnosti) prikazujemo dinamiku urbanog razvoja, koristeci  ANT pa MAS. Dodala sam ovaj momenat arhitektonskog istrazivanja, jer mi se cini da bi to bio i glavni doprinos teze - prakticna primena ove metodologije za one koji se bave gradom na bilo kom nivou odlucivanja. Ideja mi je da se naprave vizuelne interpretacije koje lako mogu da se kompjuterizuju (html5) i onda lako menjaju i na osnovu toga stalno izvode zakljucci i uvode i opisuju novi elementi.
A usput bi proistekla i ta nova definicija urbaniteta.

\subsubsection{[..] to the research objectives}

An inclusive and dynamic urban development model, as outlined herein, constitutes a challenge with regard to redefining the scientific approach to urban conflicts through a the MAS-ANT methodological approach as an urban development strategy that ultimately aims at generating a new vision of cities that is best suited not only to post-socialist cities and transitional countries.

\hl{Samardzic in Doytchinov 2015} 
The relative poverty and the negligible public influence of citizenry are partly the result of the incapability of the local
humanistic  sciences  to  become  one  of  the  means  for  the  problem  solving. 

\hl{waves of planning 2006}
in  agreement  with  Thomas [1998] it is found that the relationship between past, present and future is essential to understanding  the  evolution  process  and  products  of  various  events  and  influences.
the dynamics of the process  of  evolution  resembles  more  closely  transiting  through  various  stages  (or  waves)
than a continuous development on a unidirectional trajectory

\subsection{[..] to the research questions}

The whole of MAS-ANT visualisations of agent profiles aims at encompassing major urban planning strategies, in situ transformative forces and potentials, and follow the creative paths of urban dwellers (participatory urban design activities) for imagining new urban futures. Consequently, in post-socialist cities, the final goal of MAS-ANT applicability is an action towards harmonisation of the morphology of post-socialist urban decision making.

\hl{Vujosevic 2015 Regionalizam u Srbiji 2}
upravljanje konfliktima - planerska interakcija mora biti cvrsto strukturisana
uloga participacije and mreza u resavanju konflikata

\subsection{[..] to the methodological approach}

communicate research-based data to nonspecialists.

Flyberg: 
Misunderstanding 4: The case study contains a bias toward verification, that is, a tendency to confirm the researcher's preconceived notions.
ANT reduces researcher bias

This illustration of Savamala urban context is but only one version of visualizing agent profiles. MAS-ANT methodology enables us to make a crucial change in approaching urban dynamics in post-socialist cities. The contribution lies in the rise of explanatory framework from rigid and static to relative and dynamic and susceptible to change through continual iterations. While some trends and directions of urban development in Savamala are clear and defined, uncertainty dominates urban decision making and implementation in the turbulent environment of post-socialist cities (Nedović-Budić 2001). Exposing the power game of relations and influences among urban key agents in Savamala indicates opportunities for altering post-socialist urban planning on various levels. MAS-ANT analyses reveals: 
(1)	Processes and procedures which are interrelated within an urban agent profile (nature of agents, network of influence, function etc.);
(2)	What are assemblage network of relations between regulatory framework, urban actors, spatial issues;
(3)	what  urban  patterns  and  social  impact  may result  from  these  relations  and  induce  spatial,  social  or institutional changes.

\paragraph{ANT}
ANT approach facilitates logical argumentation for urban dynamics and enables mapping the urban development process and visualizing actors and networks through diagrams. In order to interpret urban development of Savamala, specific political, economic and cultural aspects are treated also as actors (social artefacts). Distribution of these networks are traced within the map through the identification of (1)  key actors involved, (2) levels of decision making it stems from, (3) sets of social aspects aggregated together. The key findings are articulated through a comprehensive description of on-site complexity and dynamics, which these conflictive political, economic and cultural aspects produce. In our approach we kept certain traditional concepts from urban theory and practice, but reinterpreted them in ANT logical framework. In this manner, we clarify what type of networks (urban assemblages) these conflictive social aspects address. Moreover, the users of this map (professionals) are able to indicate gaps (networks and actors) for possible operational interventions on the respective aspects. 
Urban reality and developmental circumstances in Savamala illustrated herein seem to represent bits of transitional chaos which post-socialist countries face. Not to mention that, capital cities are personification of such “development schizophrenia” (Vujosevic 2015) while being the focal political, economic and cultural nodes as a legacy of once centralized state. This exercise of visualizing through ANT methodological approach expresses an attempt to depict the complexity of urban actors, forces and artefacts to a legible scheme of links and nodes inspired by the similar endeavour from Marshall and Staeheli (Marshall and Staeheli 2015) . We recognize the quality of ANT scientific approach as an explanatory construct that studies associations and symmetrical relationality (Farias et al. 2009). Without addressing any particular state of affairs, this new perspective minimalizes the importance and influence of permanence of urban structures across time and space, and instead deals with a city as a contingent, fragmentary and heterogeneous, but persistent product of human/non-human actors, intermediary/mediator roles, concrete associations, stabilizing and destabilizing agencies and urban assemblages. We have shown that all these elements are spatially embedded and harmonized in particular physical set of Savamala. From our research results, we agree with the mentioned authors on the point that ANT interpretation may be a mere entry point or an operational agenda for further research, though we would like to argue that, on this pathway, ANT appear to have limited capacity for going any further from it has brought us by now – it appears as unable to receive practical recognition, influence the reality and go beyond identification of the obvious.
Bearing in mind that urban development of cities is has surpassed its perception as merely economic, social and cultural venues, the vitality of ANT approach lies in: (1) encompassing the active role of non-humans, (2) seeing the totality of the world as process, and (3) overreaching radical categories of time and space by representing horizontal links and associations. Although these premises grasp the core concept of urban dynamics, this methodology does not imply the capacity to deconstruct and interpret such complex aggregation of all real-life urban processes. In urban terms, ANT is, therefore, still perceived as a conceptual methodology whose integral approach works out only in confined urban environments, where it could comprise a dynamic, interactive process of interdependences and connections among all active urban actors and the formation of urban assemblages through roles, associations and agencies and their calibration within the chain of decision making.
In this respect we would like to recapitulate ANT setbacks to stand out as an overarching methodological approach for urban research:
•	ANT in its insistence on general symmetry fails to go beyond the description of the empirical reality of urban processes. Although it succeeds to include causal relations between actors, the future state of the system based on these relations stays undiscovered (Elder-Vass 2008). For example, we have identified how political, economic and cultural aspect signify and engage in several networks, but we do not have any insight on the valence of these aspects to engage in new networks, while this relation between what is and what will/would/could be is actually the essence of development.
•	Though ANT inaugurate flat ontology of the social (Latour 2005), networks are "narrated" by human constituents (Collins and Yearley 1992; Czarniawska 1997; Whittle and Spicer 2008;  Marshall and Staeheli 2015) and interpretations and translations are chiefly the product of the researcher’s positionality  (Rose 1997;  Ruming 2009). We acknowledge that our case study diagram reorder and multiply if we go from one actor’s to the other’s viewpoint or if we have it re-iterated by other researcher. Therefore, the credibility of ANT as a scientific method can be questioned as it should produce the same results regardless of iterations or agency.
•	Perceiving ANT results only as detailed empirical descriptions means discrediting “how and why” questions, leads to thinking that its sole aim is maintenance of the system   (Amsterdamska 1990; Lee and Brown 1994; Whittle and Spicer 2008), with no regard to prospects of its change or transformation(Gabriel and Jacobs 2008). Similarly, when we examine urban development diagram of Savamala neighbourhood, we see summarized the developmental flows of human collisions and coalitions, finances, practices, information and knowledge, but we do not have any tools at hand to point out where maintenance, transformation and change of the system happens. Finally, with ANT results we are incapable to intervene in an urban system – to articulate social practices, anticipate conflictual urban issues, and provide an overview of actions, solutions and changes - and we find these the pillars of urban development. 
In sum, we comply with the vision of others that ANT theoretical perspective has aspired to explain the totality of the world without relying on "other" frameworks (Lee and Brown 1994, Gad and Jensen 2010), but actually remains on the level of description that may appear insufficiently scientific for a methodological approach (Gabriel and Jacobs 2008). Our ANT diagram addresses complexity and provides framework for future extension of actors and new relations when they collide, overlap and interfere in networks. These networks represent system dynamics of cooperative, discontinuous, contradictory or even mutually exclusive relations among actors, whereby, in dynamic terms, the actors and networks are constantly changing and consequently constituting new realities. According to our interpretation, ANT scheme neither can tell us anything about this, nor can it indicate how the urban system maintain, transform or change itself. In this respect, even though we have widen the scope of ANT categorizations and interpretations and have dealt to a certain point with urban complexity, it still falls short to meet the expectations as a potential interpretive tool and urges for methodological revisions, adaptations or complements in order to facilitate an understanding of undercover processes and mechanisms or to provide explanations, recommendations or operational diagnosis on how to cope with developmental dynamics of maintenance, transformation and/or change of an urban system.


\section{Conclusions related to the theoretical framework}

Harrison 2002:
Extending the general analytic strategy across cases is referred to as replication (Yin, 1994)
Conclusions conflict with the literature. This is a challenge to
‘build internal validity, raise theoretical level and sharpen construct definitions’ (see Table 9.2).

\subsection{Urban Development Taxonomy}

•	An inclusive urban development model strengthens horizontal practices and initiatives, unlike the leading urban public institutions, which tend to verticalize this work to create cities. A participatory approach and the MAS-ANT discourse represent a relevant and up-to-date problem-solving strategy that encourages a top-down authority to horizontalize urban planning procedures, thus enabling all urban actors and stakeholders to intervene in their immediate surroundings. 

•	The dynamics of an urban development model addresses the current change in the urban planning paradigm (Rode 2006) towards an open-ended positive future concept with an emphasis on inclusive, transparent and flexible procedures. This urban development model, constructed on the principles of the MAS-ANT methodological approach, encompasses a range of urban codes and freely-available design rules which can be modified and adapted to local conditions and individual needs.

\hl{Vujosevic 2015 Regionalizam 2}
projekat - niz aktivnosti - jasno utvrdjen cilj - odredjen vremenski period - definisan budzet
politika - nacionalni prioritet, prioritet za akciju na najvisoj instanci (prim aut)
program - siroke oblasti zahtevanog rada na sprovodjenju/implementaciji odluka politike

\subsection{An Ordinary City}

In order to make transformations contextually appropriate and resistant to biased power relations and individual interests that thrive in transitional economies, it is important to continually keep track of wider social repercussions and assess risks of a range of “inter-states”, intersections of the timeline of development to indicate swift or biased socio-spatial changes in post-socialist neighbourhoods. The flexibility and the trial and error iterations of such urban transformations represent a catalyst for change and a means of seizing opportunities inside an urban environment, and converting these into development tools. Thus, such inclusive socio-spatial interventions strengthen horizontal practices and initiatives, unlike the leading urban public institutions, which tend to support vertical urban development decision-making. 

"to live is to leave traces" Walter Benjamin - testify the complexity of its life - for Belgrade it's its oriental past traces - dig for every city.

\hl{Samardzic in Doytchinov 2015} 
As in the rest of Yugoslavia the urbicide in  Belgrade,  affected  by  war  conflicts,  was  a  response  to  the  urban  culture generally, to the civic order and the value systems.

\hl{Lazarevic Bajec 2009}
understand the framework, presumptions and dynamic factors that influece its continual adaptation of UP in developed countries

\hl{Mornings after Nedovic Budic}
It is not the laws but their implementation that is based on flawed institutional processes and difficult societal circumstances.

\subsection{A Post-socialist City}

Yin
The significance of the case study selection
the case study must be significant: of general public interest, nationally important in policy and practical terms

An urban development model deals with deficiencies and incompatibilities between the reality of cities and the existing urban planning tools used. This is done by taking advantage of the incompleteness and the spontaneous urban development of environments in transitional economies: Creating the city through a myriad of strategies from above, interventions instigated by different interests and small changes from the ground up empowers the aforementioned incompleteness of cities, and this gives them their durability, flexibility and importance on the global scale. The case study of Savamala neighbourhood in Belgrade, Serbia within such a incompleteness and the on-going generation of urban conflicts and bottom-up interventions, offers the potential to be constantly remade through a summary of small movements and partial approaches released as an integrated system of urban development.

•	The case study of Savamala neighbourhood in the post-socialist city of Belgrade in the transitional country of Serbia is not only relevant for the Central and Eastern European region, but it is also applicable to other countries around the world that are undergoing a similar transition to a democratic and market-based system. Transitional economies that find themselves in an unclear social, economic and political situation are fertile ground for a fragmented, small-scale approach to urban conflicts, which could eventually produce more long-term and far-reaching results. However, research into the underlying forces that drive urban change has been limited notwithstanding the importance of cities in the overall process of economic and social transition. Building an urban development model in a post-socialist context of Savamala is a strategic basis for future interventions and didactic material for further education which surpasses the model of a post-socialist city in a transitional economy on which it has been built.

Having followed the aims and results of the bottom-up analysed activities in Savamala, we have identified the following capacities for Savamala to transform a crisis of aggregated urban conflicts into an opportunity for urban development: mobilise available local human resources, comply with current global trends in participatory urbanism, low-budget revitalisations and creative economy initiatives, educating apathetic local population on the importance of active participation in urban planning and development, critical attitude and “learning by doing” towards urban planning. It is also important to acknowledge that local citizens are not the main actors of these interventions. In this manner, the bottom-up  nature of the agency in Savamala is rather limited to the activation of alternative and non-institutionalized cultural scene with the focus on the whole city as well as the aggregation and multiplication of such NGOs in Savamala. However, the negative changes have taken place as well – the first intrinsically bottom-up organization in Savamala (“Club of Savamala funs and friends) having put in the middle of different agendas and interests ended up as a type of informal political body in party service.
In this extend, the livelihood of Savamala is still assumed to be at least disseminated from the ground up through the social bonds between different social groups (artists, youngsters, students, senior citizens) and among neighbours and locals, and achieved through the mutual efforts of participation and dialogue from these urban actors with different backgrounds. At some point, these internal relationships have surpassed all their campaigned and institutionalized initiators (UIB, Mikser festival), being followed by informal events such as: meetings of the locals in “Spanska Kuca” space, co-action of baking paprika, open access to spaces for artistic and educational purposes (KM8). Not to mention that, through several of the activities, variety of urban actors have become engaged in using these open public spaces (UIB, KM8, Mikser festival) and actively thinking and imagining what the positive future of these places might be. In this light, the major benefit that could transform socio-urban landscape of Serbian cities is the strong expression and statement of cultural and artistic interests within the agendas of these activities and raising awareness and promotion of participation in the urban domain.
Though it may also sound pretentious, intensive UIB media campaign  and the role of Goethe institute these activities have certainly paved such way for Savamala and have ensured a place for Savamala among European neighbourhood symbols of creative clusters and urban upgrade potentials  .  Respectively, it should be attentive to the possible negative effects of such trendy image that could lead to gentrification and expulsion of current population. Growing presence of Savamala in the media has also led to the exposure of its contextual resources to the powerful and uncompromising actors. Not to mention that instead of exploaring possibilities of the bottom-up, certain actors rather contributed to the commodification of culture and space and resorted to transnational companies   to support their activities.  HoweverIn sum, the lack of strategic development goals, public funding and institutionalized approach for cultural institutions and agendas certainly make these bottom-up activities seem ephemeral and sporadic. Not to mention that consequently they could be wiped off with any whim of more powerful interests and political influences focused on Savamala spatial capital.

\hl{Vujosevic 2015 Regionalizam u Srbiji 2}
upravno pravna nacela za novi sistem (Djuric) - p87-fusnota
    vertikalni odnosi - legislativa and koordinacija, respekt nizih instanci
    horizontalni odnosi - unutrasnje racionalizacije, depolitizacija, kontinuirana edukacija
    lokalna uprava and gradjani - pravila sluzbenickog ponasanja and etickog kodeksa
    
hl{ref Peric 2016}: planning was transformed into an instrument of the ruling political party elaborate
to do for planners:
     gain skills of negotiation, mediation and facilitation
    institutional capacity building of local governance
    
\hl{ETHZ 2012 Belgrade Formal Informal}
Belgrade is the link between the informal sector of the South and advanced deregulation of the North
illustrations/DIAGRAMS p84, 96,97

 	\hl{waves of planning 2006}
The  external  influences  from Europe  and  the  Middle  East  were  sometimes  self-inflicted  and  spontaneous;  sometimes imposed in combination with diplomatic efforts, religious and cultural ideas, demagogy, or political pressure; and sometimes invited and taken by voluntary action. While traditional settlements were more responsive to local circumstances, culturally grounded, environmentally sensitive and uniquely fit for the given socio-economic context, the imposed models and solutions provided for innovation and modernization efforts that were in step with the regional (e.g. European) trends. In borrowing, however, some of the local context was often overlooked in the excitement with the imported ideas and practices.

\subsubsection{Serbia - Belgrade - Savamala}
\hl{Vujosevic and Maricic 2012}
at large, the revival of urban planning and the boost of urban development depends on revival of strategic research, thinking and governance.
    
\subsection{Urbanity}

•	Building an urban development model starts out from the experiences of all urban actors and stakeholders in order to better evaluate the dynamics used in the creation and administration of their area of living and to practically apply theoretical knowledge and scientific solutions to improve life in cities. This method mobilizes the populations to form an integral part of all decision-making processes through the education and communication strategies built up on the basis of studying local cases that participants can relate to, and which serve as an experimental field. Furthermore, an inclusive urban development model in real urban environments activates unconscious creativity and it coordinates and articulates this process of collective planning procedures which depends on the individual's direct experience as well as on their sense of well-being.

\hl{Stanek 2014}
space is a physical residue of despised socio-economic and political system and it becomes principally its ideological monument

\section{Practical Implications}

The rise of the global economy is intrinsically anchored in the flow of information and the development of communication technology (Sassen, 2012). Furthermore, these factors have influenced the perception and constitution of reality, they have allowed accelerating process of globalization, a shift from traditional industrial activity to the dispersion of production, a transfer of products, hypermobility of capital and a redefinition of physical space (Firmino et al 2008). In this way, technological capacity has become one of the major premises for competitiveness at a global economic level  (Castells, 1998).
This techno-economic paradigm is a cluster of interrelated technical, organizational, and managerial innovations: First of all, technologies act on information, not just information to act on technology. Secondly, with information forming an integral part of all human activity today, all processes of individual and collective existence are directly shaped by the new technological medium. Finally, the networking logic of any modern system or set of relationships is grounded in these new information technologies where the user-friendly aspect of technological innovation has expanded the application of a user-oriented world wide web (Silva, 2010).
With this development and its global impact, information technology has become widely and deeply embedded in our daily life and much of the economic, social, political and cultural action shifts into cyberspace (Mitchell, 1996), in the form of a legitimate second reality where “single, integrated, unitary, material objects” have all been reconceived (Baudrillard, 1983)  and their interrelations revised in a new concept of “space of flows” or space of relations (Graham and Healey, 1999). Controversially, it does not make the actual places (urban spaces) redundant, but rather it initiates an active reconstruction of urban places (Graham and Marvin, 2001) as social constructs whose meaning depends on particular social contexts and their nodes of intersection (Healey, 2004).  This gradually changes the concept of ICT-oriented urban planning strategies to the ubiquitous city (Huang, 2012).
Cities tend to urbanize technologies semi-autonomously with increases in density and networked systems that the new technologies have made possible. Thus, it is necessary to shift the technological determinist concept to a more comprehensive, network-oriented vision that considers infrastructure networks, user networks, and their interfaces to generate ICT-oriented urban strategies (Huang, 2012).
Finally, when applied to transitional economies (which lag behind the successful western countries in ICT terms), this advanced technology inflow within a city is also conspicuous although the interchange between modern science and the social aspects is somewhat subtle. There emerges, however, a growing discrepancy between the dynamics of city growth and the weakness of technical supervision; not only in the course of human factors, but also in technological sources and solutions (Vauquelin, 2010). Nevertheless, as ICT means and instruments become an inherent part of modern societies in transitional economies, they can serve as an intermediary in the process of social practice actualisation within an urban environment.

\subsection{Urban Development model}

The potential lies in the dynamics of a modern scientific approach in research and how to implement it in urbanism; namely, how potent urbanism is to be semi-dynamically programmed considering, of course, a large background database. 
Corresponding to factorial analysis, defined pattern-models will be performed, so as to have visual representation of facts which are able to lead to urban progress in developing countries.
According to the analysis of characteristics mentioned above, crucial relations between terms could be established by an a posteriori approach to the vast range of collected data through an archival working process in order for their entailment, implication and structuralisaton.
What is here even more important is a multidimensional approach which should allow for the making of a graphical representation of these analyses which are by themselves complexes.

\section{Limitations of the research}

Harrison 2002:
trade off between knowledge and time
The data cannot be collected as originally planned. This may call for an
adjustment to a particular variable, or it may mean a major re-think.

Flyberg:
Misunderstanding 5: It is often difficult to summarize and develop general propositions and theories on the basis of specific case studies.

\section{Future Prospects}

The novelty of the flatten reality of human and non-human agents is not in their application in the context of urban development. Decomposing their operations, capabilities and interaction suggested an option for the future to be confronted with the facts, before imagining scenarios. The question of practical usefulness of MAS-ANT methodological approach therefore lies in its schematic interpretation of data for participatory planning and ground-up interventions. The far-reaching idea yet to be achieved may therefore be to prepare exhaustive categorization of elements and networks for possible digitalization/visualization of urban development process.

•	An appropriate technology for urban development in transitional economies is the one that carries wider social repercussions for such a specific scientific innovation (Bolay et al. 2011). The flexibility and the trial and error iterations of an urban development model represent a catalyst for change and a means of seizing opportunities inside an urban environment, and converting these into development tools. In this way, the urban planning procedure becomes an innovative, dynamic mechanism which mirrors simultaneous social and spatial circumstances in an urban context.

•	The transdiciplinarity of an inclusive urban development model is enabled through transversal collaboration among equally valuable fields and individuals where proposals and solutions are common property and responsibility and solving problems goes beyond the specialized field. Its effectiveness and driving potential lie in its capacity to exceed the abstractions of urban planning, the concrete specifics of urban design and the politicization of urban transformations and participatory processes,  bringing all of these together  into united solutions that work as bridges for realms of ideas.

•	An inclusive and dynamic urban development model fosters social inclusion and addresses informality and it also meets Millennium Goal (goal 7, target 11) prerogatives. In transitional countries, different marginalized groups and social groups that have resorted to informality constitute a significant percentage of the population. Belgrade, for example nowadays has between 1/3 and 1/2 of its housing stock informally constructed (Grubovic 2006) and Savamala, with its spatial and locational potential, is a breeding ground for real estate speculations. In this respect, if we take their needs, financial capacities and lifestyles into account as equal urban agents, this eventually leads to improvements in their living conditions. Therefore, an inclusive and dynamic urban development model is actually an articulation of human life in an urban realm which encourages active citizenship.

%%%%%%%%%%%%%%%%%%%%%%%%%%%%%%%%%%%%%%%%%%%%%%%%%%

\begin{small}
\addcontentsline{toc}{chapter}{Bibliography}
\bibliography{ThesisBib}
\end{small}

%%%%%%%%%%%%%%%%%%%%%%%%%%%%%%%%%%%%%%%%%%%%%%%%%%


\newpage
\appendix
\noappendicestocpagenum
\addappheadtotoc

\end{document}
