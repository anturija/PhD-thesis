
\documentclass[11pt]{report}
\usepackage[a4paper,margin=2cm, bindingoffset=2cm]{geometry}
\usepackage{appendix}
\usepackage{amsmath}
\usepackage{booktabs}
\usepackage{threeparttable}
\usepackage{natbib}
\usepackage{changes}
\usepackage{color,soul}
\bibliographystyle{chicago}

%to stop orphan lines
\widowpenalty=10000
\clubpenalty=10000
\raggedbottom

%line spacing
\linespread{1.3}

\begin{document}
\begin{titlepage}
\begin{center}
\large
\textsc{EPFL} \\
\ \\
\textsc{CODEV and LASUR} \\
\ \\
\textsc{EDAR}\\
\ \\
\ \\
\ \\
\ \\
\ \\
\ \\
\ \\
\Huge
\textbf{Urban Development Process}
\ \\
\ \\
\large by
\ \\
\ \\
Marija Cvetinovic
\vfill
Thesis for the degree of Doctor of Philosophy \\
\ \\
\ \\
XXX 2017
\end{center}
\end{titlepage}
\begin{titlepage}
\begin{center}
\ \\
\ \\
\ \\
\ \\
\ \\
\ \\
\ \\
\ \\
\ \\
\ \\
\ \\
\ \\
XXXX
\end{center}
\end{titlepage}

%Roman Page Numbering
\pagenumbering{roman}
\chapter*{{Abstract}\markboth{Acknowledgements}{Acknowledgements}}
\addcontentsline{toc}{chapter}{Abstract}
xxxxxxxxxxxxxxxxxxxxxxxxxxxxxxxxxxxxxxxxxxxxxxxxxxxxxxxxxxxxxxxxxxxxxxxxxxxxxxxxxxxxxxxxxxxxxxxxxxxxxxxxxxxxxxxxxxxxxxxxxxxxxxxxxxxxxx
xxxxxxxxxxxxxxxxxxxxxxxxxxxxxxx.\\


\tableofcontents
\listoffigures
\addcontentsline{toc}{chapter}{List of Figures}
\listoftables
\addcontentsline{toc}{chapter}{List of Tabes}

\chapter*{Declaration of Authorship}
\addcontentsline{toc}{chapter}{Declaration Of Authorship}
I, xxxx, declare that the thesis entitled xxxxxxxxxxxxxxxxxxx and the work presented in the thesis are both my own, and have been generated by me as the result of my own original research. I confirm that:
\begin{itemize}
\item this work was done wholly or mainly while in candidature for a research degree at this University;
\item where any part of this thesis has previously been submitted for a degree or any other qualification at this University or any other institution, this has been clearly stated;
\item where I have consulted the published work of others, this is always clearly attributed;
\item where I have quoted from the work of others, the source is always given. With the exception of such quotations, this thesis is entirely my own work;
\item I have acknowledged all main sources of help;
\item where the thesis is based on work done by myself jointly with others, I have made clear exactly what was done by others and what I have contributed myself;
\end{itemize}
\vspace{2cm}
Signed: \dotfill
\vspace{2cm}
\newline
\noindent
Date

%Include Acknowledgements in TOC
\chapter*{Acknowledgements}
\addcontentsline{toc}{chapter}{Acknowledgements}
This work was undertaken with financial support of the 

This thesis would not have been possible without the support of many people. I would like to express my sincere gratitude to:
\begin{itemize}
\item 
\item ...
\end{itemize}



%%%%%%%%%%%%%%%%%%%%%%%%%%%%%%%%%%%%%%%%%%%%%%%%%%

\chapter{Introduction}
%Arabic Page Numbering
\pagenumbering{arabic}

%%%%%%%%%%%%%%%%%%%%%%%%%%%%%%%%%%%%%%%%
Urban development is a widespread archetype for out-of-reach improvement in cities of the Global South. However, in its essence it is more a kind of constant catch up with the West and western urban paradigm than an elaborated form of intrinsic local perception, knowledge and action toward urban transitions. 
\\
In this highly competitive international arena, transitional countries experience grave consequences due to the paucity of practical experience within the dominant/ruling western ideology of the urban. They are caught in this new context of relentless rules of market economy, decentralized political and administrative powers, lack of resources, scarcity of general international investment and scant interest for dramatic shifts in all aspects of their social organization and spatial transformations. The blurred and askew morphology of post-socialist cities in transitional countries is therefore the result of continuous pressure from the negative side effects of imitating and lagging behind conventional urbanization models and accelerating globalization patterns imported or imposed by the Global North, or colloquially known as the West.
\\
The urban transformation of Serbian cities falls into this cliché of the new post-socialist urban reality, which emerged during the "transition to markets and democracy" (Tsenkova, 2006). The dismantling of the communist system during the late 1980s represented a substantial change in all aspects of social organization, the economic model and the political system. However, Serbia is still identified as a post-socialist melting pot where representative democracy, civil society and market economy principles collide and merge with authoritarianism, vertical decision making and populism practices. In such a situation, concern about the urban has been left out and given over as a battlefield for social needs in practice and technical solutions on paper and an easy prey to the exercise of power and interest. Therefore, the practice of planning and designing Serbian cities has been narrowed down to a mere technical issue most often even without an actual or adequate realization in practice. Not to mention that very few theoretical or general methodological research studies bothered to examine alternative planning modes, techniques and instruments in transition, but continued with the manner of replication from well-known counterparts of the Global North.
\\
Among others, architects have a vital role in not only directing but also framing the path of urban formation and development in post-socialist cities. Even more so, as they are primarily focused on practice and "savoir faire" about making the built environment, while acquisition of land and illegal construction are spatial interventions that have marked post-socialist production of space more than any planning or theoretical activity. Though we have different drivers on the global scale and in developed countries, there is a global trend of resorting to sociological, planning or even IC approach in scientific studies on the urban. 
In order for architectural intervention in space to compete for more relevancy and rigour,  architects all over the world have been gradually grown interest for scientific discourse on built structures, spaces and cities in general.  In the rivalry between spatial and social basis for their interpretations, the fact that the field of architectural research is not yet standardized in terms of methodologies and techniques opens the floor for experiments and innovations \hl{(ref)}. In the circumstances of developmental bouillon or "developmental schizophrenia"(\hl{Vujosevic}) at the local level and an overall urge for architectural research framework internationally, my aim is to elaborate an architectural standpoint when applied on complex post-socialist urban reality in order to establish a methodological approach suitable for architectural scientific discourse on the matter.
\\
In my striving to contribute to post-socialist architectural research, the far-reaching aim is to capture post-socialist urban dynamics  in order to skip the classical procedure of urban development based on western planning paradigm and provide its practical application on multiple levels of urban decision making. This to be achieved requires supple methodological approaches which should better correspond to post-socialist socio-spatial patterns on multiple levels (state, city, municipality, community, and neighbourhood) and explain the correlations of various urban elements. Practice-oriented, locally focused and globally tuned  approach to complex urban reality of post-socialist cities envisions embracing the dynamics of urban systems and operationality of architectural performance for circumscribing visual interpretations that enable continuous conclusion drawing and up-to-date introduction of any new element that may appear in the system.
\\
This chapter, first of all, immerses into the contextual, scientific and disciplinary discourse of the following research. It marks the research context, historical and scientific, and puts a spotlight on the importance of the research problem as well as of the purpose and adequacy of this thesis. Then, I outlines my research drive in the mentioned scope and present in a nutshell what this research is about, how it will be performed and what are the research expectations and practical results I plan to meet. 

\section{Field of Study}
Due to growing social and physical transformations that become ever more intensified as current globalization continues to spread out profit maximization, consumption patterns and information networks (Harvey 2012), the cities have been experiencing a progressive reorganization at spatial and social levels. Even though accelerating urbanisation is a worldwide process, it still assumes different forms and meanings, depending on the prevailing local conditions (Bolay, 2007). These overall circumstances of continuous urban development influence cities to serve as the primary channel linking local realities to global social, political and economic forces (Yates and Cheng, 2002; Tsenkova, 2006).
\\
Cities are not simply market products and consumption patterns, but locally customized socio-political constructs as well (Marcuse et al. 2008). These external influences form a range of qualitatively different contextual circumstances for positive urban change, i.e. urban development. Most settlements and cities from previous historical periods had reflected upon the various degrees of forethought and conscious design in their layout and function, which is referred to as a fixedly planned development, albeit many had tended to develop organically. Generally speaking, a range of urban disciplines (urban planning, theory, sociology, legislation and design) aim to decode and harmonize growing urban issues as a side-effect of the current globalization, urbanisation processes and spread of \hl{capitalism} that are mainly affecting cities and production of urban space and bid for the expertise on managing urban development (Allmendinger, 2009; Faludi, 1973). 
\\
In practice, these disciplines are embedded in a particular social context or a territorially based system of socio-economic relations. They react to the shifts in socio-economic and political settings (Tsenkova, 2006b), but have kept privileged relationship toward Western cities, which assumed to be the sources of urban creativity, vitality and innovativeness in urban domain (\hl{Robinson, 2006:2}). Accordingly, they tend to fail substantially within the range of spatially and economically different environments that have undergone highly dramatic change in political, economic and social terms. For example, urban research and practice in transitional countries in Central and Eastern Europe (CEE) should unfold to help understand these phenomena in their immediate and wider context. The crucial is to identify patterns of the dynamic reality in these cities and be more consistent with spontaneous, everyday urban transitions. Furthermore, a corresponding change in approaching urban development can then be addressed by heterogeneous, iterative and generative process of urban space production in physical and social sense that has surpassed the perception of cities as merely economic, social and cultural venues treating them as complex and dynamic urban systems. In these circumstances it is necessary to apply proper techniques and methodologies for urban research and analyses which encompass complexity and dynamics of cities for the improvement of their living conditions and the facilitation of social interactions in the process of urban development.
\\
However, each discipline keeps its own track and pace in approaching urban matters. My architectural background has moulded my own research interest towards gaining knowledge and understanding on management of space and built environment. Moreover, production of space is also the core concern for architects. Architecture is a discipline focused on practice and consequently it urges for parameters, categories and structure for its practice-based analyses. Hitherto history and theory of architecture have been the main fields of architectural research. But the ever growing relevance of architecture for transforming the general body of knowledge on cities into a real-life problem-solving strategy that address human lifestyles, social relations and the concept of space \hl{other ref than initial Castells 2000, Dijk 2002 that is addressing ICT}, insists on the advancement of architecture in terms of the relevance and reliability of the knowledge herein produced (\hl{ref}). However, missing links with the classical scientific discourse has caused a growing concern for what is research appropriate for architectural and design practice as well as for architectural stance in urban studies, especially in terms of methodologies, methods, approaches, domain and credibility (\hl{ref+Savic 2016b}). Lacking the traditional scope of analysis, architectural research has been a polygon for innovations and experiments.
\\
In terms of methods, there has been a significant number of interdisciplinary, transdisciplinary and multidisciplinary endeavours in applied research with an architectural focus in urbanism (ref). What is more, applied fields of research acknowledge the use of methodological hybrids (Datta, 1994, De Lisle 2011). This has open doors for applied social sciences to investigate new methodological opportunities when confronted with complex and multiplex social phenomena (De Lisle 2011). Even more so as methodological and epistemological rigidity leads to ignoring the realities of the practical and cause catastrophic scientific failure of practice-oriented research (Rogers 2008, De Lisle 2011).
\\
What I have recognized as a crucial change in the methodological paradigm of an applicable urban research from the architectural standpoint can then be circumscribed to the rise of the global concept from static to iterative and dynamic. Commonly speaking a static world is one in which all transitions are according to a known law and which do not give rise to uncertainty. When defining the evolution of analyzing and simulating an urban phenomenon or process, it is fundamental to state that the existence of a problem depends on the future being different from the past, while the paradigmatic possibility of finding the solution to the problem depends on the future being like the past. Therefore, a transition in some sense is a condition of the existence of any problem. The complex empirical realities of urban transitions collide therein with the powerful and dominant policy of continuous comprehensive production of knowledge. A scientific approach towards formulating the dynamics of urban transitions have to count on uncertainty as one of its fundamental facts and in this way accept and deal with an open-ended future and the limits of human knowledge about it.
\\
Gaining knowledge has come to be a strategic activity rather than a search for truth (Kirby 2013), so that science becomes incapable of controlling society and the rationalized reality appears false and irrelevant (Alfasi et al., 2004). Given these conditions, the growing gap between the formal structure and the dynamics that take place in cities triggers an internal and independent process by which the system tends to spontaneously self-organize (Portugali 2011). Therefore, a city should be conceived as an organism, not a mechanism \hl{Charles Laundry, The Creativity City}. In these terms, the city is interpreted as a living system which is constantly mutating and emitting new elements, a container for processes of coming to be, breaking up and falling out, fragmenting and recomposing. Contemporary cities tend to be concentrations of multiple socio-spatial circuits, diverse cultural hybrids, and sources of economic dynamism - a venue where the past and the present converge upon one another. The city tells a story of one society and its attempts to move towards a positive vision of the future, through complex ranges of processes that flow together to construct a single consistent, coherent, albeit uncertain, interactive and multifaceted time-space system (Graham, 1998). These ceaseless processes are the core of spontaneous, everyday of urban development. Grasping the scope of urban development occurs as a major challenge for modern science about cities.
\\
My intention is not to produce another pattern applicable to certain cities to a certain extent, but rather to apprehend a process that embodies the complexity and dynamics of the mentioned relations in a transparent way.  This framework of research enables to ponder upon means of generating a vibrant and fluid context open to permanent transformations and, most importantly, to grasp the idea of an adjusted and balanced model, adaptive to changing views and situations on accelerating urban development. (Portugali Complexity cognition and the city). This to be achieved requires supple approaches which should aim at explaining the correlations of various urban elements and to better correspond to the socio-spatial patterns of the range of urban environments. In such a plenitude of factors, I have chosen case study as an adequate research method and a neighbourhood as a relevant level of analysis.
\\
Dynamic urban context is a complex phenomenon with a plenitude of data. Case study research method enables close, in-depth and holistic examination of a great deal of data, but requires a bounded environment in order to accurately describe and illustrate such a context and to use it for broader interpretations and demystification of modern cities. Specifying physical limits is not in itself enough for circumscribing the identified complexity of urban transitions, the issue of scale is also at stake. In urban terms, different spatial and social elements are intensified or muted at different levels (global, national, regional, local). In order to acquire active follow-up, interpretation and assessment of urban issues, it is important to define a representative environment, a robust source of prominent urban "processes". Thereupon, I argue for a neighbourhood level of analysis, because it may become a paradigm for complexity and dynamics of modern urban context. It serves as an urban micro environment, which eventually increases the body of knowledge on cities concerning the methodologies used to deal with urban development and corresponding urban transitions.
\\
\textbf{"The contemporary city is a variegated and multiplex entity - a juxtaposition of contradictions and diversities, the theatre of life itself" (Amin and Graham, 1997).}

\subsection{Background}
At the beginning of the 21st century, the world experienced a progressive reorganization at an economic, political and social level: profit maximization, globalization of urban processes and the devastating history of deindustrialization (Harvey 2012) and dematerialization of the world. Nowadays, while about 50 percent of the world population lives in urban environments (United Nations 2008), the question of techniques and methodologies for urban development research and analyses should undoubtedly address these major shifts in urban life and contemporary cities (Healey 1997). Cities are rather primary venues, power poles and capacity builders of economic, social and cultural development at stake in modern societies (Castells, 1998). Conversely, cities are dynamic and diverse urban entities that are given to shaping their autonomous and innovative future on the basis of human resources and creative human potential (Knight and Gappert, 1989; Yigitcanlar, 2008). The prosperity of cities depends on how competitive they are on a global economic scale, how flexible they are in terms of adjusting to current trends and needs, and how fertile they are for the development of knowledge and the application of innovation. These major uncertainties of contemporary life, created mainly but not exclusively by the current method of production and management, are acutely symbolized by concerns about urban development (Healey 1997).
\\
Urban development is widely accepted even though also contested category usually associated with urbanisation processes in "so called" developing countries. Lots of professionals in urban research and practice use the term, not to mention the great number of people around the world affected by their work. Nevertheless, the notion of the word "development" itself means different things to most of them.
However, traditional and widespread interpretation comply with the western paradigm of development: modernisation and economic growth \hl{ref}. Interpreted in this way, the notion of urban development actually promotes the leading hierarchies and categorization of cities in the world, based on the impact of globalization, new transnational  economic  progress  and  networking  of  cities \hl{ref}. Both of these approaches impose the hierarchical relation among them, as Jennifer Robinson (2006) bluntly puts it, "while some are exemplars and others are imitators". Besides, the chronological paradigm of western urban planning dilutes when it is spatially translated to these qualitatively different environments (Robinson 2002), causing them to lose their substance as an urban phenomenon through the ill-decoded application of western patterns (Bolay 2004). In addition, modern urban thought could be stuck in this rut, inducing negative background effects on the whole gamut of urban activities (Amin et al. 1997) and causing urban conflicts to thrive on the basis of inequitable power relationships, and cultural differences, as they develop from an individual level towards a socio-urban dimension (UN Habitat 2009).
\\
In  this  respect,  Jennifer  Robinson  summarizes  that  categorization  and  differentiation  of  cities, according to Modernity and Development, are a pure product of colonial past. This actually means that in the scope of widely praised universal image of "cityness" as the final goal of the ambition of cities, successful examples of cities are included inside these categories. \underline{World cities} are defined in relation to their regional, national and international influence inside a global economy where the country categorization is transferred into this world city categorization \hl{ref}. Conversely, \underline{global cities} are categorized according to their industrial potential of transnational management and control \hl{ref}. They both focus on the characteristics of cities and potentials in the scope of global economy, its flows and networks, which has proved to be insufficient, exclusive and restrictive for cities in less developed countries, if we keep up with the same terminology at the national level. On the other hand, cities which are outside these categorizations but with a same ambition and vision of "cityness" are regarded as \underline{third World} or \underline{developing cities}. Consequently, there are even more categorizations such as those of western, wealthy, third world, developed and developing cities. However, with all of them together, there is still a vast number of cities which are left out and with barely any possibility to ever fit in any of the categories (Robinson, 2006). 
\\
\textbf{"Ordinary cities also emerge from a post-colonial critique of urban studies and signal a new era for urban studies research characterised by a more cosmopolitan approach to uderstanding cityness and city futures. This can underpin a field of study that encompasses all cities and that distributes the difference amongst cities as diversity rather than as hierarchical categories. It is the ordinary city, then, that comes into view within a postcolonialised urban studies" (Robinson, 2006)}.
\\
This brilliant insight put forward by Jennifer Robinson in her book "Ordinary cities between modernity and development" questions the geopolitics of urban theory and urban development (Fraser 2006). Taken from this standpoint, each and every city is an indicator of what an urbanized society is and what course of urban development it may take. My research scope in this thesis perseveres with post-colonial critique of urban studies and the notion of ordinary cities, introduced by Amin and Graham (1997) and further developed by Robinson (2002). This concept approaches the knowledge of diversity and complexity that exists within the world and "distributes the differences amongst cities as diversity rather than as hierarchical category" Robinson (2002). 
\\
Ordinary cities approach provides unique assemblage of internally different, distinctive and \added{context-based} urban transitions as well as overlapping space-, time- and relation- networks across cities. In other words, it is not only necessary to examine the ways in which countries/cities interface with the global economy, but also social, cultural and historical legacies that each country/city carries into the era of globalization. Within such domain for explanations, this thesis revolves around the interpretation of urban development as an answer to the question "how can cities facilitate urban transitions while also maintaining the culture and values of the community itself?" \hl{(ref article: Does Placemaking Cause Gentrification? It’s Complicated.)} The idea of indicating what encompasses urban development of an ordinary city lead to identifying its internal and external influences \added{that constitute the core of maintenance, transformation or change processes in an urban system of a city}, when treated equally within the global hierarchy of cities (Robinson 2002). 
\\
This approach makes a worldwide, broad, general and mutable process of urban development actually connected to place - making an actual urban setting a vital factor for case specific uncertainties and a polygon for transformation of global aspects to meet local specifications. My aim is to move away from the general theoretical research into an on-site practice-based investigation. Consequently, this research project attempts to show how the real-life focus on Savamala neighbourhood in Belgrade eventually increases the body of knowledge on post-socialist urban environment and the methods used to deal with complex and dynamic urban context. Complying with "ordinary cities" approach, I would like to elaborate that post-socialist cities in transitional countries meet extraordinary difficulties when copying urban models from the West.  The cause is found in the lack of the institutional infrastructure and cultural patterns essential for the functional unity present in western cities (Petrovic 2009). Furthermore, fundamentality and intensity of economic and political change in Balkan post-socialist countries may be a historic exemplary of social transition hard to find in a "typical" capitalist city (Sykora 1994). Its internal environment is in a state of flux, with the rapid adjustment of the physical, economic, social, and political structures of the city itself (Sykola, 1999).
\\
Included in this range of spatially and economically turbulent surroundings, post-socialist cities in transitional countries have undergone highly dramatic change in political, economic and social terms. The mayor consequences of such \deleted{post-socialist} transition introduce, on one hand, the disastrous effects of increasing social polarization (inequity), deinstitutionalization of socio-spatial practices (informality) and unfair wealth redistribution (poverty), and, on the other, the huge socio-cultural base inherited from the socialist period with centralised and authoritarian practices in urban governance. This has had a profound influence on the spatial adaptation and social repositioning of post-socialist cities.
\\
Turbulent social times, such as the disintegration of Yugoslavia’s political system and the introduction of new context of market economy, decentralized administrative powers and a lack of investment and resources are reflected in chaotic urban development pattern. Urban systems of post-socialist cities are highly susceptible to tense on-going transformations, diverse but reciprocal in their nature: economic transformations (transformation of production and consumption in relation to space, income polarization and poverty), political transformations (urban governance, political voluntarism, participation and decentralization), spatial transformations (demographic trend and distribution of functions) and social transformations (social exclusion-inclusion, social activism and informality). In other words, what proceeded after the end of the socialist era is a neoliberal model of urban planning with the supremacy of market-oriented solutions for urban problems (Sager 2011). Conversely, with the huge socio-cultural base inherited from the socialist period, cities in transitional countries have continued to be centres of economic growth with a variety of services, expansion, technological innovation and cultural diversity. While some trends and directions within these transformations are clear and defined, uncertainty dominates decision making and implementation in the turbulent environment of post-socialist cities (Nedovic-Budic 2001). Therefore, the post-socialist period in these cities contains prevailing characteristics of the disintegration of the preceding system rather than a coherent vision of what should follow. 
\\
In practice these conditions ended by having the strategic plan as an advisory long-term urban vision, but leaving the real actions and decision making to political and market forces. Thenceforth, urban development of post-socialist cities most often has exceeded and diluted the common strategic framework defined from top-down: to establish clear links between the process of strategy development, its institutional framework, the hierarchical structure of long-term and short-term objectives of all actors involved, and the real-time changes happening simultaneously in an urban environment. The major characteristics of post-socialist urban development are: a multitude of actors, various economic, social and political interests, social aspects and fragmentation of urban spaces. Consequently, post-socialist cities lack complex, operational logistics \hl{check if appropriate(Repetti et al. 2010)} to link top-down changes to bottom-up interventions in urban systems. There exists an growing discrepancy between the national and global levels, on one side, and city and neighbourhood levels, on the other. 
\\
The  conceptual  framework  explained  herein  pinpoints  the  blurred  and  askew  morphology of  post-socialist  cities which  requires  dynamic  solutions  in  order  to  skip  the  classical, western procedure  of  urban  formation  and development. Consequently, this particular context shows the increased need for proper techniques that are spatially and temporally adjusted to current issues. The far-reaching goal actually is to transform the negative side effects of imitating and lagging behind the western urbanization model and those of the accelerating globalization into a development impetus suited to these environments. Urban development of post-socialist cities is perceived as a dynamic concept, a multi-dimensional integrated system composed of qualitatively different and semi-autonomous processes, with the inclining tendency to improve the economic, social, demographic, political and technological state of an urban environment. In view of all this, we need an overarching theory of urban development that can encompass all discrepant decision making forces: future-oriented urban projections (urban planning strategies), in situ transformation forces and potentials (urban transformations), and follow the creative paths of urban dwellers (participatory urban design activities) for imagining new urban futures. The question of facilitating and localizing urban transitions rests with overlapping urban scenarios from dissonant levels of decision making, tracking cultural identities, requirements and needs of all urban actors, and, in general, indicating contextual processes of maintenance, transformation and change of an urban system. 

\subsection{Problem statement}
The focus of this thesis is urban dynamics of post-socialist cities. The issue is not addressed as a problem to solve, but rather as a moving target for an exploratory observation of the way how cities function and how various urban transitions condition urban development of post-socialist cities.
\\
Post-socialist cities are treated herein as a range of qualitatively distinctive cities that "deal differently with their difference" \hl{ref}. In their incompleteness, plurality and informality, post-socialist cities in transitional countries represent dynamic and diverse arenas of contemporary urban life, experience and theory. Included in this range of spatially and economically developing surroundings, transitional countries in Central and Eastern Europe (CEE) have undergone highly dramatic change in political, economic and social terms. The disintegration of Yugoslavia’s socialist system led to the destabilization of the institutions and the social value system in Serbia. Such confusing political and social circumstances have deprived an average citizen of sufficient information about the possibilities and tools to take an active part in the development of their city. These factors provoked a legal void susceptible to shady deals and questionable public-private partnership (illegality), a lack of strategically proactive urban governance which resulted in tolerance to illegal building practices (informality), and the increasing social polarization (inequity) and poverty in this region {the number of poor people had reached 100 million in CEE by 2001 (Tsenkova 2006a)}. This rather organic path of urban development leads to the classifying of post-socialist cities in transitional countries as unregulated capitalist cities (investment-led) with third world urban development elements (substantial illegal activities and informal markets) (Petrovic 2009).
\\
Conditioned by the geographic location of Serbia (CEE), murky circumstances of transition (towards liberal market, private property, profit motive and consumer sovereignty) are followed by a set of decentralization and democratization protocols for joining EU, availability of European research and civil sector funds as well as the promotion of participation and engagement from the ground up \hl{ref}. Having said that, the lack of successful urban planning models and actions make possible that the rising economy of social exchange and local capacity building could contribute to an improvement of life and functionality of urban structures and systems and effectively address the tensions between top-down and bottom-up urban planning in a post-socialist city. Tracing institutional articulation of post-socialist context involves structural analysis of administrative procedures and content analysis of policy agendas. It serves to systematically deconstruct local urban governance in terms of political, economic and cultural aspects of transition with a multitude of actors, variety of interests, conflicted strategies and fragmented implementation. In the long run, the identification of relations and influences on post-socialist urban governance examines how urban actors, space and regulatory framework rely on planning and decision support systems as means to forecast and orchestrate any movement or change of the system. In this manner, any element of urban systems, human or not, is attributed agency.
\\
Conversely, under the hood of scientific neutrality, urban development concept is critically approached, broken down and recomposed as a process of urban transitions, not as an indicator or the final product in urban practice. Urban development of post-socialist cities is seen as a complex, multifaceted network of urban transitions that evidence: 
\begin{enumerate}
\item the level of urbanity - qualitative processes of maintenance, transformation and changing processes of an urban system;
\item legitimacy of different layers of urban decision making top-down urban planning strategies, tactical urban transformations, and bottom-up participatory activities;
\item urban key agents - constitutive elements for the morphology of urban decision making;
\item numerous urban conflicts, social practices and contextual resources resulted from the incompleteness, plurality and informality of post-socialist cities.
\end{enumerate}  

\section{Thesis Aims and Scope}
\textbf{research scope:} grasp the actual urban development process in cities
\\
\textbf{research impact:} the quality of an urban system generates a vibrant and fluid context open to permanent transitions gives rise to potential to originate diverse opportunities for new rounds of exchanges among research, innovation, action and development (Bolay et al. 2011).

\subsection{Research Objectives}
\textbf{Overall objective:} encompassing complexity and dynamics of urban transitions as an urban development indicator at the local level in a rather transparent way
\\
\deleted{identify variables in objectives}

\subsubsection{RO1}
re-formulate urban development concept in terms of urban transitions to fit the idea of dynamic state of an ordinary city in its full complexity
\begin{itemize}
\item RO1a: identify what urban system complexity is \deleted{and sort out active urban key agents and contextual resources and map their interconnections and networks} 
\item RO1b: map \deleted{how socio-spatial patterns of an ordinary city are constituted in} urban networks
\item RO1c: trace the morphology of decision making 
\item RO1d: define dynamic state of a complex urban system in an ordinary city - description of empirical reality of urban networks and processes
\end{itemize}

\subsubsection{RO2}
gain an in-depth understanding of the level of urbanity in an ordinary city as an indicator of \deleted{urban development/urban transitions/}urban dynamics
\begin{itemize}
\item RO2a: elaborate the level of urbanity \deleted{connect the level of urbanity to urban dynamics socio-spatial patterns in an ordinary city}
\item RO2b: \deleted{elaborate and encode socio-spatial patterns (spatial, social and technical differences and specificity) of post-socialist context - local urban conflicts, social practices and contextual resources} \added{connect the level of urbanity to urban dynamics}
\item \deleted{RO2c: connect urban transition processes to socio-spatial patterns}
\item RO2d: contextualize the level of urbanity categories \deleted{urban transitions} in post-socialist cities 
\end{itemize}

\subsubsection{RO3}
conceptualize a methodological hybrid for tracing urban complexity and dynamics
\begin{itemize}
\item RO3a: specify a neighbourhood level of analysis 
\item RO3b: describe urban complexity - networks - to indicate the morphology of urban decision making
\item RO3c: trace the level of urbanity - urban processes - to indicate urban dynamic
\item RO3d: proceduralize urban transitions for circumscribing urban development process
\end{itemize}

\subsection{Overall Research Question}

\textbf{Overall research question:} HOW To investigate \deleted{socio-spatial patterns of} post-socialist cities in order to reinvent a more inclusive and flexible approach to understanding Urban development dynamics engaging the complexity of an urban context? 

\subsection{Main Concepts}
The city is regarded as a geographically condensed, highly structured economic, and the most complex social phenomenon (Mumford 1961). "Time" and "social interactions", in the modern qualitative sense of the term, are now the leading determinant for the way urban systems function \hl{(ref SNF1)}. Urban structures interact in an environment that is constantly undergoing transitions, as they themselves are not permanent and unchangeable. As a result, this constantly influences and changes our point of view, influencing our way of solving problems that exist in our environment, as we and all of our surroundings are in a constant state of flux (Harvey 2003). This sort of relativism, where the interactions of as many elements as they emerge determine the context in which they are placed, should be a formative factor in addressing urban complexity and dynamics in terms of urban development prospects and circumstances. The theoretical stronghold of this thesis is the interpretation of urban development, namely going away from qualitative notion of the term and indicate its operational equivalence with more neutral and relativistic idea of urban transitions. Urban transitions encompass complexity and dynamics of an urban environment within the combination of urbanity (description of the state of an urban system and the agency of its transitions) and urban decision making (sorting urban elements and transitions according to the layers of interventions - planning, investment-based transformations, and participatory activities).
\\ 
\textbf{Urban development} is rather a generic term for circumscribing the progress of and in cities addressed in the \added{blurred} field of practice-oriented research \hl{World Bank ref}. Nowadays, when cities are primary venues, power poles and capacity builders (Castells, 1998), the theorem that the growth of cities expand opportunities seems to hold up. Moreover, urban development concept has been easily mixed up with urbanization and economic growth and more often ruled out by the appealing righteousness of sustainable development trends \hl{ref}.  In this sense, urban development has been either patterned or predicted referring to whether it is the part of a model or a project for a city or an urban environment. However, in both cases it implies change. The programmed change is usually assumed positive in its intention or marked as developmental if it has positive economic or less often social outcomes \hl{ref}. In reality urban change is most often the consequence of power struggle and has conflictive outcomes on different stakeholder groups \hl{Fainstein 2010 and else ref}. Yet it has been bounded only spatially - referring to a city or a part of the city. Not to mention that today's solution may be the conflict of tomorrow \hl{Holden 2015, ref}.
Therefore, this thesis approaches urban development concept in relativistic terms\hl{(explain time-space concept in a footnote} where urban development is applied as an overarching codifier for urban complexity bounded rather as a comprehensive overlay for urban dynamics, not as its qualitative, prognostic nor delineative indicator. In other words,  urban development is circumscribed herein by a set of premises as follows: 
\begin{itemize}
\item Urban development is treated as a process of urban transitions over time;
\item Urban transitions indicate every socio-spatial reference that affects an urban system;
\item Urban transition is the consequence of urban decision making;
\item Urban transition affect the range of human and non-human elements of an urban environment;
\end {itemize}
This interpretation of urban development not only emphasizes its processual nature, but also moves away from its project- or model-based feature by incorporating locally contingent socio-spatial patterns \hl{Guy and Henneberry 2000} and non-human basis of urban agency\hl{Healey 1991 add others}. The units of analysis are temporarily and spatially bounded urban systems, either whole cities or its \added{conventional parts} \hl{ref}. \textbf{Socio-spatial patterns of urban transitions} is a provisory term  that contributes to develop an understanding of development processes beyond mere strategic economic and social framing of needs and events and taking into account sporadic and spontaneous agencies of urban systems. The sensitivity to this range of needs, events and agencies means that whatever happens refer to the state of an urban system - the processes of maintenance, transformation and/or change which we define as urban transitions \hl{ref}. Accordingly, the complexity of an urban system, that involves the unpredictable and uncertain in its structure is bridged by emphasizing the reference to its state and corresponding urban dynamics. This approach indicates political aspect of urban processes not that of urban structures. Moreover, it coincides with the political view of urban planning \hl{ref}, though it takes a more inclusive turn with all the agents of interventions, relations and events are taken into account not withstanding their nature, function or purpose. In other words, urban development becomes reconfigured to a fine-grained urban dynamics, adding up elements to the battlefield of urban decision making, but enables labeling the complexity of urban systems.
\\
In general, the main research challenge of this thesis is testing the legitimacy of urban decision making for urban development. The issue at stake is to encompasses planning, interests, design and participation in an overarching urban decision making procedure. Namely, the source of urban transitions are decisions made through these various top down, bottom up, interest-based interventions, relations and events. Political and governance practices are open and susceptible to choice, through contestation and struggle, and accident, historical or natural, "but decisions become locked in" and instigate urban transitions - maintenance, transformation or change of the current state of an urban system (\hl{Hudson and Leftwich 2014}). \textbf{The morphology or urban decision making} therefore comprises and reconciles all its different layers that spread urban transitions through and across an urban system and engages certain level of forethought. These layers are: top-down urban planning strategies, tactical urban transformations, and bottom-up participatory activities recognized on site \hl{ref}. They serve to enclose the historical continuum of global urban trends and patterns in a local socio-spatial framework and translate them into an internal, on-going interaction of individuals or constituted groups.
\\
Identified overarching decision making procedure acknowledges human agency, urban actors who, through these interactions, initiate the process of their integration into the environment through an appropriation and transformation of space. In this sense, we could refer to the classical vision upon cities as a setting that consists of: venues (their spatial and built environment) for social interactions (economic, political and cultural), social practices (policies and processes) and reproduction of social order of all urban actors (Firmino et al. 2008). The way cities function shapes the expectations and actions of all the urban actors involved, who also influence the constitution of the city itself. The network of these internal and external influences between human and non-human elements engaged in urban transitions introduces urban agency as a property of all urban key elements. Henceforth, people (urban actors and stakeholders), objects (built environment), territories (space), institutions (regulatory framework), infrastructure and social  aspects (political, economic and cultural circumstances) are all correlated through the morphology of urban decision making. They are also granted agency in urban transition where they figure as \textbf{urban key agents} \hl{(Firmino et al., 2008) and ref}. This multitude and diversity of elements is an urban system and, while emboding its dynamic state, it is rather blackboxing the agency of urban dynamics than decoding it.
\\
\textbf{Urbanity} is another rather blurry concept, applied often in architectural research and practice with the potential for decoding urban dynamics. In general terms, it relies on urban complexity as an active attribute of the overall state of an urban environment (Canuto et al. 2012). Cities are simultaneously the source of both problems and solutions of contemporary life. Cities are the polygon of contemporary decision making. Socio-spatial patterns of urban transitions bend the way how decision making layers address urbanity as its constitutive reality and its ultimate positive goal \hl{ref}. The conceptual framework of urbanity examines the urban key agents, numerous urban conflicts, social practices and contextual resources and how - in their incompleteness, plurality and informality - they form urban transitions.
\\
Moreover, this thesis argues that an overarching definition of urbanity concept improve scientific capacity for grasping urban dynamics. It elaborates how the level of urbanity figures as an indicator for contextual processes of maintenance, transformation and change of an urban system, incorporating simultaneously its state and the transitions. The relationship between the physicality of urban form and the social components of urban life generates the level of urbanity - the quality of continuous harmonization of the variety of structural elements, social factors and vested interests existing in an urban environment (Holanda 2002, Canuto et al. 2012). Moreover, all these urban key elements are assumed to be equal agents in the continuous process of urban development that has been marked by maintenance, transformation and change of the urban system in order to improve its living conditions and facilitate social interactions.

\subsection{Methodological justification}

Following contemporary relativist trends for rethinking space, time, globalization and cities, future research challenge can be defined as "visualizing cities as unformed, unorganized, non-stratified, always in the process of formation and deformation, eluding fixed categories, transient nomad space-time that does not dissect the city into either segments and ‘things’ or structures and processes" (Smith 2003:574). Accordingly, a corresponding change in approaching urban development can then be addressed by heterogeneous and iterative approach that has surpassed the perception of cities as merely economic, social and cultural venues treating them as complex and dynamic urban systems. In these circumstances it is necessary to apply proper techniques and methodologies for urban research and analyses which encompass complexity and dynamics of cities for the improvement of their living conditions and the facilitation of social interactions in the process of urban development. 
\\
Bearing in mind the complexity of such a relativist approach to the urban and the necessity of practice-oriented knowledge, I propose a mixed-method case-study approach \hl{check ref (Flyvbjerg et al, 2012)}. According to Kuhn's paradigm shift (1962) science about the city is constantly swinging as a pendulum between scientific and hermeneutics approach - quantitative analysis vs. descriptive study \hl{Portugali Complexity Cognition and the City}. Mixed research method in this case provides complementary information and in-depth knowledge of the problem. However, it has been solely moulded according to qualitative data sets. This research is influenced by the choice of an innovative methodological approach,  but  the  set of qualitative techniques  and  their  sequence  are  guided by the requirements of the research problem \hl{check ref(Flyvbjerg, 2004; Aitken, 2010)}.
\\
The choice of the methodologies is justified by the \hl{process-driven}, correlational research design and the exploratory character of the research itself. This thesis suggests the potential of the combination of multi-agent system (MAS) and actor-network theory (ANT) methodologies. ANT has been extensively applied in sociology for the analysis of cities and the urban \hl{ref}, while MAS itself is more mathematical-computational method for agent-based modellings \hl{ref}. MAS-ANT hybrid methodology  herein serves to capture local urban dynamics and reframe complexity of permanent urban transitions for urban development. This argument is built on the usefulness of ANT for describing urban reality. ANT approach with a potential capacity to afford openness and flexibility necessary for founding logical argumentation before tracing urban dynamics \hl{ref}. It will be then demonstrated that MAS adds the framework of action when applied over ANT. Finally, its application is presented on the case study of a post-socialist neighbourhood in Belgrade. In this case, I had the opportunity to be educated in Belgrade and to work in the architectural production in the Serbian capital. Therefore, the  researcher (me) is  to  some  extent  familiar  with  the  local  context  and has possibilities to access certain data.
\\
This thesis adopted MAS-ANT methodology in order to:
\begin{enumerate}
\item describe complex urban reality \deleted{socio-spatial patterns} (urban agency, decision making) \deleted{local conditions and track local urban knowledge)} in a post-socialist city (ANT);
\item understand \added{how the level of urbanity path serves for tracking socio-spatial patterns of transition} \deleted{understand the processes of urban transitions} in Belgrade, Serbia (MAS)
\item indicate \added{the processes of urban transitions} \deleted{the level of urbanity path for tracking urban dynamics} (ANT+MAS) 
\end{enumerate}
DIAGRAM

\section{Contribution}

Relate Contribution to Conclusions

The idea is to create visual interpretations that can be easily computerize (html5), and then easily changed. This enables the continuous generalizations and conclusions drawing and the introduction and description of new elements.

A by-product would be this new definition of urbanity and urban development.

This proposal aims to define a method of solving concrete problems through a process of understanding and dealing with current difficulties as they emerge and evolve.

elaborate inappropriateness of urban development concept for describing and guiding urban processes in cities outside the Global North. Especially not for tracing and directing urban transitions in the way that brings social and environmentally sustainable benefits to the inhabitants and the urban environment.

However, it is a widespread rarely criticised and unbeatable concept, especially in practice and practice-based research. Without delving into hidden motives and circumstances (economic, political, colonial), as this is rather an architectural approach to the urban, I use redefine the core of the concept, but keep its scope and aim to produce an operational methodological framework for practical investigations of the pallette of different cities around the world.

\section{Thesis Structure}

The study is structured in seven chapters.
For reaching the research objectives, it is crucial to provide the basic understanding and scientific justification of what forms and conditions urban complexity and dynamics and how the problem is approached within the limits of this research. The next \textbf{CHAPTER 2} contains an extensive literature review concerning the applicable concepts and the chosen methodologies. These concepts form the essence for categorizations with the chosen methods.
The conceptual and methodological parts build the theoretical framework for this thesis. 
\\
\textbf{CHAPTER 3} relies on the primary statements from this introductory chapter, builds on the range of indicators identified within the theoretical framework and further elaborates the methodological approach and the scientific argument of the research.
\\
In order to substantiate proposed hypothesis,  presented  theoretical framework will be tested on an elucidated case study. In \textbf{CHAPTER 4} I clarify the choice of Savamala neighbourhood in Belgrade as a case study.
\\
The following chapter moves forward to hypothesis testing and consecutive application of the chosen research methodologies. Data analysis with Actor-network theory is the core of \textbf{CHAPTER 5}.
\\
\textbf{CHAPTER 6} deals with system building according to the postulates of Multi-Agent system.
\\
\textbf{CHAPTER 7} presents the actual hybridization of two methods.
\\
In \textbf{CHAPTER 8} I pull together the outlined background  information on the theoretical framework and deconstructed MAS-ANT methodological hybrid and collected and analysed data on the cases study from  the previous  chapters  for  resulting  discussion. Based on these results, this thesis is concluded on two separate levels, regarding research and theoretical framework.


%%%%%%%%%%%%%%%%%%%%%%%%%%%%%%%%%%%%%%%%%%%%%%%%%%

\chapter{Literature Review}

%%%%%%%%%%%%%%%%%%%%%%%%%%%%%%%%%%%%%%%%%%%%%%%%%%

\section{Introduction}

This chapter outlines xxxx. ...

\section{Conceptual Framework}

\subsection{An Urban Development Process in an Ordinary City}
The concept of progress is central to modern society and it is orientated towards a positive vision of the future. In an urban scope, this concept corresponds to that of risk, where the control of all future events is calculable and predictable in probabilistic terms. This new concept of urban planning is based on the notion of an open-ended future, which implies that uncertainty must be accepted and managed, authorities and actual urban actors should be ready for new requirements and renewability as conditions change, and professionals are to increase their knowledge of risk and vulnerability in urban environments. In this sense, the planning of discourse relates to a master narrative of modernity, including ideas of rationality, objectivity, scientific evidence, values and possible control through normativity.
\\
There are different and numerous interpretations of what is and should be urban development, such as \hl{(Evropski regionalizam 1, World Bank and find other references)}:
\begin{itemize}
\item the course of a culture
\item meeting the needs of human and natural worlds
\item economic growth
\item "right to development" between the developed and developing
\item modernization
\item emphasize raising the living standards by addressing issues of health and safety, inclusion and equity
\end{itemize}  
Robinson (2006) clarifies   her  cosmopolitan  tactics  for  surpassing  hierarchical categorization of cities in the world, which in terms modernity and development of kept them apart. Her main point is to avoid the hegemony of western urban theory, as  well as the growing  strength of a discourse of development, which from 1970s onward has been emphasizing the differences between cities in the west and elsewhere, by: 
\begin{enumerate}
\item Dislocating accounts of ―urban modernity from the big cities of the west which claims to be its 
originator
\item Tracking  and  gather  differences  within  the  world  of  cities  in  order  to  justify  that  people  in different  places  have  invented  new  ways  of  urban  life  and  their  particular  production  and circulation of novelty, innovation and new fashions
\item Enriching  all  cities  with  better  future  perspective  depending  on  their  distinctiveness  and creative potential, without any hierarchical order among ordinary cities; they are all equal as 
sites of the production and circulation of modernity.
\end{enumerate}
In many scientific studies, interest lies in ordinary city constant state of change: dynamics and the structure of complex systems.

acknowledge the role of agency, decision making and urbanity

\subsection{The Constitution of Urban Key Agents/Urban Agency}


\subsubsection{Socio-spatial patterns of a post-socialist city}
In such a situation, urban planning was not a priority (Sykola, 1999), and it was not considered effective for managing local urban issues (Maier, 1998; M. Vujošević and Nedović-Budić, 2006). Therefore, planning was narrowed down to just one technical issue and very few theoretical or general methodological research studies bothered to examine alternative planning modes in transition, apart from replications of the approaches taken by neo-liberal or institutional economies (Begović, 1995).
\\
Thus, the crucial failures of post-socialist urban planning have come about through the lack of consensus on priority goals, action-oriented programs of implementation and coordination of different levels, sectors and areas. 
\\
In transitional countries, the course of merging socialist and neoliberal socio-economic condition, regulatory practices and organizational solutions led to inefficiently operationalized and inconsistently formalized institutional reforms rather known as "growth without development" \hl{Vujosevic}.
The post-socialist urban governance fails substantially through the lack of consensus on priority goals, action-oriented implementation and horizontal and vertical coordination.
\\
\added{Due to these circumstances, the urban development of post-socialist cities is perceived as a multi-dimensional integrated system composed of qualitatively different and semi-autonomous processes, with the inclining tendency to improve the economic, social, demographic, political and technological state of an urban environment.}
\subsection{The Morphology of Urban Decision Making}
On the contrary, according to one of the leading urban theories of David Harvey and Manuel Castells, urban planning cannot be seen as an autonomous process of spatial development, but rather it is situated in its political economic context and constantly overlaps current economic and social changes  (Taylor, 2006). In other words, urban planning in practice is intrinsically connected to the property market (which in turn involves a particular political ideology) and this tends to maintain current social order (Dear and Scott, 1981; Taylor, 2006), both of which are grounded in the development and expansion of industrial capitalism, neo-liberalism and consumerism (Ellin, 1999; Harvey, 1989). In other words, urban areas are, and have always been, the spatial and symbolic manifestations of broader social forces (Giddens, 1992).

\subsection{The Level of Urbanity}

Network of All Active Agents and Contextual Resources

the relation between urban life and urban form creates potential (Marcus)
level of urbanity broadens the opportunity for change (Marcus)

\section{Epistemological Framework}

A General overview of Methodologies for 
Methodologies for understanding urban development complexity and its dynamics

\subsection{Actor Network Theory}

\paragraph{general - social sciences}

\paragraph{urban studies}

\subparagraph{urban development}

\subparagraph{decision making}

\subparagraph{urbanity}


\subsection{Multi-Agent System}

xxxxxxx

\section{Theoretical Framework}


%%%%%%%%%%%%%%%%%%%%%%%%%%%%%%%%%%%%%%%%%%%%%%%%%%

\chapter{Methodology}

%%%%%%%%%%%%%%%%%%%%%%%%%%%%%%%%%%%%%%%%%%%%%%%%%%
Before delving into the data \added{sampling} and outcomes of this research, it is crucial to delineate the research process and procedures. Within the scope of this thesis, the research process involves the development of an organized body of knowledge on urban development processes in post-socialist cities. The aim of this chapter is to justify the choices made about what and how to research and the means to collect and analyze the data.
\\
The chapter starts with a presentation of a larger framework where the research objectives presented in the introduction are conducted into the context-specific research questions and working hypotheses. In the following, an explanation for the choice of case study method, the criteria for the case study selection, as well as mixed method methodological approach are explicated, along with a brief overview of the methods and techniques used.  

\section{Research Framework}

This thesis starts from the trendy term of urban development and cities in order to scrutinize urban complexity and dynamics in more operational, procedural manner. The research challenge of this thesis can be resumed to happen on multiple levels:
\begin{enumerate}
\item trace and put forward a value-neutral definition of urban development and identify the corresponding concepts that comply with it;
\item elaborate the validity of post-socialist neighbourhood as a case study that blends and exposes complexity and dynamics of modern urban context;
\item apply Actor-network theory framework for the descriptive analysis of a post-socialist neighbourhood;
\item construct MAS-ANT visual hermeneutic set, an engine for agent-based representations of urban dynamics.    
\end{enumerate}

The logistical construction of the inquiry involves the exploratory journey through facts, phenomena and theories of the conceptual framework within urban studies using the proposed methodological hybrid of Multi-agent system and Actor-network theory. The fundamental question stays the same: it is crucial to understand what is going on in cities under the hood of urban development and especially how it is going on.
Current body of knowledge on this matters gives us an input in the way how to transform and adapt general concepts presented herein into the indicators for complexity and dynamics of urban development processes. Theoretical framework has provided the foundation of facts, phenomena and theories in this direction, by acknowledging the conversion of general concepts into indicators as follows:
\begin{enumerate}
\item \textbf{Concepts into indicators}
\begin{itemize}
\item urban development - dynamics of urban processes;
\item urban agency - urban key agents and urban networks;
\item urban decision making - the morphology of urban decision making layers: top-down, real estate, bottom up;
\end{itemize}
\item \textbf{Indicators into dependent variables}
\begin{itemize}
\item dynamics of urban processes: urbanity and the morphology of urban decision making;
\item urbanity: socio-spatial patterns of urban transitions - urban transitions and socio-spatial patterns;
\item layers of urban decision making: urban key agents and urban networks;
\item urban transitions (maintenance, transformation and change processes): urban networks and socio-spatial patterns;
\end{itemize}
\item \textbf{Independent variables}
\begin{itemize}
\item human and non-human agents;
\item urban networks;
\item socio-spatial patterns (social practices, urban conflicts, contextual resources);
\end{itemize}
\end{enumerate}

Bearing in mind this extensive re-categorization and structuralization of urban development, MAS-ANT methodological hybrid proposes the road map for an inclusive and flexible approach for exploratory research - describing, tracing and representing dynamics of urban processes. Actor-network theory illustrate urban agency and decision making concepts while Multi-agent system operationalizes urbanity concept at an qualitative level and brings up the logics of the whole MAS-ANT procedure. Such provisional statements shed new light on the overall research questions proposed in the introduction and makes this thesis a methodological exploration.

Desired outcomes are dependent on the success of cross-pollination of concepts through MAS-ANT mixed research method. They are intended to influence both theoretical and practical domain. The research is guided in the way that it:
\begin{itemize}
\item questions the concepts of urban development, urbanity in general and urban decision making in post-socialist cities
\item proposes the terminology of transition which connects the processes of maintenance, transformation and change to urban conflicts, social practices and contextual resources at the local level
\item invents visual interpretations for practical uses
\end{itemize}

Consequently, the research is built on 3 hypotheses. Each hypothesis addresses a theoretical and a methodological issue and they are drawn in an consecutive order. Hypotheses justification is built gradually on describing, exploring and proceduralizing in order to master complexity and dynamics of urban development processes. 

\subsection{Context-specific Research Questions}

\textbf{Overall research question:} How To investigate socio-spatial patterns of post-socialist cities in order to reinvent a more inclusive and flexible approach to understanding Urban development processes engaging the complexity of an urban context? 

\subsubsection{RQ1}
What constitutes an inclusive approach (complexity and dynamics) to urban development?
\begin{itemize}
\item RQ1a: (indicator: figuration of human and non-human elements as urban key agents) What constitutes spatial and social differences and specificity in an ordinary city? 
\item RQ1b: (indicator: urban networks of all human and non-human elements) How do cities as specific socio-spatial phenomena are manifested through urban dynamics?
\item RQ1c: (indicator: morphology of urban decision making) Why does the morphology of urban decision-making determine pathways for urban development (urban transitions)?
\item RQ1d: (indicator: urban transitions) How do urban transitions redefine/describe urban complexity and dynamics in terms of ordinary cities \hl{doctrine}?
\end {itemize}

\subsubsection{RQ2}
Why do the level of urbanity traces determine pathways for urban development dynamics (urban transitions)? 
\begin{itemize}
\item RQ2a: (indicator: socio-spatial patterns in terms of local urban conflicts, social practices and contextual resources) What are the conditions for specifying the level of urbanity in an ordinary city?
\item RQ2b: (indicator: \deleted{contextual processes of} urban transitions) How to frame contextual processes to embody the dynamics of socio-spatial patterns in post-socialist cities?
\item RQ2c: (indicator: urban dynamics) How does the level of urbanity systematically approach urban transitions?
\end {itemize}

\subsubsection{RQ3}
How to frame urban development process to embody complexity of urban systems and dynamics of urban transitions?
\begin{itemize}
\item RQ3a: Why does tracing the level of urbanity within the morphology of urban decision making embody the dynamics of urban transitions?
\item RQ3b: How to frame the morphology of influences among different decision-making levels to describe/interpret the complexity of urban networks? \deleted{a dynamic urban development model of an ordinary city in a constant state of change} \hl{H3a: decoding urban dynamics}
\item RQ3c: How to design the framework for action in order to operationalize urban development concept?
\end {itemize}

\subsection{Research Hypotheses}
\textbf{Central hypothesis:} Urban development process, interpreted through MAS-ANT methodological approach, embodies networks of urban key agents initialized by the morphology of urban decision making and determines its level of urbanity.
Such relational \added{system} \deleted{object/structure} is a transparent engine for capturing the complexity and dynamics of \hl{urban transitions (urban development processes)}. 

\subsubsection{H1}

H1: Actor-network theory (ANT) gives an exhaustive image of the complexity of an urban environment (neighbourhood) by providing openness and flexibility for describing urban processes: figuration of human/non-human agency, blurred and askew morphology of urban decision  making and networks of all active urban key agents.
Breaking  down  the morphology of influences among different decision-making layers (top-down urban planning, real estate transformations, bottom-up participatory activities) through mapping networks of interactions and interconnections among urban key agents (urban actors, built environment, space, regulatory framework, infrastructure, 
social practices) clarifies the agency of urban transitions - urban development processes - in a post-socialist city.

\subsubsection{H2}

H2:  Urban development processes  is  determined  by  upgrading the  level  of  urbanity.  In value neutral sense, an overarching definition of the level of urbanity improves scientific capacity for grasping urban dynamics. The level of urbanity analysed through Multi-agent system (MAS) methodological approach indicates opportunities for urban transitions (maintenance, transformation, change) within socio-spatial patterns of an urban environment. 

\subsubsection{H3}

H3: Complex urban development processes set as an iterative procedure of tracing the level of urbanity within the urban agency map reinterprets a multi-layered morphology of urban decision making in terms of system flexibility and dynamics (transformation, maintenance, and/or change). A methodological hybrid that combines Multi-agent system (MAS) and Actor-network theory (ANT) offers a framework for capturing urban dynamics and reframing urban complexity at the neighbourhood level.

\section{Research Design}

The aim of this section is to present the research reasoning and the adopted methodology, namely the logical sequence that connects empirical data to the research questions, hyptheses and its conclusions. In designing the research process, the defined goals are assumed as exploratory in its nature and methodological in terms of urban studies. The study is gradually built from specific observations of the literature towards an in-depth analysis. An exploratory standpoint is chosen with regard to theoretical and practical strivings of the research. This division is crucial for establishing the research methodology. The first relies on secondary data and inputs theoretical constructs, while the second provides primary data and empirical evidence from the study field.
\\
The theoretical summary of urban development processes and the critical overview of the corresponding general concepts from urban theory (urbanity, urban decision making) is performed in the literature review in Chapter 2. This works as the structural catalyst for the chosen methodologies, as a kind of cross-pollination of concepts within the MAS-ANT methodological scope. On the other hand, MAS-ANT methodological approach is practically tested through the case study method. The application of this methodological hybrid in an hierarchical order (first ANT than MAS) for the analyses on the selected case study enables practice-oriented understanding of the situation in post-socialist neighbourhoods. Data display at the end gives an outline of the final blending of MAS and ANT methodologies and how they re-order and re-interpret the field data. This synthesis aims at transferring tacit into explicit knowledge about urban development processes in post-socialist neighbourhood in Belgrade.
\\
The so-called cross-pollination procedure justifies proposed indicators (operational definitions of the concepts used) and enables connections among independent and dependent variables constructed within the research hypotheses. This is the core logical construction of the research inquiry. The research further follows an inductive method of reasoning within the case study. The point of departure was the case study. Interpretative and participatory action research methods are used for the data collection. These qualitative methods are overlapping case study in order to support proposed theoretical categories (indicators and variables). Principal data sources were documentaries, open-ended interviews, workshops, and questionnaires, which contributed to the structuralized description of post-socialist empirical analysis performed with Actor-network theory (ANT). Multi-agent system (MAS) further made use of qualitative evidence to elaborate urban networks and the involvement of the key agents in urban affairs. Finally, MAS-ANT diagram displays the research results and facilitate interpretations of maintenance, transformation and change processes in an urban environment.
\\
The main study focus is to invent a looping procedure which examines the relations among the variety of urban elements, explores the "specificities and globalities" of the particular context, and catalyzes the framework of action on the neighbourhood level. The reach of this research is incremental, open-ended procedure-building based on pragmatic approach through iterative and collaborative techniques towards:
\begin{enumerate}
\item understanding the phenomenon,
\item creation of an overall framework,
\item identifying the pattern of dynamic reality in terms of urban transitions. 
\end{enumerate} 
The entry point for this methodological exploration is a case study.

\subsection{Case study}

This research adopted an in-dept case study inquiry as the adequate method for collection and framing of empirical data. Case study serves as a data collection engine, catalyser and boundary framework. Of particular importance is exploratory and descriptive character of the case study method. In general, the first captures the process, the second prepare and illustrate the incidence/prevalence of the phenomena \added{(Yin 1994)}. These features provide us with a comprehensive framework for describing contemporary phenomena with extensive types and sources of data \added{Feagin, Orum and Sjoberg 1991}. The goal is the holistic description of urban development and understanding the processes at stake over time \added{Wanborn 2010}. In this manner, the case study takes embedded approach with multiple units of analysis \added{(Scholz and Tietje, 2002; Yin 2009)}: urban key agents, the morphology of urban decision making, urbanity and urban transitions.
\\
The set of well-known components for designing a case study triggered its application in this research, such as \added{(Yin 2009)}:
\begin{enumerate}
\item focus on HOW and WHY questions about the researched phenomenon;
\item units of analysis, information relevant for the case construction, depend on the definition of research questions;
\item exploratory nature of research hypotheses as each proposition is built on something relevant within the scope of the study, one or more units of analysis;
\item linking findings to the hypotheses, units of analysis and theoretical background, "pattern matching" \added{Campbell 1975};
\item data collection focus for the case study while testing methodologies and existing theories provide rich theoretical framework for case studies
\end{enumerate}

However, in this thesis, the most important is the opportunity for application of multiple methods \added{Yin} and consequently methodological hybrids; as case study enables systematization and validation of data, according to the above presented features.
The justification of case study choice ????
single case design - testing methodology - extreme case/typical
holistic vs embedded
A kind of \textbf{Define sampling} - statistical analysis technique used to select, manipulate and analyze a representative subset of data points in order to identify patterns and trends in the larger data set being examined.
data obtained from the case study aim to contribute to XYZ objectives of the research??
Feagin:
case study was not only a method in sociology
case study for grounding of obversations and concepts about
1 social actions and structures in the natural setting
different sources and over a period of time - a holistic study of complex social networks and actions
case study offers discovering complex phenomena and recount their effects over time
examine the ebb and the flow of social life over time and to display the patterns of everyday life as they change.
\\
Yin Conducting case studies
sources:
documentation stable exact, bias the purpose of document and audience
archival records (service records, maps, charts, lists, survey data, personal records) precise and quantitative
interviews targeted focused insightful depend on the construction of questions weakness: reflexivity of interviewer's will inaccuracy
direct observation - visiting the site. cover changes in Savamala over time -  covers events in real time contextual cover context of events weakness: selectivity, reflexivity
participant-observation - workshops -  as direct observations, insights into motives
physical artifacts (tools, instruments, works of art)- insights into cultural features and technical operations
triangulation of data constructs validity - the events and facts of the case study supported by multiple data sources (this case)
Case study database: evidentiary base and investigator's report; notes, tabular material, narratives
chain of evidence: case study report (citations) -> case study database CSD -> citations to evidentiary sources in CSD -> case study protocol (questions to protocol topics that followed tthe procedure) -> case study questions
cross-referencing methodological procedures and the resulting evidence
triangulation (patton 1987 in Yin) - use of multiple sources of evidence - converging lines of inquiry:
of data sources
ivenstigator triangulation
theory triangulation
methodological triangulation
structure: linear analytic
\\
Yin - Reporting case study:
linear-analytic: starts with the issue of problem and review of the literature, then methods used, findings from collection and analysis, conclusions and implications
and
chronological - according to the case history, but for explanatory and causal sequences
significant communication device - description and analysis with implications about a more general phenomenon
communicate research-based data to nonspecialists.
case study as part of larger multimethod study 
case study includes data from other methods - embedded units of analysis being researched through these other methods
or when the larger study encompasses the case study
separately the data from other methods
in my case, the case study illustrates, in great depth, the unit of analysis for the methodological hybrid
REPORT:
1 general tactic - lit review -> methodology -> descriptive data about the case
2 problem and the case
3 construct validity
case study selection:
the case study must be significant: of general public interest, nationally important in policy and practical terms
CS must be complete: 1 boundaries of the case are explicit testing the boundaries, 2 exhaustive effort in collecting evidence, 3 artifactual conditions design the research according to the known constraints
CS must consider alternative perspectives: for descriptive take into account different points of view
CS must display sufficient evidence: indication of validity and maintained chain of evidence
CS written in engaging manner: clear writing style, clarity
\\
judging quality of CS:
This set of aspects based on the examination of hypothetical factors through logical argumentation and simulation will practically introduce its effect on urban development within an urban area (a city).
Construct validity defeat subjectivity - feasibility - chain of evidence and multiple sources of evidence
Internal validity - credibility and logics of explanations, no explanations in this research
External validity - transferability - on the methodological level, but needs testing
Reliability - document the procedure, this is the purpose of case study in this methodological research
Objectivity - confirmability - standardability
\\
flaws:
subjectivity and unreliability of data - deal with ANT flatten reality
credibility and research bias \added{Flyvbjerg 2006}
researcher role, interpretations and selections - dealt through the methodological rigidity, relativistic approach
the threat of generalizing in this research, analytic generalizations more on the methodological level - categories \added{(Yin 2009)}

Harrison 2002:
unit of analysis=  The boundary defined what had to be investigated in further detail, and what processes would be excluded.
analytical framework: context (global and local - outer and inner), content (process by with transitions are delivered), outcome variable (link the process of transition to different outcomes)
timeframe - longitudinal 
Pettigrew’s research design seeks to control complexity by building constancy into the study by controlling the content, and by defining a constant
unit of analysis to facilitate cross-case comparison. The outcome variables
must be clearly identified and measured. 
carrying out case study: structured data plan by a case study
dynamic phenomenon
Conducting case study: For case study work in particular, it is important to ensure variety,  and  that  you  have  sampled  sufficient  points  of  view  to  develop  a balanced picture.
selecting interviewees is Pettigrew’s concept of ‘supporters, opponents and doubters’.

Flyberg: 
Misunderstanding 2: One cannot generalize on the basis of an individual case; therefore, the case study cannot contribute to scientific development.
Carefully chosen experiments, cases, and experience were also critical to
the development of the physics of Newton, Einstein, and Bohr, just as the
case study occupied a central place in the works of Darwin, Marx, and Freud.
Misunderstanding 4: The case study contains a bias toward verification, that is, a tendency to confirm the researcher’s preconceived notions.
ANT reduces researcher bias

systematization of collected data should enable comparisons to the certain extent, generalizations or further analyses \added{Yin 2009}.
conclusion and intro do the case study selection: phenomenological approach - contemporary social phenomenon (urban development) within its real-life context (post-socialist neighbourhood) - case about events (process)

\subsection{Case study selection}

holistic approach, overview of relations, factors and influences,
understanding of the local context of the phenomena \added{Peric}

Flyberg: 
Misunderstanding 1:  General, theoretical (context-independent) knowledge is more valuable than concrete, practical (context-dependent) knowledge.
case study can be used for transferring tacit (context dependent knowledge) into explicit knowledge.

Through this research project, the problem defined will be examined in the real-life context of a city as a valid representation of a country setting. Any historical background to this research problem, its transformation as time has passed and the immediate surroundings where it emerges and changes should be taken into consideration as a chronological sequence, which - if properly described - provides an adequate capacity to explain correlational links among identified factors and elements of urban development. Given positive theory implications, it will be possible to predict future relations and behaviour of the elements in question.

Harrison 2002:
case study good where the theory base is weak and the environment under study is messy
deductive research from general statements derived form a prior logic to explain particular instances
the case study point of view
observed pehnomena lead to postulaiton of the existence of structures/mechanisms, which if exist could explain the relationships, tested by further research activities to isolate and observe phenomena and to eliminate the alternatives
Case oriented research: 
basis multiple methods to establish the research, must account for all deviating cases, few general conclusions
understanding of a phenomenon (unit of analysis) within its operating context
Case study research is flexible and can be adapted to many areas of knowledge creation. And the researcher is continuously confronted with the question ‘does this make sense?’

Flyberg: 
Misunderstanding 3: The case study is most useful for generating hypotheses; that is, in the first stage of a total research process, whereas other methods are more suitable for hypotheses testing and theory building.  
strategic case selection - information oriented case: To maximize the utility of information from small samples and single cases. Cases are selected on the basis of expectations about their information content.

in Savamala, a historical but deteriorating city quarter in Belgrade, where a set of bottom-up urban transformations and participatory spatial interventions and swift, investor-based developments intent in this respect to make these environments 
clarification of the unit of analysis - neighbourhood level and how it bounds up all recognized indicators

\subsection{Local Context}

post-socialist city and transition
legacy from the past (path dependency, socialism) and prospects for the future (transition)

\section{Adopted Methodology}

\subsection{Savamala Case study - Data collection} \label{sec:predis}

Harrison 2002:
interpretative research by direct observations is case study, explain how it is deviated in my case (figure 9.3) Meredith figure suggests that case study is an envelope for possible research processes, accurately referred as research strategy.  Thus, structured interviews, field studies and surveys are all possible methods which can be deployed under the case study banner
figure 9.4 - methodology verification -> research design case -> collection method: documentation, participant observation, interviews, questionnaires -> implementation: case study sellection (sample, scale), questionnaire, workshop, interviews organization -> data analysis

The researcher is faced with the challenge of coping with large amounts of
data, defining the scope of the study, collecting data in a coherent way,
analysing data in a replicable way and condensing the complexity into some-
thing  that  is  logical  and  understandable  to  others

Denzin et al:
process of building a case study:

bounding the case
phenomena, themes, issues
seeking patterns of data
triangulating key observations
alternative interpretations
assertions of generalization

\subsection{ANT and MAS approaches - Data analysis}

\subsubsection{Discourse analysis}
\subsubsection{Structural analysis}

An urban development model is based on:

Measuring the efficiency of urban planning

Testing the legitimacy of urban transformation interests

Recognizing the opportunities of bottom-up urban design initiatives

\subsubsection{Setting a procedure}

\subsection{System Building - Data display}

Reporting the findings...
Harrison 2002:Use techniques of data reduction and display.

%%%%%%%%%%%%%%%%%%%%%%%%%%%%%%%%%%%%%%%%%%%%%%%%%%

\chapter{Case study}

%%%%%%%%%%%%%%%%%%%%%%%%%%%%%%%%%%%%%%%%%%%%%%%%%%

\section{Background}
In post-socialist cities, urban planning should link the top-down changes (linked to national and global level) to the bottom-up changes in the urban systems of the city, by emphasizing diversity and reciprocity in the nature of the on-going transformations: economic transformations (transformation of production and consumption in relation to space, income polarization and poverty), political transformations (urban governance, participation and decentralization), spatial transformations (demographic trend and distribution of functions) and social transformations (social inclusion, social activism and informality)
While some trends and directions within these transformations are clear and defined, uncertainty dominates decision-making and implementation in the turbulent environment of post-socialist cities (Nedović-Budić, 2001). The internal environment is in a state of flux, with the rapid adjustment of the physical, economic, social, and political structures of the city itself (Sykola, 1999). This captures the pace of change and the multi-layered nature of transformation, with the focus on the process of change in the city’s economy, society, system of governance and the spaces of production and consumption.
\section{Stimulants and deterrents of decision-making tradition in Belgrade}

xxx. ...

\subsection{Post-socialist Urban Planning}

Top-down Management of Urban Conflicts

\subsection{Legitimacy of Interests in a Post-socialist City}

Tactical Urban Transformations 

\section{Dynamism of bottom-up urban agency in Savamala}

Moreover, the way cities function shapes the expectations and actions of all the urban actors involved, who also influence the constitution of the city itself. 

\subsection{Network of civic engagement}

Participatory Urban Design Operations


%%%%%%%%%%%%%%%%%%%%%%%%%%%%%%%%%%%%%%%%%%%%%%%%%%

\chapter{ANT Data Analysis}

\section{A forward-thinking overview of building an urban development model for Savamala}

xxx...

\section{Urban assemblage map: urban key agents and contextual resources}

\chapter{MAS System Building}

%%%%%%%%%%%%%%%%%%%%%%%%%%%%%%%%%%%%%%%%%%%%%%%%%%
xxxxxxxxx

\section{Profiling the agents}

xxx...

\chapter{MAS-ANT data display}
xxxx

\section{Body of urban relations: Urban Development dynamics}
Data display
\section{MAS-ANT diagram: scope and operationality}

Discussion

%%%%%%%%%%%%%%%%%%%%%%%%%%%%%%%%%%%%%%%%%%%%%%%%%%

\chapter{Conclusions}

%%%%%%%%%%%%%%%%%%%%%%%%%%%%%%%%%%%%%%%%%%%%%%%%%%

\section{Conclusions related to the research framework}

xxxx ...

\subsubsection{[..] to the research objectives}

xxxx

\subsection{[..] to the research questions}

xxxx

\subsection{[..] to the methodological approach}

xxxx

\section{Conclusions related to the theoretical framework}

Harrison 2002:
Extending the general analytic strategy across cases is referred to as replication (Yin, 1994)
Conclusions conflict with the literature. This is a challenge to
‘build internal validity, raise theoretical level and sharpen construct definitions’ (see Table 9.2).

\subsection{Urban Development Taxonomy}

xxx

\subsection{An Ordinary City}

xxx

\subsection{A Post-socialist City}

xxx

\subsection{Urbanity}

xxx

\section{Practical Implications}

xxx

\subsection{Urban Development model}

The potential lies in the dynamics of a modern scientific approach in research and how to implement it in urbanism; namely, how potent urbanism is to be semi-dynamically programmed considering, of course, a large background database. 
Corresponding to factorial analysis, defined pattern-models will be performed, so as to have visual representation of facts which are able to lead to urban progress in developing countries.
According to the analysis of characteristics mentioned above, crucial relations between terms could be established by an a posteriori approach to the vast range of collected data through an archival working process in order for their entailment, implication and structuralisaton.
What is here even more important is a multidimensional approach which should allow for the making of a graphical representation of these analyses which are by themselves complexes.

\section{Limitations of the research}

Harrison 2002:
trade off between knowledge and time
The data cannot be collected as originally planned. This may call for an
adjustment to a particular variable, or it may mean a major re-think.

Flyberg:
Misunderstanding 5: It is often difficult to summarize and develop general propositions and theories on the basis of specific case studies.

\section{Future Prospects}

xxx


%%%%%%%%%%%%%%%%%%%%%%%%%%%%%%%%%%%%%%%%%%%%%%%%%%

\begin{small}
\addcontentsline{toc}{chapter}{Bibliography}
\bibliography{ThesisBib}
\end{small}

%%%%%%%%%%%%%%%%%%%%%%%%%%%%%%%%%%%%%%%%%%%%%%%%%%


\newpage
\appendix
\noappendicestocpagenum
\addappheadtotoc

\end{document}
