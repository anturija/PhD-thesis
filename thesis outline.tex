
\documentclass[11pt]{report}
\usepackage[a4paper,margin=2cm, bindingoffset=2cm]{geometry}
\usepackage{appendix}
\usepackage{amsmath}
\usepackage{booktabs}
\usepackage{threeparttable}
\usepackage{natbib}
\usepackage[final]{changes}
\usepackage{color,soul}
\usepackage{tocloft}
\usepackage{xcolor}
\usepackage{hyperref}
\bibliographystyle{chicago}
\setcounter{tocdepth}{3}% Include \subsubsection in ToC

\newcommand{\listofdiagrams}{List of Diagrams}
\newlistof{Diagram}{ex}{\listDiagram}
\newcommand{\Diagram}[1]{%
\refstepcounter{Diagram}
\par\noindent\textbf{List of Diagrams \theexample. #1}
\addcontentsline{exp}{example}
{\protect\numberline{\thechapter.\theexample}#1}\par}
\hypersetup{
    colorlinks,
    linkcolor={red!50!black},
    citecolor={blue!50!black},
    urlcolor={blue!80!black}
}

%to stop orphan lines
\widowpenalty=10000
\clubpenalty=10000
\raggedbottom

%line spacing
\linespread{1.3}

\begin{document}
\begin{titlepage}
\begin{center}
\large
\textsc{EPFL} \\
\ \\
\textsc{CODEV and LASUR} \\
\ \\
\textsc{EDAR}\\
\ \\
\ \\
\ \\
\ \\
\ \\
\ \\
\ \\
\huge
\textbf{Urban Development Processes} 
\ \\
\huge
{A Methodological Investigation into the Complexity and Dynamics of Post-socialist Cities}
\ \\
\Large
{A Case Study of Savamala Neighbourhood in Belgrade, Serbia}
\ \\
\ \\
\large by
\ \\
\ \\
Marija Cvetinovic
\vfill
Thesis for the degree of Doctor of Philosophy \\
\ \\
\ \\
XXX 2017
\end{center}
\end{titlepage}
\begin{titlepage}
\begin{center}
\ \\
\ \\
\ \\
\ \\
\ \\
\ \\
\ \\
\ \\
\ \\
\ \\
\ \\
\ \\
\subsubsection{"Every thing possible to be believ'd is an image of truth."}

\end{center}

\begin{flushright}
― William Blake, The Marriage of Heaven and Hell
\end{flushright}

\end{titlepage}

%Roman Page Numbering
\pagenumbering{roman}

\tableofcontents
\cleardoublepage
\listoffigures
\addcontentsline{toc}{chapter}{List of Figures}
\cleardoublepage
\listoftables
\addcontentsline{toc}{chapter}{List of Tables}
\cleardoublepage
\listofdiagrams
\addcontentsline{toc}{chapter}{List of Diagrams}
\cleardoublepage

\chapter*{{Abstract}\markboth{Acknowledgments}{Acknowledgements}}

\textbf{The overall objective} of this dissertation is to critically address, break down and recompose \textit{the urban development process in post-socialist cities}
through a set of analyses that encompasses
urban planning strategies, real-estate interventions, and participatory and urban design activities.
The blurred and askew morphology of the gamut of different cities around the world requires dynamic solutions and urges for proper techniques that are spatially and temporally adjusted to local socio-spatial patterns.
\\
\textbf{The theoretical foundations} are built upon \textit{the ordinary cities approach} that provides a unique assemblage for examining how these cities interlace with the world, taking also into the account the social, cultural and historical legacies that each city carries into the era of globalization.
The urban development of post-socialist cities is therefore perceived as a complex and dynamic system evolution with the inclining tendency to encompass all discrepant layers of \textit{urban decision-making}, track \textit{the level of urbanity} through the fluctuating relations of \textit{urban agency} and socio-spatial patterns, and, in general, indicate the contextual processes of maintenance, transformation and change within an urban system.
\\
\textbf{The case study} of the \textit{Savamala neighbourhood} is a scaled example of the pre-socialist material legacy, a socialist cultural and societal matrix, a transitional reality and a condensed case of the multi-faceted circumstances of post-socialist urban development. These elements also frame the exploratory limits for the case study research that comprises a chain of decision-making and a network of interdependences and interconnections among all active urban key agents and socio-spatial patterns identified in Savamala exclusively through a qualitative inquiry.
\\
\textbf{The methodological framework}  is based on the process-driven, correlational research design through a mixed method. The hybrid method  comprises the \textit{Actor-Network Theory (ANT)}
for describing  urban complexity by founding bottom-up logical argumentation,
and the \textit{Multi-Agent System (MAS)} that adds the framework of action by tracking urban dynamics.
The overlapping of methods and visualizations through diagrams set a system of data analysis, triangulation and reduction - from complex actor roles and synthesized networks, through contextualization of interests and interventions to urban system transitions within historical processes of long duration.
%The systemic correlation between theoretical, methodological and empirical frameworks was presented as a tool for addressing urban complexity and dynamics in the Savamala neighbourhood.
\\
\textbf{The research findings} are three-folded.
Firstly, they address actors and processes at play in Savamala revealing disbalanced conglomerate of  political (power-mongers), economic (profit-seekers), professional (technicians and apparatchiks), civil and cultural sectors (on the go).
The articulation of \textit{urban agency in Savamala} confirms fundamentally authoritarian distribution of roles and decisions.
Then, the empirical results of the study contribute to the operationalization of several
theoretical constructs for practical investigations of the palette of ordinary cities around the world.
While \textit{urban development} is interpreted in terms of contextualized urban system transitions, an overarching definition of \textit{the level of urbanity} grasps the fluctuations of local socio-spatial capital.
Finally, visualizing data through MAS-ANT methodological approach depicts the complexity of urban actors, forces and artifacts and the dynamics of urban networks, interrelations and processes as a legible, data-loaded scheme of nodes and links. 
This research contributes to responding to the necessity to shift the deterministic concept of  urban research
%to a more comprehensive, network-oriented vision
in terms of finding an intermediary between the qualitative data analysis and the graphical data display.
%closely related to modern means of communication.

\cleardoublepage

\chapter*{Declaration of Authorship}
\addcontentsline{toc}{chapter}{Declaration Of Authorship}
I, xxxx, declare that the thesis entitled xxxxxxxxxxxxxxxxxxx and the work presented in the thesis are both my own, and have been generated by me as the result of my own original research. I confirm that:
\begin{itemize}
\item this work was done wholly or mainly while in candidature for a research degree at this University;
\item where any part of this thesis has previously been submitted for a degree or any other qualification at this University or any other institution, this has been clearly stated;
\item where I have consulted the published work of others, this is always clearly attributed;
\item where I have quoted from the work of others, the source is always given. With the exception of such quotations, this thesis is entirely my own work;
\item I have acknowledged all main sources of help;
\item where the thesis is based on work done by myself jointly with others, I have made clear exactly what was done by others and what I have contributed myself;
\end{itemize}
\vspace{2cm}
Signed: \dotfill
\vspace{2cm}
\newline
\noindent
Date

%Include Acknowledgements in TOC
\chapter*{Acknowledgements}
\addcontentsline{toc}{chapter}{Acknowledgements}
This work was undertaken with financial support of the 

This thesis would not have been possible without the support of many people. I would like to express my sincere gratitude to:

\deleted{SNSF
SCOPES
IAUS
APEP UCD

JCB
ZND,SZ, TM,

NDVBGD

ABigail
Marija Krstic}

\begin{itemize}
\item 
\item ...
\end{itemize}



%%%%%%%%%%%%%%%%%%%%%%%%%%%%%%%%%%%%%%%%%%%%%%%%%%

\chapter{Introduction}
%Arabic Page Numbering
\pagenumbering{arabic}

%%%%%%%%%%%%%%%%%%%%%%%%%%%%%%%%%%%%%%%%
Urban development is a widespread archetype for the improvement of outreach in cities outside the Western World. However, in its practical application, it is more of a kind of constant effort to catch up with the West and the occidental urban paradigm, than an elaborated form of intrinsic local perception, knowledge and action toward urban system evolution. 
\\

In this highly competitive international arena, transitional countries experience grave consequences due to the paucity of practical experience within the ruling western ideology of the urban. They are caught in this new context with its relentless rules of the market economy, decentralized political and administrative powers, lack of resources, scarcity of general international investment and scant interest in the dramatic shifts that occur in these societies, in all aspects of their social organization and spatial transformations. The blurred and askew morphology of post-socialist cities in transitional countries is therefore the result of continuous pressure from the negative side-effects of imitating and lagging behind conventional urbanization models and accelerating globalization patterns imported from or imposed by the Global North, colloquially  known as "the West".
\\

The urban transformation of Serbian cities falls into this cliche of the new post-socialist urban reality, which emerged during the "transition to markets and democracy" (\href{tsenkova}{\citealt{tsenkova_urban_2006}}). 
The dismantling of the Yugoslav socialist regime during the late 1980s represented a substantial change in all aspects of social organization, the economic model and the political system.
In such a situation, concern about the urban has been left out and given over as a battlefield for social needs in practice and technical solutions on paper and is easy prey to the exercise of power and interest. Therefore, the practice of planning and designing Serbian cities has most often been without even an actual or adequate goal, plan or procedure in place.
\\

Furthermore, very few theoretical or general methodological research studies have bothered to examine alternative planning modes, techniques and instruments during transition, but have continued the practice of replicating  well-known counterparts in the West.
In general, even though for construction and space organization, we have different drivers on the global scale and in developed countries, there is rather a global trend of resorting to sociological, planning or even ICT (information and communication technology) approach in scientific studies on the urban. 
\\

Within the circumstances of developmental bouillon or "developmental schizophrenia" (\href{Vujosevic}{\citealt{vujosevic_postsocijalisticka_2010}}) at the local level, my aim is to elaborate, on the standpoint of an architect, a methodological approach suitable, to a certain extent, for application upon the complex post-socialist urban reality.
Among others, architects have a vital role in not only directing, but also framing the path of urban formation and development in post-socialist cities.
This is even more the case, as they are primarily focused on practice and "savoir faire" about making the built environment.
Yet, in Serbia, acquisition of land and illegal construction are spatial interventions that have marked the post-socialist production of space more than any planning or theoretical activity (\href{grubovic}{\citealt{grubovic_belgrade_2006}}).
In order for spatial interventions to compete for more relevancy and rigour, local architects have gradually shown interest in a scientific discourse within the context of built structures, spaces and cities in general. In the rivalry between the spatial and social basis for their interpretations, the fact that the field of architectural research is not yet standardized in terms of methodologies and techniques opens the door for experiments and innovations (\href{till}{RIBA 2014}).
\\
 
In order to contribute to post-socialist urban research, my far-reaching aim is to capture the post-socialist urban complexity and dynamics in order to skip the classical procedure of urban development based on the western planning paradigm and provide a practical application to the multiple layers of urban decision-making. To be achieved, this requires supple methodological approaches which should better correspond to post-socialist socio-spatial patterns on multiple levels (state, city, municipality, community, and neighbourhood) and better explain the correlations of various urban elements. This practice-oriented, locally-focused and globally-tuned approach to the complex urban reality of post-socialist cities envisions embracing the dynamics of urban systems and the operationality of architectural investigation.
Together they serve for circumscribing visual interpretations that enable continuous conclusion drawing and an up-to-date introduction of any new element that may appear in the system.
\\

This introduction delves into the contextual, scientific and disciplinary discourse of the following research. It marks the research context, historical and scientific, and puts a spotlight on the importance of the research problem, as well as on the purpose and adequacy of this thesis. The research drive is then outlined accordingly. In a nutshell, the chapter presents what this research is about, how it will be performed and what the research expectations are and the practical results expected. 

\section{Urban development: What, How and Where to Study It}
%field of study

The city is regarded as a geographically condensed, highly structured economic, and the most complex social phenomenon (\href{ref}{\citealt{mumford_city_1961}}). "Relational space", and "social interactions", in the modern qualitative and temporal sense of the term, are now the leading determinant for the way urban systems function (\citealt{soja_socio-spatial_1980}; \citealt{low_anthropology_2003}; \citealt{amin_regions_2004}; \citealt{massey_for_2005}; \href{ref}{\citealt{khan_epistemology_2013}}). 
Due to growing social and physical transformations that have become evermore intensified as current globalization continues to spread out profit maximization, consumption patterns and information networks (\href{Harvey}{\citealt{harvey_rebel_2012}}), cities have been experiencing a progressive reorganization at spatial and social levels. Even though accelerating urbanisation is a worldwide process, it still assumes different forms and meanings, depending on the prevailing local conditions (\href{Bolay}{\citealt{bolay_slums_2006}}). 
These overall circumstances of continuous urban development influence cities to serve as the primary channel linking local realities to global social, political and economic forces (\href{yates}{\citealt{yates_north_2002}}; \href{tsenkova}{\citealt{tsenkova_beyond_2006}}).
\\

Urban development is rather a generic term for circumscribing the progress of and in cities addressed in  practice-oriented research (\href{ref}{World Bank}). Today, when cities are primary venues, power poles and capacity builders (\href{ref}{\citealt{castells_urban_1979}}), the theorem that the growth of cities expand opportunities seems to hold up. Moreover, the urban development concept has been easily mixed up with urbanization and economic growth and more often ruled out by the appealing righteousness of sustainable development trends(\citealt{christie_here_2001}; \citealt{hopwood_sustainable_2005}; \citealt{sachs_development_2010}; \citealt{sachs_millennium_2012}; \citealt{sachs_age_2015}).
In this sense, urban development has been either patterned or predicted referring to whether it is a part of a model or project for a city or an urban environment.
\\

The crucial move forward is to identify the patterns of the dynamic reality in cities and be more consistent with the spontaneous, everyday urban system transitions. Furthermore, a corresponding change in approaching urban development can then be addressed by heterogeneous, iterative and generative processes of urban space production in the physical and social sense.
Ordinary cities approach provides unique assemblage of internally different, distinctive and context-based urban system transitions as well as overlapping space-, time- and relation- networks across cities
(\href{Amin}{\citealt{amin_ordinary_1997}};
\href{Robinson}{\citealt{robinson_global_2002}}; \href{Robinson}{\citealt{robinson_ordinary_2006}}; \href{Roy}{\citealt{roy_urbanisms_2011}}; 
\href{Parnell}{\citealt{parnell_retheorizing_2012}};
\href{Robinson}{\citealt{robinson_urban_2013}}; \href{Robinson}{\citealt{robinson_comparative_2015}}; \href{Robinson}{\citealt{robinson_thinking_2016}}).
In other words, it is not only necessary to examine the ways in which countries/cities interlace with the global economy, but also social, cultural and historical legacies that each country/city carries into the era of globalization.
\\

Within such range of explanations, this thesis revolves around the interpretation of urban development as an answer to the question "how can cities facilitate urban system transitions while also maintaining the culture and values of the community itself?" (\href{ref}{\citealt{kahne_placemaking_2015}}). The idea of indicating what encompasses the urban development of an ordinary city leads to identifying the internal and external influences that constitute the core processes of maintenance, transformation or change in the urban system of a city, when treated equally within the global hierarchy of cities  (\citealt{robinson_global_2002}). 
Such an approach aims to surpass the perception of cities as merely economic, social and cultural venues and treat them as complex and dynamic urban systems. In these circumstances, it is necessary to apply proper techniques and methodologies for urban research and analyses that encompass the complexity and dynamics of cities in order to improve their living conditions and facilitate social interactions in the process of urban development.
\\

Generally speaking, a range of urban disciplines (urban planning, theory, sociology, legislation and design) aim to decode and harmonize growing urban issues as a side-effect of the current globalization, urbanisation processes and the spread of capitalism. These trends mainly affect cities and the production of urban space and bid for the expertise to manage urban development (\href{ref}{\citealt{allmendinger_planning_2009}}; \href{ref}{\citealt{faludi_planning_1973}}).
In practice, these disciplines are embedded in a particular social context or a territorially based system of social relations. They react to shifts in socio-economic and political settings  (\href{ref}{\citealt{tsenkova_beyond_2006}}), but have maintained a privileged relationship with Western cities, which are assumed to be the sources of urban creativity, vitality and innovation in the urban domain (\href{ref}{\citealt{robinson_ordinary_2006}:2}). Accordingly, these disciplines tend to fail substantially when applied to a range of spatially and economically different environments that have undergone highly dramatic change in political, economic and social terms. 
\\

However, each discipline maintains its own perspective and pace in approaching urban matters. My architectural background has moulded my own research interest towards gaining knowledgeand understanding of the management of space and the built environment. Moreover, production of space is also a core concern for architects. Architecture is a discipline focused on practice and consequently it argues for parameters, categories and structure for its practice-based analyses.  
Hitherto, history and the theory of architecture have been the main fields of architectural research. 
Nonetheless, the production of space and place-making have been the topic of architectural output since the classical period  (\citealt{Vitruvius, 20BC}; \href{ref}{Rossi, 1982}; \href{ref}{Braunfels, 1990}; from \href{van}{\citealt{van_assche_co-evolutions_2013}}), subsequently contributing to the drafting of theory in landscape architecture and urban design (e.g. \citealt{Rossi, 1982}; \citealt{Lynch, 1981}; \citealt{Braunfels, 1990}; \citealt{Child, 2010}; from \citealt{van_assche_co-evolutions_2013})
\\

In terms of methods, there have been a significant number of interdisciplinary, transdisciplinary and multidisciplinary endeavours in applied research in urbanism. What is more, applied fields of research acknowledge the use of methodological hybrids (\href{ref}{\citealt{datta_paradigm_1994}}; \href{ref}{\citealt{de_lisle_benefits_2011}}). This has opened doors for applied social sciences to investigate new methodological opportunities when confronted with complex and multiplex social phenomena (\href{ref}{\citealt{de_lisle_benefits_2011}}). This is even more relevant as methodological and epistemological rigidity leads to ignoring the realities of the practical and causes catastrophic scientific failures of practice-oriented research (\href{ref}{Rogers 2008};\href{ref}{\citealt{lisle_benefits_2011}}).
\\

In contrast, the recent growth of architects engaged in urban research makes a case for transforming the general body of knowledge of cities into a real-life problem-solving strategy that address human lifestyles, social relations and the concept of space (\href{ref}{\cite{castells 2000}}; \href{ref}{\cite{dijk 2002}}).
However, the ever-growing presence of a multidisciplinary frame of reference in research  argues for the advancement of each discipline in terms of the relevance and reliability of the knowledge therein produced.  However, missing links within the classical scientific discourse has caused a growing concern about what research is appropriate for architectural and design practice, as well as for an architectural stance in urban studies (\href{RIBA}{RIBA}). This especially concerns methodologies, methods, approaches, domain and credibility ({\citealt{savic_what_2014}; \citealt{savic_introduction:_2016}). Lacking the traditional scope of analysis, architectural research has been a paragon for innovation and experimentation.
\\

What I have recognized as a crucial change in the methodological paradigm of the applicable urban research from the architectural standpoint, can then be boiled down to the rise of the global concept from one that is static to one that is iterative and dynamic.
Generally speaking, a static world is one in which all transitions are according to a known law and which do not give rise to uncertainty. When defining the evolution of analyzing and simulating an urban phenomenon or process, it is fundamental to state that the existence of a problem depends on the future being different from the past, while the paradigmatic possibility of finding the solution to the problem depends on the future being like the past. Therefore, a transition in some sense is a necessary condition for a problem to exist. The complex empirical realities of urban system transitions collide therein with the powerful and dominant policy of the continuous, comprehensive production of knowledge. A scientific approach towards formulating the dynamics of urban system transitions have to count on uncertainty as one of its fundamental facts and in this way accept and deal with an open-ended future and the limits of human knowledge about it.
\\

Gaining knowledge has come to be a strategic activity rather than a search for truth (\href{ref}{\citealt{kirby_cities_2013}}). 
o science becomes incapable of controlling society and the rationalized reality appears false and irrelevant (\href{ref}{\citealt{alfasi_planning_2004}}). Given these conditions, the growing gap between the formal structure and the dynamics that take place in cities triggers an internal and independent process by which the system tends to spontaneously self-organize (Portugali 2011). Therefore, a city should be conceived as an organism, not a mechanism (\href{ref}{\citealt{landry_creative_2012}}). In these terms, the city is interpreted as a living system which is constantly mutating and emitting new elements, a container for processes of coming to be, breaking up and falling out, fragmenting and recomposing.
\\

Contemporary cities tend to be concentrations of multiple socio-spatial circuits, diverse cultural hybrids, and sources of economic dynamism - a venue where the past and the present converge upon one another (\href{ref}{\citealt{braudel_history_1970}}; \href{ref}{\citealt{harvey_condition_2003}}). The city tells a story of one society and its attempts to move towards a positive vision of the future, through complex ranges of processes that flow together to construct a single consistent, coherent, albeit uncertain, interactive and multifaceted time-space system (\href{ref}{\citealt{graham_end_1998}}; \href{ref}{\cite{graham_relational_1999}}). These ceaseless processes are the core of spontaneous, everyday urban development. Grasping the scope of urban development occurs as a major challenge for the modern science about cities.
\\

My intention is not to produce another pattern applicable to certain cities to a certain extent, but rather to apprehend a process that embodies the complexity and dynamics of these relations in a transparent way. This framework of research enables a consideration of the means of generating a vibrant and fluid context open to permanent transformations and, most importantly, to grasp the idea of an adjusted and balanced method, adaptive to changing views and situations of accelerating urban development. (\href{ref}{\citealt{portugali_complexity_2011}}). For this to be achieved, requires supple approaches which should aim to explain the correlations of various urban elements and  better correspond to the socio-spatial patterns of a range of urban environments. 
\\

A dynamic urban context is a complex phenomenon with a plenitude of data.
The case study research method enables a close, in-depth and holistic examination of a great deal of data, but requires a bounded environment in order to accurately describe and illustrate such a context and to use it for broader interpretations and the demystification of modern cities. Specifying physical limits is not in itself enough to circumscribe the identified complexity of urban system transitions - the issue of scale is also at stake. In urban terms, different spatial and social elements are intensified or muted at different levels (global, national, regional, local).
\\

In order to acquire active follow-up, interpretation and assessment of urban issues, it is important to define a representative environment, a robust source of prominent "urban processes". Therefore, I argue for a neighbourhood level of analysis to become the paradigm for the complexity and dynamics of the modern urban context and local specificities. It serves as an urban micro environment, which eventually increases the body of knowledge of cities concerning the methodologies used to deal with urban development and corresponding urban system  transitions.
\\

\textbf{"The contemporary city is a variegated and multiplex entity - a juxtaposition of contradictions and diversities, the theatre of life itself" (Amin and Graham, 1997).}

\section{Urban development: An Issue in Post-socialist Cities}
%problem statement

The focus of this thesis is urban complexity and the dynamics of post-socialist cities. The issue is not addressed as a problem to solve, but rather as a moving target for an exploratory observation of the way cities function and how various urban system transitions condition urban development of post-socialist cities.
\\

Following the "ordinary cities" approach, I would like to elaborate that post-socialist cities in transitional countries encounter extraordinary  difficulties  when  copying  urban  models from the West.
The cause is found in the lack of institutional infrastructure and cultural patterns essential for the functional unity present in western cities (\citealt{petrovic_cities_2009}). Furthermore, the fundamentality and intensity of economic and political change in Balkan post-socialist countries may be a historic exemplary of social transition hard to find in a "typical" capitalist city (\citealt{sykora_transitional_1999}). Its internal environment is in a state of flux, with rapid adjustment of the physical, economic, social, and political structures of the city itself (\href{ref}{ibid.}).
\\

Post-socialist cities are treated herein as a range of qualitatively distinctive cities that "deal differently with their difference" ({\citealt{amin_ordinary_1997}). In their incompleteness, plurality and informality, post-socialist cities in transitional countries represent dynamic and diverse arenas of contemporary urban life, experience and theory. Such confusing political and social circumstances have deprived the average citizen of sufficient information about the possibilities and tools to take an active part in the development of their city. 
This rather organic path of urban development leads to classifying post-socialist cities in transitional countries as unregulated capitalist cities (investment-led) with third world urban development elements (substantial illegal activities and informal markets) (\href{ref}{\citealt{petrovic_cities_2009}}).
\\

Conditioned by the geographic location of Serbia (in CEE), murky circumstances of transition (towards the liberal market, private property, profit motive and consumer sovereignty) are followed by a set of decentralization and democratization protocols for joining the EU, for the availability of European research and civil sector funds, as well as for the promotion of participation and engagement from the ground up (\citealt{vujosevic_conundrum_2012}; \citealt{vujosevic_regionalizam_2015}; \citealt{zekovic_spatial_2015}).
Having said that, the lack of successful urban planning models and actions make it possible for the rising economy of social exchange and local capacity building to contribute to an improvement of life and functionality of urban structures and systems, and effectively address the tensions between top-down and bottom-up urban planning in a post-socialist city.
\\

Tracing the institutional articulation of a post-socialist context in Belgrade involves the structural analysis of administrative procedures and content analysis of policy agendas. It serves to systematically deconstruct local urban governance in terms of the political, economic and cultural aspects of transition with a multitude of actors, a variety of interests, conflicted strategies and fragmented implementation. In the long run, the identification of relations and influences on post-socialist urban governance examines how urban actors, space and the regulatory framework rely on planning and decision support systems as a means to forecast and orchestrate any movement of the system. In this manner, each element of urban systems, human or not, is attributed agency.
\\

This thesis approaches the urban development concept in relativistic terms.
\footnote{The relativistic approach to space and time and their relations stems from modern physics and in urban research, it was introduced by David Harvey with the term "space-time compression" which explained the altered relations between space and time caused by the capitalist order (\citealt{harvey_condition_2003}). However, the series of authors speak about condensed or diluted (speeding up and spreading out) space-time relations, but they connect this with, globalization and urbanization trends and technological advancement rather than exclusively with capitalism (\href{ref}{\citealt{massey_global_2010}}).}
Under the hood of scientific neutrality, the urban development concept is critically approached, broken down and recomposed as a process of urban system transitions, not as an indicator or the final product in urban practice.
In this sense, urban development is applied as an overarching codifier for urban complexity bounded rather as a comprehensive overlay for urban dynamics, not as its qualitative, prognostic nor delineative indicator.

\section{Thesis Aims and Scope: Urban Complexity and Dynamics}

Urban structures interact in an environment that is constantly undergoing transitions, as they themselves are not permanent and unchangeable. As a result, this constantly influences and changes our point of view, influencing our way of solving problems that exist in our environment, as we and all of our surroundings are in a constant state of flux (\citealt{harvey_condition_2003}).
\\
Accordingly, the complexity of an urban system, which involves the unpredictable and uncertain in its structure, is bridged by emphasizing the reference to its state and corresponding urban dynamics. This approach indicates the political aspect of urban processes, not that of urban structures.  Moreover, it coincides with the political view of urban planning (\href{ref}{Friedman 1987}), though it takes a more inclusive turn with all the agents of interventions, relations and events taken into account, not withstanding their nature, function or purpose. In other words, urban development becomes reconfigured into a fine-grained urban dynamics, adding up elements to the battlefield of urban decision making, while it enables labeling the complexity of urban systems.
\\

This sort of relativism, where the interactions of as many elements as they emerge determine the context in which they are placed, should be a formative factor in addressing urban complexity and dynamics in terms of urban development processes, prospects and circumstances.

\subsection{Research Objectives: Urban Development as an Indicator at the Local Level}

This research aims to encompass the complexity and dynamics of urban transitions as an urban development indicator at the local level in a transparent way.

\subsubsection{[RO1]}
re-formulate urban development concept in terms of urban transitions to fit the idea of dynamic state of an ordinary city in its full complexity
\begin{itemize}
\item describe urban system complexity;
\item trace the morphology of decision making;
\item map urban networks;
\end{itemize}

\subsubsection{[RO2]}
investigate urban dynamics through the urbanity concept of an ordinary city
\begin{itemize}
\item elaborate urbanity;
\item connect urbanity to urban dynamics through the level of urbanity;
\item contextualize the level of urbanity in post-socialist cities;
\end{itemize}

\subsubsection{[RO3]}
conceptualize a methodological hybrid for tracing urban complexity and dynamics
\begin{itemize}
\item identify urban networks within the morphology of urban decision making
\item trace the level of urbanity to indicate urban dynamic
\end{itemize}

The theoretical stronghold of this thesis is the interpretation  of  urban  development,  namely  moving  away  from the qualitative  notion of the term to indicate  its  operational  equivalence  with  a more  neutral  and  relativistic idea of urban system transitions.
In this respect, an overall research question is: HOW TO investigate  post-socialist cities in order to reinvent a more inclusive and flexible approach to understanding urban development by engaging the complexity and dynamics of an urban context?

\subsection{Thesis built-in concepts}

Urban system transitions encompass the complexity and dynamics of an urban environment within the combination of urban decision-making, urban agency and urbanity.
This interpretation of urban development not only emphasizes its processual nature, but also moves away from its project- or model-based feature by incorporating locally contingent socio-spatial patterns (\href{ref}{\citealt{guy_understanding_2000}}) and a non-human basis of urban agency (\href{ref}{\citealt{healey_planners_1992}}; \citealt{mcfarlane_assemblage_2011}; \citealt{anderson_assemblage_2011};
\cite{healey_circuits_2013}; \citealt{rafiee_relationship_2014}).
\\

In general, the important research challenge of this thesis is testing the legitimacy of urban decision making in addressing urban development. The issue at stake is to encompass planning, power struggle, economic interests, design and participation in an overarching urban decision making procedure. Namely, the source of urban system transitions are decisions made through these various top down, bottom up and interest-based interventions, relations and events (\citealt{hudson_political_2014}).
Political and governance practices are open and susceptible to choice, through contestation and struggle, and accident, historical or natural (\href{ref}{ibid.}).
\textbf{The morphology of urban decision-making} therefore comprises and reconciles all its different layers that spread urban transitions through and across an urban system and engages certain level of forethought ({\citealt{healey_collaborative_1997}}; \href{ref}{\citealt{scott_seeing_1998}}; \href{ref}{\citealt{pierre_debating_2000}}; \href{ref}{\citealt{pierre_governance_2000}}; \href{ref}{\citealt{hudson_political_2014}}). They serve to enclose the historical continuum of global urban trends and patterns in a local socio-spatial framework and translate them into an internal, on-going interaction of individuals or constituted groups.
\\

The identified overarching decision making procedure acknowledges human agency. Through these interactions, urban actors initiate the process of their integration into the environment through an appropriation and transformation of space. In this sense, we could refer to the classical vision of cities as a setting that consists of: venues (their spatial and built environment) for social interactions (economic, political and cultural), social practices (policies and processes) and a reproduction of the social order of all urban actors (\citealt{firmino_pervasive_2008}). The way cities function shapes the expectations and actions of all the urban actors involved, who also influence the constitution of the city itself. The network of these internal and external influences between human and non-human elements engaged in urban system transitions introduces \textbf{urban agency} as a property of all key urban elements. The multitude and diversity of elements in an urban system, while emboding its dynamics, are rather blackboxing the agency of urban dynamics than decoding it.
\\

\textbf{Urbanity} is another rather blurry concept, applied often in architectural research and practice with the potential for decoding urban dynamics. In general terms, it relies on urban complexity as an active attribute of the overall state of an urban environment (\citealt{canuto_establishing_2012}). 
Moreover, this thesis argues that an overarching definition of the urbanity concept improves scientific capacity for grasping urban dynamics {\citealt{marcus_spatial_2007}}; \href{ref}{\citealt{zijderveld_theory_2011}}; \href{ref}{\citealt{canuto_establishing_2012}}; \href{ref}{\citealt{de_aguiar_douglas_vieira_what_2013}}; \href{ref}{\citealt{holden_justifying_2015}}).
It elaborates how the level of urbanity figures as an indicator for maintenance, transformation and change processes of an urban system, incorporating simultaneously its state and the transitions.
The relationship between the physicality of urban form and the social components of urban life generates the level of urbanity - the quality of continuous harmonization of the variety of structural elements, social factors and the vested interests existing in an urban environment (\citealt{de_holanda_exceptional_2011}; \citealt{canuto_establishing_2012}). 
\\

This approach views a worldwide, broad, general and mutable process of urban development that is actually connected to place. Appreciating an actual urban micro setting is a vital factor to understand case specific uncertainties and creates a polygon for the transformation of the global aspects to meet local specificities.  

\subsection{Methodology as a starting point}

Following contemporary relativist trends for rethinking space, time, globalization and cities, the future research challenge can be defined as "visualizing cities as unformed, unorganized, non-stratified, always in the process of formation and deformation, eluding fixed categories, transient nomad space-time that does not dissect the city into either segments and ‘things’ or structures and processes" (\href{ref}{\citealt{smith_world_2003}:574}). Accordingly, a corresponding change in approaching urban development can then be addressed by the heterogeneous iterative approach that encompass the complexity and dynamics of cities for the improvement of living conditions and the facilitation of social interactions in cities. 
\\

Bearing in mind the complexity of such a relativistic approach to the urban and the necessity of practice-oriented knowledge, this thesis proposes a mixed-method case-study approach (\href{ref}{\citealt{denzin_case_2005}}). According to \href{Kuhn}{\cite{kuhn_structure_1962}} paradigm shift (1962), the science about the city is a constantly swinging pendulum between the scientific and hermeneutics approach - quantitative analysis vs. descriptive study (\href{ref}{\citealt{portugali_complexity_2011}}). The mixed research method in this case provides complementary information and in-depth knowledge of the problem. However, it has been solely moulded according to qualitative data sets. 
\\

The choice of methodologies is justified by the process-driven, correlational research design and the exploratory character of the research itself. In this regard, the thesis suggests the potential of the combination of a Multi-Agent System (MAS) and Actor-Network Theory (ANT) methodologies. ANT has been extensively applied in sociology for the analysis of cities and the urban, while MAS itself is a more mathematical-computational method for agent-based modelings. The MAS-ANT hybrid methodology herein serves to capture local urban dynamics and reframe the complexity of permanent urban system transitions for urban development. This argument is built on the usefulness of ANT for describing the urban reality. It will then be demonstrated that MAS adds the framework of action when applied over ANT.
\\

Finally, its application is presented in the case study of a post-socialist neighbourhood in Belgrade. In this case, the researcher had the opportunity to be educated in Belgrade and to work in the architectural production in the Serbian capital. Therefore, the researcher is to some extent familiar with the local context and has the ability to access relevant data.
%DIAGRAM

\section{Contribution}

My aim with this dissertation is to mark a path for move away from the general theoretical research into an on-site practice-based investigation. Consequently, this study attempts to show how a real-life focus on the Savamala neighbourhood in Belgrade eventually increases the body
of knowledge on urban development processes, the post-socialist urban environment and the methods used to deal with complex and dynamic urban contexts.
\\

The envisoned research findings are three-folded.
First of all, the conclusions addressing actors and processes at play in Savamala.
Then the scope of several theoretical constructs is reconsidered for describing and guiding urban processes in cities outside the Global North.
The empirical results of this research contribute to the operational interpretation of urban development, in terms of value-neutral urban system transitions, and the level of urbanity, in reference to contextualized socio-spatial patterns.
Finally, visualizing data through MAS-ANT methodological approach expresses an attempt to depict the complexity of urban actors, forces and artifacts and the dynamics of networks, interdependences and processes to a legible, data-loaded scheme of nodes and links. This research contributes to responding to the necessity to shift the deterministic concept of how to approach urban research to a more comprehensive, network-oriented vision in terms of finding an intermediary between the qualitative data analysis and the data display closely related to modern means of communication.

\section{Thesis Structure}

The study is structured in seven chapters.
\\
This chapter (\textbf{CHAPTER 1}) sets the path for reaching the research objectives. Its crucial role is to provide a basic understanding and scientific justification of what forms and conditions urban complexity and dynamics and how the problem is approached within the limits of this research. The next \textbf{CHAPTER 2} contains an extensive literature review concerning the applicable concepts and the chosen methodologies. These concepts form the essence for categorizations with the chosen methods.
The conceptual and methodological parts build the theoretical framework for this thesis. 
\\
\textbf{CHAPTER 3} relies on the primary statements from this introductory chapter, builds on the range of indicators identified within the theoretical framework and further elaborates the methodological approach and the scientific argument of the research.
\\
In order to substantiate the proposed hypothesis, the presented theoretical  framework  will  be tested on an elucidated case study. In \textbf{CHAPTER 4} the choice of the Savamala neighbourhood in Belgrade is clarified and data collection procedures are summarized in the form of a linear and chronological case study report.
\\
The following chapter moves forward to hypothesis testing and consecutive application of the chosen research methodologies. Data analysis with the Actor-network theory (ANT) is the core of \textbf{CHAPTER 5} which addresses the issue of urban complexity.
\\
\textbf{CHAPTER 6} deals with system building according to the postulates of the Multi-Agent system (MAS) and provides the elaboration of urban dynamics.
\\
\textbf{CHAPTER 7} presents the actual hybridization of two methods (MAS-ANT) and display the data on urban development processes.
\\
In \textbf{CHAPTER 8} the conducted research is pulled together. The resulting discussion is drawn upon the outlined background information on the theoretical framework, the deconstructed MAS-ANT methodological hybrid, and the collected and analysed data on the case study from the previous chapters. Based on these results, this thesis is concluded on two separate levels, regarding the research and theoretical framework. The final results are also presented in the wider perspective concerning its practical application and the limits and potential of its framework.

In parallel, the story of Savamala urban development is narrated in pictures of historical processes that bound localized actors and networks across space and time.


%%%%%%%%%%%%%%%%%%%%%%%%%%%%%%%%%%%%%%%%%%%%%%%%%%

\chapter{Urban Development - Theory, Methodology and Context}

%%%%%%%%%%%%%%%%%%%%%%%%%%%%%%%%%%%%%%%%%%%%%%%%%%

Cities are rather primary venues, power poles and capacity builders of economic, social and cultural development at stake in modern societies (\href{ref}{\citealt{castells_urban_1979}}).
Cities are the polygon of contemporary decision making.
Even more so, cities are dynamic and diverse urban entities that are given to shaping their autonomous and innovative future on the basis of human resources and creative human potential (\href{ref}{\citealt{knight_cities_1989}}; \href{ref}{\citealt{yigitcanlar_knowledge-based_2008}}).
Contemporary cities are the source of both problems and solutions of contemporary life.
\\

The prosperity of cities depends on how competitive they are on a global economic scale, how flexible they are in terms of adjusting to current trends and needs, and how fertile they are for the development of knowledge and the application of innovation. These major uncertainties of contemporary life, created mainly but not exclusively by the current method of production and management, are acutely symbolized by concerns about urban development (\href{ref}{\citealt{healey_collaborative_1997}}).
\\

This chapter aims to provide a critical overview of the relevant and explanatory theories to help provide an interpretation of the urban development prospects in a post-socialist neighbourhood in Belgrade. These theories are critically reviewed through separate conceptual, methodological and contextual investigations respectively.

\section{Conceptual Framework}
%visualization

Following the methodological focus of this research project, the main concepts outlined in the introduction lead through a theoretical investigation into the current body of scientific knowledge to see to what extent they can principally help fulfill the research objectives. The conceptual framework structurally builds the bases of phenomena, facts and theories that operationalize the main concepts identified by explaining how they are converted into indicators, and from indicators into dependent and independent variables.
\\

Starting with the general notion of urban development, the narrative is constructed from a critique   of its focus on economic growth towards its processual and more inclusive nature. In this reference it also brings forth the ordinary city, a theoretical stance that emphasizes no qualitative difference among cities according to their geographical location, history and economic prosperity. Therefore, urban complexity is deconstructed in terms of urban agency and urban decision-making. Urban agency (\href{Healey}{\citealt{healey_institutional_1992}}; \href{ref}{\citealt{mcfarlane_assemblage_2011}}; \href{ref}{\citealt{delanda_new_2013}}) sets in motion urban environments.
Its distribution in cities is governed through the complex organizational logic of urban decision-making (\href{ref}{\citealt{putnam_what_1993}}; \href{ref}{\citealt{hall_political_1996}}; \href{ref}{\citealt{healey_collaborative_1997}}; \href{ref}{\citealt{scott_seeing_1998}}; \href{ref}{\citealt{pierre_debating_2000}}; \href{ref}{\citealt{pierre_governance_2000}}; \href{ref}{\citealt{hudson_political_2014}}).
On the other hand, the concept of urbanity (\href{ref}{\citealt{haussermann_urbanitet:_1992}}; \href{ref}{\citealt{gronlund_notions_2007}}; \href{ref}{\citealt{marcus_spatial_2007}}; \href{ref}{\citealt{zijderveld_theory_2011}}; \href{ref}{\citealt{canuto_establishing_2012}}; \href{ref}{\citealt{de_aguiar_douglas_vieira_what_2013}}; \href{ref}{\citealt{holden_justifying_2015}}) was the point of departure for addressing urban dynamics by linking urban agency to socio-spatial references as the source of qualitative urban transitions.
\\

In sum, this section traces the systematic view on urban complexity and dynamics within urban theory in order to form the basis, not of a goal-oriented, but rather of a process-oriented, value neutral vision of urban development.

\subsection{What Urban Development stands for?}

Urban development is a widely accepted although also contested category, usually associated with urbanisation processes in the "so-called" developing countries. Many professionals in urban research and practice use the term, overlooking the great number of people around the world affected by the term.   Nevertheless, the notion of the word "development" itself means different things to different people.
The quest for recapitulation of the urban development concept undoubtedly maintains a reference to something that was its initial concept. The phrase "urban development" consists of the words urban and development, both widespread and crammed with connotation, but loose and less inefficient when used in the narrative of actions (\href{Bolay}{\citealt{bolay_slums_2006}}; \href{Bolay}{\citealt{bolay_changements_2010}}; \href{Bolay}{\citealt{bolay_what_2012}};\href{Bolay}{\citealt{bolay_technology_2011}}).
\\

The urban is a self-contained theoretical term, yet, according to its use (\href{Brenner}{\citealt{brenner_urban_2014}}) in practice it is not a pre-given, self-evident reality, condition or form. 
However, following the authors' explanations the urban is a territorially bounded historical process where the population is concentrated and where social relations unfold (\href{Brenner}{ibid.}).
\href{Soja}{\cite{burdett_urbanization_2006}} explain the urban as a way of life, binding its spatial and social dimensions together.
Taking into account its processual nature and socio-spatial polymorphism, the urban phenomenon is acknowledged for its fluid, extensive and variable actualization in space-time (\href{Brenner}{\citealt{brenner_urban_2014}}).
\\

Concomitantly, this research followes (\href{Esteva}{\citealt{esteva_development_2010}}) explanations of development, which state that this powerful semantic constellation does not show the same eagerness in producing the substance and meaning applicable in practice. According to the author (\href{Esteva}{ibid.}), through the historical distortion of the term primarily applied in biology, development has become a conservative political project that does not effectively contribute to the visions and futures it agitates for. 
\\

In biology, development explains the process through which organisms achieve their mature natural form. Darwin already used the terms evolution and development interchangeably  (\href{Esteva}{ibid.}).
Consequently, when transferred to the social sphere in the 18th century, development was already assumed as a pursuit of appropriateness and perfection.
Harvey specifies that development is inseparable from capitalism (\href{Harvey}{\citealt{harvey_urban_1978}}).
Connecting development to capitalist ideology endows it with a set of social, economic, political, and spatial implications.
In an overarching development, taxonomy attributes it with structural transformation (economic), human development, democratic participation and improved governance and environmental sustainability (\href{Vázquez}{\citealt{vazquez_revisiting_2013}}).
\\

In its 200 year long history, the concept has been significantly updated and modified mainly in the course of development and environmental studies (\href{Vujosevic}{\citealt{vujosevic_novi_2012}}). 
%novi regionalizam knjiga 1
In this reference, \href{Walker}{\cite{walker_capitalist_1989}}
%ovde ostaje cite!!!!!!!!!!!!!!!!!!!!
discuss industrial and social development, and scientists and practitioners in the world also talk about capitalist, economic, spatial and metropolitan development, uneven development, climate change and development, and so forth. Another developmental catchword was that of sustainable development. Its theoretical scope tends to reconcile opposites growing around the restrictive requirements of the environment on one side, and, on the other, the economy devouring borders and different systems into a flexible, workable unity  (\href{ref}{\citealt{bolay_what_2012}}).
The operational definition that brings together environmental and socio-economic questions was expressed as meeting ‘the needs of the present without compromising the ability of future generations to meet their needs’ (\href{WCED}{WCED 1987}:43).
%ne diraj!!!!!!!!!
Setting development in the time perspective like this, brings after all a steady, anthropocentric frame of reference (\href{Lee}{\citealt{lee_global_2000}}), which, in a certain sense, defies the initial association of development with mainly economic growth.
\\

Being bounded in the Millennium Development Goals (MDGs), sustainability became an agenda of international pursuit (\href{MDG}{\citealt{united_nations_about_2000}}).
Placed as a goal on a vast global scale, development was in need of ever better adjustments to the local context and continual update of scientific and technological trends. Therefore, the leading paradigm sets it in the domain of moral and practical commitments, bottom-up crowd-sourcing activity, technological and social improvements and global mobilization of knowledge (\href{Sachs}{\citealt{sachs_millennium_2012}}).
\\

Still, in comparison to its all-inclusive definition, when turned into practice, sustainability also meets its limits (\href{ref}{\citealt{bolay_what_2012}}).
After multiple results and setbacks and an increasing number of field-research and action-plans, the developmental agenda has grown from the initial 8 millennium development goals (MDGs) into 17 sustainable development goals (SDGs) of the post-2015 Development Agenda (\href{SDG}{\citealt{un_post_2013}}; \href{Sachs}{\citealt{sachs_age_2015}}).
\footnote{While being critical of the MDGs in the first place, experts are even more divided about the possible effectiveness of such an extension, describing it as naively idealistic and in this sense possibly even more liable to deviations and abandonment in practice.
(\href{FinancialTimes}{\citealt{financial_times_experts_2015}})}
\\

In favour of sustainable development as a workable concept is the claim from several experts that sustainability with its environmental and timeless human awareness threatens the capitalist order of things 
(\href{Christie}{\citealt{christie_here_2001}}; \href{Hopwood}{\citealt{hopwood_sustainable_2005}}).
Even though essentially humanitarian in its meaning, the openness and lack of rigor in the definition has allowed for ambiguous interpretations. As a matter of fact, some point out that this overarching proposition in reality enables business and governments to be in favour of sustainability without any fundamental challenge to their present course  (\href{Hopwood}{\citealt{hopwood_sustainable_2005}}) and easily makes it a synonym for "sustainable growth" (\href{Daly}{\citealt{daly_sustainable_1992}}; \href{Rees}{\citealt{hamm_understanding_2001}}; \href{Dollar}{\citealt{dollar_growth_2002}}).
An extreme example in this direction may be the case of Serbia, where abstract and general criteria and principles of sustainability have been easily introduced into strategic and planning documents, but only on paper and without any actual translations of the terms into implementation-oriented regulations (\href{Vujosevic}{\citealt{vujosevic_novi_2012}}). 
%novi regionalizam knjiga 1
\\

Continually criticized and easily abandoned, these specializations of development evidence the everlasting scientific trend to follow the terminology of policies and programmes and further theoretize, improve and shift the concept.
\footnote{Recent shifts from sustainability towards urban resilience  demonstrate at least etymological alteration from "developing something"/"something is developing" towards adapting and empowering, even though the term development (and also sustainable development) is still present in the explanations and discussions of any rising concept in the domain {\cite{ref web of science}}.}
Moreover, these many attributions of development suggest a disciplinary tuning of the term, yet one thing is sure from the perspective of various disciplines - seeing the term as growth and a "growth machine",
\footnote{Growth also in a sense of betterment and improvement in the social, not only the economic sense. However, in practice it has been shown that global standards of good/satisfying/desirable are either shaped according to some, and not all, or, when presented in the global arena, cannot be devoid of its bonds to the economy’s gross national product.} 
is an unavoidable interpretation in discussing development (\href{Gottdiener}{\citealt{gottdiener_social_2010}}).
Moreover, the words development and growth are very used interchangeably across the social sciences and even more so with hardly any critical recapitulation of this matching.
\\

Subsequently, urban development also involves the ideology of growth, but in reference to space, spatial organization, space in social organization, urban space.
While some other concepts can be more suitable and workable (sustainability, resilience etc.), development and urban development are used very often and even more so as a goal positive in itself without explicating the exact meaning.
With the bonds to pro-growth boosterism, the production of space is necessarily marked as capitalist, even though capitalist relations are not reflected directly in urban forms (\href{Gottdiener}{ibid.}). 
\\

However, Marxist theory of urban development emphasizes that economic and political processes of technology, labour power and existing relations of production in cities entail contradictions and crises pertaining to the built environment and class struggle (\href{Harvey}{\citealt{harvey_urban_1978}}).
According to sophisticated architectural rhetoric, capitalist growth is also a physical act of land interventions (presence of bulldozers and modern construction materials, high-rise buildings and costly designs).
Conversely, in planning terms, it involves de-concentration and de-centralization (suburbanization and urban sprawl as the other side of the coin) (\href{Gottdiener}{\citealt{gottdiener_social_2010}}).
\\

As explained herein, there are different and numerous interpretations of what is and what should be development in general and concurrently urban development in this case.
Following the growth paradigm, the most dominant yet biased one, \href{Galtung}{\cite{galtung_peace_1996}}
%ostaje cite!!!!!!!!!!
points out several readings of what development is:

\begin{itemize}
\item the spread of the capitalist system and its values;
\item a bureaucratic evaluation of the success of projects that compete for development subsidies under the slogans of health and safety, participation, inclusion, equity, the poor etc.;
\item an appropriation of the history of civilization by the west, where western development becomes modernization, and for the rest of the world it is GNP;
\footnote{gross national product}
growth; 
\item the appropriation of economic growth to certain regions (Europe, North America, India, China), while the rest of the world is condemned to a periphery status;
\item development assistance coming from imperialist and missionary traditions.
\end{itemize}

All these interpretations, extreme in attitude, not only glorify growth, but directly define development by its opposite, the connection to the underdeveloped other. Development is therein seen as a polarized, clustering scale, following the new distribution of power after WWII. 
\footnote{The reference to underdeveloped changed the meaning of development after WWII, when President Truman in a speech put forth the thesis of making available industrial and scientific achievements (those of US and western countries) to the rest of the world (\href{Esteva}{\citealt{esteva_development_2010}}).}
Thenceforth, development has been marked as a ceaseless pursuit of escaping underdevelopment (\href{Esteva}{\citealt{esteva_development_2010}}).
\\

Instead of these practical social and economic aspects of the pursuit of gratification, the division of countries into clusters and profit orientation, the terminological core of the term development is a holistic and dynamic approach to the human condition (\href{Galtung}{\citealt{galtung_peace_1996}}). 
In accordance, its tangible definition should be contextually constructed within not single single global development, but as a sum of different developments and within its self-developmental, not transitive essence
\footnote{These interpretations are based on Galtung's grammatical explanations that development as a noun has also plural form and as a verb it is an intransitive or reflexive or reciprocal verb, but not in any case a transitive verb.}
(\href{Galtung}{ibid.}).
This interpretation shows the well-known center-periphery dichotomy in development pursuit to be obsolete  (\href{Robinson}{\citealt{robinson_ordinary_2006}}).
\\

And even if the term entails any kind of relationship or even assistance, those should be reciprocal. Development also holds a culture-centered notion of the unfolding of a culture and a needs-centered reference towards meeting the needs of the human and natural worlds  (\href{Galtung}{\citealt{galtung_peace_1996}}).
These considerations in local development both collide with the initial ideal of development as growth. Having said that, urban development is not a goal, but a process. From the growth perspective it implies favourable change, a transition to the superior and the better, as though there is a necessary, universal law by which to judge it (\href{Esteva}{\citealt{esteva_development_2010}}).
Even in its essential, value neutral sense of processual transition, development may be the result of radical change, societal or spatial.
Destitute of its simplistic market explanation, radical change therefore entails the redistribution of the population, social order or physical structures that challenges the ontologically fixed city organization (\href{Brenner}{\citealt{brenner_urban_2014}}).
Radical change is a dramatic forward-motion.
\\

On the hand, development is also a relentless churning of city morphology and management, which may be referred to as a continual socio-spatial transformation  (\href{Brenner}{ibid.}).
Furthermore, keeping a distance from the term growth and addressing its process in non-normative way should also be a part of an overall "right to development"
\footnote{From the perspective of this research, the official \href{UN}{\citealt{united_nations_declaration_1986}} "Declaration on the Right to Development" references the notion of growth and development-underdeveloped dichotomy.}
to be chosen by those who perform the self-development and who are also allowed to choose to maintain the system as is.
This means that urban development seen as a distribution of urban transitions incorporates the basic dynamics of territorial/urban systems: maintenance, transformation and change (\href{Friedmann}{\citealt{friedmann_planning_1987}}).
\\

\textbf{Urban development is treated as a process of urban system transitions over time.}
The term "transition"
\footnote{This interpretation is partly based on the theory of transition from the democratization theory, but it is not herein applied in terms of its primarily political orientation (\href{Offe}{\citealt{Offe_varieties_1997}}), but rather in the general sense of the word - "transition is the process or a period of changing from one state or condition to another" (\href{Oxford}{Oxford Dictionary}).}
is herein used to refer to a process that leads from one state of the urban system to another
(\href{Stark}{\citealt{stark_system_1992}}).
In this sense it holds the clear categorization of its pre-transitional situation - the deterministic influence of the past (i.e. path dependency) - and it includes the moment of discontinuity (\href{Thomas}{\citealt{thomas_thinking_1998}}) or the reflected continuity (\href{Nedovic}{\citealt{nedovic-budic_mornings_2011}}).
%Mornings after Nedovic Budic
\\

For better understanding, putting the equation sign between urban development and system transitions, as well as its reference to maintenance, transformation and change of the urban system should be elaborated in terms of how the transitions are qualified as continual (maintenance) vs transformative (transformation) or evolutionary (change), what determines start and end points and the moment of discontinuity / reflected continuity (\href{Nedovic}{ibid.}).
For \href{Friedmann}{\cite{friedmann_planning_1987}}
%NE DIRAJ OVDE CITE!!!!!!!!!!!!!!!!!!
maintaining forces are practices.
\footnote{In his planning oriented concept, he refers to them as usually bureaucratic in nature and articulated by the state (\href{Friedmann}{\citealt{friedmann_planning_1987}}).}
He further elaborates systemic change as a process riddled with conflict and compromise, when radical proposals become integrated within the structure of the guidance system in a society (\href{Friedmann}{ibid.}).
According to the same author, transformation is the tolerated disruptive action that is a legitimate part of the established political order (\href{Friedmann}{ibid.}).
On the other hand, to define the space-time reference of transition (start, end and the moment) it is important to take into account: its constant contingency, the complex contextual influences, the directional proceedings, continuity and its incremental nature.
\\    

Thus, it is necessary to shift the deterministic concept of urban development to a more comprehensive vision that considers complex networks and their dynamic interfaces and generates better understanding and strategizing of urban development in practice (\href{Huang}{\citealt{huang_ict-oriented_2012}}).
Apart from the confusing mix of global and local influences, the complexity of such stand-alone artifacts is encumbered with layers of infrastructure that progressively interweave and infiltrate the urban systems, life and culture in cities (\href{Graham}{\citealt{graham_end_1998}}; \href{Portugali}{\citealt{portugali_complexity_2011}}).
The powers of such networking support a complex restructuring of urban elements, along with a capacity for recombining economic, political, cultural, technical or natural factors (\href{Murdoch}{\citealt{murdoch_spaces_1998}}).
Such urban heterogeneity consists of operationalization, interrelation and interaction of socio-technical assemblies within a city (\href{Graham}{\citealt{graham_splintering_2001}}).
These become extended over the times and spaces of urban life (\href{Mitchell}{\citealt{mitchell_city_1996}}) and offer us an opportunity to construct dynamic, sophisticated and synthesized approaches to contemporary urban development.
\\

Consequently, cities are now in a constant state of flux, with the rapid adjustment of their physical, economic, social, and political structures (\href{Sykola}{\citealt{sykora_transitional_1999}}) to the information flows and infrastructural scapes.
The urban present is no longer attributed only to spatial forms, economic units and cultural formations, but also to integral and complex socio-material and socio-technical systems in cities that are attributed agency (\href{Farias}{\citealt{farias_introduction:_2011}}). 
\\

\textbf{Urban system transitions both stem from and affect the range of human and non-human elements that are thusly attributed agency in an urban environment.}
With its complex system of agents and their relations, cities are stand-alone artifacts (\href{Portugali}{\citealt{portugali_complexity_2011}}).
However, each of these agents influence the state of the urban system at a certain scale. 
Therefore, urban transitions happen in the course of the interrelationships between the current state of the system and the active agency (\href{Guy}{\citealt{guy_understanding_2000}}).
Moreover, as a consequence of transitions a layer of agency becomes an additional layer of the system (\href{Guy}{ibid.}).
Therefore, the ebb and flow of urban system transitions is rooted in the current complexity of the system, but they are also the formative factor of its future complexity.
\\

An orientation towards the future is central to the global society nowadays and even more so is its orientation towards a positive vision of the future, whatever that might be. In an organized system such as an urban one, this insistence on a future-oriented perspective of actions corresponds to that of risk, where the control of all future events is calculable and predictable in probabilistic terms. In realty, all urban agents actively respond to these everlasting condition changes which imply that, when actions target a positive future, uncertainty must be accepted and managed. In these circumstance, the morphology of how decisions are made represents how the system deals with this notion of an open-ended future. 
\\

\textbf{Urban system transitions are the consequence of urban decision making.}
Cities are constantly forming and reforming by human and non-human agency (\href{Portugali}{\citealt{portugali_complexity_2012}}).
%Complexity theory of cities
This also means that urban system transitions are ever present. Urban development is a continuous, unstoppable process, while separate transitions can be identified according to: (a) the description of the pertaining agencies, (b) identification of their roles and relations, (c) assessment of strategies and interests associated to these role, (c) how these roles are shaped by resources, rules and ideas, and (d) the relation between these resources, rules and ideas to the local social aspects  (\href{Healey}{\citealt{healey_institutional_1992}}).
In this sense, it is important to acknowledge that diminishing the growth paradigm of development does not mean the negligence of power relations that dominate the modern urban context. 
\\

Today, the term "governance" is more benevolently used and evaluated as more suitable than that of decision making when investigating how cities function.
Yet, this research project purposely avoids it.
\footnote{This standpoint and the terminological choice of decision-making instead of governance will be further developed in \href{Section 2.1.4}{Section 2.1.4}} 
Governance deals with strategies, tactics and operations on how to steer the economy and society in order to reach [common] goals (\href{Pierre}{\citealt{pierre_governance_2000}}).
However, for the topic of urban transitions, the process that precedes is not of importance - when the decision is set, then urban system transitions, whichever they are, start. Therefore, the morphology of urban decision-making, in the sense of a classification based on form, structure and function of decisions and the process ahead, explains the chronology of system transitions (beginning with a decision, discontinuity / reflected continuity, and its ending point with a new decision).
\\

Debates on sustainable development add a necessary connection to space and concerns for the natural environment, so that urban governance aims to combine environmental and socio-economic issues in determining priorities and actions (\href{ORiordan}{\citealt{oriordan_challenge_1989}};\href{Hopwood}{\citealt{hopwood_sustainable_2005}}).
However, even without the considerations for environmental effects, any urban decision influences the context, environment, space and reverberates through society.
"It is the ongoing development of that system which comes to be materialized in space at any given time, so that observable patterns of socio-spatial organization are [its] phenomenal forms" (\href{Gottdiener}{\citealt{gottdiener_social_2010}}).
The urban development process involves social and spatial relations to urban agency through the morphology of urban decision making.
\\

\textbf{Urban system transitions are contextually rooted processes that affect the space and society of local urban systems.}
In this sense, the organization of space and social relation are not independent categories, they are but a product of system dialectics, urban transitions that are simultaneously social and spatial (\href{Soja}{\citealt{soja_socio-spatial_1980}}). They are intrinsically local, stemming from context-based urban agency and urban decision-making, so that they become patterned (\href{Gottdiener}{\citealt{gottdiener_social_2010}}).
As a result, socio-spatial patterns are the product of the uncoordinated nature of urban processes through localized urban agency.
\\

Socio-spatial patterns of urban transitions is a provisory term  that contributes to develop an understanding of development processes beyond mere strategic economic and social framing of needs and events and taking into account sporadic and spontaneous agencies of urban systems. The sensitivity to this range of needs, events and agencies means that whatever refers to the state of an urban system and the processes of maintenance, transformation and/or change. They further entail urban system transitions within the morphology of urban decision-making (\href{Gottdiener}{ibid.}).
\\

Based on the previous argument built in this section, urban development as a process of systemic transitions in cities is enacted through urban agency within a complex body of urban decision-making.
In terms of sustainable development, socially and spatially restricted decisions, which come out of the balance/imbalance of local environmental and socio-economic sides, result in the status quo (the actions achieved within the current system), reform (rupture and fundamental change of the existing arrangements) or radical transformation (high level, systematic change) of an urban system 
(\href{Rees}{\citealt{hamm_understanding_2001}}; \href{Hopwood}{\citealt{hopwood_sustainable_2005}}).
\footnote{
The terminological framework of status quo, reform and transformation corresponds to that of maintenance, transformation and change.
However, the preference for the second stems from its neutrality in a terminological sense.
The first in fact has a value judgment incorporated slightly into it. For example, status quo refers to the existing state of affairs and nominally it implies system maintenance. However, it can also `refer to a situation that people find mutually undesirable but the outcome of any changes to it may be overly risky; at the same time they recognize that eventual change will occur, and openness to the potential that a better alternative solution may emerge over time.` (\href{WorldHeritageEncyclopedia}{World Heritage Encyclopedia 2001}). And similar works for the word reform.}
\\

Both reform and transformation imply change.
A reform as a programmed change usually assumed to be positive in its intention or marked as developmental if it has positive economic or, less often, social outcomes. In reality, urban change is most often the consequence of power struggle and has conflictive outcomes on different stakeholder groups (\citealt{fainstein_just_2010}). Yet it has been bounded only spatially - referring to a city or a part of the city. Further, today’s solution may be the conflict of tomorrow  ({\citealt{holden_justifying_2015}).
\\

In general, balanced relations are reproduced through urban practices, while uncoordinated and contradictory ones germinate into urban conflicts. Within this dynamics, the state of practices-conflicts proportion reveals the level of urbanity of the local context. Urban conflicts manifest themselves in space through contradictory, contentious processes (radical change), and urban practices reproduce the necessary functionality of the system (maintenance). What is more, there is a certain potentiality in the local contextual resources (social and spatial) to override this opposition and induce cycles of exchange (transformation) determined by the mature system logic (\href{Galtung}{\citealt{galtung_peace_1996}}).
\\  

In sum, urban development is anything what may happen to a city in terms of maintenance, transformation and change of its original state (\href{Friedmann}{\citealt{friedmann_planning_1987}}).
Such a context implies that physical spaces are constantly intermingling with the social constructions of these spaces (\href{Firmino}{\citealt{firmino_pervasive_2008}}), binding the territory and social aspects, agency and decision-making together.
The "city" concept is thereafter redirected from spatially bounded, people-centred phenomenon to dynamic and complex urban systems, which in their incompleteness and indeterminacy, are stages where all urban elements participate in their "making", changing and transforming.
This complex dynamics involves space and time warp and overlap.
In other words, a city is perceived as a nexus that balances relational proximity in a fast-moving world with ‘time-space extensibility’.
In cities, all human actors and material objects engaged in networks extend beyond the immediate corporeal environment (\href{Graham}{\citealt{graham_splintering_2001}}).  

\textbf{"By the beginning of the twentieth century, a new use of the term became widespread. ‘Urban development‘ has stood, since then, for a specific manner of reformulation of urban surroundings, based on the bulldozer and the massive, homogeneous industrial production of urban spaces and specialized installations." \href{Esteva}{\cite{esteva_development_2010}}}
%ostavi CITE

\subsection{An Ordinary City in a Constant State of Change}

The city is perceived as a dynamic entity which embodies the social narrative and the attempts to govern its social interactions and the spatial distribution happening within the intersection of past, present and future.
\\

In traditional urban theory, cities have been conceptualized as a natural habitat for civilized people (\href{Park}{\citealt{park_human_1952}}), the indicators 
of an urbanized society (\href{Mumford}{\citealt{mumford_city_1961}}), a symbolic expression of the progress of our present time (\href{Osborne}{\citealt{osborne_modernity_1992}}), and a 
framework for social life and its subsequent milestone of collective identity represented by the human factor in a built environment (\href{Berman}{\citealt{berman_all_1988}}).
However, because of the radical social and swift physical transformations that occur ever more rapidly as the new millennium continues to unfold, the city has slowly developed into a place of consumption just as it has started to decline as a centre of manufacturing. The modern city is a territory but it also encloses underlying relations between capital, labour and the state (\href{Castells}{\citealt{castells_city_1983}}).
With the passing of time, the city has become a venue for people to express themselves as the citizens of a consumer society (\href{Slater}{\citealt{slater_consumer_2003}}), attaining economic independence and importance in the course of this process (\href{Miles}{\citealt{miles_spaces_2010}}).
This is the capitalist city oriented towards growth and wealth creation (\href{Harvey}{\citealt{harvey_urban_1978}}).
\\   

This commodification robs cities of their particular identities and turns them into clones (\href{Miles}{\citealt{miles_spaces_2010}}).
Cities are no longer perceived as geographical entities with their own distinct identities.
They have become places of extraordinary circulation and 
interconnectivity without any clear boundaries, and the essence of urbanity has stretched beyond their spatial proximity (\href{Amin}{\citealt{amin_globalization_1996}}),  leading  to  a  "compression  of  the  world"  (\href{Robertson}{\citealt{robertson_globalization:_1992}}) or  to  put  it  in  other  words,  "world  shrinkage"
(\href{Larsson}{\citealt{larsson_race_2001}}).
This is a vivid example of rapid globalization that intensifies the perception of the world as a unitary, world society (\href{Albrow}{\citealt{albrow_globalization_1990}}) of wide social relations and interconnections (\href{Giddens}{\citealt{giddens_consequences_1992}}). 
\\

The extensive urbanization of the landscape and globalization shrinkage of social conditions to meet certain standards and trends more than ever puts the pressure on cities to fit into an overall classification and compete for a position in the overall growth machine that dominates the world under the capitalist order.
In these circumstances, urban research as well as its practice aims to tame urban dynamics and reduce urban complexity in order to provide the desirable results
(\href{Jacobs}{\citealt{jacobs_cities_1985}}).
%Jacobs 1984 Cities and the wealth of nations
The overarching classification and the overall goals suppress the intrinsic complexity and dynamics of cities and push for a predictable future which could be chosen  from a menu. However, what actually happens in most cities, especially those outside the Western world is rather a multiplicity and variety of possibilities that come up from the particularities of the local urban experience (\href{Amin}{\citealt{amin_good_2006}}).
\\

The problem might be obvious - the danger of overemphasizing and overgeneralizing from particular entities/characteristics/relations (\href{Thrift}{\citealt{thrift_not_2000}}).
%Nigel Thrift (1996a) Not a straight line but a curve
Following the theoretical concept of ordinary cities proposed by 
\href{Amin}{\cite{amin_ordinary_1997}}
%ne diraj cite!!!!!!!!!!!!!!!!!!!!!!!!!!!!!!
to represent a locally molded, just and genuine city, \href{Robinson}{\cite{robinson_ordinary_2006}}
%NE DIRAJ OVDE CITE!!!!!!!!!!!!!!!!!!
clarifies the cosmopolitan tactics for surpassing the hierarchical categorization of cities in the world, which in terms of modernity and development kept them apart. Her main point was to avoid the hegemony of western urban theory, as well as the growing strength of a discourse of development, which from the 1970s onward has emphasized the differences between cities in the west and elsewhere (\href{Robinson}{ibid.}). 
\\

The essential  resources  for  thinking  about  cities  was  the relation  between  modernity  and its  excluded  other  (tradition,  primitivism  and  difference),  as  it  was represented  in  the  dawn  of modern Western  urban  theory  of  the Chicago  school  thinkers  (\href{Park}{\citealt{park_human_1952}}; \href{Wirth}{\citealt{wirth_urbanism_1938}})
%Robert  Ezra Park and Louis Wirth)
and a German sociologist \href{Simmel}{\cite{simmel_metropolis_1971}}.
%NEK OSTANE CITE OVDE!!!!!!!!! 
The constitution of Western urban thought further on was threaded within a dialectics of modernity and tradition represented in the work of \href{Benjamin}{Walter Benjamin (1935)}.
%NE DIRAJ!!!!!!!!!!!!!!!!!
While influenced by Marxism, he agitated 
for revolutionary consciousness about the problems of the present city and wrote about the dynamic, potentially transformative interplay of modern innovations (new technologies, commodities and inventions).
Accordingly, he emphasizes that all cities that occupy the same historical time, while being open  to  various  interconnections,  contribute  to  the  variety  of  inventive  modernities  of  the  present 
(\href{Robinson}{\citealt{robinson_ordinary_2006}}). 
\\

The traditional and widespread interpretation complies with the western paradigm of development: modernisation and economic growth.
Interpreted in this way, the notion of urban development actually promotes the leading hierarchies and categorization of cities in the world, based on the impact of globalization, new transnational economic progress and the networking of cities ref. Both of these approaches impose the hierarchical relation among them, as (\href{Robinson}{\citealt{robinson_ordinary_2006}}) bluntly puts it, "while some are exemplars and others are imitators".
Besides, the chronological paradigm of western urban planning is diluted when it is spatially translated to these qualitatively different environments (\href{Robinson}{\citealt{robinson_global_2002}}), causing them to lose their substance as an urban phenomenon through the ill-decoded application of western patterns (\href{Bolay}{\citealt{bolay_urban_2004}}). In addition, modern urban thought could be stuck in this rut, inducing negative background effects on a whole gamut of urban activities (\href{Amin}{\citealt{amin_ordinary_1997}}), causing urban conflicts to thrive on the basis of inequitable power relationships, and cultural differences, as they develop from an individual level towards a socio-urban dimension (\href{UN}{UN Habitat 2009}).
%ne diraj ovo bez ref
\\

This line of thought belongs to the critics of colonial urbanism and the idea that  urban  modernity belongs to the cities of the West, while development is left to the other cities as "third world cities" (\href{Robinson}{ibid.}).
In this respect, Jennifer Robinson summarizes that the categorization and differentiation of cities according to Modernity and Development is a pure product of the colonial past.
\footnote{
In terms of the reference to the experience of colonialism, \href{Scott}{\citealt{scott_nature_2015}} claim that the theory has also appropriated its theoretical base from the early postcolonial urban theory of \href{Abu­Lughod}{\citealt{Abu­Lughod (1965)}}, \href{Jacobs}{\citealt{Jacobs (1998)}} and \href{King}{\cite{King (1976}}.}
%NE DIRAJ!!!!!!!!!!!!!!!!!!!!
This actually means that within the scope of the widely praised universal image of "cityness", which is imposed as the final goal and ambition of cities, successful examples of cities are
included in the following categories.
\\

It is also echoed in "worlding cities", a methodological striving to refer to the global in the investigations on ordinary cities (\href{Roy}{\citealt{roy_worlding_2011}}).
\underline{World cities} are defined in relation to their regional, national and international influence  inside  the global  economy  where  the  country is categorized according to its status in global economy  (\href{ref}{\citealt{sassen_global_1991}}). Conversely, global cities are categorized according to their industrial  and  communication  potential  for  transnational  management and control  (\href{ref}{ibid.}). Both categories focus on the characteristics of cities and their potential within the scope of the global economy, its flows and networks. This approach has proven to be insufficient, exclusive and restrictive for cities in less developed countries, particularly when we maintain the same terminology at the national level. On the other hand, cities that are outside these categorizations but hold a similar ambition and vision of "cityness" are regarded as  \underline{third World} or \underline{developing cities}. Consequently,  there  are  even  more categorizations  such as those of western, wealthy, third world, developed and developing cities. However, taking all these terms together, there is still a vast number of cities that are left out and have barely any possibility of ever fitting into any of these categories (Robinson, 2006). 
\\

Jennifer Robinson in her book "Ordinary cities between modernity and development" (2006) further develops this insight and questions the geopolitics of urban theory and urban development (\href{ref}{\citealt{fraser_globalization_2006}}). Taken from this standpoint, each and every city is an indicator of what an urbanized society is and what course of urban development it may take.
In any case, this theory has been further adapted and improved by Jennifer Robinson
(\href{Robinson}{\citealt{robinson_global_2002}}; \href{Robinson}{\citealt{robinson_ordinary_2006}}; \href{Robinson}{\citealt{robinson_cities_2011}}; \href{Robinson}{\citealt{robinson_urban_2013}}; \href{Robinson}{\citealt{robinson_comparative_2015}}; \href{Robinson}{\citealt{robinson_thinking_2016}}),
%Robinsons all ref
as well as by other human geographers
(\href{Amin}{\citealt{amin_ordinary_1997}}; \href{Roy}{\citealt{roy_urbanisms_2011}}; \href{Roy}{\citealt{roy_worlding_2011}}; \href{Parnell}{\citealt{parnell_retheorizing_2012}}).
%Roy 2011 Slumdog
%Amin&Graham, Roy&Ong, Parnell
This concept approaches the knowledge of diversity and complexity that exists within the world and "distributes the differences amongst cities as diversity rather than as a hierarchical category" (\href{ref}{\citealt{robinson_global_2002}}).
The most recent trends are the merging of ordinary cities theoretical background into Actor-network theory and assemblage theory analyses
(\href{ONg}{\citealt{collier_global_2008}}; \href{Robinson}{\citealt{Robinson2004}};
\href{Sassen}{\citealt{Sassen2008}}; \href{McFarlane}{\citealt{McFarlane2010}}; \href{Farias}{\citealt{farias_introduction:_2011}}; \href{Rankin}{\citealt{rankin_assemblage_2011}};
\href{Scott}{\citealt{scott_nature_2015}}).
%Ong  and  Collier,  2004;  Robinson,  2004;  Sassen,  2008;  Farías  and Bender, 2010; McFarlane, 2010; 2011; Rankin, 2011.; Scott and Storper
\\

The theoretical stance of ordinary cities is built between the critiques of urban modernity and developmentalism.
Robinson identifies urban modernity as a pure product of western cities and other cities became - "the objects for developmentalist intervention" - meaning that the modernity of poor cities is always about development (\href{Robinson}{\citealt{robinson_global_2002}}; \href{Robinson}{\citealt{robinson_ordinary_2006}}).
Therefore, according to Robinson, the current categorization and differentiation of cities is a product of the colonial past and the capitalist present. In these circumstances, cities are under constant threat of catching-up in an increasingly hostile international, economic, and political environment. In this respect, the first step for addressing urban development is re-framing the concept to meet the challenges of both the wealthiest and the poorest cities.
\\

Bearing  in  mind  \href{Escobar}{\citealt{escobar_encountering_1995}} critique  of  power  and  knowledge  concerning  development  and developmentalism, the  ideology  of  "ordinary  cities"  serves as an equalizing  element  for  all  cities  to  become  able  to  shape  their  distinctive  futures/  developments whatever political, economic and cultural position they hold in relation to other places. 
Within  the  compass  of  "ordinariness",  cities are viewed  as  multi-dimensional  integrated  systems composed of qualitatively different and semi-autonomous processes.
According to  \href{Robinson}{\citealt{robinson_ordinary_2006}}, cities feature as  platforms facilitating  diverse  economic  activities  and  social  life,  sites for  reformulation  and redistribution of social needs and arenas for political contestation.
In such context, "ordinary cities" theory confirms and relies upon \href{Sassen}{\cite{sassen_global_1991}}  vision of the global city, where global economic connections cannot be neglected for city's structural  and  social  formations.
\\

Taking into account both global and local, multiple geographies and temporalities of the urban, concrete context is not enough for analyzing urban systems.
Having said that, deterritorialized external influences also shape specific local urban system transitions (\href{Robinson}{\citealt{robinson_urban_2013}}).
Therefore, envisioning concrete urban complexity and tracing its dynamics encompasses:

\begin{enumerate}
\item Dislocating urban development history, narrative and record from western urban practice which claims to be its originator;
\item Sensitivity to the specificity, diversity and specialization of the city;
\item Justifying how  people  in particular urban settings  have  produced their  particular  production modes and circulation of novelty, innovation and values;
\item Responding to a city's relations within wider networks of circulations, competition and power; 
\item Letting cities  shape their developmental perspectives depending  on  their  distinctiveness  and creative potential, without any hierarchical order among the cities.
\end{enumerate}

If following the assumption of \href{Amin}{\citealt{amin_globalization_1996}} that the city‘s boundaries have become permeable and stretched, both geographically and socially, and that "the city is  now  everywhere  and  everything", it is apparent how, in the world of interconnected cities, the knowing and understanding of different urban lifestyles and forms is essential for dealing with heterogeneous urban contexts. Therefore, the constant state of change of ordinary cities is the dynamics of its complex systems. 
\\

Robinson's theoretical turn in human geography is seen by some to beas revolutionary as that of Lefebvre's historical notion of space
\footnote{\href{Lefebvre}{\cite{lefebvre_production_1974}}
%Ne diraj u picku materinu, pokvario si brdo ovih cite-a!!!!
differentiates  three aspects of space: perceived, conceived and lived space.}
(\href{Chapelin}{\citealt{gintrac_les_2014}}).
\cite{fraser_globalization_2006} supports this approach by elaborating that  in  terms  of  urban  theory  the  main  trap  has  been  the "spatial  turn"  in  theorizing  about  cities  where  they  are  actually  reduced  to  global  issues  and  their competitiveness  in  ‘globality‘ and with Robinson's "ordinary cities", instead of ‘western centrality‘, "the world is [actually] being flattened".
\\

However, its comparative methodological framework engender several theoretical traps: how far-reaching are concepts generated from an instance, what are the methodological tools for navigating amongst instances, and how to distinguish repetitive and distinctive causes and outcomes across cities (\href{Robinson}{\citealt{robinson_comparative_2015}}).
This has been partly resolved by keeping the initial reference to the notion of the "ordinary city", but moving away from comparisons and generalizations through flattened networks of both human and non-human elements
( \href{Robinson}{\citealt{Robinson2004}};
\href{Sassen}{\cite{Sassen2008}}; \href{McFarlane}{\cite{McFarlane2010}}; \href{Farias}{\citealt{farias_introduction:_2011}}; \href{Rankin}{\citealt{rankin_assemblage_2011}}).
%imaju sve ref
This is a rather descriptive approach of actor-network theory (ANT) and assemblage theory that aims at constructing universal typologies of urban agency based on their associations \href{Scott}{\citealt{scott_nature_2015}}.
\\

Although this approach may  be  confronted  with  certain practical constrains, it surely holds the potential to better address the local when viewing all cities as ‘ordinary‘. Moreover, in acknowledging the role of urban agency it also addresses the complex present and the dynamic course of an open-ended, future development. 

\textbf{"Ordinary cities also emerge from a post-colonial critique of urban studies and signal a new era for urban studies research characterised by a more cosmopolitan approach to uderstanding cityness and city futures. This can underpin a field of study that encompasses all cities and that distributes the difference amongst cities as diversity rather than as hierarchical categories. It is the ordinary city, then, that comes into view within a postcolonialised urban studies" \href{ref}{\cite{robinson_ordinary_2006}}}

\subsection{The Constitution of Urban Agency}
%addresses urban complexity

Weber already identified the urban as a place of civility, civics and culture confronting it with disorder and chaos. Urban living is therefore a social construct, the human environment characterized by layered complexity and apparent entropy entailed from extreme dynamics.
Avoiding straight-forward explanations coming from class, market relations and individualism, questions arise about what the means and modes of acting are that generate the urban condition.
\\
An overarching answer to this question from the social sciences perspective was the introduction of the notion of agency.
In general, agency is the capacity to act (\href{Oxford}{Oxford dictionary}).
%ne diraj!!!!!!!!!!!!!!!!!!!
Consequently, broadly and intuitively speaking urban agency is defined as the source of action that brings in and inducts urban system transitions in cities.
\\

Yet, general theories on agency in social sciences always involve structure as an micro-macro, element-system or individual/society dichotomy when talking about agency.
Even though the concept of human agency was already displayed in René Descartes' phrase Cogito ergo sum, it was recently renewed in post-WWII sociology.
This renewed structure-agency reflects the return to individual capacity and its wider social repercussions in contrast to structuralism, functionionalism and system theories.
\\

Giddens' structuralization theory claims that structure and agency are dual and mutual and interrelated through the distribution of power in society (\href{Giddens}{\citealt{giddens_constitution_1984}}).
While institutions hold structural principles with the greatest time space extension, agency implies power through actions that challenges a preexisting "state of affairs" (\href{Giddens}{ibid.}).
Giddens' theory addresses the process of institutionalization over time.
\\

Contesting this approach, \cite{archer_morphogenesis_1982} introduces culture (non-material phenomena and ideas) within the structure (material phenomena and ideas) and explains how transformation, but also maintenance of the systems are produced through the actions and interactions within  an agency-structure dichotomy.
The idea was further developed by \href{Bourdieu}{\cite{pierre_distinction-social_1979}} in his theory of practice.
For him, the structure of the social world forms the setting and the constraints (usually that of power and class) for the perception of actors, but is simulataneously contested by an agent's capital (social, economic and cultural) (\href{Bourdieu}{ibid.}).
Structure, in other words, holds the supremacy and logic of social patterns \href{Rafiee}{\cite{rafiee_relationship_2014}}.
Finally, in communication, \href{Habermas}{\cite{habermas_theory_1985}} explains that, agency is human centered and the system (structure) is the macro setting where these interactions are embedded. And the action is actually an interaction, a two-sided agency.
\\

As already mentioned, all these agency-structure theories have a human as their conceptual core. Still, it seems that this structure-agency integration returns to the systemic view of the world rather than defying it. Yet, it is more of a counterpart for micro-macro integration, where the previous division (emphasized by the systemic integration) has already been made. In this regard, I argue, agency-structure and the system and micro-macro as levels are in themselves the theoretical means for different types of clustering (such as those of cities), as well as of values chains and goals sets (urban development as growth). However, the acknowledgment of institutional roles and relations, the maintenance and transformation processes of social systems, everyday practices and interactive associations among human agencies bring into discussion the question of how the social is distributed.
\\

In this respect, when considering the urban, the notion of urban complexity and its dynamics shifts the view from the perspective of human agency to that of assemblages. Taking into account the designated social distribution, assemblage thinking puts an emphasis on multiplicity and indeterminacy in this setting and its heterogeneous composition of diverse socio-spatial elements and socio-temporal orders (\href{Anderson}{\citealt{anderson_assemblage_2011}}).
Assemblage therefore assigns certain characteristics to the urban totality that intrinsically contain the references to the complexity and dynamics of the system: the incorporation of the local and global, different velocities and scopes in reference to space and time, and the constant generation of new actors and organizational logic (\href{Sassen}{\citealt{sassen_sociology_2007}}).
\\

Moreover, assemblage brings about the notion of process, "the agency of assemblage emerges in process, in bringing different actors together, in their dissolution,  contestation  and  reformulation"
(\href{McFarlane}{\citealt{mcfarlane_assemblage_2011}}).
These conceptions attribute creative, contingent, and un-structured characteristics to agency (\href{Rafiee}{\citealt{rafiee_relationship_2014}}) and focus on the individual sources of agency as well as on the interactive whole \href{McFarlane}{\citealt{mcfarlane_assemblage_2011}}.
It could be reasoned then that any social being has agency. To be more precise, agency could be ascribed to any human and non-human, organic and inorganic, technical and natural element of the system (\href{Anderson}{\citealt{anderson_assemblage_2011}}).
\\

Following the frame of reference indicated earlier, urban agency is the initiator and conductor of urban system transitions. Consequently, urban agency is defined as the action or the force that brings about the particular state of the city as an urban system, but in its heterogeneous and processual nature (\href{Farias}{Farias 2011:15}). 
%NE DIRAJ!!!!!!!!!!!
So, while bearers of urban agency could be materialities as well, the interactions and distributions that produce complex and dynamic urban systemS happen not only between humans and humans and non-humans, but also in the interactions of materials themselves (\href{De Landa}{\citealt{Manuel De Landa 2000}}).
\\

Assigning agency in the form of acts, to the whatever it could be, actually alludes that the world is not flat  \href{Roy}{\citealt{roy_21st-century_2009}}, while the domain of thick description of relations between past, present and future ("history and potential") becomes such (flat) \href{Rankin}{\citealt{rankin_assemblage_2011}}.
Therefore, bringing agency to socio-material interactions is also a way to approach ordinariness. In the perspective of ordinary cities, urban agency decoupled from the presuppositions of intentionality, subjectivity, and free-will enables seeing the processes of difference, in terms of where and how they are emerging (\href{Sayes}{\citealt{sayes_actor-network_2014}}; \href{McFarlane}{\citealt{mcfarlane_assemblage_2011}}).
It is not only intentional action that matters, but a causal one as well  
(\href{Sayes}{\citealt{sayes_actor-network_2014}}).
\\

In this respect, the overarching description of cities is constituted both from the general models of agency and agent behaviour and from local interpretations of authority, behaviour, events and materialities in this process (\href{Healey}{\citealt{healey_institutional_1992}}).
Still, understanding what urban is includes rendering cities as complex systems and explicating the dynamics of urban  system transitions.
The knowledge about cities therefore, in order to be pertinent, must be inclusive and flexible in allocating urban agency in local contexts. 
As a result, people (urban actors), objects (the built environment), territories (space), institutions (the regulatory framework), infrastructure and social practices are all assumed to be equal agents in the dynamic process of contemporary urban development (urban system transitions) and therefore should be equally treated in delegating them key roles.
The interaction and interconnections among these key agents in an urban environment result in the diversity of social practices and battlefields of urban conflicts, which are the major issues that influence urban decision-making. 

\subsection{The Morphology of Urban Decision-Making}
%addresses urban comlexity

The city, when viewed from the presented framework of overarching and contingent urban agency, is not simply an output or resultant formation,  but  an ongoing  construction (\href{McFarlane}{\citealt{mcfarlane_assemblage_2011}}).
In this sense, urban decision-making is not only a formal conduct of visible and institutionalized power, but is  complemented with informal and less obvious sources of agency \cite{(Lukes 1974 from Healey 1997)}.
According to contemporary trends in social sciences as well as in practice, these circumstances are generally associated with the term governance (\href{Pierre}{\citealt{pierre_governance_2000}}).
\\

Much like urban development, the term governance is used in different contexts and with different meanings, mainly because of its capacity to cover a vast arena of formal and informal relations and structures (\href{Pierre}{\citealt{pierre_governance_2000}}).
In this respect, we can refer to four generations in the theory of governance: 
(a) governance for development management,
(b) the political and historical nature of governance,
(c) the political economy approach to governance,
and
(d) the return of politics into governance through the notion of power, idea, agency and contingency (\href{Hudson}{\citealt{hudson_political_2014}}).
However, the recent practice-oriented trends also showed the tendency to have governance "measured" and the World Bank Group (WBG) has presented a report proposing worldwide governance indicators, which address the quality of governance at the national level. The data were based on the data from an enterprise, citizen and expert survey covering over 200 countries during the last 20 years. The proposed indicators 
\footnote{These indicators cover six dimensions of governance: voice and accountability, political stability and absence of violence, government effectiveness, regulatory quality, rule of law and control of corruption (\href{wgi}{WGI 2016}).}
cover mainly political, institutional and economic spheres, yet such limited reliance of governance within the social sphere is not an official view.
\\

In urban discourse, the term urban governance covers a range of social and political processes through which the management of urban issues is conducted (\href{Healey}{\citealt{healey_collaborative_1997}}).
Governance itself is a process of steering the economy and society to manage the interplay of rules, interests and values in cities conducted through hierarchies, markets and needs (\href{Pierre}{\citealt{pierre_debating_2000}}; \href{Pierre}{\citealt{pierre_governance_2000}}) - a battlefield of policy-driven, profit oriented and just city.
\footnote{The phrase "just city" is used to designate the criteria of democracy, diversity and equity proposed by  \href{Farnstein}{\cite{fainstein_just_2010}}.}
%OSTAJE CITE!!!!!!!!!!!!!!!
And it is set up through the human actions, that of individuals, organizations, and coalitions (\href{Hudson}{\citealt{hudson_political_2014}}).
In short, governance is how decisions are made.
\\

In political terms, governing (steering) at the city level is brought forward from either the top-down or bottom-up, with the authorities at the one end and the communities at the other (\href{Hudson}{ibid.}).
However, the current trends of globalization have also contributed to moving the steering poles from nation state to the international level.
Namely, the concept of underdevelopment
\footnote{This concept is generally applied to Third World countries, both during colonization and after national independence. Nevertheless, following the discussion from (\href{Section 2.1.1}{Section 2.1.1}, \href{Section 2.1.2}{2.1.2}), it is also applicable to all countries and cities outside the Western world.}
guided through the global system and development agencies (e.g. the World Bank and International Monetary Fund) tend to produce benefits for powerful countries and multi-national corporations through economic, political, and military actions.
\\

Consequently, international organizations and international corporate capital can easily override the state and come down directly to the city level.
And the picture becomes ever more complex with private investors and non-governmental sectors growing in importance and moving in the play.
For that reason, \href{Pierre}{\cite{pierre_governance_2000}}
%NE DIRAJ CITE!!!!!!!!!!!!!!!!!!!!
advocate that steering society and the economy is increasingly moving up, down and out from  state institutions as its supreme authority.
\\

While politics is open, governance is a directional process and decisions become locked in (\href{Hudson}{\citealt{hudson_political_2014}}).
In this respect, the goal of governance is to: contribute to conflicts resolution, capture and apply multiple  forms  of  knowledge, facilitate local capacity building and enable inclusive decision-making (\href{Mathur}{\citealt{mathur_defining_2007}}).
According to the global predominance noted earlier,  governing is moving up (international organizations), down (regions, localities, communities) and out (NGOs, corporatizations and privatizations) to enable these multifaceted and complex negotiations (\href{Pierre}{\citealt{pierre_governance_2000}}).
\\

On the other hand, decision-making is an effective exercise of judgment and the first step in implementation by taking the consequences of decisions (\href{Knight}{\citealt{knight_risk_2012}}).
Decisions involve risk and uncertainty, but they also imply resulting actions, risk of errors and responsibility for their correctness (\href{Knight}{ibid.}).
Actions, risk, errors and responsibility are managed differently when decisions are made by those who stand to bear main consequences (\href{Meppem}{\citealt{meppem_planning_1998}}), but may turn out to have grave consequences when that is not the case (\href{Friend}{\citealt{friend_planning_2005}}).  Unclear values, lack of information and inefficient coordination are the mayor causes of disastrous effects of inefficient and top-down risk management (\href{Friend}{ibid.}).
In short, while governance aims to resolve a conflict, decision-making solves or produces it; governance instates practices and decision-making produces or reproduces them; and finally, governance captures socio-spatial resources, while decision-making either neglects or exploits them.
\\

Therefore, the question of the morphology of urban decision-making is not about how, but about by who and through which mechanisms decisions are made.
In terms of urban agency, the morphology of urban decision-making acknowledges the role of ‘non-humans’, ‘knowledge’, ‘techniques’ and ‘rationalities’ (\href{Healey}{\citealt{healey_circuits_2013}}), which come from policies, institutions, documents, places and events.
The resulting framework for the broad arena of up, down and out governance, are decisions coming from top, bottom or outside and they are implemented top-down, bottom-up or through and across the system
(\href{Hudson}{\citealt{hudson_political_2014}}).
Urban planning mechanisms, real estate transformations and participatory practices will be elaborated respectively as the layers of urban decision-making.

\subsubsection{Urban Planning Strategies}

With the morphology of urban decision making broken down into three layers that catalyse governance mechanisms into operational decisions, the scope of the urban  planning  framework aspires  to  generate  an  action  plan  for  development  that  achieves  common viewpoints,  goals  and  priorities  within  a city,  as  well  as  a  set  of  strategies  optimized  over  time  within  the institutional mechanisms for their implementation, monitoring and evaluation (\href{Fisher}{\citealt{fisher_building_2001}}).
Urban planning is, therefore, a future-oriented activity for managing urban development (\href{Nedovic}{\citealt{nedovic-budic_mornings_2011}}). 
\\

It was only from the beginning of the 20th century onward that the scientific discourse of planning became distinguished from that of sociology and from urban design and the traditions of landscape architecture
\footnote{Urban design and landscape architecture stem from architecture and design, which were established as disciplines since Greek times
(\href{Handlin}{\citealt{Handlin and Burchard, 1966}};
\href{Allmedinger}{\citealt{allmendinger_planning_2002}};
\href{Hall}{\citealt{hall_cities_2002}};
\href{Van}{\citealt{van_assche_co-evolutions_2013}})} 
Consequently, urban planning theory has always complied with the prevailing theoretical framework of social studies (\href{Portugali}{\citealt{portugali_complexity_2011}}).
On the other hand, in order for planning activities to be practically effective, they should be embedded in a particular socio-spatial context, and they should react to the shifts in socio-economic and political settings.
\\

Having the operational framework of urban planning defined as such, it becomes conspicuous how its practice has always complied with an overall planning paradigm, being simultaneously, intrinsically  connected to the property  market and tending to maintain the current social order (\href{Taylor}{\citealt{taylor_urban_2006}}). 
In respect to the master narrative of modernity installed after WWII, urban planning has been continuously enriched with the ideas of rationality, objectivity, scientific evidence, values and possible control through norms. Scientific research in the field has been operating within this framework and during the same period it evolved from normative to communicative planning. At first, it turned to a normative planning model based on a top-down decision-making process. Then, when the diversity of values, meanings, and interests emerged more vigorously, collaborative and communicative planning took up and promoted the changed role of the urban planner from being a technical expert to a mere facilitator (\href{Taylor}{.ibid.}).
In this respect, urban planning is governance-style activity that involves knowledgeable reasoning and argumentation: scientific/professional/expert knowledge that can transcend over space and time, as well as institutional knowledge that embraces systemic and functional logic and the managerial capacity of organisations and institutions (\href{Healey}{\citealt{healey_collaborative_1997}}; \href{Getimis}{\citealt{getimis_comparing_2012}}).
\\ 

Furthermore, according to one of the leading urban theories of David \href{Harvey}{\cite{harvey_right_2008}} and Manuel \href{Castells}{\cite{castells_urban_1979}}, urban planning cannot be seen as an autonomous process of spatial development, but rather it is situated in its political and economic context, where current economic and social organizations constantly overlap (\href{Taylor}{\citealt{taylor_urban_2006}}).
Urban planning in practice is intrinsically connected to the property market (which in turn involves a particular political ideology) and this tends to maintain the current social order (\href{Dear}{\citealt{dear_urbanization_1981}}; \href{Taylor}{\citealt{taylor_urban_2006}}), both of which are grounded in the development and expansion of industrial capitalism, neo-liberalism and consumerism (\href{Ellin}{\citealt{ellin_postmodern_1999}}; \href{Harvey}{\citealt{harvey_urban_1989}}).
In other words, the urban planning system is, and always has been, the spatial and symbolic manifestations of broader social forces  (\href{Giddens}{\citealt{giddens_consequences_1992}}), following the societal evolution to various degrees and at various speeds (\href{Flyvbjerg}{\cite{Flyvbjerg, 1998}}).
\\

The socio-spatial organisation of cities therefore involves high-level coordination of policy and practice, procedures and content (\href{Van}{\cite{(Van Assche and Verschraegen, 2008}}), so that urban planning attempts to balance the roles of authorities and markets, legal, economic, and political perspectives from national and sub-national levels (regional, city, community) (\href{Nedovic}{\citealt{nedovic-budic_mornings_2011}}).
Urban planning is therefore regarded as an institutionalized profession, and more often than not its decision-making mechanisms are deployed from top-down.
\\

In simplistic terms, the urban planning agency covers the interaction between policies and institutions and for designing frameworks of action in space (\href{Getimis}{\citealt{getimis_comparing_2012}}).
Historically speaking, the complex coordination in planning systems was first constituted within the hierarchical model of the centralized state to later become even more complexified with its democratization and the involvement of new actors and actor groups 
(\href{Van}{\citealt{van_assche_co-evolutions_2013}}).
The planning system comprises the urban regulatory framework steered through the local planning culture (\href{Getimis}{\citealt{getimis_comparing_2012}}; \href{Peric}{\citealt{peric_evolution_2016}}).
The planning culture implies local styles, norms, values, belief systems, visions and frames, but not only in terms of the behaviours and actions of human agents, but it is also embedded in planning frameworks - rules, plans and programs, procedures and organizational structure are locally nested products of evolution (\href{Moroni}{\citealt{moroni_evolutionary_2010}}).
Various authors speak differently about the urban planning system’s organization. \href{Getimis}{\cite{Getimis XX}} advocated  for the fusion of the urban planning system and urban culture. 
\href{Alexander}{\cite{alexander_institutional_2005}} elaborated the thesis of institutional design that comprises processes and structures of legislation, policy making, planning and programs and implementation, where as \href{Portugali}{\cite{Portugali_Self-planned_2011}} identified power with systemic elements - legislative, judiciary and executive authorities within planning institutions.
\\

Bearing in mind the vast pool of ordinary cities and taking into account the figuration of both human and non-human agency, a general distribution of top-down decision-making in the urban regulatory framework is distributed through the following channels:

\begin{itemize}
\item \textbf{institutional framework} consisting of public administration and urban planning authorities and addressing political and administrative jurisdiction;
\item \textbf{legal framework} that comprise rules and procedures and deals with social, economic, political and ethical implications of planning interventions;
\item \textbf{planning framework} with policy agendas and technical documentation for programming and implementation in planning.
\end{itemize}

The top-down authority in planning also involves bringing together, structuring and regulations of relations within a planning system between public, private, common, corporate and collective interests
\footnote{In ideal conditions, these interests are distributed such that the state represents public interest, the market private interests, communities have common interests, companies - corporate interests and associations and organizations - collective interests. But in real and, even more so, in turbulent political systems the distribution of interest is disrupted and confused among these initial patron groups of actors (\href{Vujosevic}{\citealt{vujosevic_regionalizam_2015}}).}
(\href{Maksic}{\citealt{maksic_european_2012}}; \href{Vujosevic}{\citealt{vujosevic_regionalizam_2015}}).
The growing dominance of markets, and the power of private and corporate investors brought a change from ethically based to profit-oriented decision-making in urban planning, especially in weak political contexts (\href{Lazarevic}{\citealt{bajec_rational_2009}}).
In such circumstances, power manages to manipulate knowledge and to determine`what kind of interpretation attains authority as the dominant interpretation` (\href{Getimis}{\citealt{getimis_comparing_2012}}).
In this sense, the political poles and financial centers are brought into decision-making ravishing through and across the structures of the system.

\subsubsection{Real Estate Bias}

As has already been mentioned, urban decision-making relies on much more than strategic urban planning.
Several authors challenge the concept of ‘the public interest’ and its perseverance during the planning process and in planners' actions
(\href{Alexander}{\citealt{alexander_XXX_2002}};
\href{Campbell}{{\citealt{campbel_XXX_2012}};
\href{Moroni}{\citealt{moroni_XXXX_2004}}};
\href{Sager}{\citealt{sager_logic_2006}}). 
%Sager 2006, Alexander, 2002; Campbell and Marshall, 2002; Moroni, 2004}.
\href{Sager}{\cite{sager_logic_2006}} claims that there is neither guarantee nor restrictive rules that would force planners to "act in the interest of the public rather than in self-interest or partisan interest".
\\

Every urban issue relies directly on the economy  and  the  mode  of  production  and  consumption  in  modern  cities.  Namely,  the  capitalist  economy  needs urbanization to absorb surplus products, so that the deregulation of land use and property markets is the precondition for capitalist accumulation and thereafter for proceeding to economic growth (\href{Harvey}{\citealt{harvey_rebel_2012}}).
Following Harvey’s line of thought, the power  extracted  from  the  exclusive  control  over  property  or  land  is  the  source  of  capital/income  produced  by  its  locational, infrastructural, social or cultural capacity.
In other words, the contextual resources of an urban environment make it appealing for incoherent distribution of resources and responsibilities (\href{Bolay}{\citealt{bolay_urban_2005}}).
\\

Historically speaking, spatial organisation and design have been strongly tied to top-down political and institutional powers, but also more indirectly, over time the presence and increase of patronage have influenced urban interventions
(\href{Van}{\citealt{van_assche_power_2014}}).
In extreme circumstances, influential economic and political actors tend to abuse their powers and appropriate urban space.
When the regulatory framework is blurred, biased or, in extreme cases, corrupt, economy and politics tend to take over urban decision-making.
In other words, their projects and interests intervene in planning frameworks
(\href{Hudson}{\citealt{hudson_political_2014}}) and therefore they are associated with urban agency.
\\

Their actions usually defy public interest and tend to endanger the rights and interests of less powerful and more often marginalized groups (\href{Sager}{\citealt{sager_logic_2006}}).
This interference in the urban planning framework is enabled through social networks, social distribution of interests, and above all through the relations these actors have with politicians and officials
(\href{Healey}{\citealt{healey_collaborative_1997}}).
This is \href{Healey}{Healey's (1997)} definition of clientelism - economic actors gain profit in return for political favours through this personal patron-client relation.
\\

However, this is only one form of corruption.
Other forms of corruption that dominate urban planning are:
legislative and regulatory corruption,
\footnote{Influence on rules and legislation (\href{Chiodelli}{\citealt{chiodelli_corruption_2015}})}
bureaucratic corruption,
\footnote{Corrupt acts of bureaucrats within their domains of work (\href{Chiodelli}{ibid.})}
and
public works corruption
\footnote{"The systemic graft involved in building public infrastructures and services" (\href{Chiodelli}{ibid.})}
(\href{Chiodelli}{\citealt{chiodelli_corruption_2015}}).
Corruption involves intentional  non-compliance  with  formal  rules by the public power holders for personal or related private benefits
(\href{Kaufmann}{\citealt{kaufmann_XXX_1997}};
\href{Begovic}{\citealt{begovic_corruption_2001}}; \href{Grubovic}{\citealt{grubovic_belgrade_2006}};
\href{Chiodelli}{\citealt{chiodelli_corruption_2015}}).
\\

According to
\href{Chiodelli}{\cite{chiodelli_corruption_2015}} corruption is endemic in the land-use planning field.
In this respect, the determinants of corruption are:
(1) discretionary allocative power; (2) economic rents, taxation and redistribution programmes associated with discretional power; and (3) the low probability for perpetrators to be detected and punished
(\href{Healey}{\citealt{healey_collaborative_1997}}; 
\href{Chiodelli}{\citealt{chiodelli_corruption_2015}}).
\\

In local frameworks, this means that the decision-making is moved up (to international corporate capital) and out (private investors).
One powerful means for profit and source of corruption that stems from such shifts of decision-making is megaproject developments in metropoles around the world.
Within local contexts where urban planning frameworks are weak and inconsistent, they usually jeopardize the position of low-income people and marginalized groups, minimize public amenities and entail gentrification, large-scale unitary projects, exclusive developers and nonexistent and insufficient public participation (\href{Fainstein}{\citealt{fainstein_just_2010}}).

\textbf{"Politicians like to benefit their political supporters. Officials may come under pressure to bend rules to favour the friends of politicians and disadvantage their enemies. Individuals with particular interests may lobby vigorously for 'special attention' in relation to their development project." \href{Healey}{\cite{healey_collaborative_1997}}}
%NE DIRAJ!!!!!

\subsubsection{Traces of Participation}

Public participation generally encompasses bottom-up, action-orientated and socially inclusive engagement of all individuals or constituted groups
(\href{UN}{\citealt{UN_building_2009}}). %NEMA REF
What is more, successful participation firmly relies on the accessibility, transparency, responsiveness, and accountability of all institutional processes. The active urban agency of individuals or constituted groups places them into political and economic processes, and therefore deliberately includes them in the future of their society
(\href{Arnstein}{\citealt{arnstein_ladder_1969}};
\href{Fisher}{\citealt{fisher_building_2001}}).
\\

Following Arnstein's ide of the ladder of participation, each society is left to mix and match the participatory processes that meet its needs, influence power and relations (\href{Fisher}{ibid.}).
Accordingly, including citizens priorities, values and needs aim at achieving certain end-results, contributing to the efficiency of society as a whole in a process for accumulating social capital, and creating institutionalizing networks of civic engagement (\href{Putnam}{\citealt{putnam_making_1993}}).
This means that every society would be able to produce its own space with a strong impact of its ideology and cultural spheres, and thereby control its urban development (\href{Lefebvre}{\citealt{lefebvre_production_1974}}). 
The identity of an ordinary city constantly in flux is then defined as the process of self-understanding, self-creation and self-representation of an operating urban environment by its urban actors, all of whom are mobilized to intervene responsibly and who willingly integrate their customs and needs into this process
(\href{Bolay}{\citealt{bolay_urban_2004}}).
\\

Participation in planning is a process that is usually designed to address urban conflicts with the aim of resolving or exploiting it successfully (\href{Fisher}{\citealt{fisher_building_2001}}).
Namely, from the introduction of the communicative planning approach in the 1980s onward, public participation has been set in motion as a tool for exploiting the democratic capacities of modern society in order to locally mobilize all available human resources, develop discourses and practices, change institutional conditions to transform a crisis of aggregated urban conflicts into an opportunity for urban development
(\href{Healey}{\citealt{healey_communicative_1996}}; \href{Scharpf}{\citealt{scharpf_XXX_1997}}).
\\

However, practical application has shown that public participation in Arnstein’s terms lacks popular sovereignty in order to place all urban actors and stakeholders equally within the decision-making process
(\href{Mouffe}{\citealt{mouffe_which_2002}}).
The situation has been particularly aggravated by thriving neo-liberal market policies (\href{Mouffe}{ibid.}).
The influence of this trotting up and down the ladder of participation is especially accentuated in ex-authoritarian states.
In this sense, the trends of commercialization and free market policies led to the decline of the public realm, the deconstruction of urbanity and the abuse of public space
(\href{Hirt}{\citealt{hirt_landscapes_2008}}).
\\

An alternative vision was recently set in practice with the paradigm of tactical urbanism whose main goal is to set forth economic, political, cultural and spatial transformation in global cities by instigating creative interventions that guide their change, giving them unique identities
(\href{Lydon}{\citealt{lydon_tactical_2012}}).
The conceptual core of such an approach circumvents involvement of the least powerful urban actors in decision-making, encourages them to creatively trace their cultural identity through adequate professional supervision and bring positive change, develop social capital and organisational capacity that involves shaping the physical and social component of cities 
(\href{Bolay}{\citealt{bolay_planning_1996}}).
Moreover, their needs should also be modified to what the urban planning framework can actually offer; they need to act or interact with the world around them, which is in flux
(\href{Harvey}{\citealt{harvey_condition_2003}}).
Moreover, the way cities function shapes the expectations and actions of all the urban actors involved, who also influence the constitution of the city itself.
\\

The identity of a city in flux is defined as the process of self-understanding, self-creation and self-representation of an operating urban environment by its urban actors, all of whom are mobilized to intervene responsibly and who willingly integrate their customs and needs into this process
(\href{Bolay}{\citealt{bolay_urban_2005}}).
In this way, these individuals or constituted groups may become the actual "makers of the city".
They determine space as a social product of their values, the logic that pilots them, the relationships and representations that influence them and the aspirations that motivate them (\href{Lefebvre}{\citealt{lefebvre_production_1974}}).
Individuals reproduce and adopt explicit or implicit local/everyday/milieu knowledge (nature, culture, and social conceptions within a social network) and build non-institutional frames (based on their expectations, beliefs, attitudes and values) (\href{Getimis}{\citealt{getimis_comparing_2012}}).
However, in recent scientific studies in this domain, the  positive image of the urban and participation is rather characterized as unrealistic and even Utopian
(\href{Lindner}{Lindner 2009}).
%NE DIRAJ!!
\\

Yet the trends of globalization and urbanization have moved up and out along the poles of decision-making and have brought new sources of urban agency into local arenas, which also plead for participatory purposes - that of international organisations (IO) and local and global non-governmental organizations (NGOs) and sporadic private elements. Apart from social issues, planning-as-design has been an important entry point for these actors to engage in local urban decision-making 
(\href{Van}{\citealt{van_assche_co-evolutions_2013}}).
The focus on content-focused, incremental changes  and the attractiveness of place is the point of their engagement in particular interventions in cities by advocating participatory urban design.
\\

Yet, instead of enhancing the micro urban environment for locals and train a populace to know, show and actively express their needs and directly apply them in space interventions and social practices
(\href{Ostrom}{\citealt{ostrom_governing_1995}}),
these attractive activities very often yield the increasing commodification of urban public space
(\href{Lehrer}{\citealt{lehrer_old_2008}}),
gentrification and commercialization of urban neighbourhoods
(\href{Zukin}{\citealt{zukin_cultures_1995}}; \href{Lloyd}{Llloyd 2006})
%ne diraj ove reference, nek ostanu bez biblio, ne mogu da ih trazim
and low-wage services as only new source of economic activity
(\href{Basett}{Basett 1993}).

\subsubsection{Urban Decision-making Layers}

Cities are not simply market products and consumption patterns, but locally customized socio-political constructs as well (\href{marcuse}{\citealt{marcus_spatial_2007}}). These external influences form a range of qualitatively different contextual circumstances for a positive urban system transition, i.e. urban development. Most settlements and cities in previous historical periods reflect the various degrees of forethought and conscious design in their layout and function. This approach was referred to as a fixedly planned development, although many cities still tended to develop organically.
\\

Identifying actors, actor groups, institutions, rules, policies, documents and events, local and global ones, and acknowledging their agency in urban decision-making is not per se enough for approaching urban dialectics.
It is essential to discern their roles and the effects of their agency in order to understand the evolution of urban system transitions.
In this respect, the heterogeneous distribution of urban agency was structured according to the congregation of decision-making from top-down, bottom-up and across the system.
Even though these structural layers enclose power relations and the elements of social order, the purpose of the morphology of urban decision-making is to deploy unstable urban development modalities 
through the rhetoric of urban agency, self-descriptions of current actors and the discourse on non-human elements governing their interactions (\href{Van}{\citealt{van_assche_co-evolutions_2013}})
\\

Much like planning systems alone, whole structures of decision-making develop continuously throughout space and time in the world through the waves of innovation,  imposition,  borrowing  and  adjustment
(\href{Nedovic}{\citealt{nedovicbudic_waves_2006}}).
Concurrently, there are certain global trends which cannot be denied
(\href{Van}{\citealt{van_assche_co-evolutions_2013}}):
(1) pulling for institutionalization and coordination of planning systems;
(2) regulating the overlap of markets and planning;
(3) professional and disciplinary articulations and the legitimacy of spatial interventions;
(4) the role of aesthetics in socio-spatial interventions, and
(5) increase in transparency and public participation in socio-spatial transformations.
This may suggest that the distribution and structuralization of urban agency and the corresponding networks of influences [from top, bottom and outside
\href{Section 2.1.4}{Section 2.1.4}], by belonging to the same cluster of elements, produce the effects and contexts which can be placed and put in hierarchical order. 
\\

Nevertheless, in particular local contexts, this is not the case. The specific mixtures of internal and external influences result in a concrete local organisational system, a style of action and quality that evolves over time according to its intrinsic, ‘ordinary‘ logic. While complexity and dynamics of urban systems resemble one another, if they are let to, they all may mature differently constituting their intrinsic socio-spatial patterns.

\subsection{How to frame urbanity to grasp system dynamics}

Side by side with the historical continuum of global development patterns, the socio-political framework at the neighbourhood level is shaped by human integration into the local environment.
Appropriations, adaptations and modifications of space are the main agencies of physical interventions, which are followed by continual adjustments of their political, economic, and cultural structures (\href{Sykola}{\citealt{sykora_transitional_1999}}).
This process captures the pace of change and the multi-layered nature of transformation, with the focus on transitions in the local economy, society, system of governance and the spaces of production and consumption. A systematic approach to such dynamics should integrate different modi operandi, transcend multiple scales and recognize the temporality of information, actions and intentions that are followed up by satisfactory results  (\href{Tardin}{\citealt{tardin_landscape_2014}}).
\\

While today about 54\% of the world population lives in cities (\href{UN}{UN 2016}), it has also been argued that "not all or even the greater part of these existences [and circumstances] can be described as being intrinsically urban"
(\href{Scott}{\citealt{scott_nature_2015}}).
In urban studies the concept of urbanity is used as a parameter for the quality of urban spaces, but the particular meanings of the concept vary across contexts and disciplines (\citealt{bisson_urbanity:_2016}).
According to the following argument, this rather "fuzzy concept" (\citealt{Bourdin 2010} could also be used to grasp and operationalize urban dynamics, if taken in the course of Actor-network theory and updated to better indicate the parameters of urban dynamics through urban system transitions of maintenance, transformation and change.
\\

From the historical viewpoint, the term "urbanity"
\footnote{Urbanitaet means urbanity in German and urbanité in French. In all 3 languages it refers to either cities or urban life with slight differences according to the scientific domains where it is used.
This research prefers to stay away from etymological and discursive interpretation of the term, but tend to emphasize that it is bounded theoretical concept often used/referred to separately/independently from the notion of the word urban.}
is used as a qualitative indicator for physicality in cities and urban life. 
Based on the language discourse, different disciplines moulded the meaning of the term.
Therefore, it is possible to distinguish two different scientific views of urbanity - theoretical and practical, sociological and architectural.
\\

The sociological approach in urban studies loosely associates urbanity with the city and addresses the urban way of life, referring to its original definition from the Oxford Dictionary. This definition dates back to the 16th century French interpretation of the Latin word ”urbanitas” which denotes politeness. Its use in this sense is very common within the English scientific context in urban sociology. In this way, urbanity is closely related to civility and indicates the "cultural dimension and symbolic infrastructure of cities" (\href{Zijderveld}{\citealt{zijderveld_theory_2011}}).
In French language discourse, this sociological notion of the term is also found in urban geography. However, herein it incorporates both the materiality (urban structures) and the substance (social interactions) in the city (\citealt{bisson_urbanity:_2016}). 
\\

Accordingly, within the German tradition of urban studies, the term has been a more common reference for urban policy and planning, where it denotes the urban way of life and typical structural properties of a traditional "European city" (\href{Prigge}{\citealt{Prigge_urbanitaet_1996}}; \href{wust}{\citealt{wust_urbanity_2005}}; \href{Lossau}{\citealt{lossau_new_2008}}). 
As an overall definition in sociological and planning traditions of the 20th century (Georg Simmel, Louis Wirth, Robert Park, Jane Jacobs, Henri Lefebvre, Thierry Paquot, Richard Florida among others), urbanity is a condition and setting that makes possible a specific way of life characterized by the city (\href{levy}{\citealt{levy_liens_2013}}). 
\\

In contemporary urban theory, urbanity is identified as the state of a city/space coming out of the balance  between the physical and social. On one hand, urbanity figures  as a conditional arrangement for arguing and evaluating actions towards an overall vision of the common good (\href{Holden}{\citealt{holden_justifying_2015}}).
This rather contested and open-ended articulation of contemporary urbanity aims at balancing top-down, market and civic-based governance roles, responsibilities and outcomes within the interrelations of biophysical and socio-cultural urban elements - people and spaces, the regulatory framework and urban structures (\href{Groth}{\citealt{groth_reclaiming_2005}};\href{Tardin}{\citealt{tardin_landscape_2014}};\href{Holden}{\citealt{holden_justifying_2015}}).
On the other hand, urbanity as a state of the urban is deconstructed through the density and diversity of urban areas (\href{Levy}{\citealt{levy_mesure_1997}};\href{Levy}{\citealt{levy_liens_2013}}). In this reference, it could also be graded and thereafter characterize cities through  "the level of urbanity" (strong/weak urbanity) (\citealt{levy_liens_2013}).
\\

On the other hand, the sense of urbanity as a way of life (urban life) and culture (urban culture) emphasizes the temporal property of the concept (\href{Farias}{\citealt{farias_introduction:_2011}}).
In light of assemblage and ANT theory, urbanity emerges within socio-spatial networks at multiple scales (\href{Kamalipour}{\citealt{kamalipour_assemblage_2015}}).
This approach emphasizes the idea that a city might be just a fixed actualization of urbanity in a particular space-time bound (\href{Farias}{\citealt{farias_introduction:_2011}:297}).
The biggest challenge for urban analysis, addressed by the practitioners of ANT, is the definition of "rational urbanity" that addresses urban experience over time.
Proposed by Spanish anthropologist Delgrade (\citealt{farias_introduction:_2011}:211), it embodies fluid, unstable and ambiguous forms and principles of life in the space-time flux of a city.
\\

In the English speaking context, urbanity is usually linked to physical components of the city within the domain of architecture and consequently urban design.
In this reference, the concept is generally the indicator of urban quality. For architects urbanity sounds widespread and familiar at least in its normative sense as an articulated, "zero-friction" vision of urban development (\href{ref}{\citealt{hajer_zero-friction_1999}}; \href{wust}{\citealt{wust_urbanity_2005}}).
In architectural research this concept is very often assumed and used without being defined.
Several authors use it for their analysis without clearly stating its meaning, although slight differences in interpretations can be detected between authors as well.
But in general terms, the architectural interpretation is narrow and tends to tame urban complexity, and debunks its relevance to interpret urban dynamics  and to deal with diversity, the unexpected and the non-planned in cities (\href{Groth}{\citealt{groth_reclaiming_2005}}, \href{Wuest}{\citealt{wust_urbanity_2005}}). 
\\

The rich meaning of urbanity as a structural continuity of spaces in cities has been adequate for application in urban design practice. Historically speaking, this viewpoint was backed up by theories from Camillo Sité, Jane Jacobs, Kevin Lynch and Christopher Alexander.
This line of interpretation has led to the formulation of a parametric vision of urbanity as an architectural category for spatial configuration of urban spaces.
The idea of breaking urban space down into components is bounded up in the space syntax set of theories and techniques for the practical application of the concept of urbanity.
\\

The most common definition of urbanity in the space syntax domain explains it as "the generic need for people and societies to access differences as a means for social, cultural and economic development" (\href{Marcus}{\citealt{marcus_spatial_2007}}:10).
In this respect, the operational definition of urbanity stems from its integration in urban morphology  and refers to it as accessible diversity and efficient integration to locally capture the spatial capital (\href{Marcus}{\citealt{marcus_spatial_2007}}).
This interpretation might be easily politicized and rather reductionist by seeing urbanity as nothing more than an instrumentalized, aestheticizing perception filter (\href{Wuest}{\citealt{wust_urbanity_2005}}).
\\

In this battle for dominance of either social or spatial (theoretical and practical) references, the first invokes the socio-cultural dimension of cities (\href{ref}{\citealt{haussermann_urbanitet:_1992}}; \href{ref}{ \citealt{christiaanse_auf_2000}}), while the second turns to their architectural and design qualities (\cite{Neuffer 1976}).
However, all of them agree to a certain point that acknowledging difference and heterogeneity as well as embracing fragmentation and contradictions in a social and spatial sense are the prerogatives for accessible diversity and therefore the quality of the urban (\citealt{(Durth 1986: Krämer-Badoni 1996} \href{Wuest}{\citealt{wust_urbanity_2005}}; \href{Markus}{\citealt{marcus_spatial_2007}}). 
\\

Taking into account a close scrutiny of these polarized explanations, there has recently emerged a tendency for an operational combination of sociological and architectural notions of urbanity to stand not for an urban condition but for a dynamic urban process.
While sociological interpretation exposes its processual potential, the practical focus of its application in architecture connects it to the local setting.
Urban reality in this way was amplified towards a heterogeneous, dynamic set of flows (\href{de Aguiar}{\citealt{de_aguiar_douglas_vieira_what_2013}}).
\\

The initial step forward in this direction was a recent influx of theory in space syntax research and the tendency to explain urbanity as an experience that incorporates urban agency of all human and non-human actors and to analyse it with ANT (\citealt{rheintantz_narratives_2012}).
On the other hand, there is also an increase in addressing place making, not only the common good and the social, when evaluating urbanity from the sociological perspective (\href{Holden}{\citealt{holden_justifying_2015}}). 
The multifaceted character of the human-environment (nature) and nature-culture interactions that are addressed by this conceptual overlap of urbanity captures dynamic urban reality over time (\href{Tardin}{\citealt{tardin_landscape_2014}}).
\\

This line of research aims at combining the parametric nature of urbanity with the ANT description of all key urban agents and tracing it within the level of urbanity. In this way, the test of an urbanity level justifies opportunities for socio-spatial continuations, options, and turnovers to reach the collective demand towards the common good in the public sphere (\href{Holden}{\citealt{holden_justifying_2015}}:4). 
When defined as such, urbanity may serve as a valid scope that offers categories for restricting complexity towards a structural unity of urban elements and comprises a sum of social interactions that "enable people  to  live  together,  without  conflicts,  in  dense places"
(\citealt{bisson_urbanity:_2016}).
Concomitantly, the level of urbanity incorporates dialectics of values, identities and relations that make the urban system nurtured locally and open to constant, flexible, spontaneous system transitions (\href{Groth}{\citealt{groth_reclaiming_2005}}).
In these circumstances, networks of urban agency and contextual elements define urbanity through the character of space and the corresponding agenda for social encounters.

\paragraph{Contextual resources}

In general, the character of space captures its spatial capital. 
Moreover, it has been widely accepted that spatial, social and human capital has been created from the accessible difference/discrepancy/change (\citealt{becker_human_1993}) \href{Coleman}{\citealt{coleman_social_1988}}, or urbanity as accessible diversity from the space syntax point of view (\href{Marcus}{\citealt{marcus_spatial_2007}}). In other words, contextual resources are formed by making use of available territorial capital.
\\

Territorial capital is a localized set of objects and relations (\href{Camagni}{\citealt{camagni_regional_2013}}):
(1) localized externalities (economic and technological),
(2) localized actions (activities, traditions, skills and know-hows),
(3) localized relations (a system of rules and practices),
(4) all the material and non-material elements physically produced, supplied by history or derived from nature, in an intentional or unintentional manner. In sum, territorial capital consists of social, human and spatial capital  (\href{Gronlund}{\citealt{gronlund_notions_2007}}).
While human capital refers to knowledge (intellectual capital), personality (habits and personal attributes) and the creativity (creative capital) of an individual (\href{Becker}{\citealt{becker_human_1993}}), social capital is mobilized through social networks and relations (\citealt{Bourdieu 1986}), and might be followed by economic (monetary income and  financial  assets) and cultural (demonstrated through education and aesthetic values) (\citealt{rerat_spatial_2011}).
\\

Contextual resources have an extraordinary transformative character that enables horizons of possibilities in spatial and social sense.
These resources reside in real and symbolic reconstructions and restructurings of everyday urban life and permanent urban forms and could therefore be addressed as spatial capacities and social potentials at the local level (\href{Swyngedouw}{\citealt{swyngedouw_glocal_2003}}).
%Swyngedouw and Kaika 2003)

\paragraph{Urban Practices vs Urban Conflicts}

On the other hand, urbanity is traced from the intensity of harmonizing and clashing social encounters which are determined by urban practices and conflicts produced by urban agency in the particular setting. 
\\

A constant change of urban actors and urban structures also accelerates flows of social practices (policies, actions and processes) that together induce the complexity and diversity of city life, and build urban experiences and urban capacity (\href{Robinson}{\citealt{robinson_ordinary_2006}}).
Urban practices contribute to the reproduction of the current urban order through the institutional capital of formal and informal institutions within private, public, voluntary sectors (\href{Vujosevic}{\citealt{vujosevic_regionalizam_2015}}), daily life, local rituals, ambiance and atmosphere, and a sense of belonging (\href{ref}{\citealt{volic_belgrade_2012}}).
\\

On the contrary, urban conflicts are defined as a confrontational relation between at least two actors, actor groups or categories of actors with respect to the management of urban issues (\href{ref}{\citealt{aznar_quels_2006}} {\href{ref}{\citealt{renau_nimbysm_2016}}).
Conflicts in general are incorporated in the systems of rules, habits, norms, even though they represent the threats and challenges to the current social order (\citealt{sears_good_2005}).
Urban conflicts derive from and target the relatively stabilized, routinized particular spatio-temporal set (\citealt{Brenner and Theodore 2005}), and therefore have intrinsically local character (\href{Sassen}{\citealt{sassen_toward_2007}}).
\\

Today there is an extensive set of urban conflicts stemming from the global level.
Many authors are of the opinion that conflictive interactions are embedded in the capitalist order and neoliberalism.
Urban conflicts thrive on discriminatory power dynamics, clashes of cultural differences and a series of confrontations of opposing viewpoints within a city and they tend to progress from a personal level to a socio-urban dimension.
\\

Nonetheless, in general terms, in a top-down manner the conflicts may arise from the external interventions in the local political-economic conditions, regulatory arrangements and power poles (e.g. conflicts between the business coalition and local residents and the role of the institutions in them) (\citealt{brenner_neoliberalism_2005}).
However, conflicts are usually paired with injustice, poverty and racism in urban sociology (\citealt{hubbard_revenge_2004}).
\\

Generally speaking, law and science aim to resolve conflicts by discovering truths (\citealt{brenner_neoliberalism_2005}).
However, there is another interpretation that material and human capital might be activated in conflict resolution (\href{Coleman}{\citealt{coleman_social_1988}}), in creating new tools, skills, capabilities and fundamentally transforming the existing relations into new, productive ones (\ref{ref}{Sears 2008}).
%NE DIRAJ
One such example is the recently popular boost of creativity as an engine for economic development and change (\href{Sears}{ibid.}).
Therefore, "favouring the dynamic, spontaneous nature of the urban process proposes a positive vision of urban conflicts" 
(\href{Sevilla-Buitrago}{\citealt{evilla-buitrago_debating_2013}}).

\paragraph{Urban System Transitions}

Bearing in mind this complex vision of urban reality, it is the micro level, in this case it is a neighbourhood, where the test of urbanity may find its fullest expression in terms of the societal challenges and the production of urban spaces (\href{Blotevogel}{\citealt{blotevogel_new_2008}}).
Namely, the relation between urban life and urban form creates potential/opportunity for urban system transitions (\href{Marcus}{\citealt{marcus_spatial_2007}}).
As was elaborated in [\href{Section 2.1.1}{Section 2.1.1}], system evolution is entrenched in the context through the activation of contextual elements .
\\

The operationalization of the concept of urbanity is then possible in monitoring the level of urbanity of a local context as an overview of processes fueled by urban practices, contextual resources, and urban conflicts.
Urban agency activates the context and inscribes local urbanity measured through its fluctuating degree (the level of urbanity) (\href{Marcus}{ibid.}).
Instead of taking a predefined political position by applying reasoning based on social order, power and class, the level of urbanity figures as an indicator of urban dynamics that combines:
(1) the resilient processes of the old socio-spatial order (maintenance),
(2) flexibility of structures and behaviours towards a "projected future" (transformation), and
(3) contradiction between preservation forces and radical differences (change).
Benefiting from urban practices maintain the urban system (\href{Taradin}{\citealt{tardin_landscape_2014}}), harnessing contextual resources indicate the possibility for transformations, and urban conflicts convey radical changes arising from the resolution of conflicts. Therefore, the level of urbanity reflects the multilateral, multichannel nature of cities that incites not only the constellation of social practices and harnessing contextual resources, but also evidences the production and the challenge of urban conflicts.
\\

Socio-spatial patterns of urban system transitions bend the way how decision making layers address urbanity as its constitutive reality and its ultimate positive goal.
This is the theoretical ground on which the proposed methodological hybrid traces the elements of a particular context for its categorical convergence and maps their interconnections and contributions to continuations, transitions and turnovers within urban development processes at the neighbourhood level. 
\\

\textbf{"The disadvantaged have structural interests that run counter to the status quo, which, once they are assumed, will lead to social change. Thus, they are viewed as agents of change rather than objects one should feel sympathy for." \citealt{Sears 2008}}

\section{Epistemological Framework}
%visualization
The presented conceptual framework imposes a scientific background on the analysis and delivery of data.
Its relevance is built on further elaboration from empirical and analytical work within a real-life experiment (\href{Parnell}{\citealt{parnell_retheorizing_2012}}).
\\

In this thesis, the research on urban development is divided between labeling urban complexity and tracing urban dynamics in the case study of a neighbourhood in Belgrade.
The chosen neighbourhood is a vivid and resourceful representation of post-socialist developmental circumstances
They are examined from the point of view of ordinary cities, without setting it into a matrix of evaluation in reference to the general, western model of development.
Therefore, the source of urban agency is attributed to 
all human and material, social and technical elements. They are assumed to contribute together to continuous urban system transitions.
\\

In urban studies, there is but one method that acknowledges the active role of non-humans in urban systems - Actor Network Theory (ANT).
All other methods that analyze actors' constellation are exclusively oriented towards human elements, to name but a few: various types of stakeholder analysis [impact/priority matrix, power/interest matrix, readiness/power matrix, support/opposition, importance/influence matrix,
constructive/destructive matrix] (\href{Mathur}{\citealt{mathur_defining_2007}}), The Institutional Analysis and Development (IAD) Model (\href{Ostrom}{\citealt{ostrom_governing_1995}}; \href{Ostrom}{\citealt{ostrom_upravljanje_2006}}).
\\

In urban studies, ANT has been already extensively applied as the explanatory construct that studies the associations and symmetrical relationality of all active elements of an urban environment (\href{Farias}{\citealt{farias_introduction:_2011}}).
This initial choice of ANT is further backed up by its
potential capacity to afford such openness and flexibility that is well suited within the scope of ordinary cities
\footnote{This has already been applied within this frameworok \href{Section 2.1.2}{Section 2.1.2}.} 
and enables deconstructing complex urban systems without assigning to them preexisting explanations.
Although ANT enables the exhaustive systematic description of an urban system, in concrete case studies, researchers meet their limits as it very often does not bring up new causalities in terms of facts, analysis, and conclusions. It provides a detailed description of confined urban environments, but meets its limits when confronted with complex real-life urban processes.
\\

Bearing in mind that this methodological approach hesitates to offer explanations and to analyse individual behaviours, it tends to fail at an operational level.
Moreover, when addressing an extensive practical application, reliability and credibility of data are not sufficient, it is also important to provide the generalizability of its results.
While no single method is without its limitations, it was crucial for this research project to keep in mind not limiting the research to the shortcomings of only one method.
\\

Being aware of the advantages and shortcomings of ANT, the mixed method approach is chosen in order to provide adequate scientific discourse and an operational framework for the research question to be answered in a satisfactory manner.
ANT's urge for methodological revisions, adaptations or complements in order to facilitate a wider understanding of the undercover processes and mechanisms is therefore resolved with its complementation from the Multi-agent system (MAS).
In their symbiosis, the application of these methods proposes a new reading of urban development processes in a post-socialist neighbourhood.

\subsection{ANT in analysis of urban development}

In recent urban studies, the grasped complexity and dynamics of the networked urban system has been extensively reinterpreted by Latour’s Actor-network theory (ANT), with all human, social and technical elements that are symmetrically treated within a system. All these entities contribute together to a dynamic perpetual networking, where an understanding of all phenomena, including the social ones, lies in the associations among them  (\href{Latour}{\citealt{latour_reassembling_2005}}). Differently put, it brings up the reproduction of inherent complexity and incompleteness of urban development in three gradual steps:  (A) labelling all active elements of an urban system (B) identification of their roles, and (C) focusing on the associations among them (\href{Chart 1}{Chart 1}). The contribution of ANT lies in: (1) instating the socio-material topology of urban networks, (2) navigating the interpretative dualism of urban theory (nature/society, local/global, action/structure), (3-3) elaborating the supremacy of associations that configure the relational understanding of the city, (3-4) overcoming spatial hegemony in a complex urban reality, and above all (3-5) radicalisation of the symmetry principle for human actions and non-human materials that allows tracing the consistency and extensibility of urban phenomena beyond its spatio-temporal manifestation (\href{Latour}{\citealt{latour_we_1993}}, \href{Murdoch}{\citealt{murdoch_spaces_1998}}, \href{Farias}{\citealt{farias_introduction:_2011}}) (\href{Table 1}{Chart X}).
\\

Even though the human is still the essential, inseparable urban element, this blending establishes new interpretation of cities as a composite entity where all objects (physical spaces and structures, tools, technologies, data, formulae and regulations, institutions and, of course, humans) are mutually constituting through enactment, interaction and translation of different elements (\href{Farias}{\citealt{farias_introduction:_2011}}) in  (\href{Latour}{\citealt{latour_reassembling_2005}:71}) words - "any thing that does modify a state of affairs by making a difference is an actor". An actor is granted activity by others, and can be the subject or object of an activity  (\href{Latour}{\citealt{latour_actor-network_1996}}). As such, the heterogeneous body of associations and symmetrical treatment of humans and non-humans contribute to place action outside the actors where ‘[a]n "actor"... is not the source of an action but the moving target of a vast array of entities swarming toward it’ (\href{Latour}{\citealt{latour_reassembling_2005}:46}). The figuration of a relation is what counts, not its nature, function or purpose; the network is established when arrangements between actors produce stable patterns of performance and practice (\href{Smith}{\citealt{smith_world_2003}}).
\\

ANT methodology redraws principal concepts of urban theory in actor-network terminology, and naming only a few, these include: social order; scale, power, decision making, governance, and urban development. The wide field of ANT application in urban research and practice addresses the urban core by encompassing not only analytical views on theory and research  (\href{Boelens}{\citealt{boelens_theorizing_2010}}), but also planning methodologies, policy and practice recommendations, and development prospects (\href{Healey}{\citealt{healey_circuits_2013}}). All these works adhere to the basic ANT principles: (1) treatment of material objects and representations through actor-networks; (2) reduction of well-known dualities and general concepts to in situ actors and networks; (3) and the nature and process of networking in terms of associations and translations (\href{Table ANT table}{Chart X}).
\\

Anchored in science and technology studies (STS), an early application dealt with the nature of human/non-human exchange in mapping land cover projects and GIS, allowing the reconciliation of data with different ontologies and addressing "nodes, links and type of links" terminology factor-networks  (\href{Comber}{\citealt{comber_actornetwork_2003}}). The analytical lenses in architectural, housing and planning studies have focused on materiality/artefacts/objects and an up-to-date fruitful application of ANT: 
(1) for identifying non-human actors which happen to be missing, silenced, or even rendered invisible in the practice of the housing system, markets and policy (\href{Gabriel}{\citealt{gabriel_post-social_2008}}); (2) as an interpretative tool for processes and mechanisms under review distinguishes active mediators and passive intermediaries (\href{Cowan}{\citealt{cowan_nominations:_2009}}); (3) as a theory of action for interpreting complex associations of people and things in architecture (\href{Fallan}{\citealt{fallan_architecture_2011}}); (4) for demystifying the complexity of stabilizing/destabilizing object enactment mechanisms as a way to readdress the position of ‘plan’, ’implementation’ and ‘design’ in governance and planning process (\href{Van_Assche}{\citealt{van_assche_co-evolutions_2013}}); (5) for assessing the relational aspect of assemblages as a way of explaining the influence of innovative tools for spreading explicit and tacit knowledge in planning and building sustainable cities (\href{Georg}{\citealt{georg_building_2015}}) (\href{Table 1}{Chart 1}).
\\

Furthermore, the network related ANT framework has been stretched to analytical research tendencies toward urban practices. Doak and Karadimitriou rely on Callon’s four steps in actor-network translations  (\href{Callon}{\citealt{callon_elements_1986}}) to map complex redevelopment processes once reduced to a set of associations in social relations and material objects and stabilized by intermediaries (\href{Doak}{\citealt{doak_re_2007}}). On the other hand, \href{Holifield}{\cite{holifield_actor-network_2009}:647}  advocates for the version articulated by (\href{Latour}{\citealt{latour_actor-network_1996}}) and the political usefulness of ANT and suggests using  intermediary/mediators' roles in risk assessment changes as a tool for "tracing the (contested) assembling without taking the existence of social relations" like capitalism and class "for granted". A similar stance has been taken by (\href{Boelens}{\citealt{boelens_theorizing_2010}}) to promote a relational view on spatial planning and how it interacts with the behavioural urban regime so that ANT serves to identify actors and see how they organize from the ground up, and not be identified from above through an objective, vision or plan. ANT seems to have been recently gaining attention as part of the wider poststructuralist approach to cities  (\href{Smith}{\citealt{smith_ordinary_2013}}), therein further emphasizing its role in the process of production and acceptance of associations in terms of evaluating the positionality of researcher agency in human geography (\href{Ruming}{\citealt{ruming_following_2009}}), and  reflecting the process of production and the acceptance of associations in urban enclaves (\href{Wissink}{\citealt{wissink_enclave_2013}}).
\\

Even though this post-structuralist ANT tenet mainly holds on a flattened, network-oriented interpretation of system dynamics, it has been recently argued that the role of material objects must also be acknowledged in all its vigour and heterogeneity. Tracing back non-human elements from Latour to Foucault, it is obvious that material objects can be everything but passive and have been playing various social roles such as: (2-1) reflecting and maintaining the social order, (2-2) facilitating social relations, (2-3) moral and political signposts, and (2-4) acting asintermediaries of the social across space and time (\href{Sayes}{\citealt{sayes_actor-network_2014}}; \href{Van Assche}{\citealt{van_assche_power_2014}}) (\href{Table 1}{Chart X}). Henceforth, non-humans, when granted agency, become intermediaries/mediators and actors and their active engagement in urban development refers back to various levels of urban decision making: (X) upholding legitimacy of urban planning, (XX) underpinning multiple realities of real-estate interest, and (XXX) personalizing participatory urban transformations through actor-network perspective (\href{Latour}{\citealt{latour_actor-network_1996}}; \href{Rydin}{\citealt{rydin_actor-network_2010}} \href{Van Assche}{\citealt{van_assche_power_2014}}) (\href{Table 1}{Chart X}).
\\

ANT seems to continue to provide a conceptual framework for interpreting and guiding various ways of examining networks and has demonstrated a substantial coherence as "a pragmatic approach to study actual practice in concrete sites and situations" (\href{Farías}{\citealt{farias_culture_2015}}:526), which affords focusing on a description of the performativity of the black-boxed social world through: (1) the active role of non-humans, (2) the sociology of translations (3) free associations, (4) inseparable actor-networks, (5) urban assemblages (\href{Latour}{\citealt{latour_actor-network_1996}}). The concept of assemblages is aptly after capturing the complex relationality of a dynamic urban system, though it fails to go beyond ‘following the actors’ technique of examining human-human-nonhuman interactions (\href{Cowan}{\citealt{cowan_nominations:_2009}}) and to facilitate wider understanding of their normative and transformative nature (\href{Gabriel}{\citealt{gabriel_post-social_2008}}).
\\

The rudimentary yet hyper dynamic circumstances of transitional societies offer an insight from within the network on how the body of norms, projections and structures of urban development unfold and upon the network of how the associations and translations of basic elements are formed and developed. In Serbia, the urban planning framework withstands a complex and elaborated institutional legacy, yet holds rather a symbolic meaning  (\href{Nedovic}{\citealt{nedovic-budic_adjustment_2001}}), The fragmented and uncontrolled spatial transformations are governed by a constellation of different, often illegitimate, interests  (\href{Petrovic}{\citealt{petrovic_cities_2009}}), and, on site, the spectrum of active but powerless urban actors struggle to develop flexible social patterns and networks (\href{Cvetinovic}{\citealt{cvetinovic_engine_2013}}). Therefore, the case study of a post-socialist neighbourhood in the capital of Serbia is a good illustration for observing the relationships between top-down urban planning, interest-based urban transformations and bottom-up urban design activities. Moreover, very few methodological research studies have bothered to examine urban  development modalities  in  transition,  apart  from  replications  of  the  approaches  taken  by  neo-liberal  or institutional economies (\href{Tsenkova}{\citealt{tsenkova_urban_2007}}). In this respect, we aim to examine the utility of ANT analysis for understanding the developmental reality of the Savamala neighbourhood in Belgrade.

\subsection{Multi-Agent System for Process Simulation}

Very important for developing hybrid methodologies is the correspondence of the individual epistemological framework of the methods (\cite{mixed method}). In this respect, the researcher has realized that the dynamics of urban reality interpreted by ANT matches the concepts of agency, communication, cooperation and coordination of actions, where all elements influence each other simultaneously (\href{Ferber}{\citealt{ferber_multi-agent_1999}}).
This interpretation corresponds to the Multi-agent system (MAS) approach for complex computing systems.
\\

This approach has already been applied in urbanism as a simplified problem solving strategy primarily used for urban decision-making processes. The multiple urban actors and stakeholders are first converted into agents. Then, they are used for simulating social organisations in which these agents are embedded (\href{Bousquet}{\citealt{bousquet_multi-agent_2004}}). A multi-agent paradigm is actually very useful as a structuring method that gradually builds the capacity and flexibility of systems.
Its potential lies in analysing the operationality, functionality, usability and extensibility of decision-making mechanisms on urban land use and land cover (\href{Brown}{\citealt{brown_path_2005}}), housing market dynamics (\href{Diappi}{\citealt{diappi_smiths_2008}}) and Planning Support Systems (PSS) (\href{Saarloos}{\citealt{saarloos_multi-agent_2008}}). The MAS methodology is, in fact, a process generation tactic based on the principles of  ecosystem management (levelling, fluctuating, evolving). It applies the technique of categorizing the process infrastructure with apparatuses (the set of fields of influences and major forces) and procedures (the set of operational agencies)  (\href{Bousquet}{\citealt{bousquet_multi-agent_2004}}).
\\

In general terms, MAS goes along with ANT as it also aims to explore and understand the system, not to predict the future.
However, the radical difference that contributes to its operationality is the focus on spatio-temporal dynamics. MAS tests the impact of interactions and structures that emerge from these interactions (\href{Crooks}{\citealt{crooks_multi-agent_2014}}). It can therefore serve to complement actor-networks (ANT) with a systematic framework where MAS analysis of agent behaviours provides fine tuning for qualitative discrepancies in the system. 
\\

The characteristics of MAS that are very useful for an operational update of ANT in exploring urban dynamics are:

\begin{enumerate}
\item Profiling elements as agents: 

With agent profiles, the complex system is divided into subcategories, all of which are identified as independent subunits (agents) and then the activity among these subunits is coordinated. This allows for agent typology, "an object-oriented approach and, as such, [enables] to distinguish actors, activities, flows, investments, facilities, regulations, rights, issues, forces, opportunities and constraints" (\href{Hopkins}{\citealt{hopkins_structure_1999}}; \href{Saarloos}{\citealt{saarloos_multi-agent_2008}}). 
Moreover, distinguishing active-passive roles of agents (proactivity, sensibility, capacity for interaction) may be crucial for representing real forces in an urban environment.
 
\item Describing the impact of procedures/agencies by categorizing the agents accordingly (\href{Arsanjani}{\citealt{arsanjani_spatiotemporal_2013}}). These agencies are usually system transitions.

\item Exploring a generative bottom-up typology of the system by defining rules that govern urban dynamics (\href{Bretagnolle}{\citealt{bretagnolle_simulating_2010}}).

Identify the rules facilitates bridging the gap between top-down (evaluation of global trends) and bottom-up agent behaviours (local decisions which lead to emerging landscape patterns over time) (\href{Bone}{\citealt{bone_modeling---middle:_2011}}). 

\item Analysing complex systems through the agent-based view on urban decision-making (links among agents' perceptions, representations and actions), control (hierarchical relations among agents) and communication (the syntax of the interaction between human decision-makers and biophysical changes) (\href{Bousquet}{\citealt{bousquet_multi-agent_2004}}; \href{Brown}{\citealt{brown_exurbia_2008}}).

\item A multi-agent model for simulating system dynamics

The aim is to understand and explore the system. The model is able to describe the emergent phenomenon and the dynamic behaviour of the system,  and to draw  consequences on the environment and agent behaviours (system dynamics) (\href{Diappi}{\citealt{diappi_smiths_2008}}). The primary modules of this model are borrowed from its application for programming systems in the computer sciences.
Accordingly, they envision (\href{Brown}{\citealt{brown_path_2005}}):
(1) environment;
(2) assembly of agents;
(3) set of objects;
(4) assembly of relations, the agent’s interaction with the environment (agent behaviour);
(5) assembly of operations making it possible for the agents to perceive, produce, consume, transform and manipulate objects through their relational behaviours;
(6) laws of the system, the reaction  of  the  environment  to  this  attempt  of  modification.
\end{enumerate}

This is the theoretical ground on which the hybrid methodology could be built. The combination of MAS and ANT methodological approaches takes into account all active agents regardless of their sort (ANT), their interdependencies and interconnections (ANT and MAS), and maps their contributions (MAS) to urban system transitions  in a post-socialist context at the neighbourhood level.

\section{Local framework}
%visualization

Even though globalization and urbanization are unavoidable,  worldwide,  broad,  general  and  mutable  processes, they still  contain  a  necessary connection to place.
Bearing in mind the ordinary city argument, every urban environment engenders a specific set of internal and external influences.
If anything, global aspects at least spontaneously transform to meet local specifications and local circumstances distort and sprout in reaction to global ones. The particularities of the local urban setting are vital factors with regard to dealing with the uncertainty of its urban development processes.
\\

While the evolution of urban systems very often happens in a series of cycles, the periods of fundamental transformations and radical changes in micro environments therefore become a congregated space-time bound to observe how urban agency is aggregated and articulated through the morphology of urban decision-making (\href{Watson}{\citealt{watson_planning_1998}}; \href{Nedovic}{\citealt{nedovicbudic_waves_2006}}).
Bearing in mind the proliferation of global solutions, copying strategies and imitating models, difficult circumstances and hybrid situations may provide a resourceful insight in how urban development processes unfold (\href{Harrison}{\citealt{partington_case_2002}};  \href{Nedovic}{\citealt{nedovicbudic_waves_2006}}).
\\

The choice of post-socialist cities as a localized typology of analysis responds to both the need for an intensive, diverse, localized dynamics and a complex set of various factors, facts, forces and prospects. Moreover, as the country of Serbia does not belong to either the Western world or among their colonies, its developmental path may reveal certain conditions and processes which could not be researched otherwise.
In compliance with with \href{Sjoberg}{Sjöberg's} presumption that an emerging scope of contemporary urban studies is now at play that comprises both post-socialist/post-communist and post-colonial study research, the case study in the Serbian Capital therefore contributes to "building a de-Westernised version of urban theory" (\href{Sjöberg}{\citealt{sjoberg_cases_2014}}:8).
\\

Namely, current circumstances in the neighbourhoods in Belgrade are referenced as not only post-socialist, but also transitional.
In general, the theory of transition usually addresses democratization processes in a country context and emphasizes the links between the current transitional moment with the pre-transition past (\href{Thomas}{\citealt{thomas_thinking_1998}}; \href{Holmes}{\citealt{dryzek_post-communist_2002}}).
In the domain of this research, the time frameworks of post-socialist and transitional conditions and processes are revised to comprise both past-based and future-oriented circumstances.

\subsection{Socio-spatial Patterns of Post-socialist Cities}

As already indicated [\href{(ref introduction section)}{Section 1.X}], transitional countries in CEE have undergone severe social changes on their way from socialism to capitalism after the fall of the Berlin Wall and the dismantlement of Yugoslavia.
Naturally, cities in these countries have followed a similar transitional path to move away from their socialist past
\\

Included in this range of spatially and economically turbulent surroundings, post-socialist cities are under significant political, economic and social pressure.
These factors have provoked a legal void susceptible to shady deals and questionable public-private partnerships (illegality); a lack of strategically proactive urban governance, which has resulted in tolerance to illegal building practices (informality); increasing social polarization (inequity); and poverty in this region (the number of poor people reached 100 million in CEE by 2001)  (\href{ref}{\citealt{tsenkova_beyond_2006}}).
On the other, the huge socio-cultural base inherited from the socialist period with centralised and authoritarian practices dominate post-scoialist urban governance. This has had a profound influence on the spatial adaptation and social repositioning of post-socialist cities.
\\

In urban studies, transition is explained as a process of transformations from "a ubiquitous socialist city" to "a capitalist city"as the final goal of this process (\href{Nenadovic}{\citealt{nedovic-budic_mornings_2011}}).
In theoretical terms, the transitional process considers radical shifts from (\href{Petrovic}{\citealt{petrovic_cities_2009}}): 

\begin{itemize}
\item The totalitarian to democratic political system;
\item Planned to market-based economy;
\item Public to private property ownership; 
\item Supply to demand driven economy; 
\item Industrial to service based society;
\item Isolated to integrated position in the world economy.
\end{itemize}

The major characteristics that theoretically predefine transition is the non-democratic political past
\footnote{According to \href{Linz}{\cite{Linz and Stepan (1996)}}, non-democratic political system are classified into: authoritarian, totalitarian, post-totalitarian or sultanism.}
and the consolidation of the new order through five arenas of democracy:
(1) a civil society and freedom of speech;
(2) a political system based on free elections,
(3) the rule of law,
(4) an consistent and legitimate state apparatus, and
(5) economic society with an institutionalized market (\href{Thomas}{\citealt{thomas_thinking_1998}}).
Nonetheless, in practice, important macro integration factors that initiated transition represent not only the change in the political regime, but also the crisis of cultural identity and, above all, structural economic change and the shift in the mode of production (\href{Thomas}{ibid.}).
\\

Generally speaking, transition marks the period of substantial reconstruction of the pillars of the system and society, turbulent economic processes and fluctuating  urban development circumstances (\href{Nedovic}{\citealt{nedovic-budic_mornings_2011}}).
While political and social changes tend to happen slowly, if happen at all, economic transition conquers the local context very fast. 
In this respect, the urban economy becomes a playground for: (1) macro-economy policy reforms at the national level, (2) installment of free-market ideology, (3) the first steps in privatization, (4) promotion of consumerism, and (5) dismantlement of industrial production (\href{ref}{\citealt{world_bank_cities_2000}}).
\\

These adaptations and transformations happen under the watchful eye of international actors, while foreign influences become crucial for the actual conduct of the transitional processes.
Namely, several authors mention the ambiguous influences from Western states in the CEE context, declaring that it might be that these post-socialist states serve as  a laboratory of change for the west (\href{Maier}{\citealt{maier_czech_1998}}; \href{ref}{\citealt{vujosevic_post-socialist_2010}}).
Therefore, even though the essence of transition is rooted in a pre-transitional past, the processes at play are rather future-oriented targeting the core of the capitalist value system and social order.
\\

However, with the huge socio-cultural base inherited from the socialist period, cities in transitional countries tend to partly continue with their pre-transitional past. They usually maintain their dominant positions and stay the centres of economic growth  with  a  variety  of  services, national priority for expansion,  technological  innovation  and  cultural  diversity. 
Therefore,  the  post-socialist period in these cities contains prevailing characteristics of the disintegration of the preceding system rather than a coherent vision of what should follow (\href{Stanilov}{\citealt{stanilov_post-socialist_2007}}). 
\\

The concept of "path dependency" has become crucial to express the influence of the past on the current urban processes (\citealt{stark_system_1992}; \href{djordjevic}{\citealt{djordjevic_system_2009}}).
In a post-socialist city, these are the institutions and practices that have survived the system transition as well as the initial defensive policies of socialist states (\href{Thomas}{\citealt{thomas_thinking_1998}}): 
(1) acceptance of market  forces and regulatory efforts of the state towards it;
(2) basic coordination with international laws and regulations;
(3) efforts toward the reduction of public spending;
(4) acceptance of welfare state limits;
\footnote{As the famous quote has it: "the welfare  state can be defended  but  not  extended." \href{Thomas}{\citealt{thomas_thinking_1998}}}
(5) acceptance of  privatization  as unavoidable;
(6) challenge to equality with incentives and competition;
(7) acceptance of international finance through international agreements;
(8) acceptance of the opportunities contained in European  integration programmes.
\\

The major characteristics of post-socialist urban development are: a multitude of actors, various economic, social and political interests, the social aspects and fragmentation of urban spaces.  Consequently, post-socialist cities, similar to developing ones, lack complex, operational logistics (\href{ref}{\citealt{repetti_icts_2010}}) to link top-down changes to bottom-up interventions in urban systems. There exists a growing discrepancy between the national and global levels, on one side, and city and neighbourhood levels, on the other.
\\

According to \href{Sykora}{\cite{sykora_multiple_2012}},
%NE DIRAJ!!!!!!!!!!!!!!!!!!!
the transition generally attacks the institutional framework, urban practices and the built environment, and these are also the domains where the post-socialist setting fights back.
The confusing overlap of post-socialist traditions and transitional values in these sectors provoke a legal void susceptible to a variety of influences and interests that have profound influence on the spatial adaptation and repositioning of post-socialist cities in terms of (\href{Stanilov}{\citealt{stanilov_post-socialist_2007}}): 

\begin{enumerate}
\item \textbf{Urban management} - political and financial powers profit from the weak institutional framework and extend their wealth and influence by further loosening official urban planning framework and practice; 
\item \textbf{Urban patterns} - illegal construction reduces the spatial scale and spatial formalism of urban structures; 
\item \textbf{Urban impact} - urban practices marked by inequity lead also to social and spatial stratification of urban structures.
\end{enumerate}
 
In other words, what proceeded after the end of the socialist era is a neoliberal model of urban planning with the supremacy of market-oriented solutions for urban problems (\href{Sager}{\citealt{sager_neo-liberal_2011}}).
In such a situation, urban planning was not a priority (\href{Sykora}{\citealt{sykora_transitional_1999}}), and it was not considered effective for managing local urban issues (\href{ref}{\citealt{maier_czech_1998}}; \href{Vujosevic}{\citealt{vujosevic_planning_2006}}).
Therefore, planning was narrowed down to merely a technical issue and very few theoretical or general methodological research studies have bothered to examine alternative planning modes in transition, apart from replications of the approaches taken by developed countries (\href{Begovic}{\citealt{begovic_ekonomika_1995}}).
Thus, \href{Zekovic}{\cite{zekovic_spatial_2015}}
%NE DIRAJ!!!!!!!!!!!!!!!!!!!!!!!
mark the points where the post-socialist urban planning system breaks down through the collisions and mixtures of :
(1) comprehensive vs incremental planning; 
(2) centralized vs decentralized decision making;
(3) top-down vs bottom-up approach;
(4) interventionist vs entrepreneurial urban management.
\\

Post-socialist urban decision-making substantially fails through the discordant visions and interests, nonexistent implementation and authoritarian vertical coordination. Furthermore, very few theoretical or general methodological research studies have examined alternative modes for urban development in transition or post-socialist urban planning, apart from replications of the approaches taken by neo-liberal or institutional economies (\href{Tsenkova}{\citealt{tsenkova_urban_2007}}).

\subsection{Post-socialist Urban Framework in Belgrade}

Urban systems of post-socialist cities are highly susceptible to tense on-going transformations, diverse but reciprocal in their nature:
(o) economic transformations (transformation of production and consumption in relation to space, income polarization and poverty),
(o) political transformations (urban governance, political voluntarism, participation and de-centralization),
(o) spatial transformations (demographic trends and distribution of functions) and
(o) social transformations (social exclusion-inclusion, social activism and informality). In other words, what proceeded after the end of the socialist era is a neoliberal model of urban planning with the supremacy of market-oriented solutions for urban problems (\href{ref}{\citealt{sager_neo-liberal_2011}}).
\\

Conversely, with the huge socio-cultural base inherited from the socialist period, cities in transitional countries have continued to be centres of economic growth with a variety of services, expansion, technological innovation and cultural diversity. While some trends and directions within these transformations are clear and defined, uncertainty dominates decision making and implementation in the turbulent environment of post-socialist cities (\href{ref}{\citealt{nedovic-budic_mornings_2011}}). Therefore, the post-socialist period in these cities contains prevailing characteristics of the disintegration of the preceding system, rather than a coherent vision of what should follow.
\\

The  dismantling  of  the  socialist system
\footnote{The system in Socialist Federate Republic of Yugoslavia (SFRY) was addressed as socialism or communism in various works of urban studies and urban planning research. However, after an extensive review of the literature and consultations with local experts, this research has applied the term socialism. Namely, its Constitution and the name of the state indicate a socialist political system  (\href{Ustav}{Ustav Jugoslavije [Yugoslav Constitution] 1963}).}
in Yugoslavia during the late 1980s was the entry point for substantial transitional change within all aspects of the economic model, the political order and the social organisation in Serbia. 
However, after 25 years, Serbia still finds itself in a post-socialist proto-democracy without functional, reliable, and developed institutions of a representative 
democracy, civil society and market economy (\href{Vujosevic}{\citealt{vujosevic_postsocijalisticka_2010}}). 
Therefore, in the course of merging socialist and neoliberal socio-economic condition in Serbia, regulatory practices and organizational solutions have led to inefficiently operationalized and inconsistently formalized institutional reforms rather known as "growth without development" (\href{ref}{\citealt{vujosevic_collapse_2010}}).
\\

During the socialist period, general urban planning in the former Yugoslavia incorporated the process of paradigm change, in Kuhn’s sense of the word (\citealt{kuhn_structure_1962}), and set a specific pace of progress, disjunctive with that of the Western countries of the time and dependent on the current state of socio-economic and political affairs at the national and city level of the time (\href{Bajic}{\citealt{bajic-brkovic_city_2002}}).
While it initially started as a top-down comprehensive planning practice, it was significantly developed toward a decentralized and participatory planning procedure by the mid 1980s. Its bottom-up nature and multidisciplinary practice was acknowledged and praised by developed planning cultures from Western countries (\href{Cullingworth}{\citealt{cullingworth_planning_1997}}).
The discrepancy between theory and practice in the late 1980s initiated the abandonment of this planning model with a fixed future vision of the urban environment. Yet the real dissolution of the planning paradigm started in the 1990s due to the disintegration of Yugoslavia’s socialist system and the destabilization of the institutions, which brought in the lack of legitimacy in urban planning in the post-socialist cities of the newly established state  (\href{Vujosevic}{\citealt{vujosevic_post-socialist_2010}}).
\\

Accordingly, the urban planning system in the Serbian capital of Belgrade had presented a high number of strategies and their practical implementation during the previous socialist regime, while during the post-socialist period of the 1990s implementation became continuously hindered by political instability, convergent socio-economic forces and inconsistent planning models.
In practice these conditions have resulted in placing the strategic plan as an advisory long-term urban vision, but leaving the real actions and decision making to political and market forces.
The situation was not any better when more intensive transition started in the early 21st century. Even then, it was rather qualified as a slow socio-economic transformation with a low rate of foreign investments, dominated by the flurry of wild capitalist and stumbling rudimentary democracy.
\\

While the western planning paradigm involves corrective factors for urban failures inherited from the free market and democratic principles (\href{nedovic}{\citealt{nedovic-budic_adjustment_2001}}), the path dependency tradition of urban planning in Serbia nurtured an insufficient, ineffective and biased urban  planning  framework.
As a result, urban development of Serbian cities, and above all its capital, most often has exceeded and diluted the common strategic framework defined from top-down: to establish clear links between the process of strategy development, its institutional framework, the hierarchical structure of the long-term and short-term objectives of all actors involved, and the real-time changes happening simultaneously in an urban environment.  
Complex institutional  legacies  influenced  the  behaviour  of  all  urban  actors,  preventing  the  development  of  flexible  social  patterns  and networks and falling short of providing overall legitimacy for the constellation of different interests in the post-socialist context of Belgrade (\href{Petrovic}{\citealt{petrovic_cities_2009}}).
\\

%already have it in case study chapter
These circumstances imply that urban decision-making in Belgrade is very often performed outside the institutions of the regulatory framework. Even though local authorities and the civil sector possess the legal empowerment and prescribed procedures of institutional control, they lack adequate and operational instruments for exerting their power in everyday practice   (\href{Bajec}{\citealt{bajec_rational_2009}}).
In addition, public interest in local authority services is a result of the direct influence of the political programs of those who are involved in local authorities and who are, at the same time, active protagonists within the global and national political scene  (\href{Djokic}{\citealt{djokic_political_2007}}). The pervasiveness of such uncontrolled and even illegal development leads to the deconstruction of urbanity (\href{Vujovic}{\citealt{vujovic_belgrades_2007}}).
\\

Urban decision-making in Belgrade is more reactive to the interests of capital investments, as well as being more tolerant of illegal practices than it is strategically proactive.
Thusly the produced results lead to:
(1) organic rather than comprehensive entrepreneurial city development (\href{Petrovic}{\citealt{petrovic_cities_2009}}),
(2) a laisser-faire economy and
(3) a global consumer culture which dissolves the democratic 
capacity of countries in transition (\href{Ellin}{\citealt{ellin_postmodern_1999}}).
The main characteristics of such urban system transitions  are marked by: 

\begin{itemize}
\item \textbf{investor urbanism} stemming from a loose regulatory framework and vulgar economy patterns (\href{Vujosevic}{\citealt{vujosevic_post-socialist_2010}});

Inadequately regulations and inconsistent management of urban land and undefined procedures of property ownership  changes pave the way for the powerful to extend their activities and profit, but hinder those defending the public interest and citizen rights to oppose such deeds (\href{Vujovic}{\citealt{vujovic_belgrades_2007}}).
Consequently, the authority to direct interventions and interventions in urban spaces belongs to a handful of powerful political and economic actors enabling various spatial manipulations and reinterpretations of planning outcomes to conform their interests (\href{Van}{\citealt{van_assche_co-evolutions_2013}}).

\item \textbf{pluralist political life and political voluntarism} which dominates the implementation of laws (\href{Djokic}{\citealt{djokic_political_2007}});

In Serbia, the oligarchy from the 1990s was and still is influential. Even though the regimes have changed, the "buddy" and brotherhood networks are making their ways through inefficient and corrupt public institutions to satisfy their own interests (\href{Vujovic}{\citealt{vujovic_belgrades_2007}}).
Corruption, manipulation and clientelism have governed most of the institutional relations and practices in the 
public domain, where political actors have become powerful economic actors within an un-transparent and 
semi-legal system (\href{Vujovic}{ibid.}).

\item \textbf{citizen resignation and political passivity} holding back the expansion of participation (\href{Vujovic}{\citealt{vujovic_belgrades_2007}}). 

Within such a blurred institutional framework urban actors with no political or economic power become marginalized and deprived of their rights to be actively involved in designing their urban environment (\href{Bolay}{\citealt{bolay_urban_2005}}).
These circumstances, strongly demotivate citizens to participate in society and to even defend their interest in a system where the cumbersome procedures and political connections are those that matter (\href{Vujovic}{\citealt{vujovic_belgrades_2007}}).
\end{itemize}

While the positive traits of Yugoslav self-management were easily abandoned, the negative ones were kept and extensively used (e.g. paternalism, manipulation, clientelism) (\href{Vujosevic}{\citealt{vujosevic_conundrum_2012}}.
According to \href{Vujosevic}{\cite{vujosevic_regionalizam_2015}},
%NE DIRAJ CITE!!!!!!
these flawed practices were further boosted by:
(1) extensive administrative centralization;
(2) the territorial integrity of policies and projects at the national level, but selective decentralization of territorial governance;
(3) the lack of mechanisms for the articulation of the common interest, yet dominant party-political affiliation of interest;
(4) a complicated system of budget allocation between the Republic and municipalities engendering regional development differences;
(5) flawed land/construction management and evaluation tools and mechanisms;
(6) a biased and unclear local stakeholder matrix and interest map (e.g. cultural, economic, political, military etc);
(7) the absence or inconsistency of the "Rule of law" principle within the urban planning system;
(8) spatial chaos (low construction and urbanization, illegal construction)
(9) an unfavourable demographic structure and demographic recession (influx of refugees, aged population and extensive brain-drain).
In these circumstances, any substantial societal change has been  degraded  and  misinterpreted  with  superficial  economic  liberalization  and  hyper  production  of ungrounded formalizations (emergence of new institutions and numerous policy agendas). 
\\

Due to these circumstances, the urban development processes of post-socialist cities are perceived as multi-dimensional, intense and semi-autonomous.
Based on the example of Belgrade, they reveal the inclining tendency to generate basic conceptualizations and typologies that could be used to deconstruct the economic, social, demographic, political and technological complexity and dynamics of post-socialist urban systems.
  
\subsection{The Neighbourhood is the Local Unit}

Even though post-socialist cities represent a research challenge for understanding urban development processes, taking into account the urban agency of all types (human and nonhuman), its substantial complexity and dynamics might be hard to follow in order to produce meaningful results. In this reference, the neighbourhood unit, as part of the city with a distinct spatial and social identity and functional coherence (\href{Merlin}{\citealt{choay_dictionnaire_2010}}), is a favourable choice as a representative micro environment.
Moreover, the neighbourhood unit has already been used as an indispensable tool for the analysis of development processes and the practice of city organization around the globe (\href{Meenakshi}{\citealt{meenakshi_neighborhood_2011}}).
\\

The neighbourhood as a fact of nature - a place where people live together with common origins or common purposes - is an old definition  \href{Mumford}{\cite{mumford_neighborhood_1954}}
%NE DIRAJ CITE!!!!!!!!!!!!!!
that has still been used in theory and in practice. The neighbourhood is not an intrinsically urban phenomenon, it can be found in both urban and rural environments
(\href{Merlin}{\citealt{choay_dictionnaire_2010}}).
\\

Throughout history, it aggregated spontaneously, while now neighbourhoods are built systematically.
Yet in all times the neighbourhood has represented a strong sense of attachment, identity, admittance and belonging for inhabitants (\href{Meenakshi}{\citealt{meenakshi_neighborhood_2011}}).
Having said that, the concept of neighbourhood is now extensively used in urban design and planning as a tool for providing a sense of place (\href{Patricios}{\citealt{patricios_urban_2002}}) and social well being (\href{Meenakshi}{\citealt{meenakshi_neighborhood_2011}}).
\\

However, a neighbourhood must have mobile borders and it figures in this way as a part of the city. It is an aesthetic unit, but the neighbourhood should not be self-contained or self-enclosed (\href{Mumford}{\citealt{mumford_neighborhood_1954}}).
A neighbourhood is distinguished by its: (1) particular topography, (2) common history, (3) specific built environment typology, and (4) characteristic social distributions
(\href{Merlin}{\citealt{choay_dictionnaire_2010}}).
It is not necessary that it is an administrative unit (\href{Merlin}{ibid.}).
Yet they do require minimal autonomy and space-time continuation (\href{Merlin}{ibid.}).
\\

The idea of a neighbourhood is also a symbolic manifestation of the ideology of community and togetherness  (\href{Lefebvre}{\citealt{lefebvre_production_1974}}).
Furthermore, the growing interest in neighbourhoods and the sense of neighbourhood is a part of the new traditionalism approach for preserving the socio-cultural values of the past in contrast to current technologized and virtualized sense of being and place (\href{Meenakshi}{\citealt{meenakshi_neighborhood_2011}}).
Neighbourhoods are the representatives of urban life, specificities and differences of particular lives. Particular neighbourhoods altogether create the spirit of the city. Neighbourhoods are always recognizable, a specific set of spatial attributes and historical consistency  (\href{Merlin}{\citealt{choay_dictionnaire_2010}})..
Paired with distinguished urban life, these characteristics make of a neighbourhood a key landmark in the city (\href{Merlin}{ibid.}).

\textbf{"le quartier (...), organise par les forces sociales qui ont modele la ville et organise son developpement (...) est une forme d'organisation de l'espace et du temps de la ville (..). Il serait la moindre difference entre les espaces sociaux multiples et diversifies, ordonnes par les institutions et centres actifs. Il serait le point de contact le plus aise entre l'espace geometrique et l'espace social, le point de passage de l'un à l'autre" Citing Lefebvre in \href{Merlin}{\cite{choay_dictionnaire_2010}}}
%NE DIRAJ!!!

\section{Theoretical Framework}

The theoretical framework of this research is based on the superposition of the conceptual layer over the layer of local context with help from the epistemological framework of the mixed method.
\\

The local context represents the practical base of the theoretical framework. The relevance of post-socialist cities is demonstrated through the multitude of data that comply with the categorical requirements of both conceptual and epistemological frameworks. On the other hand, a further reduction to the neighbourhood level is argumented by the need for controlled circumstances and therefore bounded micro-environment. What is more, the theoretical elaboration of the neighbourhood as a socio-spatial element of a city also provides a theoretical reasoning on the neighbourhood level and its relevance for this research. 
 The issue will be further elaborated in [\href{Chapter 3}{Chapter 3}].
\\

The methodological focus of this thesis limits the conceptual framework to the set of rather general, widespread terms of urban studies (urban development, agency, decision-making, urbanity, socio-spatial patterns).
In the scope of this research, these concepts have been traced within the scientific literature of urban theory, urban studies and urban planning. Their operational definitions were chosen to best correspond to the requirements of the methodology.

\begin{itemize}
\item Urban development is addressed as a sum of context-specific processes of urban system transitions.
\item Urban agency is everything that makes a difference and that engages with other elements in an urban system (human/non human, material/non-material).
\item Urban decision-making represents a social distribution of urban agency in an urban system.
\item Urbanity links contextual properties (socio-spatial patterns) with urban agency and indicates paths for urban system transitions.
\end{itemize}

Moreover, the binding frame for these concepts is found in "ordinary cities" theory. On the one hand, the ordinary cities concept supports a value neutral vision of urban development. On the other hand, it covers a missing link between these concepts and the methodological framework, namely through the theoretical relations already established between ordinary cities research and ANT.
\\

The adequacy of the epistemological framework is initially based on the contribution of ANT in terms of:
(1) the role of non-humans,
(2) approaching the environment as a relational process, and
(3) mapping the transitions through horizontal links and associations among actors (\href{Latour}{\citealt{latour_reassembling_2005}}).
However, its operationality is enabled through the support of MAS in terms of agent-based mapping and the framework for the interpretations of agent behaviours. 
\\

The elaborated theoretical reasoning frames the theory behind the research objectives (\href{Section 1.X.X}{Section 1.X.X}), backs up the general research question defined in (\href{Section 1.X.X}{Section 1.X.X}), indicates its relationship to the research problem (\href{Section 1.X.X}{Section 1.X.X}) and explicate the path for the drafting of hypotheses (\href{Section 3.1.2}{Section 3.1.2}) within the specified relations of the conceptual framework, their compatibility with the local context and their applicability within the chosen hybrid method (\href{Section 3.2}{Section 3.2}).

\textbf{"Realities are not flat. They are not consistent, coherent and definite. Our research methods necessarily fail. We need to understand that our methods are always more or less unruly assemblages." \href{Law}{\cite{law_making_2007}}}

%visualization of the framework and conclusions

%%%%%%%%%%%%%%%%%%%%%%%%%%%%%%%%%%%%%%%%%%%%%%%%%%

\chapter{The Journey through Methodology}

%%%%%%%%%%%%%%%%%%%%%%%%%%%%%%%%%%%%%%%%%%%%%%%%%%
Before delving into the data sampling and outcomes of this research, it is crucial to delineate the research process and procedures. Within the scope of this thesis, the research process involves the development of an organized body of knowledge on the urban development processes in post-socialist cities. The aim of this chapter is to justify the choices made about what and how to research and the means to collect and analyze the data.
\\

This chapter starts with a presentation of the larger framework in which the research objectives presented in the introduction are conducted into context-specific research questions and working hypotheses. Next, an explanation for the choice of the case study method, the criteria for the case study selection, as well as the mixed method methodological approach are listed, along with a brief overview of the methods and techniques used.


\section{Research Framework}

This thesis starts from the trendy term of urban development in order to scrutinize urban complexity and dynamics in a more operational and procedural manner. The following layers of this research, reflect its challenging nature:

\begin{enumerate}
\item trace and propose a context-based definition of urban development and identify the corresponding concepts that comply with it;
\item elaborate the validity of a post-socialist neighbourhood as a case study that blends and reveals the complexity and dynamics of a modern urban context;
\item apply the Actor-network theory framework for the descriptive analysis of a post-socialist neighbourhood;
\item construct a MAS-ANT visual hermeneutic set as an engine for agent-based representations of urban complexity and dynamics.    
\end{enumerate}

The logistical construction of the inquiry involves an exploratory journey through facts, phenomena and theories of a conceptual framework within urban studies using the proposed methodological hybrid of Multi-agent system and Actor-network theory. The fundamental question stays the same: it is crucial to understand what is going on in cities under the hood of urban development and, even more, how it is occurring.
\\

The current body of knowledge on this matter provides input on how to transform and adapt the general concepts mentioned earlier into the indicators of the complexity and dynamics of urban systems and corresponding development processes. The theoretical framework has provided the foundation of phenomena, facts, and theories in this direction, by acknowledging the conversion of general concepts into indicators as follows (\href{Section 2.1}{Section 2.1}):

\begin{enumerate}
\item \textbf{Abstract concepts are reinterpreted through complex indicators, which may be traced through the relations of field-data:}

\begin{itemize}
\item urban development = urban system transitions (dynamics of urban processes);
\item urban decision making = engagement of urban agency within urban networks;
\item urbanity = engagement of urban agency with socio-spatial patterns
\end{itemize}

\item \textbf{Indicators constructed as dependent variables:}
\begin{itemize}
\item urban system transitions - the dynamics of urban processes of maintenance, transformation and change relies on the level of urbanity and the articulation of the morphology of urban decision making;
\item the level of urbanity - is indicated through the socio-spatial patterns of urban system transitions;
\item the morphology of urban decision-making - the engagement from urban agency aggregated within the layers of urban decision making;
\end{itemize}

\item \textbf{Indicators reduced to independent variables:}
\begin{itemize}
\item urban agency consist of human and non-human actors;
\item urban networks are the assemblages of urban relations identified through the layers of urban decision making (top-down, real-estate, bottom up);
\item socio-spatial patterns are local contextual elements identified as urban practices, urban conflicts, contextual resources;
\end{itemize}
\end{enumerate}

Bearing in mind this re-categorization and structuralization of the concept of urban development, the analytical tool in this research is the MAS-ANT methodological hybrid (\href{Section 2.2}{Section 2.2}). 
It provides the road map for an inclusive and flexible approach to exploratory research - describing, tracing and representing the dynamics of urban processes. The Actor-network theory illustrates the urban agency and decision-making concepts while the Multi-agent system operationalizes the concept of urbanity at a qualitative level and brings about the logic of the whole MAS-ANT procedure. Such statements shed new light on the overall research questions and have turned this thesis into a methodological exploration.
\\

The research output depends on the success of the cross-pollination of concepts through the MAS-ANT mixed research method. They are intended to influence both the theoretical and practical domain. The research is guided in this way in that it:
%research objectives - contribution
\begin{itemize}
\item questions the concepts of urban development, urbanity in general and urban decision making in post-socialist cities;
\item •	proposes the terminology of urban system transitions, which connects the processes of maintenance, transformation and change to urban conflicts, social practices and contextual resources at the local level;
\item invents visual interpretations for practical use.
\end{itemize}

The research is thus built on three hypotheses. Each hypothesis addresses both theoretical and methodological issues. They are drawn in a consecutive order. The justification of the hypotheses is gradually built by means of describing, exploring and proceduralizing in order to master the complexity and dynamics of urban development processes. 
%visualization of the framework

\subsection{Context-specific Research Questions}

\textbf{Overall research question:} How to investigate the socio-spatial patterns of post-socialist cities in order to reinvent a more inclusive and flexible approach to understanding urban development processes engaging the complexity of an urban context? 

\subsubsection{[RQ1]}
What constitutes an inclusive approach to urban development?
\begin{itemize}
\item figuration of human and non-human elements as urban key agents;
\item manifestation of urban key agents in urban networks;
\item pathways of urban transitions within the morphology of urban decision-making;
\end {itemize}

\subsubsection{[RQ2]}
Why does the level of urbanity determine pathways for urban dynamics? 
\begin{itemize}
\item What are local contextual conditions for specifying the level of urbanity in an ordinary city?
\item How does the level of urbanity embody the dynamics of urban transitions in post-socialist cities?
\end {itemize}

\subsubsection{[RQ3]}
How to frame the urban development process to embody the complexity of urban systems and the dynamics of urban transitions?
\begin{itemize}
\item How to connect urban complexity and dynamics by tracing the level of urbanity within the morphology of urban decision-making?
\item How to design the framework for action in order to operationalize the urban development concept?
\end {itemize}

\subsection{Research hypotheses}
\textbf{Central hypothesis [RH]:}
MAS-ANT methodological approach captures urban development processes in terms of urban system transitions by giving an exhaustive image of urban complexity and providing transparency and flexibility in describing urban dynamics in a post-socialist city.

\subsubsection{[RH1]: Urban complexity}

The Actor-network theory (ANT) clarifies urban complexity through the networks of urban key agents initialized by the morphology of urban decision-making in a post-socialist city.

\subsubsection{[RH2]: Urban dynamics}

Multi-agent system (MAS) expresses urban development dynamics by tracing the level of urbanity within the relations of urban agency and contextual elements in a post-socialist city.

\subsubsection{[RH3]: Urban system transitions}

Urban development processes set as an iterative procedure to trace the level of urbanity and the morphology of urban decision making within the urban agency map capture urban dynamics and re-frame urban complexity at the neighbourhood level by using a methodological hybrid that combines the Multi-agent system (MAS) and the Actor-network theory (ANT).

\section{Research Design}

The aim of this section is to present the reasoning behind the research and the adopted methodology, namely the logical sequence that connects the empirical data to the research questions, hypotheses and conclusions. In designing the research process, the defined goals are assumed to be exploratory in nature and to address methodological investigations and testing in urban studies. The study builds gradually from the specific observations of the literature towards an in-depth analysis. An exploratory standpoint is chosen with regard to the theoretical and practical goals of the research. This division is crucial for establishing the research methodology. The first, theoretical part relies on secondary data and is based on theoretical constructs, while the second, practical one provides primary data and empirical evidence from the field study.
\\

The theoretical summary of urban development processes and the critical overview of the corresponding urban theory concepts (urban development, urbanity, urban decision making) is carried within the literature review  (\href{Chapter 2}{Chapter 2}). It functions as the structural catalyst for the chosen methodologies, as a general cross-pollination of concepts within the MAS-ANT methodological scope. The MAS-ANT methodological approach is practically tested through the case study method. The application of this methodological hybrid in a hierarchical order (first ANT than MAS) to analyse the selected case study enables a practice-oriented understanding of the situation in post-socialist neighbourhoods. The final data display blends both MAS and ANT methodologies and underlines how the field data are re-classified and re-interpreted. This synthesis aims to turn tacit knowledge on socio-spatial patterns in Belgrade into explicit knowledge about urban development processes in post-socialist neighbourhoods.
\\

The so-called cross-pollination procedure justifies the proposed indicators (operational definitions of the concepts used) and enables connections between the independent and dependent variables constructed within the research hypotheses. This is the core logical construction of the research enquiry. The point of departure was the case study. The research further follows an inductive method of reasoning within the case study. Interpretative and participatory action research methods are used for the data collection. These qualitative methods overlap with the case study to validate the proposed theoretical categories (indicators and variables). The principal data sources included documentaries, open-ended interviews, workshops, and questionnaires, which contributed to the structuralized description of the post-socialist empirical analysis performed with the Actor-network theory (ANT). The Multi-agent system (MAS) made further use of qualitative evidence to elaborate urban networks and reveal the involvement of urban key agents in urban affairs. Finally, the MAS-ANT diagram displays the research results and facilitates interpretations of the maintenance, transformation and change processes in an urban environment.
\\

The main study focus is to invent a looping procedure which examines the relations among a variety of urban elements, explores the ”specificities and globalities” of the particular context, and catalyses the framework of action at the neighbourhood level.  The scope of this research is an incremental, open-ended procedure-building based on a pragmatic approach through iterative and collaborative techniques towards:

\begin{enumerate}
\item understanding the phenomenon,
\item creating an overall framework,
\item identifying the pattern of dynamic urban reality in terms of urban system transitions. 
\end{enumerate} 

The entry point for this methodological exploration is a case study.

\subsection{Case study}

This research adopted an in-depth case study inquiry as the adequate method for collecting and framing empirical data. The case study serves as a data collection engine, a catalyser and a boundary framework. The method’s exploratory and descriptive nature are of particular importance. In general, the former captures the process, while the latter prepares and illustrates the incidence/prevalence of the phenomena (\href{Yin} {\citealt{yin_applications_2011}}). These features provide us with a comprehensive framework to describe contemporary phenomena with extensive data-types and sources of data (\href{Feagin}{\citealt{feagin_case_1991}}). The goals focus on a holistic description of urban development and gaining an understanding of the processes at work over time (\href{Swanborn}{\citealt{swanborn_case_2010}}). In this manner, the case study takes an embedded approach with multiple units of analysis (\href{Scholz}{\citealt{scholz_embedded_2002}}; \href{Yin} {\citealt{yin_case_2009}}): urban key agents, the morphology of urban decision making, urbanity and urban system transitions. These units of analysis define the scope of the investigation - which elements are to be studied in detail and which processes are to be excluded (\href{Harrison}{\citealt{partington_case_2002}}).
\\

Therefore, the case study observes the complexity of urban development processes and recounts their dynamics by adding the dimension of time to the analysis (\href{Feagin}{ \citealt{feagin_case_1991}}).
The case study research process is broadly divided into three parts: designing, conducting and reporting. 
\\

However, a set of well-known components of \textbf{designing case study} have triggered its application within this research, such as (\href{Yin} {\citealt{yin_case_2009}}):

\begin{enumerate}
\item a focus on HOW and WHY questions about the researched phenomena;
\item the units of analysis, the information relevant for the case construction, depend on the definition of research questions;
\item the exploratory nature of research hypotheses, as each proposition is built on something relevant within the scope of the study or for one or more units of analysis;
\item the linking of findings to the hypotheses, units of analysis to theoretical background, i.e. "pattern matching" (\href{Campbell}{\citealt{campbell_iii.degrees_1975}});
\item the data collection focus for the case study; testing methodologies and existing theories therefore provide a rich theoretical framework.
\end{enumerate}

The case study is commonly but not exclusively applied in anthropology and sociology.
In general, it is used to ground observations and concepts about social phenomena in their natural setting. Consequently, it has been increasingly applied in other disciplines including urban studies and architecture (\href{Feagin}{\citealt{feagin_case_1991}}). Even though major criticism is levied towards a single case research focus and doubts are voiced about the scientific generalizations based on an individual case, this research builds on (\href{Flyvbjerg}{\citealt{flyvbjerg_five_2006}}) that a careful and strategic choice of cases, and thereafter the units of analysis, contributes to the collective process of knowledge accumulation.
Advocating the scientific relevance of the case study, (\href{Flvybjerg}{\citealt{flyvbjerg_five_2006}}) distinguishes several selection strategies: random, extreme, multiple, critical and paradigmatic cases. A selection that is based on expectations about information content is the most proper strategy for the scope of testing methodologies. For example, extreme case circumstance enable close examination of the embedded units of analysis.
\\

\href{Flvybjerg}{\cite{flyvbjerg_five_2006}} also states that the descriptive manner chosen herein puts forward the path for scientific innovation, which in this thesis is the hybridization of methods for urban data analysis. Henceforth, the most important herein is this opportunity for application of multiple methods (\href{Yin}{\citealt{yin_case_2003}}; \href{Yin} {\citealt{yin_case_2009}}) and consequently methodological hybrids, MAS-ANT.
Data obtained from the case study aim to contribute to objectives of the research by providing the local layer with real-life data. Accordingly, the case study enables testing this methodological approach through the systematization and validation of the case study data analysed by the methods involved, the Actor-network theory and the Multi-agent system. In these circumstances, the case study is referred to as a sort of data sampling strategy, used to select, manipulate and prepare a representative subset of data points for analyses by the chosen methods. It delivers patterns, trends and structures in the larger data-set afterwards.
\\

Then, in \textbf{conducting case studies}, the most important is to ensure not only the variety but also the convergence of data. It is essential to have sampled  sufficient  points  of  view  to be able to develop a balanced picture (\href{Harrison}{\citealt{partington_case_2002}}), but also to provide converging lines of inquiry within the multiple sources of evidence (\href{Yin} {\citealt{yin_case_2009}}).
A case study usually involves a variety of data sources, both human (interviews, workshops) and non-human (documentation, archival records, direct observations and physical artifacts).
With this abundance of data, the phenomena and the processes become underpinned by multiple data sources and ensure the construction of validity through triangulation (\href{Denzin}{\citealt{denzin_research_2009}}). In this research, triangulation is applied at two levels (\href{Patton}{\citealt{patton_qualitative_2001}}):

\begin{itemize}
\item data triangulation,
\item methodological triangulation. 
\end{itemize}

Therefore, one of the main reasons for choosing the case study method are: (1) a structured and bounded data plan and (2) the incorporated units of analysis, (3) the cross- referencing of methodological procedures and (4) the resulting evidence triangulation with the mixed method (\href{Patton}{\citealt{patton_how_1987}}; \href{Yin} {\citealt{yin_case_2009}}).
\\

For this thesis, the case study is part of a larger multi-method study and \textbf{reporting} is reduced to the general structuring tactic for the descriptive data about the selected case.
However, documenting relationality between the research problem and the case and constructing validity is elaborated within the reasoning for case study selection.
This entire research design is linear-analytic in its structure: it starts with the issue of the problem and the literature review, then presents the logic of its research design and chosen methods, the findings from data collection and the analysis, conclusions and implications.
\\

The data collection process through the case study method retains the same linear-analytic manner in its descriptions and implications in a broader scope and takes a chronological course according to the causal sequences within the case history. In order to maintain the chain of evidence, the case study presentation must (\href{Yin} {\citealt{yin_case_2009}}):

\begin{enumerate}
\item explicate and justify the boundaries of the case;
\item design the research according to the known constraints;
\item indicate an exhaustive data collection process;
\item consider alternative perspectives and different points of view;
\item display sufficient evidence.
\end{enumerate}

The case study should illustrate, in great depth and clarity, the embedded units of analysis, which are being researched through the MAS-ANT methodological hybrid. Such research design encircles hypotheses testing by logical argumentation for building the methodological framework and the simulation of how the framework is applied to the case study. The choice of case study method for data collection is justified by its feasibility for structuring the chain of evidence and the confirmability of data by triangulation. On the other hand, the reliability of the case study method is determined by its ability to document the methodological procedure with data and its external validity is proved by the transferability of the procedure in other contexts and cases.
\\

Limiting the case study method to the data collection reduces the risks of its common deficiencies. The unreliability of soft data is dealt with through the ANT approach of flattened reality, while a researcher’s subjectivity in interpretations and selections should be prevented with the methodological rigidity in the mere classification and interconnection of data. Finally, in multi- method research there is no need for explanations and internal validation of the case study logic. Moreover, generalizations are reduced to the analytical ones on the methodological level and in terms of categories and networks. In this research, the systematization of the collected data is used for further analyses and the case study is rather a narrative of urban development as a contemporary social process within its real-life context. Therefore, the selected paradigmatic case should be a valid representation of a setting suitable for extensive application of the proposed methodological hybrid within data analysis and data display procedures.
\\

\textbf{"The case study can enable a researcher to examine the ebb and the flow of social life over time and to display the patterns of everyday life as they change." (\href{Feagin}{\citealt{feagin_case_1991}}})

\subsection{Case study selection}

As previously stated, this research is established on the basis of mixed method and its adequacy for research on complex and dynamic urban phenomena. Verifying such a methodological hybrid in practice means that the proposed categories and mechanisms within the methodological procedure should be tested by further research activities. These activities include: (1) observing a real-life context, (2) putting forward the coined phenomena and relations and (3) postulating the correspondence of proposed methodological structures to the reality, which, if they exist, answer the research question. In a nutshell, this methodological framework acts as an a priori logic to explore particular instances, but it still must account for their various deviations and aim at a few conclusions that contribute to the general, scientific body of knowledge on the urban. The case study point of view is herein deduced from general statements and signified as a derived, localized, contextualized form of researched phenomena, in this case urban development processes. It gives an overview of relations, factors and influences in a holistic manner, providing understanding of phenomena (units of analysis) within its operating context  (\href{Harrison}{\citealt{partington_case_2002}}).
\\

The case study is used in the first stage of the research process, whereas other methods (Actor-network theory and Multi-agent system) are suited for hypotheses testing and conclusion drawing. However, a strategic case study selection is crucial for this research in order to maximize the utility of information from a single case and a small sample of the units of analysis  (\href{Flyvbjerg}{\citealt{flyvbjerg_five_2006}}).
Therefore, the case is selected on the basis of expectations about the correspondence of its data content to the proposed methodological hypotheses. The elaboration of paradigmatic case study for this research is based on (\href{Yin}{\citealt{yin_case_2009}}) criteria:

\begin{enumerate}
\item an exhaustive data collection process with sufficient interpretative and artefactual evidence,
\item the multiplicity and variety of data sources, especially human;

but with - 
\item explicit case boundaries,
\item precise data constraints.
\end{enumerate}

\textbf{(1)/(2) }The case study database is built upon the investigator’s report (narrative, notes, tabular material, diagrams etc.) and the quality  of reporting depended on an extensive evidentiary base. In order to provide exhaustive evidence, the case study choice relied on heterogeneous data sources (\href{Yin}{\citealt{yin_case_2003}}):

\begin{itemize}
\item exact documentation, archival and qualitative data and records documentation (service records, maps, charts, lists, survey data, personal records) (\href{Bibliography}{Bibliography - Legal and Planning Documents; Grey Literature; Media Sources});
\item physical artifacts (tools, instruments, works of art) - insights into cultural features and technical operations;
\item insightful, targeted interviews dependent  on the construction of research questions (\href{Appendix}{Appendix A});
\item participant-observation - workshops -  as direct observations, insights into motives (\href{Appendix}{Appendix C});
\item direct observation - visiting the site in order to cover changes in Savamala over time
- following events in real time and in its context, the choice of events was susceptible to the researcher’s selectivity and reflexivity.
\end{itemize}

Bearing these in mind, favourable circumstances for a comprehensive data collection is knowledge of the local language, previous general knowledge of the context, professional connections and extensive site visits. Even though foreknowledge can impact the neutrality of the researcher, within this research project, a quick and systematic understanding of the local urban development processes facilitated data collection and improved flat classification, a well-known feature of Actor-network theory. 
\\

\textbf{(3)} In his argumentation on case study method in management research, (\href{Harrison}{\citealt{partington_case_2002}}) argued for its maximal benefits in the circumstances "where the theory base is weak and the environment under study is messy".
This also contributed to rejecting the general misconception stating that theoretical (context-independent) knowledge is more valuable than concrete, practical knowledge. Moreover, following (\href{Flyvbjerg}{\citealt{flyvbjerg_five_2006}}), the case study can be extremely useful for transferring tacit, context dependent knowledge, into explicit, general knowledge. When explicating the domain of practical knowledge, any historical background to the research problem, its time-space transitions and the immediate political, economic and cultural circumstances where it emerges and evolves should be taken into consideration as a chronological sequence. Fine tuning of these various factors and processes presented on site and - if properly described - provide an adequate capacity to explain correlational links among the identified urban key agents of urban development.
\\

\textbf{(4) }Finally, by placing high priority to provide an abundance and precise calibration of data, it becomes less likely that an overall scrutiny of numerous relations, behaviours and processes could be possible in a wider context with an extensive complexity of datasets. Regions, metropolitan areas and cities can be difficult to handle through the embedded units of analysis. Therefore, the neighbourhood level of analysis is already fixed by the hybrid method.
\\

Consequently, the case study choice was retained on the neighbourhood level of analysis, as the choice that gathers up all recognized indicators of the urban development process. On the other hand, the best option is that the researcher is to some extent familiar with the local context and is capable of accessing particular data. My native country of Serbia with its turbulent burgeois and socialist past and the transition of today is taken as an exemplary case for the intensive congregation of factors, trends and power struggles in a single place. The adopted case study field is the Savamala neighbourhood in Belgrade, a historical but deteriorating city quarter in Belgrade, where a set of bottom-up urban transformations and participatory spatial interventions are colliding with the top-down imposition of master planing and swift, investor-based developments in the area. This multitude of influences with different sources and an extensive yet limited time-span provide an opportunity for a holistic study of complex social networks and processes.

\textbf{"Case study research is flexible and can be adapted to many areas of knowledge creation. And the researcher is continuously confronted with the question ‘does this make sense?’" (\href{Harrison}{\citealt{partington_case_2002}})}

\subsection{The Local Context of the Savamala Neighbourhood Case Study}

The choice of case study method for data collection is most suitable when the  contextual  conditions  are  believed  to  be highly relevant for the phenomenon being explored (\href{Robson}{\citealt{robson_real_1993}}; \href{Yin}{\citealt{yin_case_2003}}). The hypotheses of this research were examined within the real-life context of the Savamala neighbourhood in Belgrade as an exploratory basis for building the methodological framework of analysis for urban development processes. 
\\

The selected case study of the Savamala neighbourhood should feed the MAS-ANT analytical framework with site-specific data on (\href{Harrison}{\citealt{partington_case_2002}}):
\begin{enumerate}
\item the context - global and local, outer and inner in reference to time and space;
\item the content - urban key agents and urban decision-making layers that put forward urban development processes;
\item income and outcome variables - link urban system transitions to socio-spatial patterns and urban networks.
\end{enumerate}

The boundaries of this research are spatial, though the Savamala neighbourhood is not an administrative unit nor does it have its own local authorities. It is rather more a place on the mental map of Belgrade and an important landmark of the city, than an official unit. Consequently, the exact spatial boundaries are drawn according to the survey conducted among professionals and citizens (\href{Appendix}{Appendix B}).
\\

This neighbourhood is a scaled example of a pre-socialist material legacy, a socialist cultural and societal matrix, a transitional reality and a condensed case of the multi-faceted circumstances of post-socialist urban development (\href{Table space-time}{Chart X}). These elements also frame the epistemological constraints for the case study research in this case.
\\

The units of analysis comprise a knowledge-based chain of decision-making and a dynamic, interactive process of interdependences and interconnections among all active urban key agents and socio-spatial patterns identified in Savamala exclusively through a qualitative inquiry.
\\

In summary, urban agency and urban networks are not spatially bounded phenomena, they develop as products of interaction between human and non-human elements in particular localities. In this respect, they contribute to an understanding of the broader urban systems and enhance theoretical frameworks (\href{Giddens}{\citealt{giddens_constitution_1984}}; \href{Grubovic}{\citealt{grubovic_belgrade_2006}}).

\section{Adopted Methodology}

This thesis adopted an in-depth case study research design with the hybrid methodology approach. The reasoning behind the selection of methods is elaborated in relation to the objectives, questions and hypotheses of the research (\href{ref}{Figure 1}).  
\\

For this purpose, this research is performed in two stages: the methodological and the case study level. It relied on qualitative data, collected from an extensive literature review, expert interviews and participatory workshops. The application of these data sources depends on the stage of the research process: case study data collection, ANT and MAS data analysis and MAS-ANT data display. The case study is limited to the collection of qualitative data through a range of techniques:  (a) extensive review of written sources, (b) interviews,
(c) workshops, and (d) questionnaires. Conversely, data analysis presents a combination of: (1) theoretical concepts as the indicators of urban system transitions; (2) theoretical stances for exhaustive description from ANT; (3) the operational categories used in MAS. Finally, data display gives an overview of the urban complexity and dynamics through MAS-ANT methodological cross-pollination.
\\

The initial stages of this research started with qualitative inquiry. Data are collected in human and non-human chunks of analysis. Content analysis of data sources (urban planning policies and agendas, urban planning documentation, archival and media sources etc.) provided an insight into the local context of urban planning institutions and land devel- opment practices. Interpretative research by direct observations through semi-structured interviews and participatory-action research directed data analysis and further stratified the categories within the mixed-method approach. Online surveys validate the findings based on the working assumption that the soft data give a valuable insight into the complexity and dynamics of urban development circumstances in the local context. These research stages form the basis for ”MAS on ANT” analysis of the obtained data and the final systematization of urban system transitions in reference to urban development.

\subsection{The Savamala Case study - Data collection} \label{sec:predis}

The case study research design is adopted as the most comprehensive one, to a certain extent as a research strategy, for an overview of possible categorizations and linkages in terms of complexity and the dynamics of urban development processes  (\href{Meredith}{\citealt{meredith_theory_1993}}; \href{Harrison}{\citealt{partington_case_2002}}). In short, case study research design seeks patterns of the available and myriad data in the bounded space-time of the selected case (\href{Denzin}{\citealt{denzin_research_2009}}). A systematic approach for this empirical research is founded on the verification of hybrid methodology. The data collection process has been recurrent, iterative consultations and interpretations of qualitative data. 
%{(Figure XXX which is based on interpretative research explains how it is deviated in my case figure 9.3 and figure 9.4. for the systematic approach for empirical research Harrison 2002 Flynn 1990)}.
Thus, participant observations, interviews, and surveys are all eligible methods which can be deployed in these circumstances.
\\
The interpretative itinerary directs how the soft data are collected and built into the artificial reconstruction of the developmental reality through the MAS-ANT methodological approach (\href{Figure xxx}{Figure X}). The implementation of the case study (case study protocol) is executed in circles:

\begin{enumerate}
\item preliminary identification of the morphology of urban decision making, urban key agents and urban networks at the local level from the scientific literature, official documentation and records, and media coverage;
\item recognising urban key agent and urban networks and further structuralisation through human perception of the objective reality by the participatory action research method and semi-structured interviews;
\item verifying urban key agents, socio-spatial patterns and urban transitions through on-line surveys for different professional cliques;
\item triangulating key observations and data sources;
\item examining alternative interpretations and assertions of generalisation for all elements and networks through interviews with key-informants (members of different interest, knowledge and action groups).
\end{enumerate}

The initial chain of evidence is presented in the longitudinal distribution of case study data, with time-frame (chronological) and linear-analytic (causal) references. Therefore, the case study report is structured according to the articulation of three layers of the morphology of urban decision making in Savamala: top-down urban planning, real-estate transformations and participatory bottom-up activities. The morphology of urban decision making is the bounding factor for all phenomena, themes, and issues built into the case study.  This manner of systematisation for the collected data corresponds to all case study protocol topics (which are also the research indicators/variables). Moreover, it also re-ordered the protocol procedure differently so that it formed the basis for methodological analysis with the Actor-network theory and Multi-agent system. Therefore, this case study account sought to ”catch” urban development processes in Savamala by building MAS-ANT patterns into the empirical data within the case study narrative. Thereafter, the outcome variables were clearly identified and interpreted with MAS-ANT data display. 

\subsubsection{Qualitative inquiry}

\textbf{"Science is not achieved  by distancing oneself from  the world; as  generations of scientists  know,  the greatest conceptual and methodological challenges come from  engagement with the world" (\href{Whyte}{\citealt{whyte_participatory_1991}})}
\\

Qualitative inquiry is applied as a research instrument which enables scientific processing of soft data - meanings, experiences and descriptions (\href{Yin}{\citealt{yin_case_2003}}).
In this thesis, it serves to incorporate the socially constructed knowledge of urban phenomena into MAS-ANT modelling (\href{Mertens}{\citealt{mertens_introduction_1998}}; \href{Flick}{\citealt{flick_introduction_2009}}; \href{Grubovic}{\citealt{grubovic_belgrade_2006}}).
In order to ensure replicability of the hybrid method, it is crucial to collect data in a coherent way and condense the complex spectrum of issues into a logical unity familiar not only to the researcher. A high research priority herein is to cover the wide gamut of humanly moulded data. The proposed reporting scope of the morphology of urban decision making puts forward the data structure through: the top- down management of urban issues, the legitimacy of real-estate interests and the dynamism of bottom-up urban agency. To do so, a qualitative inquiry followed the case study protocol proposed in the previous section ((1) documentary analysis, (2) preliminary interviews, (3) workshops, (4) surveys, (5) in-depth interviews); while the range of data sources within these separate inquiries should coincide with the concept of supporters, opponents and doubters for any recognized data point of importance (\href{Pettigrew}{\citealt{pettigrew_character_1992}}; \href{Harrison}{\citealt{partington_case_2002}}). 
\\

A common criticism revolves around the internal and external validity of qualitative data (\href{Flyvbjerg}{\citealt{flyvbjerg_five_2006}}). Within this methodological research, the validity issue is not particularly at stake as the perceptions and interpretations of urban actors (human factor), whatever they may be, influence urban systems and networks in their raw format, just as they do in interviews/discussions/surveys. The major threat has been either researcher bias (in directing the interactive data collection processes towards confirming the researcher’s preconceived notion) or the reflexivity of an interviewee’s interests rather than the statement of their perceptions or opinions. However, the triangulation of qualitative data as well as the conducted iterative case study should have reduced these negative effects.
\\

The iterative nature of the case study protocol also helped in the continual evaluation and update of data sources and circular data collection for MAS-ANT data analysis. The evidence and circumstances under which the data were collected are summarized within the following data collection procedures: 

\paragraph{(1) Documentary analysis} 
[\href{Bibliography}{Bibliography - Legal and Planning Documents; Grey Literature; Media Sources}]
\\

Documentary analysis is a rather discursive research technique for identifying and interpreting documentary evidence in order to support and validate facts and incorporate them within scientific research. In this thesis, documentary data are used directly for data collection within the initial research phase. By addressing the first research question -

[\textit{RQ1: "What constitutes an inclusive approach (complexity and dynamics) to urban development?"}]-

the documentary research method provided insight into the first round of independent variables.
Not only that interviews and surveys may not be appropriate and useful in this phase (\href{Mogalakwe}{\citealt{mogalakwe_use_2006}}), but the early documentary analysis also improved the preconditions for interviews by introducing basic issues and concepts, indicating who the potential interviewees might be and setting the path for open- response dialogues (\href{Robson}{\citealt{robson_real_1993}}; \href{Grubovic}{\citealt{grubovic_belgrade_2006}}).
Documents are naturally occurring objects, independent and beyond particular scientific production within a research project. Through their concrete existence beyond their production and its context, they indirectly narrate the circumstance of the social world as well as the actors and circumstances of their production at a specific time and space (\href{Jary}{\citealt{jary_harpercollins_1991}}; \href{Payne}{\citealt{payne_key_2004}}; \href{Mogalakwe}{\citealt{mogalakwe_use_2006}}).
\\

It must also be recognised that documentary narratives may be inaccurate, fragmented and subjective (\href{Foster}{\citealt{foster_power_1994}}). 
Therefore, an early data validation was performed following (\href{Scott}{\citealt{scott_matter_1991}}) criteria for assessing documentary sources and data: authenticity (genuine, original and reliable material), credibility (fateful explanations and accuracy), representativeness (reliability for the research), meaning (whether the documents are clear and comprehensible). In this respect, reinforcing the robustness and rigour of this research in the first place is enabled by a preliminary investigation of documentary data sources, which is followed by examination, categorization and the identification of their limitations accordingly (\href{Scott}{ibid.}) (\href{Table Data sources}{Table XX}). Moreover, the documentary data are continually adapted and validated during the data analysis through the process of triangulation of data obtained from other qualitative research techniques (interviews, workshops and surveys) 
 (\href{Yin}{\citealt{yin_case_2003}}).
\\

Generally known documentary sources encompass: public records, the media, newspapers, official gazettes, minutes of meetings, reports and blueprints, visual documentation etc; and they are broadly categorized according to their proprietary rights in: personal, public, and private sources (\href{Payne}{\citealt{payne_key_2004}}). According to the ANT approach, the focus has been put on public and publicly available private sources to provide adequate context for data analysis. In the course of the case study, a variety of formal and informal documents and their figuration and impact on the current developmental reality in Savamala (2010-2016) were examined, including:

\begin{itemize}
\item post-socialist urban and scientific literature;
\item legal documents: laws, by-laws, strategies, official gazettes etc.;
\item technical documents: spatial and urban plans and projects;
\item internal contracts, reports, projects, meeting minutes etc.;
\item media coverage. 
\end{itemize}

As a part of the preparation for interviews, documents were reviewed in order to provide a provisional overview of independent research variables, further developed through ANT and MAS data analysis and MAS-ANT data display. Documentary evidence was also used later to supplement detail and to expand upon and support or challenge points raised during interviews. They were also utilized to generate additional questions or themes for investigation. In this thesis, documentary evidence has been the data core to establish the iterative and continuous nature of the research process. 

\paragraph{(2) Preliminary interviews} 
[\href{ref}{Appendix A}]
\\
The ethnographic nature of Actor-network theory requires the usage of soft data and the application of qualitative data collection techniques that are as diverse as possible. Consequently, the preliminary interviews were held in a rather unstructured manner, characterized by an open-response style. Open-ended interviews were focused on a certain topic, but were usually directed in a rather non-predefined way.  Even though the questions may have been scripted, the interviewer usually could not predict what the contents of the response would be. Open-ended interviews might be: informal, semi-restrictive or structured.
\\

The preliminary interviews set out the framework for constructing dependent variables of this research in the context of the second research question -

[\textit{RQ2: "Why do the level of urbanity traces determine pathways for urban development dynamics (urban transitions)?"}] -

and the third research question -

[\textit{RQ3: "How to frame the  urban development process to embody the complexity of urban systems and the dynamics of urban transitions?"}].

?”). In this case, the interviews were targeted local experts in urban research and practice, the staff of the local and city authorities, as well as activists operating on the ground in Savamala. Several interviews are conducted in iterations. At times, interviews also took a snowballing course, when the informants were identified by the interviewees from the previous round. The interviews were carried out in circles and the level of dialogue restriction varied according to the occasion and interview iteration. The focus was usually on the interviewee’s thoughts, experiences, knowledge, skills, preferences, and ideas. All the interviews are undertaken face-to-face.
\\

These interviews were carried out during an extensive period of time and overlapped with other data collection processes, even during the first stages of data analysis. The timing was also influenced by the on-going processes in Savamala, as well as by the formal and informal channels for establishing  contacts with these interviewees. In several cases, the interviews were directed in an informal and least restrictive way, in the form of a dialog, without actual preparation, and by  asking questions spontaneously in the course of the wider context of the research topic. Subsequently, no two informal interviews are alike. The interviews with experts and activists were conducted in a semi-restrictive manner. These interviews follow a general outline of issues of interest, but several questions were generated spontaneously or were transferred to other topics when interviewees’ answers prompted a need for more connected information (\href{ref}{\citealt{payne_key_2004}}). The research questions were grouped into X major themes (\href{Appendix}{Appendix A}). Only a few interviews were recorded, but were approved by the interviewees. These interviews were used for the further identification of major concepts within the research variables, while the clarity and understanding of the interview content was sufficient. 

\paragraph{(3) Workshops} 
[\href{ref}{Appendix C}]
\\
Topic-oriented workshops are qualitative research techniques moulded for this research in combination with participatory action research (PAR) and simple participant observations. They were applied for the re-problematization of research questions in light of the critical reflection and dialogue between and among participating actors (\href{Mc}{\citealt{mcintyre_participatory_2008}}). The aim of this technique is to transfer tacit knowledge concerning the reality of urban development in the post-socialist context of Belgrade into explicit indicators in order to overcome the single- discipline limitation. It eventually led to rethinking the definitions and conceptualizations of the independent variable, dependent variable and their interconnections within this research project (\href{Whyte}{\citealt{whyte_participatory_1991}}). 
\\

The context-specificity of participatory action research provides multiple opportunities for practitioners and scientists to construct knowledge and integrate theory  and  practice within a "theory of possibility" (\href{Mc}{\citealt{whyte_participatory_1991}}), in this case on the prospects for urban development in the Serbian capital. Combined with the participant observation approach, these workshops were explored and valued to understand how the experience of the selected participants influenced their collective realities. In this context, the gathered information and constructed explanations from participants were used to test the systematization of the MAS-ANT approach in terms of the morphology of urban decision making, urbanity and urban system transitions. In order to construct adequate circumstances for such workshops, it was essential to believe that the participants were capable of creating and influencing the urban condition (\href{Mc}{ibid.}), while the actual data were evaluated afterwards in terms of their confidentiality, trustworthiness, and credibility.
\\

Within the scope of this research, three workshops were organized: expert, PhD and student workshops. Two of them (the expert and student workshops) were on a precise, research oriented topic and one (the PhD student workshop) was conducted with a broader context in mind.
\href{Appendix C}{Appendix C} gives an overview of the organized workshops and their role in building a relationship between theory and practice according to the criteria mentioned above. All the workshops were moderated by local experts, while the participants of the student and PhD workshops were engaged in using urban research methods for constructing knowledge on the topic. Even though the active role of participants was crucial for the success of the workshop, the uncertainty of whether they would choose to participate was also very important (\href{Mc}{ibid.}).
In terms of participants’ control of the conduct of the workshop, they chose to react, interact or stay passive. Workshop data also tested the credibility of data collected with two previous qualitative research techniques (documentary analysis and preliminary interviews).

\paragraph{(4) On-line Surveys}
[\href{ref}{Appendix B}]
\\
An online survey is a type of questionnaire. Questionnaires are better suited to collecting factual information (\href{Payne}{\citealt{payne_key_2004}}). However, they can be used for provisory verification of subtle and complex data sets, if the questions are formulated in a simple, non-technical and unambiguous way, and if they use  language easily understood by all participants.
\\

The online surveys were conducted simultaneously with the workshops, but over an extensive period of time (before and after the workshop events). The structure of the online surveys varied slightly and corresponded to the workshops. They were adapted to each of the workshop audiences: the experts, PhD students and young professionals, and future professionals (students). As the workshops followed the initial documentary analyses and interviews, the surveys were intended to encircle most of the previous results and test their manifestations in MAS-ANT methodological scope. The survey results were separately evaluated and partially quantified, if possible, on several topics. Anything ambiguous, biased or likely to arouse anxiety was avoided by substituting direct questions with indirect ones  (\href{Payne}{ibid.}).
\\

The surveys were prepared through the Survey Monkey online service. The survey content was built up as a combination of open-ended and closed questions, which were categorized according to the prevailing topic. They were structuralized in a linear, complexity-growing manner - from general to more specific questions. The order of questions had an important influence on the answers. At several points, leading questions, where certain answers had been expected, were used. At other points, filter questions were used, where respondents were only required to answer certain questions if they had answered a previous question in a particular way. The online, internet polling nature of the surveys was especially beneficial for the self-completion parts of the questionnaires, where the question sequence depended on respondent answer. 
\\

Yet, an important constraint was the requirement of brevity, especially for the surveys targeting experts. Namely, respondents’ attention spans are usually short. However, the wording and clear instructions were recognized as vital for the success of the surveys, as well as the possibility for respondents to easily navigate through questions and their meanings and to answer them without special effort and in a clear and simple way. To a certain extent, these constraints reduced the impact of these online surveys on data analysis.

\paragraph{(5) In-depth Interviews}
[\href{ref}{Appendix A}]
\\
In-depth interviews pertain to the final stage of the data collection process. They were carried out on specific topics within the research framework in order to provide an in-depth account and validation of research hypotheses. All the interviews were face-to-face encounters, but not all of them are recorded. Recorded or note-based conduct was left as an option for interviewees to decide. The few recorded interviews were summarized and cross-referenced with the notes taken during the non-recorded ones to ensure clarity of interview content and verification of research data.
\\

The collection of data herein was predominantly carried out through semi-structured interviews. However, the sequence of addressed topics was more rigid and corresponded to that of the research framework and the structuralization of research questions and hypotheses [RH1, RH2, RH3].
The question grid was restrictive, and the exact same questions on specific topics were prepared for each interview, even though slight variations occurred in response to the rather open-response style of the dialogues. The careful wording of questions prepared in advance contributed to the avoidance of ambiguity or specifically undesired connotations. Nevertheless, the semi-structured nature of interviews enable the researcher to develop a positive rapport with the interviewee and vice versa. Even though the semi-structured nature of interviews is more adequate when the range of interviewee accounts and overviews about the research topic are not well known in advance (\href{King}{\citealt{cassell_qualitative_1994}}), in this case it was useful to obtain the factual information on the matter and to facilitate the data collection with  interlocutors from different professional and disciplinary backgrounds. 
\\

The choice of interviewees was such that their responses covered the systematic categories of informants identified with previous collection techniques and corresponded to the proposed  MAS-ANT categorization. \href{Appendix A}{Appendix A} presents the interview grid and qualitative evaluation of the capacity of the data collected herein to be used for MAS-ANT data analysis in terms of their reliability and credibility for this research project.

\subsection{ANT and MAS approaches - Data analysis}

The hybrid field of overlapping MAS and ANT methodological approaches proposes an innovative concept to define causal relationships among urban key elements in order to elaborate the process of urban system evolution (urban dynamics) in terms of prospects for maintenance, transformation and change. This is the theoretical ground on which our hybrid methodology identifies the concepts for its categorical convergence. In the course of the case study, the combination of MAS and ANT methodological approaches involves taking into account all active agents regardless of their sort (ANT), their interdependencies and interconnections (ANT and MAS), and map their contributions (MAS) to continuations, transitions and turnovers of urban system transitions at the neighbourhood level.    

\subsubsection{ANT Discourse analysis}

According to the interpretation provided in (\href{Section 2.2.1}{Section 2.2.1}), ANT is addressed herein neither as a network in the technical sense, nor a theory in the social sense (\href{Latour}{\citealt{latour_actor-network_1996}}), but as a methodological approach which prioritizes  `relations over their characteristics` (\href{Cerulo}{\citealt{cerulo_nonhumans_2009}}: 536) and `action over mind` (\href{Cerulo}{ibid.}: 543). These relational and operational elements that mould urban development circumstances in Savamala were explained with ANT. The focus is on an actual post-socialist urban setting and the generation of the maintenance, transformation and/or change of the current state of affairs when global aspects are transformed to meet local specifications and vice versa. In terms of post-socialist cities, copying urban models from the West meets extraordinary difficulties because these cities lack the institutional infrastructure and cultural patterns essential  for  the  functional  unity  present  in  western cities (\href{Petrovic}{\citealt{petrovic_cities_2009}}). Furthermore, the fundamentality and intensity of economic and political change in Balkan post-socialist countries may be a historic exemplary of social transition hard to find in a ‘typical‘ capitalist city (\href{Sykora}{\citealt{sykora_transitional_1999}}).
\\

In Savamala, the researcher confronted with a dynamic reality, a battlefield of different influences, interests and interpretations which determines the future of the urban system itself. The qualitative data collection through case study with interpretative and participatory action research provided the background for ANT analysis enforced with correlational study. Moreover, the engineering approach of logical argumentation and schematic interpretation was later used for the dissemination of data. The principal data sources were both human and non-human. They were collected through a range of collection methods: [(1) extensive review of written sources, (2) interviews, (3) workshops, and (4) questionnaires], provided from key informants [(A) experts, (B) young professionals, (C) participatory activities, (D) Belgrade Waterfront Project] (\href{Table data sources}{Chart XXX}). These key informant categories also matched the aggregated human and non-human bearers of action and meaning (\href{Latour}{\citealt{latour_science_2005}}) among the traces of relevant influences, interests and interpretations in Savamala.
\\

On the level of data analysis, diagnosing urban development circumstances could be determined through a transposition of the current state of this neighbourhood into the elements which could denote urban flux. These elements, when gathered together into functional networks, form a unique set that indicate the factors of maintenance, transformation and/or change of the system, which, in this case, is the neighbourhood of Savamala. The successful application of ANT for these purposes involved transposing the terminology of ANT from (\href{Section 2.2.1}{Section 2.2.1}) into urban development factors. Furthermore, it was followed by an exploratory analysis that traces these factors within a real-life context. In order to apply the identified theoretical principles for the on-site analysis of a dynamic urban reality at the neighbourhood level, they were reformulated into a step-by step methodology, which was the following: (1) identifying human/ non-human entities; (2) flattening the reality of
intermediaries’ figuration and translations between mediators; (3) traceable associations among those who were constituted as actors; (4) tracking stability/instability of agency among actors; (5) (5) simplify and functionalize relations in urban assemblages based on established roles and the nature of links among them (\href{Table ANT discource analysos}{Chart XXX}). As part of a broader study on post-socialist urban development, the object of examination included the actor-networks in Savamala rendered from the composition of different layers of decision making that, through coordination and predominance, bring up urban dynamics. The level of analysis is the neighbourhood. 
\\

The central methodological issues for translating ANT terminology onto an urban environment indicate:
 
\begin{enumerate}
\item \textbf{All human and non-human actors:}

From an ANT viewpoint, the source of an action accounts equally for humans and non-humans, and only the action/relation counts (\href{Latour}{\citealt{latour_actor-network_1996}}). Animals, objects, texts, symbols, events, even mental concepts may be actors depending on their activities and/or relations (\href{Cerulo}{\citealt{cerulo_nonhumans_2009}}).
Based on the case study, the analysis distinguished the figuration of all particular human and non-human entities that are subjects of translations at the neighbourhood level of Savamala. The argument was grounded in the local context of planning procedures and practice concerning the Serbian urban system and the post-socialist neighbourhood level, as well as bottom-up activities in Savamala. In this manner, the reseacher pondered the complexity of the case study of a neighbourhood to be made up of human - people (urban actors and stakeholders), and non-human entities - urban structures and territories (natural and urban space), institutions and policy agendas, urban and communication infrastructures (\href{Mitchell}{\citealt{mitchell_city_1996}}; \href{Firmino}{\citealt{firmino_pervasive_2008}}) and social aspects (economic, political and cultural) (\href{Table ANT discource analysos}{Chart XXX}). These operational categories of urban key agents were traced through the extensive content analysis within the theoretical scope of this urban study. All case-specific entities are identified within the content analysis of various sources on the morphology of urban decision making in Savamala (post-socialist urban theory, planning legislation and documentation, media sources) and from on-site examinations.

\item \textbf{Intermediaries and mediators:}

Following \href{Latour}{Latour’s definition (2005)}, these human and non-human entities become "the means to produce the social" (\href{ref}{ibid.:38}) only when their roles in the system enact them as intermediaries or mediators (\href{ref}{ibid.}). In his words, intermediaries are simple bearers of meaning and mediators actually change the actions/relations they are engaged in. Based on the content analysis of scientific, legislative, operational and media data, the results of the analyses recognized that certain elements only through certain figurations in networks take the actor roles (\href{Table ANT discource analysis}{Chart XXX}). Four element types (entity, human, artifact, and event) were distinguished respectively. For example, all 4 mattered differently, depending on whether they were taken individually or in a set/group. In terms of artifacts, it was crucial to consider whether they were of a strategic, technical or repository type. In terms of illustrations, the shape of the nodes depends on what figuration of an element makes it an actor.

\item \textbf{Free associations:}

One of major achievements of ANT is its attempt to redefine sociology as the tracing of associations and thereafter designating social not as a quality of an element- entity, but as "a type of connection between things that are not themselves social"  (\href{Latour}{\citealt{latour_science_2005}}:5).
The urban key actors (urban actors, spatial forms, regulatory framework, and social aspects), after being denoted as mediators, have an active role in the networks, and in ANT terminology this is referred to as the performance of subjects (human entities) and the enactment of objects (non-human) (\href{Callon}{\citealt{callon_elements_1986}}; \href{Farias}{\citealt{farias_introduction:_2011}}). As part of ANT data analysis, the recognized entities were juxtaposed and converted into actors.  The established actors were those who associate and form networks (\href{Table ANT discource analysos}{Chart XXX}). The reason to reinterpret classical categories of scale, structure and the social in network terms was grounded in qualitative inquiry from experts (\href{Table data sources}{Chart XXX}). These categories were not taken for granted but applied only when they influenced actors’ relationality.

\item \textbf{Stabilizing and destabilizing agencies:}

When applying ANT for urban analysis, the importance lies in avoiding pre-established social science categories (\href{Farias}{\citealt{farias_introduction:_2011}}). It is essential to refer to agency as a relation that connects multiple actors and distributes causality and explanations across networks in a stabilizing or destabilizing manner  (\href{ref}{ibid.}). 
Based on expert insights and data from PhD workshops and documentation on local, regional and national levels, the researcher examined the complexity and interactions among the actors at the neighbourhood level - how they cooperate/coordinate/negotiate/collide and organize into networks according to their roles (\href{Table ANT discource analysos}{Chart XXX}). In graphical terms, node colours correspond to the agency of actors and the active, but standardized, networks they are engaged in. The difference between association and agency in this interpretation lies in their dynamics - these networks, though standardized have the bipolar potentialability to influence actor-networking.

\item \textbf{Urban assemblages:}

An urban assemblage is a trendy term for aggregating relations of heterogeneous urban actors (\href{Muniesa}{\citealt{muniesa_introduction_2007}}). It also address `relations of exteriority` based on actor-networks (\href{Farias}{\citealt{farias_politics_2011}}:15).
According to ANT social and structural descriptions of urban dynamics, data which had been validated in workshops with researchers, professionals, activists, young professionals and citizens were channeled visually through an actor-network diagram. The body of actor-networks was comprehended without any preconceptions of society, social realm, social context and/or social ties. They were visualized through the size of nodes (actors) and the colour of links between them (networks) (\href{Table ANT discource analysos}{Chart XXX}). The size of the node equals the centrality of an actor and its influence. The actor’s influence was assigned approximately according to its presence in time, the number of its relations, and the types of these relations. Conversely, the colour of the connections relate to the nature of links in which these actors engage and produce specific social effects.
\end{enumerate}

This 5-step ANT framework aims to illustrate the urban development of a post-socialist neighbourhood in Belgrade - Savamala. For the logical argumentation on network formation and development, the researcher accepted the basic rules that correspond to the major ANT assumptions: (1) everything that matters is an actor and therefore engaged in a network(s); (2) there is no context or any non-associated element in the system. In this respect, the ANT diagram visualized the Savamala urban development circumstances (all context-related, history-related, on-site and documentation-related data) in terms of actors (human and non-human) and the nature of links they are engaged in, relative to their activities, priorities and relationships.

%visualize ANT legend

\subsubsection{MAS Structural analysis}

According to the Multi-agent system (MAS) approach, system dynamics relates to individual elements and their communication, free will, belief, competition, consensus and discord etc. (\href{Ferber}{\citealt{ferber_multi-agent_1999}}).
For the application of this method in urban analyses, all these elements must be taken into account, as it actually is in its application in computer programming. The system of analysis therefore consists of:

\begin{itemize}
\item E(nvironement) - environment: static, defined by the level of analysis - this research: post-socialist neighbourhood;
\item A(gents) - assembly of agencies: static state of urban key agents;
\item O(bjects) - set of objects: passive contextual elements - this research: socio-spatial patterns of spatial capacities and social potentials;
\item R(elations) - assembly of relations: active elements - this research: socio-spatial patterns of urban practices and urban conflicts;
\item Op(erations) - assembly of operations: active morphology of urban decision making - this research: top-down urban planning, real estate transformations; participatory bottom-up activities;
\item S(ystem) - reaction of the context: static, research goal - this research:  urban system transitions in terms of maintenance, transformation and change processes.
\end{itemize}

These elements interpret the totality of a system, its functioning and its evolution.
The data analysis process sorts out data from [(1) written sources, (2) interviews, and (3) workshops] according to the above introduced system set.
This initial categorization was tested - in practice - by the questionnaires, which targeted [(A) experts, (B) young professionals, (E) students] (\href{Table data sources}{Chart XXX}).
\\
In more technical terms, the basis for MAS-ANT cross-pollination is the MAS level of agent profiles:

AGENT PROFILE = AGENT STRUCTURE + AGENT PREFERENCES + AGENT BEHAVIOUR

On the level of the MAS method, actors were readdressed as agents and thereupon agent profiles came up. They were configured from a combination of agent structure, agent preferences and agent behaviour.
\\

However, MAS data analyses address the character of links among different elements of the urban environment in terms of agent preferences. These qualifications of links is defined relative to elements identified as objects (contextual resources) and relations (urban conflicts and urban practices) in the MAS data circle. These interpretations indicate the state of urbanity of an urban environment (in this case, Savamala), while the level of urbanity is recognized in their further operationalization through urban system transitions. The issue of agent profiles and urban system transitions were further elaborated and examined for the MAS-ANT data display.

\subsection{System Building - Data display}

The reporting of findings is done in a visual manner, where all the data were categorized and built into the visualized system of urban system transitions. In this manner, data visualization techniques were used for data reduction and the operational display of data.

\subsubsection{Setting a procedure}

The data analysis processes were carried out iteratively from: (1) collecting context-based information and knowledge; (2) ANT classification of the data at the local context; (3) MAS analysis of agent behaviours. The key informant categories were identified after the traces of relevant influences, interests and interpretations in Savamala. The descriptive nature of ANT premises enabled data structuring in terms of the set of human and non-human agents and urban assemblages at the neighbourhood level. After that, the behaviours of agents were identified by qualitative surveys.

Each agent profile element refers back to both ANT and MAS categories:

\begin{itemize}
\item AGENT STRUCTURE addresses the ANT categories of: (a) the nature of actors; (b) the structure and networks of influence; (c) the secondary and socially functional networks. The sum of these interpretations translated actors into agents and constituted an assembly of agencies in MAS. 
\item AGENT PREFERENCE(S) take (d) social artefacts (political, economic and cultural) from ANT and involve them in the operationalization of objects - O - (contextual resources) and relations - R - (urban practices and urban conflicts) from MAS.
\item AGENT BEHAVIOUR(S) are the ANT (e) networks of translations and the MAS explanation of HOW (pro-activity, sensitivity, interaction) these agents reference back to the system development (maintenance, transformation and change) - U from MAS terminology.
\end{itemize}

Sorting all the data about an agent in these categories provided the full description of how this urban system works. ANT analysis furnished an exhaustive categorization of elements and networks - the detailed image of agent structure and the field of their influence (networks). Agent preferences were represented through object and relation categories from MAS and corresponded to the state of urbanity’s theoretical stances. In practice, it signifies that all urban conflicts and social practices can be identified as directed relations (R). Conversely, spatial capacities and social potentials were referred to a set of objects (O) (O)	to be activated. Agent behaviours are the products of multi-criteria MAS analysis  (\href{Arsanjani}{\citealt{arsanjani_spatiotemporal_2013}}). In practice, urban dynamics is defined as how the agents - A - behave in order to perceive, produce, consume, transform and manipulate objects - O - and engage in relations - R - in order to enable maintenance, transformation or change of the system.
\\

The display of these data is executed in hierarchical order. ANT served for the identification of all actors (human/non-human, material/non-material) and flattened the social into a panoptic internalized ontology. MAS traced the character of their appearance in networks and their internal relations and connections. In this way, ‘actors (ANT) are transferred into ‘agents‘ (MAS), while the theoretical layer represents a generative body of concepts suitable for tracing urban dynamics. Such triangulation is carried out in three steps (\href{Table MAS structural analysos}{Chart XXX}):

\begin{enumerate}
\item Interpreting agent structure:

All context-related, history-related, on-site, literature-related and empirical data were sorted according to ANT principles - so that everything that matters is an actor and therefore engaged in a network(s) and that there is no context or any non-associated element in the system. 

\item Connecting urban agency with socio-spatial patterns:

Urban practices, contextual resources and urban conflicts were recognized and connected to the bearers of urban agency. In theoretical terms, the topology of ANT was transferred into the topography of MAS: agents were assigned behaviours in relation to their agent structure and their preferences.

\item Evaluate state and the evolution of the system (maintenance, transformation, change):

From the approach presented above, tracing urban dynamics comprises structuring an urban environment according to clear categories as well as simulating autonomous actions and interactions in order to study blurred processes of constant system evolution. The set of networks involve the heterogeneous distribution of urban key elements acting at sites (human and non-human) and entangle causes and consequences of their actions and forces.  All these urban key elements are assumed to be equal agents in the reproduction of urban practices, operating contextual resources and dealing with urban conflicts.  These continuous processes of maintenance, transformation and change reflect agent behaviours and contribute to the system dynamics.
\end{enumerate}

Referring back to ANT assemblage networks and MAS analysis of agent behaviours in networks, the results were presented in terms of scenarios and system transitions. The scenarios deal with: measuring the efficiency of urban planning, testing the legitimacy of real estate interests and recognizing the opportunities of bottom-up design and participatory initiatives. On the other hand, the estimation of urban dynamics traced the path of system resilience, flexibility and/or dynamics in terms of the maintenance, transformation and change processes.


\subsubsection{Data visualization with infographics}

This thesis adopted MAS-ANT methodology in order to:
(1) describe complex urban reality in a post-socialist city (ANT);
(2) understand how the level of urbanity serves for tracking socio-spatial patterns of transition (MAS);
and
(3) indicate the processes of urban system transitions (ANT+MAS).
In order to provide efficient classification of great amount of data and to enable new reading of the recognized interconnections and processes in this research, the data and the results are continually reinterpreted also in a visual manner.
Persistent visual representation of data collection, analysis and data display enables new level of data triangulation and data systematization in graphical terms.
\\

Visual interpretation of data is as old as all the other knowledge production mechanisms, dating back to (on cave walls) in prehistoric period to contemporary ideograms on modern signs (\href{Krum}{\citealt{krum_cool_2013}}).
Visual interpretation of information, data and knowledge have always been useful for revealing data, presenting a lot of data in a small space and making large datasets coherent (\href{Tufte}{\citealt{tufte_visual_2001}}).
Graphic design combines illustrations, text, maps and images to tell a story that stands out constructed from data. 
In this way, there are two types of static visual interpretations (\href{Krum}{\citealt{krum_cool_2013}}):
(1) infographics - short form from information graphics (visual representation of any type of information, data and knowledge), and
(2) data visualization (mainly for qualitative data).
\\

In this research, infographics are used to filter data overload and summarize and cluster them according to ANT, MAS and MAS-ANT categories.
The transferring process from data to diagrams included: recognizing, classifying, filtering, interpreting, arranging, connecting, condensing, understanding, explaining, integrating,extrapolating and generating (\href{McCandless}{\citealt{mccandless_knowledge_2014}}).
Infographics are constructed in such way that they propose the readable  language (forms - vectors - colours) and recognizable patterns (scale, proximity, links and visual emphases) within the same set of diagrams.
Visual investigations in this research project adopted a unified graphical language in terms of the identity of vector-images, patternized forms and an universal color register.
Furthermore, the nature of the diagrams applied in this research could be:
\begin{itemize}
\item relational diagrams for visualizing agents, networks and processes [i.e. ANT (\href{Chapter 5}{Chapter 5}), MAS (\href{Chapter 6}{Chapter 6}) and MAS-ANT (\href{Chapter 7}{Chapter 7})];
\item time-lines that serve as databases for historical data and processes represented in a linear manner (\href{Rosenberg}{\citealt{rosenberg_cartographies_2012}});
\item spatialized qualitative data and structuralized spatial data: data-maps (\href{Raisson}{\citealt{raisson_2033_2010}});
\item data frameworks built according to overlaps, proximity and complementarity of concepts in theoretical, methodological and contextual sets (\href{Caraes}{\citealt{caraes_images_2011}});
\item diagrammed data charts that structuralize different data types together (\href{Yau}{\citealt{yau_visualize_2013}}).
\end{itemize}

The choice of diagrammic interpretations is very common visual language of architecture (\href{Chaplin}{\citealt{chaplin_architecture_2015}}); used to extend the availability of data (\href{Hemmersam}{\citealt{hemmersam_exploring_2016}}).
While there is still a certain bias coming out from translations of textual data into visualizations, easy update processes enable continual integration of new descriptions and analysis (\href{Tufte}{\citealt{tufte_visual_2001}}). 
Moreover, visual data display is chosen as it is much more easily recognized and memorized (\href{Krum}{\citealt{krum_cool_2013}}), which may be very useful for practical use of the proposed hybrid method.

\section{Methodological Framework}

This chapter dealt with the technical explanation of reasoning and scientific domain and interpretative tools applied in this research. In this respect, the methodological framework of this thesis followed the logic of translation from the research framework into the research design, instrumentalized through the adopted methodologies. The identified research questions and hypotheses were developed from the initial research aims and further adapted in reference to the theoretical frameworks of the main concepts and methods. They were presented in consecutive order, then bounded within the scope of the chosen case study, and finally methodically traced within the elaboration of the methodological approaches and their deployment in data collection, analysis and data display.
\\

In other words, the methodological framework also presents the space-time articulation of the research process: (1) case study data collection takes up an historical and analyticial approach; (2) ANT data analysis is focused on descriptions and the present moment; (3) MAS data analysis is future-oriented and a relational study; and (4) the MAS-ANT data display serves to connect past, present and future, and to address processes and prospects.
\\

In sum, the presented methodological elaboration and the coherence of this research set the basis to weave the research problem, its scope and aim through the systemic theoretical framework with the field data, in order to indicate the paths for conclusions in terms of the research framework, the theoretical scope and practical implementations, the limitations and the prospects of this thesis.

%visualization of the framework and conclusions

%%%%%%%%%%%%%%%%%%%%%%%%%%%%%%%%%%%%%%%%%%%%%%%%%%

\chapter{Historical Urban Processes: From the State to the Neighbourhood [Serbia - Belgrade - Savamala]}

%%%%%%%%%%%%%%%%%%%%%%%%%%%%%%%%%%%%%%%%%%%%%%%%%%
As has already been pointed out, urban development is a general and broad concept as long as it is detached from an actual urban setting. In a local context, the importance of urban development is overthrown by the concern for how global aspects meet local specifications and how on-site forces and uncertainties govern the concrete dynamics of the urban system. In order to move away from the abstractions and generalizations bounded in methodological terms, all the theoretical stances that address urban complexity and dynamics are traced in a real-life context.
\\

Any historical background to the research problem, its transformation over time and the immediate surroundings where it emerges and changes, should be considered as a chronological sequence. This, if described properly, provides a suitable means for explaining causal links among the identified factors and elements of urban development processes. Given the implications of positive theory, it will be possible to predict the future relationships and behaviour of the elements in question.
\\

This chapter shows an overview of socio-spatial circumstances in their autochthonous discourse, their direct manifestations in the capital city of Belgrade and their final repercussions in the Savamala neighbourhood. The context is attenuated throughout the chapter from the nation-state aspect, a citywide dimension to the neighbourhood level. A linear, factorial analysis is performed at the neighbourhood level in order to distinguish the layers of its urban decision making. These organizational levels (national, city, neighbourhood) and directional layers (top-down, real estate, civic engagement) interlace the space-time boundary of the context for tracing its intrinsic complexity and dynamics. This contextual narrative is conducted in two directions: the chronological discourse on urbanity and the causal links within the morphology of urban decision making.

\section{The State of the Society and the Urban}

%{table diagram on different periods
% intro + topics: population info, political and social, urban culture, urban form, urban planning

Investigating autochthonous urbanity in the chosen context implies attributing socio- spatial circumstances to the continuous urban transitions. The versatile geographical landscape of the Balkan Peninsula has harboured an amalgam of cultures and religions and produced a condensed history of the social, political and economic turmoils that compose urban  structures and processes (\href{Nedovic}{\citealt{nedovicbudic_waves_2006}}).
The key factor of this historical account is understanding how the structural unity of urban key elements has been deployed in different space-time frameworks and how they reference back to the social and urban processes of maintenance, transformation and change (urban system transitions).

\subsection{The Evolution of the State Affairs}

Historically speaking, the first permanent human settlements in this region were established during the  Neolithic  period (\href{Krstic}{\citealt{krstic_planerski_1972}}). The Vinca archaeological site near Belgrade provides material evidence of cultural patterns in terms of the settlement and the behaviour of its inhabitants between 5700 and 4500 BC  (\href{Srejovic}{\citealt{srejovic_vinca_1984}}).
\\

However, the chronological discourse on urbanity in this research is built upon the following periods:

\begin{enumerate}
\item the Ottoman dominance;
\item the Serbian state (1804-1914);
\footnote{Even though Serbia officially gained independence from the Ottoman empire only in 1878 and was pro-
claimed the Kingdom of Serbia in 1882, the continual change in urban and social issues started from the first Serbian uprising in 1804 and had been gradually changing when Serbs took power over the capital city between 1807 and 1813 and later continued in parallel with the establishment of Serbian control and state authorities.}
\item Yugoslav self-managed socialism, the union of Southern Slav nations and the Federation of 6 republics (1945-1990);
\item the post-Yugoslav, post-socialist transition (1990-).
\end{enumerate}

The structural unity of urban key elements is bounded in the broad, factual and chronicled identity of the pertaining urbanites (urban actors).
The choice of periods does not strictly follow the historical continuum, \footnote{Exclusion of antique and medieval historical heritage is based on its incoherence with contemporary urban key elements crucial for this analysis. In the same terms (urban key agents), the period of the Kingdom of Serbs, Croats and Slovenes (The first Yugoslavia) shows continuity either with its predecessor (the Kingdom of Serbia) or its successor (SFRY).}
but relies on the socio-spatial constituents of the contemporary urbanites. The urbanity in this area is reflected through the fluctuating relations with the European cultural and geopolitical realm, the strongly spatialized identity of the local population (\href{Savic}{\citealt{savic_where_2014}}) and a continuous revolution   through the transitions of the historical periods listed above. Henceforth, these periods indicate the key points of the alterations and development in terms of: the political sphere, a socio-economic realm, formation of urban actors, urban culture, urban professions and the distribution of urban forms.

\subsubsection{The Ottomans}

The Ottomans  dominated  the  central  part  of  the  Balkan  peninsula for around 500 years  (today’s  Bosnia- Herzegovina, Macedonia, Serbia, and Kosovo in particular).
\footnote{The year 1459 is marked as the year when the  Serbian despotate was officially overthrown by the Ot-
tomans, and in 1804 the Serbian revolution started against Ottoman rule. However, the very first Ottoman penetration into the area of the Balkan medieval states followed the Battle of Kosovo Polje in 1389. Serbia officially gained its independence again during the Congress of Berlin in 1878. (\href{Corovic}{\citealt{corovic_istorija_1997}}).}
Since the Ottoman empire was an Islamic theocratic state, decision-making was conducted through the complex centralized administrative structure under the supremacy of the Sultan. The local Christian majority was marginalized and the rural population was ruled by the constant threat of extinction. Urban nexuses were populated mainly by the ruling social class of Ottomans and local converts. As the Ottoman rule of this region could be characterized as an authoritarian imposition, any type of development was slowed down and reduced to a minimum (\href{Nedovic}{\citealt{nedovicbudic_waves_2006}}).
\\

It was an archaic agrarian society. The territory of contemporary Central Serbia was the border zone between the Ottoman and Austrian empire, the Muslims and the Christians, the East and the West, Europe and the Other ref. The Drina and Sava rivers established the natural borders of the two civilization patterns. A few, predominantly small cities in the area, were susceptible to the frequent destructions of war, either in the Turkish-Austrian wars or during population revolts. The fortress of Belgrade, standing at the borderland, was vulnerable to the frequent conquests and invasions of the two empires. The cities developed organically, under the influence of the Middle East, Islamic planning and building (traditions
of planning) (\href{Nedovic}{\citealt{nedovicbudic_waves_2006}}). Regardless of size, cities were the centers of social and cultural life of this state. Unhygienic, haphazard construction and urbanization patterns are typical of Ottoman cities in the Balkans (\href{Kadijevic}{\citealt{kadijevic_jedan_2007}}). 
\\ 
The Christian population was either excluded from social structures and any form of decision making or gathered around Christian neighbouring states from the West and participated in their warfare against the Ottomans. Consequently, local religious powers (whether Catholic or Orthodox Christians) managed the societal context of the subdued population.
The Serbs cherish the negative collective memory of this period, which resulted in the extended dissolution of the Ottoman cultural and urban heritage and a disregard for and subsequent dilapidation of the Ottoman architecture, while unconsciously retaining certain social, urban and decision making practices (\href{Blagojevic}{\citealt{blagojevic_urban_2009}}).

\textbf{"The settlements (called kasaba if small, varos if large) had a distinct structure including: a central section (carsija) for public functions like baths (amam), schools (metresa), coffee houses and entertainment places (kafana), buildings of worship (dzamija), crafts and trading posts (bazar), and travel inns (han); and a residential section (mahala) separated into two parts – the upper for Muslim residents and the lower for the Christian population. Residences were built around yards (avlija) surrounded by high walls used to protect the privacy of the extended family." (\href{Nedovic}{\citealt{nedovicbudic_waves_2006}})}.
 
\subsubsection{The Serbian State}

The  liberation  of  Serbs  slowly  but  surely  started  with  the  uprising  in  1804.   In parallel with  their  efforts  toward  building a  nation  state,  educated  Serbs  followed  the  European  wave  of enlightenment  and  joined  efforts  to  standardize  the  national  language  and  set  a  national educational framework. The University of Belgrade was founded in 1808,
\footnote{At first it was the Belgrade Higher School in revolutionary Serbia (1808-1813) and in 1838 it merged with the Kragujevac-based departments into The Lyceum of the Principality of Serbia (1838-1863). It was renamed the Second Higher School (1863-1905) when natural and technical sciences units joint philosophical and legal departments.}
The National Theatre was established in 1869 and The National Museum was opened on the 10th of May 1844.
\\

From 1882 onward, Serbia was recognized as a constitutional monarchy with a capitalist social order. Even during the constitutional years before the official recognition of the country, the population of Serbia rose from 678,192 in 1834 to 1,669,337 in 1884 (\href{Zavod}{\citealt{zavod_za_statistiku_i_evidenciju_nr_srbije_stanovnistvo_1953}}).
The population reached 2,922,258 in 1910 (\href{Zavod}{ibid.}). 
In parallel with wiping out the surface signs of the Ottoman legacy, the mentality of the deprived, rural majority of the indigenous population under the theocratic Ottoman regime survived and contributed to the patriarchal, paternalistic and authoritarian political model
\footnote{with the huge rural hinterland and its traditional notions and influences}
at stake, more often than not, in the Kingdom of Serbia (\href{Vukmirovic}{\citealt{vukmirovic_city_2013}}). Moreover, the state was poor, dominated by a weak and vulnerable, autarchic economy based either on trade or foreign investments and closely related to the state and to privileged groups (\href{Vukmirovic}{\citealt{vukmirovic_city_2013}}; \href{Samardzic}{\citealt{doytchinov_belgrade_2015}}).
\\

\href{Stojanovic}{\cite{stojanovic_ulje_2010}}, a well-regarded historian who has investigated the Serbian state’s constitution during this period, has explained that state affairs, the civil sector and social circles developed without any real interconnections from the very beginning of the Serbian modern nation state. She has further developed the thesis that the state is usually taken for granted as a society in itself, the supreme driver of development and modernization, and the most important source of influence/authority/status/wealth for any individual who belonged to this governing society. In other words, the political ruled out and dominated the economic and the social during the rise of the modern Serbian state  (\href{Stojanovic}{ibid.}).
\footnote{\href{Stojanovic}{\cite{stojanovic_ulje_2010}} gave an example of how the state authority was dominating the political discourse during 1882-1914 period: the reasoning behind the issue of significant liberty of press in the kingdom of Serbia at the beginning of the 20th century was such that the press did not have the force of public, so that the authorities did not pay any attention on the press releases and the topics therein covered.}
\\

However, the Kingdom of Serbia was also slowly gaining its status within the European realm, forming close ties with European centers, either by educating professional staff there or by importing models, systems and structures from the West ref. This phase was also crucial for 
the foundation of a Serbian elite and the modest emergence of a local ”nouveau” bourgeoisie and an aristocracy based principally in the capital. Those circles were filled with soldiers, clergy, a small intelligentsia and highly obedient bureaucrats, while the first signs of Civil Society (traders, scholars, officers, civil servants) also emerged, along with new challenges and opportunities for development (\href{Vukmirovic}{\citealt{vukmirovic_city_2013}}).
The increasing number of engineers educated in the West (mostly in Austria) led to the construction of the city towards western models and an extended disregard for the origanization and building principles of the Ottoman city and the subsequent dilapidation of any leftovers of Ottoman architecture (\href{Blagojevic}{\citealt{blagojevic_urban_2009}}).
\\

The  urban  population  in  the  Kingdom  of  Serbia  was  also dispersed  in  small  cities that counted around 13\% of the total population (\href{Kraljevina Jugoslavija}{\citealt{drzavna_statistika_kraljevine_jugoslavije_definitivni_1932}}). Still, even the capital city of Belgrade was characterized during those times as a rather large and insufficiently regulated dorp.  This period is marked by radical changes in all domains of urban and social life.
(\href{Samardzic}{\citealt{doytchinov_belgrade_2015}}) suggests that the most important benefit of the Serbian national revolution of 1804-1830 was gaining authority over the city of Belgrade. The task of constructing the new state through building, organizing and modernizing its capital city posed a challenge to the backward peasant society, as Serbia was in those times  (\href{Samardzic}{ibid.}). Namely, cities with their pluralistic, diverse cosmopolitan character were identified as the enemies of the nation and the church ref. They tended to keep their traditional, rural flavour with the permanent problem of over-population of available housing, as well as illegal and non-quality constructions in the suburbs (\href{Doytchinov}{\citealt{doytchinov_modernization_2015}}).
\\  

The urban morphology of Serbian cities in the 19th and the beginning of the 20th century was clearly strongly affected by Western European ideas. This random, undiluted borrowing of European building principles, methods, and techniques transformed cities into modern towns  (\href{Nedovic}{\citealt{nedovicbudic_waves_2006}};\href{Kadijevic}{\citealt{kadijevic_jedan_2007}}). Architectural and city planning practice gained momentum from an emphasis on urban growth with a new orthogonal street network and central places (piazza, square, place) and within a modular grid of plots, blocks and streets (\href{Kadijevic}{\citealt{kadijevic_jedan_2007}}). An additional functional layer of traffic, sanitation and utility infrastructure contributed to the modernization of urban structures and the harmonization of urban systems and networks.

\paragraph{Urban Regulatory Framework and Practice}

The pioneers of this new wave of construction and activity were individuals educated abroad (mostly in Vienna, Budapest and Praque, and later in Paris and Germany) and foreign experts under the auspices of the state (\href{Maksimovic}{\citealt{maksimovic_idejni_1978}}). They usually received state scholarships to attend elite European universities and were given the opportunity to engage in projects straight afterwards, even without any significant professional experience (\href{Mladjenovic}{\citealt{mladjenovic_novija_2010}}).
\\

However, town planning was often used as a controllable tool at hand to consolidate political power by high authorities (Governments and even rulers). 
\footnote{There is an anecdote that Milos Obrenovic, the first ruler of the not yet independent state, who happened to be illiterate, controlled planning and building documentation and visited construction sites with the
engineers in order to supervise them, intervene and implement his own spatial visions}
The Ministry of Construction/Civil Engineering, backed up later with a specialized architecture section, was the leading actor in architecture and planning at the time. The young state administration was controlled by centralized political will, directing the development of architecture, adapting it to the needs of the newly established, young Serbian state (\href{Maksimovi}{\citealt{maksimovic_idejni_1978}}; \href{ref}{\citealt{nedovicbudic_waves_2006}}).
However, in such a small, young state, human resources with high expertise are often  scarce,  so  that,  when  the  Faculty  of  Architecture  was  founded  in  Belgrade  (1897), the  same  professionals  working  for  the  Ministry  also  obtained  teaching  positions  - pursuing in this manner parallel careers. In general, the private architectural practice of the time was highly underdeveloped (\href{Mladjenovic}{\citealt{mladjenovic_novija_2010}}). Such circumstances of centralized power, the overlapping of jurisdiction and competence and biased relations between political and economic strata, opened the door for corruption, privately driven initiatives and land speculations by powerful and rich individuals.
\\

This institutional framework and these building practices were followed by the constitution of a regulatory and legal framework of urban development, planning and construction. In the Kingdom of Serbia and its successor the Kingdom of Yugoslavia,
\footnote{As well as The Kingdom of SHS.}
the legal framework for urban planning was based and developed according to the European traditions of continental law (\href{Zekovic}{\citealt{zekovic_historical_2016}}).
The first adopted document was a sort of regulation plan - The Regulation Line for Construction of Private Buildings (1864), followed by the adoption of the following laws - The Law on Construction of Public Buildings (1865) and the Law on Expropriation of Private and Real Estate Property for Public Use (1866) (\href{Nedovic}{\citealt{nedovicbudic_waves_2006}}).
\\

The planning and construction activity during this period had a decisive influence on the Europeanisation of the Serbian development path. Nevertheless, factors such as lying on the south-eastern European periphery, the strong nationalist political and social discourse and the colonial imperialism in culture and the arts also hindered urban emancipation and institutionalization (\href{Vukmirovic}{\citealt{vukmirovic_city_2013}}).
\\

The response to such conditions was the continual re-establishment of tabula rasa urban actions which were viewed as the only means of intervention in historical time, which may eventually have rendered false (\href{Blagojevic}{\citealt{blagojevic_urban_2009}}). In this manner, the egocentric-megalomaniac attitude to urban planning during the 1912-1922 period reflected state ambitions to take advantage of the changed conditions of state sovereignty, which had significantly enlarged after the First World War (\href{Blagojevic}{\citealt{blagojevic_urban_2009}}).
\\

Both the periods of The Kingdom of Serbs, Croats and Slovenes (1918-1929) and what was known as the Kingdom of Yugoslavia (and also known as ”the first Yugoslavia”) from 1929 until the German occupation during the Second World War are excluded here as they followed the trends set forth during the previous period (1804-1914) with small insignificant shifts and certain emancipatory and modernization efforts and successes. In fact, the country comprised almost the same  territory as its descendant SFRY with a population of 11,984,911 inhabitants in 1921. The most important document of the time was the Building Act (1931), an up-to-date urban regulatory document. It features the Regulatory Plan as the main instrument of urban development, regulates technical building details and prescribes the format of planning documents and planning procedures (\href{Nedovic}{\citealt{nedovicbudic_waves_2006}}).

\subsubsection{SFRY}

The Second World War on the territory of Serbia and Yugoslavia ended in April 1945 when Yugoslav Partisan forces liberated the country and even occupied parts of German (Austrian) and Italian territory. During the Second World War, Yugoslavia lost a significant portion of its population and numbered only 15,772,098 inhabitants in 1948.
\\

After the Second World War, a new political order was installed. The first Yugoslav post- World War II elections
\footnote{The elections were secret and generally speaking fairly conducted, but there are several accounts doubting the regularity of the campaign.}
were held in November 1945 and the results were undoubtedly in favour of the coalition of parties backing the Partisans.
\footnote{The People’s Front (Narodni front, NOF) coalition was led by the Communist Party of Yugoslavia (KPJ) and
represented by Josip Broz Tito.}
On 29 November 1945, the Constituent Assembly of Yugoslavia formally abolished the monarchy and declared the Federal People's Republic of Yugoslavia (FPR Yugoslavia, FPRY)
\footnote{In 1945, the monarchy was first replaced with  first Democratic Federal Yugoslavia (DFY).}
with six "People's Republics" (Slovenia, Croatia, Bosnia and Herzegovina, Serbia, Montenegro and Macedonia).
In the post-war geopolitical divide, FPR Yugoslavia expanded beyond the borders of its predecessor (the Kingdom of Yugoslavia).
\\

The socialist period in Yugoslavia is best known for its self-managed socialism (1953-1990).
\footnote{Self-managed socialism or market socialism are popular terms for the type of socialism applied in SFRY and is known as a purely Yugoslav brand (\href{Estrin}{\citealt{estrin_yugoslavia:_1991}})}
The political regime was decentralized yet authoritarian with both capitalist and socialist elements. It was a single-party system with a president for life (Josip Broz Tito). However, decision making was partly shared between the central and supreme authority of the government and the republics, municipalities and even several public enterprises. The economic reforms put to work were quasi-market and quasi-liberal and included self-management within most  enterprises, societal ownership over large industrial enterprises, and a number of small, private businesses (services and crafts). The economic system survived through non-market mechanisms and administrative decisions (\href{Estrin}{\citealt{estrin_yugoslavia:_1991}}).
The economic system survived on non-market mechanisms and administrative decisions (\href{ref}{ibid.}).
Its most important trait was "social  ownership" , i. e.  workers' self-management and control of  enterprises (\href{ref}{ibid.}).
Finally, cultural exchange was significant both with the West and the East  (\href{Hirt}{\citealt{hirt_belgrade_2009}}; \href{Vujosevic}{\citealt{vujosevic_conundrum_2012}}).
The country experienced successful economic reforms, cultural revival, labor productivity, urban and infrastructural development and construction in its first years (until the late 1970s).
\\

During the 1970s, Yugoslavia was internationally recognized as an industrialized and middle income country. The urban milieu of socialist Yugoslavia was significantly egalitarian and diverse with a high standard of living. Socialist values led to educational, housing and health policies that treated all citizens equally and provided them with basic services, while industrialization of the country helped raise living standards for ordinary citizens. Consequently, these social circumstances contributed to less marginality and fewer social class disparities. Cities were extensively built and modernized for its citizens with equal rights and more or less equal opportunities, contributing to the low level of under-urbanisation, and consequently fewer autonomous and heterogeneous urban forms (\href{Vujovic}{\citealt{vujovic_belgrades_2007}}; \href{Stanek}{\citealt{stanek_urban_2014}}).
\\

However, the dysfunctional combination of these initially positive factors and the unfortunate course of events generated by deep a economic crisis (unemployment, inflation), the decline of legislative power and ever-increasing regional disparities marked its final years (1980-1990)  (\href{Estrin}{\citealt{estrin_yugoslavia:_1991}};\href{Stamolieva}{ \citealt{stambolieva_welfare_2013}}).
\\

These social, political and economic circumstances also reflected onto cities and urban life. The overall concepts of urban development and planning in SFRY were based on CIAM principles and established various forms of urban standardization, a center-periphery urban dichotomy, based on concepts of the local community and decentralization in decision-making (self-management at various levels - community, municipality, enterprises, organizations etc.) (\href{Fisher}{\citealt{fisher_planning_1962}}; \href{Nedovic}{\citealt{nedovicbudic_waves_2006}}).
Urban institutions, regulations and professions in general functioned on quite a mature, comprehensive, and multi-disciplinary basis. The systemic institutional approach distributed planning tasks from the top-down, involving experts from various fields with initiatives to apply the self-management paradigm and participatory practice in urban planning as well.  This advanced planning system with its biased implementation reflected the political and economic shifts in the course of Yugoslav state development, often "failing in implementation just as the state socialism itself" (\href{Nedovic}{\citealt{nedovicbudic_waves_2006}}). These changes were grounded upon the legislative transformations of the Yugoslav period (\href{Nedovic}{\citealt{nedovicbudic_waves_2006}}):
\begin{enumerate}

\item  \textbf{Post-war reconstruction (1945-1953);}
\\
During the first years, SFRY was under strong influence of Soviet political ideology. These were also the years of reforms and the deployment of ideology in socio-political, economic, and cultural terms within the newly established state. However, with the slow but significant separation from the Soviet model, Yugoslav administration put to work the principle of workers’ self-management within its enterprises (\href{Zec}{\citealt{zec_economic_2012}}). The principle was that employees had a key role in the decision-making structures of their enterprises. While in practice decisions were guided by management departments, workers were usually involved in dealing with socio-economic issues
\footnote{Questions of welfare, employment and pay}
(\href{Lydall}{\citealt{lydall_yugoslav_1986}}; \href{Estrin}{\citealt{estrin_yugoslavia:_1991}}).
Such a structure proved unsatisfactory and inefficient in creating effective capital and labor markets from the very beginning.
\\

After WWII, land and premises in private possession were expropriated and spatial resources were solely under state control. Therefore, the state retained considerable control over the country’s development by allocating investment funds centrally and very often by putting urban planning in the service of the regime  (\href{Estrin}{\citealt{estrin_yugoslavia:_1991}}). The structure of urban institutions was strongly centralized (central-command planning) and planning instruments were directed to support socio-economic development plans (\href{Borovnica}{\citealt{borovnica_osvrt_1980}}; \href{Pajovic}{\citealt{pajovic_pregled_2005}}; \href{Nedovic}{\citealt{nedovic-budic_mornings_2011}}; \href{Peric}{\citealt{peric_evolution_2016}}). 

\item  \textbf{Institutional decentralization(1953-1963)}
\\
This period was initiated by the constitutional change of 1953. The reform for social instead of state ownership was introduced in 1952 and came to the fore only in correlation with the new legal framework. Such "social ownership" and nominal workers’ control over the surplus were effectively in place as a form of  of non-ownership, with the plurality of self-management interests, delegate structure, accumulation of decision-making, but no legal individual rights over the assets (\href{Estrin}{\citealt{estrin_yugoslavia:_1991}}, \href{Zec}{\citealt{zec_economic_2012}}).
However, apart from the primarily state property ownership and market control, private ownership of small and medium business enterprises was permitted (\href{Hadzic}{\citealt{hadzic_yugoslav_2002}}).
In parallel, the country’s external affairs were marked by the politics of neutrality and initiating the foundation of the Non-aligned Movement (NAM). The policy of institutional decentralization put forward the first generation of urban planning laws in the same manner (ref). As this new legal framework was enacted, a set of decentralization practices emerged in professional urban planning, especially from 1959 to 1970. In this light, new professional organizations spread in the urban centres. The planning profession profited through the organizational division of urban planning structures and the dispersion of roles among newly established entities (1954-1959) (\href{Nedovic}{\citealt{nedovic-budic_mornings_2011}}). 

\item  \textbf{Strengthening of legislation at the republican level  (1963-1973);}
\\
In 1963, the Constitution of the Socialist Federal Republic of Yugoslavia and the Constitution of Serbia were adopted. Moreover, in compliance with pervasive constitutional reforms, the name Socialist Federal Republic of Yugoslavia (SFRY) was introduced. This phase was the emancipatory phase of Yugoslav self-managed socialism marked by a strong economic factor in terms of the decentralization from state investment funds to socially owned banks (1965). The reform aimed to transition to market liberalization. It was grounded upon the belief that, under liberal market conditions, enterprises based on social ownership would behave like those that were privately owned  (\href{Estrin}{\citealt{estrin_self_1983}}; \href{Zec}{\citealt{zec_economic_2012}}).
These principles consequently influenced the urban development of Yugoslav cities and contributed to the golden age of the Yugoslav planning profession under the second generation of urban laws adopted during this period.
\\

Urban planning discourse at the time was grounded on a highly comprehensive, integrated, and fully decentralized process closely coupled with the economic and social spheres, with a high level of public participatory programmes concerning physical development (\href{Nedovic}{\citealt{nedovic-budic_mornings_2011}}).

\item \textbf{Dissolution, deficit and tensions (1974-1989);}
\\
Public ownership and the non-property system in the economy resulted in an irrational practice of public spending, budget allocation and investment and the lack of sufficient accumulation and the absence of effective financial discipline. The grave social outcomes were declining labor productivity that made socially owned enterprises inefficient. Self-management practices were basically a participatory instrument in decision-making. Yet, as a side effect, such decentralization entailed a plenitude of documents (mainly social agreements and self-management contracts) that were hard to manage within the economic system. In the aftermath, they brought the emancipatory model of self-management socialism to the verge of collapse (\href{Lydall}{\citealt{lydall_yugoslav_1986}}; \href{Lydall}{\citealt{lydall_yugoslavia_1989}}).
The state of crisis exerted tensions among the republics and further state decentralization, agreed upon by all the interested parties, was welcomed. This brought about a rise in ethnic nationalism and was later perceived as a means to guide the federal political structure into the post-Tito era (\href{Estrin}{\citealt{estrin_yugoslavia:_1991}}; \href{Vujosevic}{\citealt{vujosevic_postsocijalisticka_2010}}).
\\

Even though urban planning organizations profited from the state’s position between the East and the West combining experience and knowledge from both sides,
\footnote{Especially in the domain of environmental protection whose importance was not yet acknowledged by Western countries. In this case, Yugoslav planners relied on Soviet regulations for "ecological zoning especially for capital investment projects and agricultural plots"}
in these times of financial crisis they were also confronted with new economic conditions and the market. The attempts to adjust brought in the third generation of laws with a significant proliferation at the republic level of a hyper-production of urban statutes and regulation (\href{Borovnica}{\citealt{borovnica_osvrt_1980}}; \href{Pajovic}{\citealt{pajovic_pregled_2005}}; \href{Nedovic}{\citealt{nedovic-budic_mornings_2011}}; \href{Peric}{\citealt{peric_evolution_2016}}).
The politics of self-management was reflected in urban planning through the significant decentralization of the planning process, so that plans were subordinated to social contracts and any problems of incompatibility between the plans, the public spending programmes, and the budget caused a halt or deviations in the implementation phase .

\item \textbf{Unfortunate split and warfare (1989-1992);}
\\
After Tito’s death in 1980, the level of political differentiation increased. These final years at first indicate that, most of all, the peculiarities of the Yugoslav economic order led to the decision of high political (party) structures that Yugoslavia abandon this unique socio-political system and move towards a western version of capitalism (\href{Estrin}{\citealt{estrin_yugoslavia:_1991}}).
On the contrary, economic questions were increasingly overshadowed by ethnic tensions. In these circumstances, the extensive and ungrounded decentralization after the Tito’s death pushed the federal government against the re- publican decision-making bodies and brought the federal country to a critical breaking point (\href{Estrin}{\citealt{estrin_yugoslavia:_1991}}, \href{Vujosevic}{\citealt{vujosevic_post-socialist_2012}}).
\end{enumerate}

In sum, the communist institutional and ideological framework had entered into every sphere of professional and private life in SFRY
\footnote{One author pushes this interpretation to the limit explaining that the political was seen as a religion in Yugoslavia, names it the "political religion of Marxism" (\href{Doytchinov}{\citealt{doytchinov_belgrade_2015}}).}
and this was found, above all, in the urban planning systems in the state and cities.

\paragraph{Urban Regulatory Framework and Practice}:

The  practice  of  planning  and  its  implementation  in  SFRY  can  be  circumscribed  as the  development  of  a single  paradigm,  in  its  various  theoretical  and  practical  manifestations  (\href{Vukmirovic}{\citealt{vukmirovic_city_2013}}).
The hierarchical political structure and the supremacy of the communist party dominated urban decision-making.
Even though urban planning expertise was socially accepted and highly valued, all expert suggestions and actions were subjected to the permanent supervision of the state leadership (\href{Piha}{\citealt{piha_osnove_1986}}).
\\

In any case, the SFRY socialist regime put forward significant legislation and practices that made Yugoslav urban planning competitive at a global scale, such as: (1) the establishment of professional agencies, bureaux and institutions at all decision-making levels (national, republic, local);
\footnote{In Belgrade it was JUGINUS 1957, ZUKD 1962, IAUS 1958}
(2) national professional associations were founded to certify and moni- tor these activities (\href{Bakic}{\citealt{bakic_prostorno_1988}}); (3) continual education and knowledge exchange platforms were encouraged and supported by the authorities; and (4) local experts participated in educational and professional programmes in Western Europe and North America (\href{Nedovic}{\citealt{nedovicbudic_waves_2006}}).
Moreover, such a knowledge-based and capacity-building attitude enforced the emergence of the interdisciplinary character of Yugoslav urban planning (\href{Cavric}{\citealt{cavric_perspectives_2000}}).
Consequently, Yugoslav urban planners, urbanists and architects were the pillars of urban development and transformations in a wide range of Asian and African countries, particularly other members of the Non-Aligned Movement (NAM).
\\

In terms of the local context, the urbanity of Yugoslav cities, as a result, were strongly tied to the underlying political organization and socio-economic order of the time. Consequently, the transitions in the matrix of urban actors and stakeholders, pertaining cultural patterns and urban forms, as well as the professional approach were linked to the most influential political and economic shifts and were actually based on the legislative reforms (1945, 1953, 1963, 1974 and post 1989) (\href{Pajovic}{\citealt{pajovic_pregled_2005}}):

\begin{enumerate}
\item \textbf{Post-war reconstruction (1945-1953):}
\\
During the first post-war years, the planning model was in fact a locally-adopted Soviet model, such as the hierarchical control mechanisms legally bounded in 5- year plans (\href{Vujosevic}{\citealt{vujosevic_planning_2006}}).
Therefore, the urban planning system functioned in the top-down tradition and in compliance with the contemporary institutional and ideological frameworks of socialism under Soviet supervision (\href{Dawson}{\citealt{dawson_yugoslavia_1987}}; \href{Papic}{\citealt{papic_z._stanje_1988}}; \href{Peric}{\citealt{peric_evolution_2016}}).
Its purpose was to centrally administer and physically plan economic and urban growth and rational use of resources (\href{Dawson}{\citealt{dawson_yugoslavia_1987}}; \href{Papic}{\citealt{papic_z._stanje_1988}}; \href{Vujosevic}{\citealt{vujosevic_planning_2006}}). 
\\

This approach was the manifesto of the new social organization (\href{Dobrovic}{\citealt{dobrovic_konture_1946}}; \href{Nedovic}{\citealt{nedovicbudic_waves_2006}}).
However, the 1931 Building Act was applied until the the Master Urban Planning Regulation was passed in 1949 (\href{Nedovic}{\citealt{nedovicbudic_waves_2006}}).
This document showed the first traces of influence from western European planning legislations (\href{Peric}{\citealt{peric_evolution_2016}}).

\item \textbf{Institutionalization (1953-1963):}
\\
The combination of western planning models and socio-economic elements of Yugoslav self-managed socialism paved the way for a distancing from the Soviet centralized planning model and toward a participatory system of comprehensive planning
(\href{Nedovic}{\citealt{nedovic-budic_mornings_2011}}).
The rationalist planning model was applied to a set of laws and bylaws that regulated construction on the recently nationalized land (\href{Vesna}{\citealt{cagic_zakoni_2014}}).
This modernist practice of planning marked the beginning of the golden era of spatial and urban planning in SFRY
(\href{Vujosevic}{\citealt{vujosevic_racionalnost_2004}}; \href{Nedovic}{\citealt{nedovicbudic_waves_2006}}).
The system of plans reflected significant improvements toward political decentralization and economic liberalization within the limits of self-managed socialism that started to rise (\href{Peric}{\citealt{peric_evolution_2016}}).

\item \textbf{Further decentralization (1963-1973):}

The decentralizing political and economic measures
\footnote{Strengthening the role of the Federal units and semi-market economic system}
contributed to the introduction of integrated and comprehensive planning in the Yugoslav context. Its main achievements are: (1) an administrative hierarchy and distribution of plans, (2) an interdisciplinary planning practice, (3) increased public participation, and (4) a social approach through the mass provision of affordable housing (\href{Vesna}{\citealt{cagic_zakoni_2014}}; \href{Peric}{ \citealt{peric_evolution_2016}}). This model was introduced as the ‘Basic Policy on Urbanism and Spatial Ordering’ and passed by the State Parliament in 1971 (\href{Nedovic}{\citealt{nedovicbudic_waves_2006}}).
contributed to the introduction of integrated and comprehensive planning in the Yugoslav context. Its main achievements are: (1) an administrative hierarchy and distribution of plans, (2) an interdisciplinary planning practice, (3) increased public participation, and (4) a social approach through the mass provision of affordable housing (\href{Simmie}{\citealt{simmie_self-management_1989}}; \href{Vujosevic}{\citealt{vujosevic_planiranje_2003}}, \href{Peric}{\citealt{peric_evolution_2016}}).

\item \textbf{The prime golden age of Yugoslav planning and its decline (1974-1989):}
\\
At this  time, the cross-acceptance  planning  principle, the  operationalization  of  integrated social,  economic, environmental and spatial aspects in urban policies and the hyperproduction of detailed plans resulted in the high point of Yugoslav planning (\href{Nedovic}{\citealt{nedovic-budic_mornings_2011}})
Its decentralized systems of planning and policy and ”bottom-up” participatory approach were at the core of the planning system, at least nominally. They preceded similar practices in developed western countries for about a decade (\href{Cullingworth}{\citealt{cullingworth_planning_1997}}; \href{Vujosevic} {\citealt{vujosevic_collapse_2010}}).
\\

The 1985 Law on Planning and Spatial organization was the manifesto of a planning practice in which local communities (or communes) were the main planning and implementation authorities (\href{Vujosevic}{\citealt{vujosevic_planning_2006}}). This meant an effort toward the harmonization of interests through: (1) a bottom-up consensus building, (2) inclusion of environmental protection prospects, (3) the distinction within land and building ownership (societal, municipal, national for land; and societal, enterprise, municipal, cooperative, and private for buildings) and (4) the acknowledgment of the variety of urban actors (citizens, workers, professional control, local authorities, socio-political organizations and the civil sector) (\href{Nedovic}{\citealt{nedovic-budic_adjustment_2001}}).
However, by the end of the 1980s, these innovative features of urban planning system and practice were diluted within the hypertrophied and bureaucratized social and political system (\href{Peric}{\citealt{peric_evolution_2016}}). They eventually became cumbersome and dysfunctional and the dissolution of the urban planning system coincided with that of society, politics and the state.
\end{enumerate}

%conclusion
No matter how single-minded the paradigm of socialist urban planning was, the regime actually managed to produce a diversity of planning systems:  from the initial centralized, urban and economic growth-based planning to an integrated, fully decentralized, participatory process and a physical development susceptible to economic and social circumstances (\href{Haussemann}{\citealt{haussermann_socialist_1996}}).

\textbf{"Planning  in  Socialist  Yugoslavia  was  the  dominant  type  of  regulation  and  control  of  modern  society, economy and urban space." (\href{Vukmirovic}{\citealt{vukmirovic_city_2013}}})

\subsubsection{Post-socialism}

As explained, the multi-sectorial crisis brought the self-managed socialist system and SFRY to an end. The dissolution of Yugoslavia coincides with the breakdown of other communist regimes in Eastern Europe. Conflicts and warfare led to the disintegration of Yugoslavia into several sovereign states. Only two of the federal republics, Serbia and Montenegro, maintained the SFRY state legacy and formed the Federal Republic of Yugoslavia (FRY). Later, it was transformed into a confederation - the State Union of Serbia and Montenegro (SCG) until the state referendum in Montenegro in 2006, followed by its separation and independence. During and since this geopolitical restructuring, the term "former Yugoslavia" has been commonly used retrospectively to designate the socialist state of 1945-1992.
\\

These political transformations were broadly present in all ex-communist countries. They were denoted as a policy of openness towards economic, political and cultural influences from the West, and consequently a transition towards markets and democracy (\href{Vujosevic}{\citealt{vujosevic_conundrum_2012}}). In contrast to other East European countries, the years that followed showed significant stagnation and regression in most of the ex-Yugoslav republics. In Serbia, post-communist transition can be distinguished into two phases (\href{Nedovic}{\citealt{nedovic-budic_mornings_2011}}).

\begin{itemize}
\item 1990s with strong re-centralization, a multi-party system, sharp economic decline and international isolation;
\item 2000s with regime change, re-decentralization and progress towards democracy, wild proto-capitalism and unprotected public interest
\end{itemize}       
     
The urban transformation of Serbian cities may be declared as typical of the "transition to market-driven economy and democracy" (\href{Tsenkova}{\citealt{tsenkova_beyond_2006}}) except in certain ways. The dismantling of the socialist system during the late 1980s represented a substantial change in all aspects of the economic model, the political system and social organization. Serbia was confronted with national wars in  neighbouring states and the three-month bombardment by the North Atlantic Treaty Organization (NATO) forces in 1999.
\\

During the 1990s, the European Union flourished economically and culturally with peace, prosperity and order installed across its territory. Nonetheless, Serbia missed the opportunity to join the vibrant and healthy realm of European Union (EU) and was submerged into isolation and wars. 
\footnote{Serbia was under the full UN sanctions from 1992 to 1995.} 
This initial transitional period in Serbia is known as the "blocked transformation" (\href{Lazic}{\citealt{lazic_nation_2009}}; \href{Vujosevic}{\citealt{vujosevic_conundrum_2012}}).
The economy was isolated, grey and semi-martial, stagnating with the disastrous effects of high unemployment, hyper-inflation, pauperization and vanishing production.  Post-socialist economic regulations were both inconsistent and unstable as the institutional business regulations kept changing in response to the overall political situation. The concrete economic conditions of the exchange rate and prices were susceptible to the circumstances of international isolation and the expansion of the informal sector.
\\

There were hardly any large-scale domestic and foreign investments in the local market, industry, businesses and construction (\href{Vujosevic}{\citealt{vujosevic_planning_2006}}).
Socially owned enterprises became private through murky privatizations focused on building individual success through political and business connections rather than professionalism.
Surrounded by wars and participating in them clandestinely, politically powerful actors in Serbia set up a centralized decision making system for all political, economic and cultural topics of high interest (\href{Nedovic}{\citealt{nedovic-budic_mornings_2011}}).
The political scene was dominated by one figure, Slobodan Milosevic. He ruled over the Serbian and partly over the FRY realm through a type of nationalist dictatorship, which was followed by the political unrest of a dissatisfied population and organized opposition supported by the West. This top-down style of government was backed up by the newly established economic elites, a powerful interest group, who supported the authoritarian style of governance (\href{Vujovic}{\citealt{vujovic_belgrades_2007}}).
The situation of halted political decentralization caused institutions to collapse under the centralized authority, while political and economic pressures also deepened societal regressions (\href{Nedovic}{\citealt{nedovic-budic_mornings_2011}}). 
\\

The Serbian social structure crumbled accordingly. The 1991 census documents that Serbia with Kosovo numbered 9,791,475 inhabitants (7,836,728 without Kosovo) (\href{ref}{SFRY Census 1991}), while in 2002. the number of inhabitants in Serbia without Kosovo declined to 7,498,001 and it further plundered to 7,186,862 inhabitants (\href{ref}{Census 2002}).
In addition, the poverty rate skyrocketed- according to 2002 statistics, 20\% of inhabitants lived below the poverty line (\href{Vlada}{\citealt{strategija_vlade_r._srbije_strategija_2003}}).
%{ \citealt{Vladina strategija za smanjenje siromastva CH4 practice-based}}.
\\

The rise in crime, drug trafficking and corruption, the vast number of refugees migrating from war-zones, numerous local soldiers coming back from war with social disorders, the immense emigration of the young and educated population, the growth of an unprofessional, private media all influenced social relations in the cities (\href{Doytchinov}{\citealt{doytchinov_urban_2015}}).
Uncontrolled immigration, counter-productive emigration, and ethnic cleansing made the cities fertile ground for social and ideological hatred. The result was the acculturation and political, ideological and social terror from the backward and paternalistic political actors in conjunction with the nouveau-riche, resulting in the further dissolution of the intrinsic urban quality achievements in the previous periods (\href{Doytchinov}{\citealt{doytchinov_belgrade_2015}}).
The same trend dominated urban planning. ’deplanification’ and ’de-professionalization’ were legitimized through top-down decision making resulting in urban deterioration, degradation, space occupation and illegal construction (\href{Vujovic}{\citealt{vujovic_belgrades_2007}}, \href{Vukmirovic}{\citealt{vukmirovic_city_2013}}).
\\

The dismantlement of the socialist regime in the 1990s radically altered the function of public space by making it a venue for the struggle for human rights, freedom of speech, movement and actions. Expecting to gain more freedom, people fought to replace socialism with a new neo-liberal order. Unfortunately, this new transitional episode, which is still in progress, has brought subtly controlled and carefully restricted freedom
(\href{Cvetinovic}{\citealt{bolay_instrumental_2014}}).
With the end of Milosevic's regime in the year 2000, the country entered a fairly dynamic,
\footnote{With 5\%  yearly growth of GDP during the first years}
but insufficient economic recovery.
This transformation primarily targeted the banking sectors and the privatization of socially owned enterprises. Transitional processes in the early 2000s were designated by marketization, privatization, and deregulation - the instruments of neoliberal capitalism. Proto-capitalist accumulation and the dominance of tycoons, who had been made rich in the 1990s, were the main catalysts of a grand redistribution of social wealth and a re-allocation of assets and income after the destruction of the former economic system (\href{Vujosevic}{\citealt{vujosevic_collapse_2010}}).
The nearly destroyed industrial production, the high rate of unemployment, a debt crisis, and a significant income gap contributed to the unstable socio-economic situation and long-term unsustainable prospects for development (\href{Vujosevic}{\citealt{vujosevic_postsocijalisticka_2010}}; \href{Vujosevic}{\citealt{vujosevic_novi_2012}}).
The growing financial debt and the strong influence of international political, economic and financial actors, In particular, positioned the country rather as a semi-colony more than as a prosperous environment on the development track.  
Serbia has become a kind of new, inner periphery of Europe, as
\href{Göler}{\cite{goeler_south-east_2004}} puts it (\href{Vujosevic}{\citealt{vujosevic_post-socialist_2012}}).
\\ 
    
%political
This colonial attitude also reflects the ideological background of post-socialist transition. Serbia was a poor country with a regressive rather than progressive tendency in development, stuck on the verge of social, political and economic crisis. Over the last 50 years, its geopolitical profile has changed multiple times. In 2003, the Yugoslav Federation (the third Yugoslavia) was replaced by the Confederation of Serbia and Montenegro. With the independence of Montenegro, Serbia also received its new country status in 2006. Kosovo has been under UN protectorate according to Security Council Resolution 1244 since 1999. It declared its independence in 2008. In 2009, Serbia applied for candidacy status for EU, and in 2012 this was approved. 
\\

Being on the European path, but not yet a EU member, places Serbia in a position to reconsider its position within global geopolitics and the economy and accordingly adopt its priorities and strategies for development. Its economic dependence on international capital and a non-critical and non-strategic attitude towards the political, social and cultural spheres of European integration suggest that alternative options were neither framed nor researched nor taken into account at all (\href{Vujosevic}{\citealt{vujosevic_post-socialist_2012}}).
With a cumbersome institutional structure and an incompetent administration inherited from socialism, the principles of governance were not high-end priorities in Serbian political culture  (\href{Trkulja}{\citealt{trkulja_serbian_2012}}).
When the capacity for research, strategic thinking and governance was reduced to zero, the country became a blind follower of global forces and a polygon for power and interest struggles (\href{Vujosevic}{\citealt{vujosevic_conundrum_2012}}).
Thus the majority of reforms were exclusive (\href{Vujosevic}{\citealt{vujosevic_novi_2012}}), revealing extreme asymmetry of power over their creation and implementation in practice (\href{Vujosevic}{\citealt{vujosevic_regionalizam_2015}}).
\\

In the course of history, manipulation, paternalism and clientelism have been carefully and gradually braided features, expressed to their fullest in the "systematic and organized mobilization of interests and bias" of the Serbian political domain today (\href{Vujosevic}{\citealt{vujosevic_post-socialist_2012}}).
In sum, with the lack of broader political and societal dialogue, transitional reforms were imposed by political and economic elites, while the corruptive channels of the same elites have dominated decision-making processes and everyday practices (\href{Vujosevic}{\citealt{vujosevic_conundrum_2012}}, \href{Vujosevic}{\citealt{vujosevic_novi_2012}}).%(\href{ref}{Vujosevic and Maricic 2012, Vujosevic et al 2012}). 
\\

The post-Milosevic reconstruction of Serbian society contains prevailing characteristics of the disintegration of the preceding system rather than a coherent vision of the future (\href{Stanilov}{\citealt{stanilov_post-socialist_2007}}). 
%social
Even with an overall mantra of re-decentralization and democratization, the top-down and centralized approach has been manifested at all levels,  national, regional, and municipal (\href{Vukmirovic}{\citealt{vukmirovic_city_2013}}). CCentralization may be linked to the most powerful regional deindustrialization in Europe, occurring in Serbia with significant territorial disparities among regions, particularly between the inland and the capital, or the "Serbian spatial banana"
\footnote{The term was used by \href{Vujosevic}{\cite{vujosevic_collapse_2010}}
%{Vujosevic 2010 sptrategic} 
to mark the broad metropolitan area of Belgrade and Novi Sad where most of the Serbian population and economic activity are located.}
(\href{Zekovic}{\citealt{zekovic_regionalizacija_2009}}; \href{Vukmirovic}{\citealt{vukmirovic_city_2013}}).
The main victims of transition have been ordinary citizens, disempowered and impoverished during extensive periods of conflict, crises and turmoil.
Difficult living conditions have caused prolonged demographic recession, which started in the 1990s (brain drain, an aging population, refugees) .
In pursuit of a better standard of living, a significant percentage of the population has migrated toward economic centers, in particular the capital (\href{Vukmirovic}{\citealt{vukmirovic_city_2013}}).
\\

%urban actors and urban culture
The social structure in Serbia currently resembles that of a third-world country more than either a European welfare state or its middle-class predecessor SFRY. A tiny layer of wealthy people, a weakened and decimated middle class and a rising poor population are the urban actors in Serbian cities. Parallel to the still present nationalist discourse, the values of mass consumerism and globalization entered Serbian society and cities after 2000. The loss of certain traditional values may not be a problem per se, as new trends also force the affinity for authoritarianism and cultural and ethnic isolation to be exchanged for cultural diversity, active communication, collaboration and participation in urban affairs and decision-making. However, several authors argue that the installation of the values of contemporary neoliberal democracy and globalization contributes to the economic and cultural erosion of the middle class social values inherited from socialism, threatens attitudes of solidarity and empathy and dissolves the sense of overall public good and community bonds  (\href{Cvejic}{\citealt{cvejic_suzivot_2010}};  \href{Vukmirovic}{\citealt{vukmirovic_city_2013}}; \href{Doytchinov}{\citealt{ doytchinov_belgrade_2015}}).

\paragraph{Urban Regulatory Framework and Practice}:

The post-socialist circumstances in urban planning, in the contemporary Serbian context, actually represents the legitimization of newly established societal circumstances and needs. The period started with the end of the Cold war and the global dominance of the market-based economy, and in the Serbian context with the authoritarian political regime under the supreme authority of the president Slobodan Milosevic. This situation created fertile ground for various malfeasance, and urban planning was not exempted from this.
\\

The top-down, but rather fragmented and uncoordinated planning approach that marginalized the position of planners was justified by the lack of funds, ineffective regulations, slow administrative procedures and inadequate tools for implementation
(\href{Peric}{\citealt{peric_evolution_2016}}). 
Namely, urban planning in Serbia was suffocating under a range of contextual difficulties in terms of (1) the disregard for public interest or strategic national development policy, (2) no participation or transparency in planning practice, and (3) the lack of planning expertise and administrative capacity at local and regional levels (\href{Stojkov}{\citealt{stojkov_neue_1998}}; \href{Vujosevic}{\citealt{vujosevic_planiranje_2003}}; \href{Vujosevic}{ \citealt{vujosevic_planning_2006}}).
Planning lost its overall legitimacy and was reduced to a technical practice of land-use distribution exclusively driven by private investments. 
\\

These factors provoked a legal void susceptible to shady deals, a lack of proactive urban governance, and increasing social polarization (\href{Tsenkova}{\citealt{tsenkova_beyond_2006}}). 
%legal
Several legal documents enabled this dubious degradation of urban planning practice and the state of the urban environment in post-socialist Serbia.
First of all, the Law on the Basis of Property [Zakon o osnovama svojinskopravnih odnosa] from 1990 enabled housing to be turned into private property. . It masked the terrible economic and environmental situation happening in front of the citizens by giving them the right to purchase their homes at discount prices. Furthermore, the establishment of 29 regions in 1992 was the act of pseudo-decentralization where the regions were just the territorial sub-branches of the state without any legal and administrative authority of their own (\href{Vujosevic}{\citealt{vujosevic_regionalizam_2015}}).
\\

Finally, the 1995 Law on Planning and Arrangement of Space and Settlements of the Republic of Serbia formalized the centralization of urban decision making (\href{Nedovic}{\citealt{nedovic-budic_mornings_2011}}). However, the document kept the state ownership structure for land, but the state, municipal and private property modes for buildings. Moreover, it also promoted, at least nominally, the rational use of space and transparency through professional control and public review of plans (\href{Nedovic}{\citealt{nedovic-budic_adjustment_2001}}).
\\    
  
%after2000
In the same manner, political elites were also indifferent towards urban policy, leaving it at the mercy of the principles of wild capitalism that dominated Serbian post-2000 discourse (\href{Vujovic}{\citealt{vujovic_belgrades_2007}}).
In fact, the newly installed neoliberal paradigm outdid the institutional capacity to support it (\href{Peric}{\citealt{peric_evolution_2016}}).
The political, in fact, dominated other spheres of urban life. The deficiency of the planning system supported from political elites de-prioritized spatial development and, instead, spatial transformation became a side-effect of political and economic decisions (\href{Stojkov}{\citealt{stojkov_neue_1998}}). 
The lack of policies, instruments and institutional measures resulted in a lack of horizontal, vertical and cross coordination and little increased knowledge and data on the urban, which allowed reinvigorated illegal construction practices and illegitimate real estate transformations (\href{Trkulja}{\citealt{trkulja_serbian_2012}}; \href{Cities}{\citealt{world_bank_cities_2000}}).
\\

Within the regulatory framework, the most disastrous effects are felt on urban land. Urban land is territorial capital neglected during socialism, when there was no competitive land market and restricted private property ownership
\footnote{Until 2003}
and the unresolved issue of property restitution,
\footnote{As a precondition for the joining EU}
which ultimately hindered direct foreign investment (FDI) (\href{Vujosevic}{\citealt{vujosevic_planning_2006}}).  
Consequently, the deficiency of property policy, laws and institutions as well as the lack of substantial urban policy, urban land management approaches and measures and urban land use strategies and rules are the major deficiencies in the current urban planning, regulatory framework and practice (\href{Zekovic}{\citealt{zekovic_spatial_2015}}).
%{Zekovic et al. 2015 spatial regularization)}).
\\

This new generation of urban regulatory framework started with the 2003 Urban Planning Act being further supported by the 2004 Privatization Law and the 2009 Law on Land Conversion aimed at adjusting it to European regulations by re-decentralizing political and administrative power and resolving the land property issue (\href{Cagic}{\citealt{cagic_zakoni_2014}}). However, these policy agendas actually could not break the strong relations between the politicians and local tycoons built during the 1990s  (\href{Peric}{\citealt{peric_evolution_2016}}).
The political discontinuation when the Serbian Progressive Party won government elections in 2012 stopped the practice of the collaboration with tycoons and turned to foreign investors putting to work new measures of distortion onto urban regulatory framework and practice in Serbia (\href{Peric}{ibid.}).
            
\subsection{Urbanity in Serbia through the lens of the Capital}

Having an overview of the historic development of the Serbian state and its urban regulatory framework, it must be noted that it lacks a continuity of good practices and transformative actions. The trend of running down the inventions of the previous period in terms of the planning process and planning practice strongly reduced the developmental potential of Serbian cities (\href{Peric}{\citealt{peric_evolution_2016}}). However, the planning context has kept its culturally moulded coherence, even though different political systems have been at play at the national level.
This rather organic path of urban development led to the classification of post-socialist cities in transitional countries as unregulated capitalist cities (investment-led) with third world urban development elements (substantial illegal activities and informal markets) (\href{Petrovic}{\citealt{petrovic_cities_2009}}).
\\ 

These historical processes of urban transitions affect Serbia’s competitiveness on a global scale by problematizing and reducing its local structural qualities and territorial capital (\href{Vujosevic}{\citealt{vujosevic_conundrum_2012}}). Local experts describe the current state of Serbian society as a whole as "growth without development" or even "developmental schizophrenia"
\footnote{This idiom is often used by \href{Vujosevic}{\cite{vujosevic_postsocijalisticka_2010}} to explain the behaviour of Serbian authorities in transition characterized by empty signifiers, unclear planning, confused strategies and an overall failure to understand what is at play in the local context.} (\href{ref}{\citealt{vujosevic_collapse_2010}}).
%Vujosevic 2003 ili nesto slicno????
%political-economic
Its source may be traced back to the first years of the Serbian state (the Kingdom of Serbia), when the political overruled any other social domain (\href{Section 4.1.1}{Section 4.1.1}). This disposition of factors is still at play. In light of recent party pluralism in Serbia, the multiplication of political actors do not bring democratization of the political sphere and constructive dialogue.   On the contrary, political discord in Serbia is performed as a struggle between enemies, while their objective is to dislodge others in order to occupy their place. In reality, all political parties work together to preserve the dominant hegemony and stagnant power relations (\href{Mouffe}{\citealt{mouffe_which_2002}}).
\\

Instead of showing rigor and determination in finding creative, strategic, locally adapted solutions, the path of development in Serbia is marked by western imports, stale dogmas and over-represented international influence. However unlikely, this attitude also dates back to social and urban practices of the previous periods and regimes. During most times (Ottoman, the Serbian Kingdom, the communist regimes), imposition was the rule of the thumb for urban transformation (\href{Nedovic}{\citealt{nedovicbudic_waves_2006}}). Even though importing, at least on the conceptual level, has had been continuously at play, at certain points in history locally grounded contemporary systems
\footnote{For example, the local administrative commitment to planning in the Kingdom of Serbia; locally adapted and intrinsically unique world trends by distinguished local professionals (1914-1940); and interdisciplinary and participatory approaches in integrated planning process during the socialist era (\href{Nedovic}{ibid.}).}
emerged (\href{Nedovic}{ibid.}).
%urban profession
\\

%social
%urban culture
In this light, the social and urban revitalization in the 2000s from the wars of the 1990s, has been and still is encumbered with an inherited local tradition of nationalism, hierarchy, authority and parochialism coupled with consumerist and neoliberal global forces (\href{Stupar}{\citealt{stupar_recreating_2004}}).
In addition, the recent rise of new-right in Europe also found its counterparts in the Serbian context. The urban quality of Serbian cities still suffers from constant waves of economic migrants from rural areas who have difficulty identifying with a modern capital, just as was the case during the first years of formation of the Serbian capital (\href{ref}{\citealt{doytchinov_planning_2015}}).
A strong nationalistic, spatialized identity (\href{Savic}{\citealt{savic_where_2014}}) also threatened the multicultural and multi-ethnic urban fusion cherished during the socialist period (\href{Stupar}{\citealt{stupar_recreating_2004}}).
\\

These historically bounded socio-spatial patterns embody the actual urbanity of Serbian cities. As already mentioned, Serbia is under the strong influence of social, spatial and, in general, an overall policy of centralization. In this respect, the capital city of Belgrade is a condensed and depicted example of urban transitions present in Serbian cities. As of the imagination of a famous contemporary Serbian author,  \textit{"Belgrade is a mill for producing the urbanite "psychological amalgam" out of the autochthnous peasant Serb from the mountains with the civilization washing over him from the northern plains"} (\href{Velmar}{\citealt{jankovic_pogled_1992}}). Not only a capital or economic center, Belgrade is the model and the microcosm of the nation (\href{Zivkovic}{\citealt{zivkovic_serbian_2011}}).

\textbf{"Za Srbiju moze vaziti jedno opste pravilo: cija vlada toga i drzava, cija vlast toga i sloboda" [in Serbia there is a rule of thumb: Those who form the government, lead the country and rule over freedom] Dubravka Stojanovic, a well known Serbian historian, cited with these words an anonymous MP during the period of the Kingdom of Serbia (\href{Stojanovic}{\citealt{stojanovic_u_2010}})}

\subsubsection{Belgrade - a City in Constant Transition}

The city of Belgrade has had a fixed location in the northwestern part of the Balkan Peninsula and Southeastern Europe for centuries. It was built at the borderline of two large geographical areas, where the Pannonian Plain meets the Balkans. It overlooks the confluence of the Danube and the Sava rivers and spreads along their banks to the north, south, east and west. The first traces of its existence date back to around 5000 BC, from the Vinca culture that developed nearby. The Celtic settlement of Singhidunum, Belograd was mentioned in the Papa Iohannes letters in the 9th century , Alba Graeca, Alba Bulgarica, Nandor Alba, Griechisch Weissenburg, and Castelbianco are the variety of names used for the city, depending on its current ruler, up until the 19th century when the current name of the modern Serbian capital - Belgrade was adopted. 
\\

From the time that the Roman empire split, and even more so during Ottoman rule, the city was the crossroads between the East and the West. Comparably, its contemporary location is the junction of two pan-European transport corridors (Corridor VII from Romania to Germany, and Corridor X from Greece to Austria and Germany). As a node of regional importance, Belgrade also belongs to the category MEGA 4 of the European areas of growth and development.
\\

Its extraordinary location has made Belgrade suffer from continuous invasions and consequent waves of destruction and rebuilding. In the course of its history, Belgrade has been heavily destroyed and rebuilt forty times, bearing in mind that it was the only European capital to be bombed at the end of 20th century (\href{Doytchinov}{\citealt{doytchinov_belgrade_2015}}). Therefore, it is hard to speak about the city of Belgrade in historical terms as there are several dozen consecutive Belgrades fading and reappearing over time (\href{Grozdanic}{\citealt{grozdanic_belgrade_2008}}).
\\

Belgrade has been the historical capital of Serbia since the constitution of the Serbian nation state in the 19th century (1867). Its geographical position has always been close to its national borders, a location easy to reach and occupy (\href{Doytchinov}{\citealt{doytchinov_capital_2015}}).
Nonetheless, it was also the capital of Yugoslavia, throughout that country’s existence, as well as the Kingdom of Serbs, Croats and Slovenes (1918-1947), the Socialist Federal Republic of Yugoslavia (1945- 1992), the Federal Republic of Yugoslavia (1992-2003), and the capital of the State Union of Serbia and Montenegro (2003-2006), before the final split and the re-establishment of the Republic of Serbia (2006-). After the period of the national revolution, abrupt shifts in its status and field of influence
\footnote{From the 19th century onward, Belgrade was at first the capital of a small nation-state, then the capital of a large federation and afterwards the capital of an ever declining territory, finishing finally as a nation-state capital once again (\href{Hirt}{\citealt{hirt_belgrade_2009}})}
complicated its ideological, political and cultural articulation in local terms and towards national and global discourses.
\\

Equally in historical terms, there have been only three periods of peaceful and prosperous urban transitions:

\begin{itemize}
\item when Ottoman garrisons left the city and before the outbreak of WWI (1867-1914)
\item the period between the two World Wars (1918-1941)
\item  the period of Yugoslav self-managed socialism (1941-1991)
\end{itemize}

%3. influence of Serbian society on urbanity + urban culture
\deleted{2. population - urban actors}
%population rise diagram
Even during the periods of conflict, the number of Belgrade’s inhabitants constantly rose. When the Serbian principality gained control over the city, it was rather an Ottoman dorp (kasaba) with 27,000 inhabitants. Before WWI, the population rose to 90,000 (1910.) and reached more than 300,000 at the brink of WWII. During the Yugoslav period the population almost tripled (from 397,911 in 1948. to 1,133,146 in 1991.).
\\

Its metropolitan area has been expanding accordingly, but at a different pace and rate than population growth. Namely, low urbanization has been caused by the agrarian societal basis and, consequently, the poor industrialization during the first years the capital was a nation-state. Low-rise, low quality, often illegally built residential areas were at the core of Belgrade’s urban expansion at those times. Belgrade has been densifying, expanding and settling down the layers of spiritual and material heritage, with various human factors directing its urban transitions.
\\

Urban culture and city form are an expression of social and spatial continuity as well as of destruction and discontinuity (\href{Grozdanic}{\citealt{grozdanic_belgrade_2008}}). In the case of Belgrade, the Roman fortress and initial street grid are still recognizable in the central area of the city. The foundation of Belgrade’s landmark Kalemegdan Fortress dates back to the Roman period, but it also maintains traces of a medieval town within the walls during the Serbian medieval state. However, in this research the historical development of Belgrade is described within an extended discourse of urbanity through:

\begin{enumerate}
\item Ottoman Belgrade (1521-1804)
\item The capital of the kingdom state (1804-1941)
\item The socialist city (1945-1991)
\item Post-communist urban path (after 1991)
\end{enumerate}

\paragraph{Ottoman Belgrade}

During the Ottoman rule of the Balkan peninsula, Belgrade paid a toll for its borderline position during the ceaseless Austro-Ottoman wars. The city frequently passed from Ottoman to Habsburg rule, yet it kept the overall flavour of an Ottoman border town with its urban structure of an Oriental city (\href{Hirt}{\citealt{hirt_belgrade_2009}}).
According to statistics, the city population did not surpass 25,000 during this period.
\\

Even though the Habsburgs occupied Belgrade several times (1688-1690, 1717-1739, 1789- 1791), the city received some of the basic features of European urban patterns only during their longest rule between 1718 and 1739 (\href{Doytchinov}{\citealt{doytchinov_belgrade_2015}}).
Strict Austro-Hungarian administration also imposed urban regulations in terms of the equal plot sizes, street patterns and building shapes with necessary facilities and numerous ornaments of the then-popular European architectural styles (\href{Kadijevic}{\citealt{kadijevic_jedan_2007}}; \href{Mladjenovic}{\citealt{mladjenovic_novija_2010}}).
The Habsburgs also introduced focal public spaces into the settlement structure with important civic and community buildings built around it (\href{ref}{ibid.}).
\\

The city was continuously rebuilt under Islamic principles, with a central street [carsija] and organic street network, mosques [dzamija], and market places [bazar]) every time it passed back under Ottoman jurisdiction.
Yet Brush’s map from 1789 reveals great mixture of both the Ottoman and Habsburg urban matrix - an overlapping of narrow, curved streets with several parts with the straight street grid and the Great Oriental Market located in the central square of the city (\href{Doytchinov}{\citealt{doytchinov_modernization_2015}}).
At the end of the 18th and the beginning of the 19th century, most sources confirm that the city was a rather a small, ruined and neglected Ottoman fortification with civilian neighbourhoods suffocating under an uneven, congested and unhygienic urban structure, while the rare, sporadic Christian neighbourhoods were scattered around and immersed in the suburban landscape (\href{ref}{ibid.}).
Yet in the end, the city that the Ottomans left behind after the symbolic "city key delivery" in 1867 was actually a hybrid Oriental-Occidental city (\href{Blagojevic}{\citealt{blagojevic_urban_2009}}).

\paragraph{Belgrade (1804-1940)}

The first record of Belgrade’s population at this time is from 1838. There were 8,483 Christians, 2,700 Muslims, 1,500 Jews and 250 foreigners, in total – 2,963 people. This record dates back to the times between the enactment of the Turkish Law (Hatisherif) in 1830, which institutionalized Belgrade as the seat of both the Serbian and Turkish administrations, and the official establishment of Belgrade as the capital of the Ottoman vassal state of Serbia (1841).
\footnote{In 1841. Prince Mihailo Obrenovic moved the capital from the city of Kragujevac to Belgrade.(ref "History (Important Years Through City History)". Official website.}
In the time span before the city evolved into the capital, the spatial concept and the building construction patterns still followed vernacular Ottoman traditions (\href{Doytchinov}{\citealt{doytchinov_modernization_2015}}).
\\

When Serbs gained limited authority over the city (rather a town at that time), they addressed their construction efforts more toward European-like architectural design than toward planning. The town was divided into three parts: (1) the town encircled by the Moat with a predominantly Muslim population and minorities (Jews), (2) the Fortress with the Turkish garrison and village-suburbs outside the Moat populated by Christians (\href{Blagojevic}{\citealt{blagojevic_urban_2009}}).
The Turkish Plan of Belgrade made in 1863 actually legitimizes the religion-based population division (\href{Roter}{\citealt{doytchinov_modernization_2015}}).
\\

The growing interference by the Serbian side in city administration and management resulted in these individual, chiefly civic buildings, newly built and decorated in various historical styles, which stood for a strengthening Serbian national identity and its obvious yearning to join the broader context of European civilizations (\href{Hirt}{\citealt{hirt_belgrade_2009}}).
In this respect, the yawning gap between the Oriental and Serbian (more European-like)
\footnote{The very first urban planning efforts were oriented towards the outer city development of Christian/Serbian neighbourhoods, for example on the river port and outside the Moat. (\href{Blagojevic}{\citealt{blagojevic_urban_2009}})}
part of the town, in terms of urban matrix, structure, and even urban culture, was obvious even before Belgrade became the capital of the Serbian principality.
\\

Therefore, even before the official takeover of the city from Ottomans, the new Serbian capital had asserted itself as the supreme national administrative, economic and cultural center.  The period of Serbian state construction was also the period of shaping the European identity of its capital city and the gradual application of European planning ideals therein.
\footnote{Introduction of squares and plazas in the urban matrix, building ornamental fountains and placing sculptures glorifying national heroes on horseback were among the indices of European urban design trends to date (\href{Hirt}{\citealt{hirt_belgrade_2009}}).}
The year of the official departure of the Ottoman administration and military from Belgrade (1867) coincided with the date of the very first attempt at a General Urban plan of the city.
\\

The necessity to have an urban plan that regulated the development of the capital city originated from Prince Mihailo Obrenovic’s vision of putting Serbia in the cultural and social league of other Central and West European countries. A detailed geodetic survey of Belgrade’s soil was undertaken by Emilian Josimovic himself in 1864. A mathematician by education and an engineer by profession, the plan he prepared in 1867 was labeled as the “Regulatory” Very technical in its approach and with significant practically oriented data, this plan corresponded to what today is a combination of a general and master plan, a plan of an implementation or feasibility study.
\footnote{The plan contained many numeric data and comparative tables; proposals  for  implementation with procedures and responsible institutions, dynamics of works, principles for calculations of value of land and cost of regularisation (\href{Blagojevic}{\citealt{blagojevic_urban_2009}})}
Josimovic’s plan envisioned re-unification of the two parts (the free Serbian city and the Ottoman fortress) clearly separated within the previous Ottoman plan from 1963 (see above). The plan can be considered to be in line with the essentials of the European planning paradigm of the period.
\footnote{Application of the ring zone}
\\

An eminent local architectural theorist and urban historian (\href{Blagojevic}{\citealt{blagojevic_urban_2009}}), however, states that with the proposition of an idiosyncratic urban structure, Josimovic's model goes beyond simply reproducing European models.
He insisted on a network of open and free green areas (parks and town wreaths) created for the sole purpose of circulation, recreation and leisure to all its citizens. (\href{Blagojevic}{\citealt{blagojevic_urban_2009}}) argues that by emphasizing these new public and social spaces, Josimovic was actually praising the liberated 19th century Serbia. In opposition, it is generally accepted within Serbian urban planning discourse that Josimovic’s plan rid the city of its former identity.
\\

In fact,  Josimovic pioneered a long-lasting paradigm of de-Ottomanization (or de-Orientalization) and Europeanization that was first imposed upon Belgrade, then on other cities in Serbia and finally on the society as a whole. Establishing an official policy of destroying the Ottoman urban legacy and traditional urban structures originating from the 16th century, the General Urban Plan of 1867 stripped Belgrade of segments of its collective memory and left it susceptible to "tabula rasa" approaches when solving urban conflicts - this may be the harshest criticism made by its opponents  (\href{Doytchinov}{\citealt{doytchinov_modernization_2015}}).
However, a group of authors argue that Josimovic instituted a paradigm of uncompromising radicalism which has nurtured generations of urban professionals in Serbia (\href{ref}{ibid.}).
\\

In  practice,  the General  Urban  Plan  of  1867  was  coupled  with  the  Law  on  Regulation  of the  Town  of  Belgrade.    However,  both  the  plan  and  the  law  were  rejected  in  Parliament under politically biased circumstances.
\footnote{Land and property owners at those times were influential enough to lobby against the law as they were reluctant to accept any changes that would threaten their premises.}
The parliamentary decision was also a relief for Belgrade Municipality authorities and its administration as they technically lacked the skills to accomplish the demanding reconstructions proposed by the plan. It appears that the multiple urban stakeholders of the time hailed the prolongation of the status quo, so that any further urban regulations did not have an appropriate legal basis (\href{Doytchinov}{\citealt{doytchinov_modernization_2015}}). It was therefore no surprise that the town still greatly expanded beyond control, but mostly in terms of low quality illegal construction in the suburban areas (\href{ref}{ibid.}).
\\

In 1896, the Belgrade Building Law was adopted (with amendments in 1898 and 1901) and from 1897 onward, the Building Code for the Town of Belgrade 69 regulated all issues relating to construction in various town zones.
By the end of the century, two more urban plans were proposed - Stevan Zaric’s(1878) and Jovan Beslic’s (1893) plans - and the Construction law for the city of Belgrade was adopted in 1896. These urban plans tended to build on Joksimovic’s plan and to continue with his trends toward European modernization.  However, with the gradual abandonment of the core innovation of Joksimovic’s plan - the idiosyncratic idea for a network of urban parks and town wreaths -	these plans showed more of a copying of western models than a fine tuning of the traditional and modern within the boundaries of the newly established country and its rising capital  (\href{Blagojevic}{\citealt{blagojevic_urban_2009}}). In fact, the adopted law (1896) was not at all modernizing, but served to legitimize the systems and interests at play.  This was a populist compromise to leave the state of affairs as it was, with no sanctions for not abiding with the law and where the corrupted judiciary failed to apply the laws in practice (\href{Stojanovic}{\citealt{stojanovic_kontroverze_2015}}). 
\\

The regime change
\footnote{In 1903. after the military coop when both king Aleksandar Obrenovic and queen Draga Masin were executed, the Karadjordjevic dynasty was set on the Serbian throne.}
and the extended period of peace (more or less from 1817  to the Balkan wars that began in 1812)  gave rise to social prosperity, a cultural upswing and a stabilized institutional framework at the beginning of the 20th century.   Consequently, this period brought a multiplicity and heterogeneity of urban forms, the renovation of public spaces and large-scale projects.
\\

Coupled with military success in the Balkan wars, the overall circumstances established the very  last  years  before  WWI  as the  peak  of  Serbian  social  and  cultural  revival  of  the  time. Instated  financial   mechanisms,   formalized  bureaucratic  procedures,   trained  administration  and  improved  public  service  facilities,  brought  success  to  several  projects  taken  up during  these  years  (1905-1912).
Finally,  the  Master  Plan  of  Belgrade of 1912  was  prepared  by  a  young  Parisian  engineer  Alban  Chambon  in  the  typical  manner of  the  European  academic  tradition  of  the  19th  century.   The  plan  was  a  symbol  of  the rising social potential and manifested the majestic ambition of the ruling class to be part  of  Europe  and  Europe  only   
(\href{Blagojevic}{\citealt{blagojevic_urban_2009}}; \href{Doytchinov}{\citealt{doytchinov_modernization_2015}}).
Notwithstanding the authorities, local experts asked for preservation of the inherited urban pattern and local expertise. They opposed "haussmannization" of the town and the demolition of the city's heritage (\href{ref}{ibid.}).
\\

The pre-war Master Plan of 1914 settled these tensions by locally framing the tradition of the Monumental City design 
\footnote{Monumental City design refers to orthogonal street system, distribution of urban parks, multiple long diagonal vistas and spectacular public plazas at the intersections (\href{Hirt}{\citealt{hirt_belgrade_2009}})}
(\href{Perovic}{\citealt{perovic_iskustva_2008}}).
This planning trend was influential during the first years of the new, larger state - The Kingdom of Serbs, Croats and Slovenes. The General Plan of Belgrade of 1923 was the first urban plan to be officially adopted. It was of the same nature as the Master Plan of 1914, but even more megalomaniac and with an egocentric attitude boosted by the significance of the new larger state.  The expansion of the state also made the city expand territoriality toward the north to include Zemun, which had a largely Slavic population that earlier had been ruled by the Hapsburgs. The plan targeted radical, after-war reconstruction with a dense urban fabric and medium-scale residential and mixed-use buildings. However, the plan was prepared by a team of exclusively foreign architects. This fact may explain why the proposed interventions actually negated Belgrade’s topography and its urban character  (\href{Grozdanic}{\citealt{grozdanic_belgrade_2008}}; \href{Blagojevic}{\citealt{blagojevic_urban_2009}}).
The General Regulation Plan of 1939 reflected the state affairs before WWII.
\\

\textbf{Summary on this period}
In the course of the 19th century, Belgrade paved its way as a national capital that faced Europe from the remnants of an Oriental border- town. However, the swift transformations and abrupt changes of its urban system were rather supplemented by sluggish adaptations, imported innovations and generally the maintenance of un-institutionalized practices that included a variety of nepotistic relations within decision-making structures. The urban development of Belgrade as the capital of the first Serbian nation state therefore seemed like a bouillon of doings and not-doings in the city that nevertheless eventually produced results. For example (\href{Dubravka}{\citealt{stojanovic_kontroverze_2015}}):

\begin{itemize}
\item urban transformations - sluggish adaptations:
\\
\begin{itemize}
\item Partial decision making: hyper-production of solutions with  a total lack of strategy and systematic approach resulted in ungrounded and nonfunctional urban projects and consequently in the doubt and reluctance to complete them.
\item 
\end{itemize}

\item urban change - imported innovation:
\begin{itemize}
\item Apotheosis of the western models;
\item Intervention initiatives left to private individuals or funds; 
\item Economic and political actors and interests braced together;
\item Disregard for the opinion of local experts;
\item Importing grandiose, inadequate, self-glorifying ideas; %discourse of smallness
\end{itemize}

\item urban maintenance - the maintenance of stagnant and backward practices
\\
\begin{itemize}
\item In the lack of political will for implementation, decision-makers consistently stick to "temporary solutions" while the costs of implementation and adjustments rise.
\item Incompatibility of political ideology and the economic model at play - the politics of urban growth was thwarted by the inefficient economic model (no budgetary allocation for the capital city,
the tax system was not adapted to urban environment prerogatives,
\footnote{The tax system did not stimulate construction.}
no incentives for urbanization and construction).
\item Translation of the tradition of Ottoman nepotism into the party state: decision making in the multi-party nation state reduced to the party level.
\item Party interest held supremacy over any other interest - strong liaison between the political and the urban in the partocratic state.
\item Disregard for the public interest - the policy of obstruction and destruction as a party campaign.
\item Reproduction and expansion of corruption mechanisms - the culture of populist measures for the party's sake. 
\end{itemize}
\end{itemize}

In general, the major question of the time was the relationship between the city and the policy, the authorities and the professionals.

\textbf{"It happened that in the capital the opposition often held the power, obstructed by state authorities and the breakdown of the decision making system. When the party in power changed, it adopted new standards and was forced to dismantle all the structures established by the previous regime" (\href{Dubravka}{\citealt{stojanovic_kontroverze_2015}})}

\paragraph{Yugoslav Capital (1945-1990)}

In SFR Yugoslavia, Belgrade became an important multinational and multifunctional metropolis. During these 50 years, the image of a dorp between the East and the West was transformed into a modernist city with a cultural scene spreading its tentacles both toward the East and the West. Belgrade was and is the symbol of Yugoslav self-managed socialism. The system reflected itself in the social layer and in the spatial structures of its capital city.
\\

Consequently, all political and planning decisions could be traced also within transitions to the urban system happening in the Belgrade of those times.

\begin{enumerate}
\item \textbf{Urban planning in the service of the regime  (1945-1953)}
\\
Initial post-war goals were simple and straight-forward: to rebuild the war-damaged urban fabric. As with the Soviet model, the first post-war local plans were to strictly follow the orders provided in the 5-year national economic plans. 
\\

The Design of the General plan for the city of Belgrade in 1948 was the first post- war urban planning document prepared by Nikola Dobrovic, the director of the Serbian Urban planning institute. The plan primarily dealt with the transport system in order to propose a new transportation network more suitable for the expected population growth. The design of the plan was preceded by several transportation studies (for all types of transport). The  design that was presented seemed radical and unrealistic and resulted not only in its rejection, but in the chief architect Professor Nikola Dobrovic being removed from his position as director of the Urban Planning Institute of Serbia and transferred to the Faculty of Architecture.
%effective, comparing to the legacy of bombardment in 1999 - conclusions

\item \textbf{Professionalization of planning (1953-1963)}
\\
With this reconstruction task accomplished across a decade, the multidisciplinary teams (comprising planners, architects, engineers etc.) set to work on the construction of massive industrial complexes in order to cater to the exploding population growth (\href{Hirt}{\citealt{hirt_belgrade_2009}}). The result was the construction of New Belgrade, an urgent, gigantic mass-housing estate, built according to CIAM principles and in compliance with the Athens charter..
\\
It may also be said that this phase actually started with The General Urban Plan of Belgrade of 1950 by Milos Somborski. The plan endorsed the urbanization of New Belgrade (Novi Beograd). However, the plan dealt not only with the expansion of the city to the left bank of the Sava river and the projects for Novi Beograd, but also proposed the reconstruction of the central zone of Belgrade (\href{Grozdanic}{\citealt{grozdanic_belgrade_2008}}).

\item \textbf{A comprehensive, integrated planning process at work (1963-1974)}
\\
The ideas and principles of comprehensive and integrated urban planning found its actualization in The General Urban Plan of Belgrade adopted in 1972. The authors, Aleksandar Djordjevic and Milutin Glavicki, by honoring the values of the past, called for historic preservation and architectural contextualisation of Belgrade’s urban fabric. Moreover, they advocated for more rational use of land and the integration of central urban functions by making the final decision on the Ada Ciganlija zone and reserving it exclusively for leisure and recreation use. They also took into account that some relatively large industries were located in attractive parts of the city (e.g., in Novi Beograd) as a result of the socialist policy of prioritizing industry over other land uses and thereupon proposed more equitable distribution of new infrastructural projects and related facilities and better transport connections between urban areas (\href{Hirt}{\citealt{hirt_belgrade_2009}}).

\item \textbf{The pioneers of urban planning decentralization(1974-1989)}
\\
Yugoslavia’s continuing political decentralization and democratization in the 1970s was mainly evident through local level decision-making, successful participatory initiatives and the multiplication of projects and their implementation. In Belgrade’s urban fabric from the 1970s, this was reflected in the break with the severe principles of Modernism and a timid introduction of new building styles (\href{ref}{ibid.}).
\\
The Modifications  and  Supplements/Annexes  to  the  General  Urban  Plan  of  Belgrade  up  to 2000 were adopted in 1985. The author was Konstantin Kostic.
The Plan did not differ much from its predecessor, but its purpose was to propose and implement more realistic solutions (\href{Grozdanic}{\citealt{grozdanic_belgrade_2008}}).
\end{enumerate}
 
\textbf{Yugoslav urban discourse}

Urban transitions during the self-managed socialist era were in fact interventions in the urban fabric of Belgrade that clearly broke with pre- WWII spatial and builing patterns (\href{Hirt}{\citealt{hirt_belgrade_2009}}). The superior quality of architectural design and progressive trends in urban planning caused those districts in Belgrade built during social- ism to be globally recognized and attributed to Yugoslav socio-spatial discourse. To a certain extent, it may be stated that the modernist building style for public buildings and large housing estates were adapted to local conditions in Yugoslavia and Belgrade.
\\

It may also be argued that the essence of Belgrade’s contemporary urbanity is based on the management of conflicts and resources and the production of urban practices during the SFRY period. The city had undergone constant transformations while the corresponding system of planning was evolving - from the initial phase of selective borrowing, then through a system of transformations by internal adjustments, and finally to the synthetic innovation represented in its own model of integrated, participatory planning (\href{Nedovic}{\citealt{nedovicbudic_waves_2006}}).

\paragraph{Post-socialist Belgrade}

The post-socialist period influenced Belgrade just as it did the rest of the country, if not even more intensely. Urban Belgrade suffered a certain decline from the post-socialist transition, a sharp one at first and a questionable recovery with several periods of prosperity later on. This periodization goes along with that indicated in the country as a whole:

\begin{itemize}
\item The capital of the 3rd Yugoslavia in post-communist circumstances (1990s)
\item The post-socialist capital in transition to markets and democracy (2000s)
\end{itemize}

\subparagraph{The isolated metropolis of the 1990s}

The political and social circumstance of the Yugoslav ”break-up” changed the climate in the capital of the crumbling state and of the new, shrunken Yugoslavia-to-be. Under the umbrella of protecting the sovereignty, territorial integrity, and functional unity of the truncated state, the country was re-centralized. As a consequence, the constitutional role and the authority of the local, regional and city levels were weakened and reduced to a minimum (\href{Vujosevic}{\citealt{vujosevic_regionalizam_2015}}). 
\\

In times of raging civil wars in the region, the Serbian and ex-Yugoslav capital was caught in a stagnant and even backward position. The major population increase occurred during the 1990s when refugees of Serbian origin came to Serbia from war-affected regions in Slovenia, Croatia, and Bosnia and Herzegovina. Most settled in Belgrade because, in the re-centralized country in a state of economic crisis, only the capital appeared to offer much possibility for those evicted from their homes and deprived of their possessions to start a new life.
The antithesis of the fascinating developments of communist Belgrade was its ideological degradation based on regional militarization, nationalization, ruralization, pauperization and, in general, a state of corruption, crime and chaos (\href{Samardzic}{\citealt{doytchinov_belgrade_2015}}). 
\\

In times when the previous political ideology were falling apart, the necessity to produce an appealing "credo" for the imploding populous had retrograde social effects. In the 1990s, the ideology of self-managed socialism in the social realm was replaced by earlier forms of social relations, namely by traditional models, mythicizations of the nation-state and exceptionalist discourses of heroism and smallness,
\footnote{There is an indication that the specific position of Serbia on the border between the East and the West has created a unique local identity as a crucial feature of geopolitical exceptionality (\href{Savic}{\citealt{savic_where_2014}}).}
(\href{Doytchinov}{\citealt{doytchinov_belgrade_2015}}). 
Surprisingly, the actors and regulatory frameworks involved in both ideological concepts were the same.
The middle class, already deprived of its economic assets and the acquainted cultural matrix, became confused and apathetic.
The influx of refugees migrating from war-zones paired with extensive brain-drain also complicated the social structure of the city (\href{Doytchinov}{\citealt{doytchinov_urban_2015}}).
\\

Excluded from the map of global cities during this nationalist regime, Belgrade experienced (1) a growth in illegal construction, (2) the naked vandalism of overbuilding and inappropriate occupation of public space, and (3) the flourishing of informal business practices, the product of a crony economy at high levels (\href{Norris}{\citealt{norris_belgrade_2008}}).
At the city level, the results included a rise in crime, drug trafficking, and corruption, as well as an overall state of moral decay in local communities in general  (\href{Prodanovic}{\citealt{prodanovic_stariji_2004}}).
\\

During a decade of continuous crises, the physical structures in the city began to deteriorate.
The refugee crisis, the lack of any official construction projects and the explosion in the number of illegally built dwellings
\footnote{A vast number of illegal buildings in Belgrade were homes for the upper classes and Milosevic's elites constructed and ornamented in rather lavish and kitschy styles that best represent the state of values and qualities of these "nouveau riche" profiteers of transition (\href{Hirt}{\citealt{hirt_belgrade_2009}})}
signified an extensive state of shock for the city.
However, the NATO bombing in 1999 was the true peak of the crisis. The bombs brought real war over the rooftops of Belgrade and collateral civilian damage in the city. A number of buildings still bear signs of the damage, the most significant one is the Building of the Yugoslav Ministry of the Defence in the central urban area of Belgrade. 
\footnote{The building was designed to symbolize the decisive WWII battle when partisans defeated Hitler’s forces in the canyon of Sutjeska. The form of the two buildings represent the canyon itself. The designer was the famous Serbian architect Nikola Dobrovic.}
\\

Taking all this into account, the picture of urban actors and urban culture of the 1990s in Belgrade was a gloomy, poor, and silenced image of stagnation. While the planning profession suffered from a major crisis of legitimacy at the state level, the local planning regime in Belgrade was also in a state of collapse. The institutional framework and planning practice in Belgrade had rather a symbolic and superficial role, or even worked as a means at hand of the politicians (\href{Vujosevic}{\citealt{vujosevic_planning_2006}}). As such, the system of urban transitions, that the city was undergoing, either happened spontaneously or they were directed from outside the corresponding regulatory framework.
%pro-liberalne snage; naopako shvacen americki model, negativna primena iskustva – prokazivanje planiranja, prepustanje stihiji trzista

\subsubsection{Belgrade Now}
%1. conclusions from the previous historical part in intro in 2000s
Belgrade stepped into the 21st century in a state of prolonged emergency. The city went through a state of war under the NATO bombs in 1999. A state of emergency almost occurred again in October 2000 when the opposition took over power after the citizen revolt and mass demonstrations. And then once more in 2003 the official state of emergency was lifted after the assassination of the Serbian prime minister Zoran Djindjic. The years 2003, 2006 and 2008
\footnote{2003 was the year when Yugoslavia was officially replaced by the state union of Serbia and Montenegro. In 2006 Montenegro became an independent state. In 2008 Kosovo proclaimed independence}
also figure as points of discontinuation in Belgrade’s history as the state capital. To be brutally frank, this meant that Belgrade lost its significant Europe-wide role as a metropolis where the West and the East meet and the iron curtain is broken  (\href{Grozdanic}{\citealt{grozdanic_belgrade_2008}}).
\\

%the role of the capital
Respectively, the city’s role as a capital of an unstable geopolitical realm has placed the city time and again in a position to rule over minorities of different nationalities and lifestyles in the more remote areas of the country (the 1st, 2nd and 3rd Yugoslavias). If for no other reason, these circumstances have strongly endangered the acceptance of Belgrade as the symbol of the state. In addition, interests have clashed in reaching a broad societal accord for the privileges it, as a capital, deserves.
\\

In general, such circumstances suggest ethnic diversity and multicultural urban fusion. Unfortunately, Belgrade  has dominated these state  regions with a negligent imposition and pressurized assimilation of an overall poor rural environment, under the obsession of nationalist agendas, and in forms that have fostered ethnic misapprehensions (\href{Samardzic}{\citealt{doytchinov_belgrade_2015}}).
Therefore, over the course of its different states, Belgrade has always been perceived as predominately the centre of Serbian-dom (\href{Savic}{\citealt{savic_where_2014}}; \href{Heppner}{\citealt{doytchinov_capital_2015}}).
In apparent retaliation, Belgrade has often been  governed in such a way that its own ruling class has jeopardized its development through a current of alienated, estranged decisions (\href{Doytchinov}{\citealt{doytchinov_belgrade_2015}}).
\\

Its vulnerable geo-strategic location, turbulent history and unsettling societal framework have made the city struggle to establish its modern identity and accordingly to defend the landscape and public interest of all its inhabitants. As a consequence, Belgrade has always been a combination of the rural in physical and organizational terms and its strong tendencies to rise to cultural and ethnic cosmopolitanism. The respect for the urban memory of the city is still present through the centrality of its diversified cultural and historical matrix, from the center (Knez Mihajlova street) through New Belgrade to Zemun. Its unique and strongly bonded unity of historic and architectural heritage and its vibrant civic life lead to having Belgrade voted as Southern European City of the Future in 2006-2007  (\href{Hirt}{\citealt{hirt_belgrade_2009}}).
On the other hand, international real-estate market forces, which were introduced in Belgrade with the liberalization and democratization following the political changes in the year 2000, press for de-industrialization and the aesthetics of globalization in the capital city  (\href{Grozdanic}{\citealt{grozdanic_belgrade_2008}}).
\\

%urban form - conflicts and resources
Regardless of  its relative de-industrialization, Belgrade  still  employs the largest amount ofthe country’s industrial labor-force (20\%)  (\href{Hirt}{\citealt{hirt_belgrade_2009}}).
Adding to this, the ever increasing commercialization of the urban fabric has been underway since 1989. While in the 1990s small and local retails were dominant chiefly around the city center, in the 2000s malls and hyper-markets sponsored by a combination of Western and Serbian capital began to thrive in the greenfield areas at the fringes of the central urban zone (New Belgrade) (\href{ibid.}{ibid.}).
As a sign of the cultural and real estate revival of the new political regime (2000-2004), these commercial zones as well as several interesting locations in the historical cores of Belgrade and Zemun were topics of numerous architectural competitions 
\footnote{Architectural competitions were open for: Dorcol marina (2007), multifunctional business center 'Usce", the historical core of Zemun, sport complex Tasmajdan, numerous central squares, pedestrian streets, new office blocks, affordable housing etc.} %pronaci godine za konkurse}
(\href{Stupar}{\citealt{stupar_recreating_2004}}).
The goal was to put all available creative and expert forces to work on the renewal of urban culture to improve living conditions in the city and preserve the oases of nature in its urban fabric (riverbanks, parks within city blocks, green areas, urban forestry) (\href{Grozdanic}{\citealt{grozdanic_belgrade_2008}}).
However, the issue of the inefficient transportation system and traffic congestion has been systematically neglected. In the rush to build more and quicker, the new structures revealed a subtle eclecticism of styles and scales, sometimes even tricking the regulatory rules, or were built illegally or informally. These uncontrolled and impulsive actions and faulty procedures have resulted in (1) urban expansion (over-consumption of agricultural land), (2) rampant sprawl and (3) the loss of public space. Coupled with the unresolved transition of property from the socialist period, Belgrade has been slowly growing into a spatially inefficient city on the road to suburbanization (\href{Zekovic}{\citealt{zekovic_spatial_2015}}).
%sprawl
\\

%population & urban actors
%population rise diagram
In 2002, Belgrade covered 3.6 percent of Serbian territory and 17.3\% of Serbian population lived in its metropolitan area (\href{Cities}{\citealt{world_bank_cities_2000}}).
The Belgrade metropolitan area also accommodates the highest share of the highly-educated population in Serbia (13.76\%) (\href{Vukmirovic}{\citealt{vukmirovic_city_2013}}).
According to the 2011 census, the city has a population of 1,166,763.
The population of the metropolitan area
\footnote{the administrative area of the City of Belgrade}                 
stands at 1,659,440 people and accounts for 17 municipalities.
Ten  of them are classified as "urban" and seven are "suburban" municipalities, whose centres are smaller towns.
In fact, the  city  is additionally burdened by joining the predominantly rural  settlements  and  conglomerates  of Barajevo, Grocka,  Lazarevac,  Mladenovac,  Sopot,  Surcin  and Obrenovac (\href{Doytchinov}{\citealt{doytchinov_belgrade_2015}}).
These changes follow  the  devastating legacy of the 1990s that threatened the urban culture generally and eroded the civic order and the value systems enstated during the communist era. Notable socio-spatial stratification with the formation of very expensive districts (the historical core and several traditionally wealthy neighborhoods) and very poor ones (near the large industrial estates and in the farther city outskirts) enlarged social  divisions  and  collisions  of  interests  and  life styles (\href{Hirt}{\citealt{hirt_belgrade_2009}}).
\\

%urban culture - conflicts & resources
With  an  at  least  nominal  revitalization  of  democratic  diversity  and  a  rising  influx  of global  trends,  Belgrade  has once  again  become  a  place  of  striking  extremes  and  contrasts. The spontaneous  revival  of  civic,  cultural,  and artistic  activities  in  Belgrade  happened  with  the support and interest of international organizations and similar initiatives. Moreover,  the  presence  of  a market-oriented  value  system  and  capital  prompted  the  lively lifestyle of the Serbian capital to transform into a famous and infamous European destination for those seeking a mischievous, casual and exciting nightlife (\href{Doytchinov}{\citealt{doytchinov_urban_2015}}).
However, the rise of civilian values have been continuously  threatened by  non-urban  tendencies,  with disordered,  chaotic and violent practices occurring as a result  of  weak,  biased  or  completely absent institutions and regulatory frameworks. The current urban identity of Belgrade is a combination of the city’s position and politics, of urban culture and traditional values, and it is full of gaps and deficiencies that determine its urban future.
\\

%urban development
Tracing the urban development of Belgrade reveals challenges and traumas both from its long history of deconstruction and reconstruction with explicit repercussions in the very recent events and the mentality of its urbanites (urban actors). The unrealistic perceptions of scarce and unorganized elites is such that the clash of Eastern and Western lifestyles and urban patterns, which are happening in Belgrade as the largest city along their border, have hindered the regularization, institutionalization and articulation of urban forms and practices (\href{Samardzic}{\citealt{doytchinov_belgrade_2015}}).
Evidently, Belgrade is still on the European periphery and has a marginal role within the European urban network (\href{Vujovic}{\citealt{vujovic_belgrades_2007}}). 
As a matter of fact, \textit{"more  than  a  "global  city",  Belgrade  is  a  [post-rural]  conglomerate characterized by the visual, emotional, ideological and material traumas of wars, poverty, lack of efficient institutions and rule of law, a micro-culture of  individual  irresponsibility  and  incompetent  development  solutions."} (\href{Doytchinov}{\citealt{doytchinov_belgrade_2015}}) 
\\

\paragraph{Urban Regulatory Framework and Practice}

Accordingly, although the case of Belgrade presented a high degree of urban planning strategies and their practical implementation during the previous socialist regime, urban planning was continuously hindered by political instability, convergent socio-economic forces and inconsistent planning systems during the transitional period of the 1990s and the early 21st century. The example of the Serbian capital testifies to the fact that the country still finds itself in a post-socialist proto-democracy without the developed institutions of a representative democracy, civil society and market economy (\href{Vujosevic}{\citealt{vujosevic_collapse_2010}}):
 
\begin{itemize}
\item Urban planning has not been a priority (\href{Sykora}{\citealt{sykora_transitional_1999}}) and planning documentation has already been turned into symbolic documents (\href{Nedovic}{\citealt{nedovic-budic_adjustment_2001}});

\item •	Urban transformations mainly concerned land use and changes in property ownership overwhelmed with powerful economic actors who take advantage of the undefined environment in order to protect and promote their own activities and extend their property ownership;
 
\item The topology of powerless urban actors (ordinary citizens and the civic sector) is such that they have almost no prospects for meaningful social participation and nobody is defending their rights therein (\href{Vujovic}{\citealt{vujovic_belgrades_2007}}).
%(\cite{Vujovic et al. 2007});

\item Fragmented spatial development dominated by informality and confused eclecticism shows the characteristics of urban design bricolage rather than the purposeful stratification of socialist and post-socialist layers upon the urban fabric (\href{Hirt}{\citealt{hirt_belgrade_2009}}). 
\end{itemize}

These circumstances imply that decision making in urban terms is performed through negotiations between investors and local governments, where local authorities and the civic sector possess rights but not the means for exerting their power and acting as equals in the negotiation process  (\href{Bajec}{\citealt{bajec_rational_2009}}).
In addition, public interest in local authority services results more from the political party hierarchical structures, interest and programmes than from legal binds and legitimate goals (\href{Djokic}{\citealt{djokic_political_2007}}).
%connect with intermediary roles
Uncontrolled, chaotic and often illegitimate development paths eventually lead to the deconstruction of urbanity (\href{Vujosevic}{\citealt{vujovic_belgrades_2007}})
%(\href{ref}{Vujovic et al. 2007}).
In such an environment, the path dependency in the tradition of urban planning in Belgrade is further halted and hinders the effective management of local urban issues (\href{Nedovic}{\citealt{nedovic-budic_adjustment_2001}}).
Complex institutional legacies influence the behaviour of all urban actors, an imposition of the neoliberal model restricts the flexibility of social patterns and networks and a prevailing chaos of traditions and models, rights and interests reduce the capacity to examine alternative modes for urban development in transition, apart from replications of the western paradigm and corresponding approaches  
(\href{Tsenkova}{\citealt{tsenkova_urban_2007}}; \href{Petrovic}{\citealt{petrovic_cities_2009}}).
   
\textbf{Alberto Moravia: Belgrade is a rare city at the confluence of two big rivers which also represent a synthesis of several world metropolis.}

\subsection{Savamala}

Savamala is a neighbourhood of Belgrade's central zone. It is situated on the southern bank of the Sava River in the old part of Belgrade (\href{Figure 2}{Figure 2}).
According to the 2002 Census, the neighbourhood had 4,500 inhabitants at that time (\href{ref}{2002 Census}).
The neighbourhood of Savamala is rather a place on the mental map of Belgrade and an important landmark of the city, than an official administrative unit  (\href{Figure Savamala boarders}{Figure X}). Its name means "Sava neighbourhood", and it is derived from the Turkish word for neighbourhood "mahala", combined with the name of the river whose bank it is situated on.
\\

The first official mention of Savamala was around 200 years ago after the resolution of city authorities to spread urban structures to the river in order to set forward its urban development. But the World Wars, authoritarian rule and the current economic crisis have left their marks. Savamala is now a traffic bottleneck with intense pollution and urban noise. For decades its existing spatial conflicts and socially disadvantaged population have been neglected by both the authorities and professionals (\href{Urban}{\citealt{kamenzid_urban_2013}}).
Before the spin-off of cultural organizations, activities, and conversions of old neglected houses to trendy cafés and restaurants in the neighbourhood, Savamala had a reputation as a home to outcasts, prostitution and criminality.
\\
During all these years Savamala has been a venue with a plausible collision (traditional/modern; past/present) rich in tradition, history and heritage.
Its physical layout can be described as:

\begin{enumerate}
\item an appealing location almost in a geometrical centre of the physical layout of the city of Belgrade now,
\item an attractive but deteriorating neighbourhood with irrevocable potential for renovations and refurbishments,
\item an area within the walking distance from the city centre but still aloof from its ever-growing hustle and bustle.
\end{enumerate} 

In a nutshell, this neighbourhood is a scaled example of the pre-socialist material legacy, socialist cultural and societal matric, a transitional reality and a condensed case of its multi-faceted circumstances of post-socialist urban development (\href{Chart Savamala Space-time}{Chart X}):

\begin{itemize}
\item The pre-socialist past marks its presence in Savamala through architectural and cultural heritage (\href{Figure 3}{Figure 3});
\item The cultural and behavioural patterns from the Yugoslavian socialist regime;
\item Post socialist backtracking;
\item Transition prospects in terms of its potential to become an attractive urban area for investments according to the recently established economic constellation.
\end{itemize}

All these circumstances bring to light that Savamala has kept its shape over time, but that various social conditions have influenced its development. Namely, four crucial political periods have left their marks on Savamala: (1) pre-socialist, (2) socialist, (3) post-socialist and (4) transitional. All cultural and architectural heritage dates back to the pre- socialist period when Savamala was promoted as a major trade and artisanal area and a communication hub with the bus and train station in its proximity. Noise and pollution have been caused by its role as a passageway for heavy transit introduced during socialism. Therefore, we could summarize its life-cycle as follows:

\begin{itemize}
\item Pre-socialist period: an amorphous urban form of the neighbourhood, a recognizable cultural  and  architectural  identity;

\item Socialist period: disintegration of tradition and heritage, middle-class society and marginalized groups living in the area;

\item Post-socialist period: lack of data on social structure, deteriorating industrial area and abandoned buildings, and leasehold of empty plots to private investors without transparent bidding procedures;

\item Transitional period: market-led economy, dominance of private ownership, vivid night life, creative cluster and limited citizen participation governed by non-governmental sector, and, last but not least, the start of a huge redevelopment project initiated by a foreign investor.
\end{itemize}

However, the symbol of Savamala’s early existence are the underground passages and dungeons. These passages are part of the pre-Ottoman history of Belgrade, but they have served several other purposes over the course of the years, for example as wine cellars or large refrigerators. During World War II, Jewish families used them to hide or escape from the Germans.
\\
\textbf{Ottomans}
\\
Ottoman Belgrade had as many as 25,000 inhabitants before the Serbian authorities gradually took over the rule of the city. It was populated by Ottoman garrisons, Turks, Greeks, Vlachs, and Jews. After the 1804 uprising, the Serbian minority settled down around the town gate and the church at Kosancicev Venac (\href{Doytchinov}{\citealt{doytchinov_modernization_2015}}).
%(\cite{Roter Blagojevic in Doytchinov 2015}).
Savamala terrain was a marshy land usually flooded in spring and known as the Venice Pond. It was mostly uninhabited or temporarily settled by gypsies. During the Habsburg rule in the early 18th century, the Austrian authorities built the first neighbourhood on Savamala land, the New Lower Town, but it was destroyed during the Ottoman conquest in 1737. The only permanent settlement emerged in the early 19th century along Karadjordjeva Street, around the Little Market and on the slopes above the flooded area. It was inhabited mostly by Christians.
\\
\textbf{The Serbian State}
\\
After the successful political deal that followed the Second Serbian Uprising (1815), Prince Milos was granted the Savamala land by the Grand Vizier of Belgrade. As he wanted to keep himself away from Ottoman eyes, he positioned himself in Topcider and allowed for the Savamala territory to be freely settled. This new suburb developed quickly above the marshy terrain, but was initially inhabited by gypsies and the poor. Prince Milos had a strategy to raise an economically powerful Serbian Belgrade around the Turkish settlement (\href{Krusche}{\citealt{krusche_bureau_2015}}).
The port in Savamala and its docks were the main trade point and the only connection between Belgrade with the city of Zemun and the European neighbors (\href{Doytchinov}{\citealt{doytchinov_modernization_2015}}).
The prince planned to develop a new mercantile district on the Sava slopes around the port. 
\\

Karadjordjeva street was the first road built along the Sava by the engineer Zujovic in 1828. However, Savamalska (Gavrila Principa) and Abadzijska (Kraljice Natalije) were the first straight, wide traffic corridors in Belgrade. All Serbian merchants and craftsmen were intended to be settled in modern, new houses along these streets (\href{Doytchinov}{ibid.}).
%(\cite{Roter Blagojevic in Doytchinov 2015}).
Consequently the Savamala poor were evicted to Palilula. 
In realty this new city commercial center was begotten by the act of the Prince to forcibly move tailors (abadzije) here.
In 1842 the Serbian capital was moved from Kragujevac to Belgrade (\href{ref}{ibid.}).
Therefore these streets represente Prince Milos’s urban visions for the future capital of the future Serbian state (\href{ref}{ibid.}).
They were also part of the Janke's urban plan for the development of the outer city in the 1830s  (\href{Blagojevic}{\citealt{blagojevic_urban_2009}}). 
%Anastas Jovanovic view on Savamala drawing - in Nestorovic 2006.
\\

As merchants started settling in Savamala with their families and their commercial premises, the Little Market commercial area was the heart of Savamala. Little Market stood approximately where the roundabout in front of the Belgrade Cooperative and Hotel Bristol is now. The construction of the Belgrade Cooperative was financed by the famous merchant and humanist Luka Celovic. It was build in 1907 in the academic style by the architects Andra Stevanovic and Nikola Nestorovic. It was the seat of the Belgrade financial institution for traders, craftsmen and clerks. After WWII the building became the seat of the Geodetic authority. The building is a prime example of Belgrade classical architecture, but it underwent serious degradation during the 1990s.  The Bristol Hotel was build in 1911. It was also financed by Luka Celovic with the support of the brothers Krsmanovici and designed by Nikola Nestorovic. Placed next to the older and significantly smaller Hotel Bosna, the Bristol became the best hotel in the city and kept this title for a long period of time. Other examples of secession architecture were the home of Luke Celovic-Trebinjac (1903) and the house of Vuca (1908).
\\

The house of the Greek Manojlo Manak was built in 1830 and is today a great and rare example of the Balkan style of architecture in the area during the 19th century. It is located at Gavrila Principa 5. However, the very first official building to be built was the old customs office (Djumurkana). It was erected in 1835 and designed in European classical style by Hadzi Nikola Zivkovic under the guide of Franz Janke (\href{Blagojevic}{\citealt{blagojevic_urban_2009}}). It stood on the land where KC Grad is now. Djumurkana served various purposes, as the first theatre show was also performed there in 1841. In 1937, other modern buildings appeared: the Prince's Palace,
\footnote{Milos’s Hamam is the only  part of his mansion that is left in Savamala. However it is now outside what we consider Savamala. It is set on Admirala Geprata street and parts of it can be seen within the interior
of the restaurant Monument.}
the Country Court and the Great Breweries. Further development  of  the  neighbourhood  was  enabled when a drainage programme  was  put  in place  in  1867.  The  Sava  riverbank  was  linked  to the city center by the "Big Stairs", ordered and invested in by Prince Mihajlo, Prince Milos’s successor. Consequently, Savamala  extended along Sarajevo Street (now Gavrila Principa Street) in the 1870s  (\href{Krusche}{\citealt{krusche_bureau_2015}}).
The railway station was built in 1884 and designed by Viennese architect von Flattich. At that time, it was understood to be part of Savamala, which now might not be the case, according to our surveys (\href{Questionnaire Experts Savamala}{Questionnaire X};\href{Questionnaire Experts Savamala}{Questionnaire Y};\href{Questionnaire Experts Savamala}{Questionnaire Z})
%(\href{rQuestionnaire}{Questionnaire X, Y, Z}).
As a result, at the end of 19th century Savamala was established as the financial and trade center of the kingdom  and the merchants who settled there wanted to build proper mansions. It is also important to note that cafes were always an important part of the identity of Savamala  (\href{Nusic}{\citealt{nusic_kafane_2013}}). A living memory from this period is the Bosiljcic candy shop on Gavrila Principa Street that has kept its interior style and business model from those times.
%map and add it to the MAS-ANT map
\\
After WWI, the most important construction venture was building the bridge over the Sava in order to connect Belgrade with Zemun (1931-34). The bridge was also the symbol of bonding with the newly extended territory of the state and it brought a new lifeblood to Savamala. Namely, the neighbourhood experienced a construction boom during this period. Unfortunately, the German bombardment in 1941 and even more the American air raids in 1944 razed the Savamala area to the ground.
\\
\textbf{SFRY - socialist regime}:
\\
Immediately  after  WWII,  the Savamala  docks  had an  important  role  for  naval  transport.  Therefore, dock workers became the first post-war citizens of Savamala.  They had their own hostel, restaurant,  cinema,  and  even  a  monument.  However,  in  1961  a  new  Belgrade  port was built on the Danube and the social and material structure of the neighbourhood changed. The ships, warehouses and the workers vanished.  Savamala became even less than a regular  Belgrade  neighbourhood;  during   socialist  rule  Savamala  was  disregarded  as  a  legacy of the capitalist era.
The seat of the Yugoslav River Shipping company was in the area. The Tito Shipyard, where most of the leading freight river boats were designed and constructed, was also located nearby on the Sava river (\href{Markovic}{\citealt{markovic_wide_2013}}). 
\\

The significant functional change came with the construction of city’s central bus terminal next to the railway station. For lack of a ring road for heavy traffic to bypass the city and for the needs of the traffic hubs located in Savamala, the area was turned into a transit roadway  surrounded  by  corresponding building stock (warehouses and manufacturing houses). Despite  these  signs  of  an  old  and  dirty urban  neighbourhood,  Savamala  still  kept  a  part  of  its  residential  and  commerce  nature. A symbol of the old-fashioned socialist approach to services and catering is the old bakery Crvena Zvezda that still stands in Karadjordjeva street.
\\
\textbf{Post-socialism - 1990s}:
\\
The 1990s left its mark on Savamala as well. The physical structures in the neighbourhood kept deteriorating during these years. As an example, instead of the busy docks, the Savamala riverbank became a ship graveyard. A few famous ships from WWI and WWII were located here until recently when they were secretly removed for the purposes of the Belgrade Waterfront Project  (BWP).
\\

The steam freighter Zupa had been a Hungarian naval ship in WWI and it was again used in WWII by the Soviet troops. Zupa was under state protection as a cultural heritage of extraordinary value. The SIP concrete ship was a cargo ship from WWII. It was used as a home for refugees until it half-submerged in the water. There are no records of how it arrived in Savamala. Krajina was King Alexander I Karadjordjevic’s naval residence, and later used by Tito until his death in 1980. Because of its outstanding design, the ship was used for movie setting purposes until it caught fire and was heavily destroyed on one such occasion.
\\

On the other hand, the social structure of Savamala was marked by stagnation, war migrations and the grey economy. Apart from other landmarks of Savamala, the proximity of the bus terminal determined the social and business model of the neighbourhood at the time. The emergence of small service and trade businesses for lower classes who used the public city and regional transportation and the grey economy meeting points were the major activity generators in the area.
\\
\textbf{Transition - after 2000}
\\
After the major political shift in the year 2000, the attractive location of this neighbourhood was put at risk to become a playground of interest for corrupted public authorities and powerful private developers working together under the guise of urban development and economic prosperity.  Despite a change in ownership, Savamala was saved for a time from this newest development trend, mostly because of its long-term decay that made it a complicated case for the limited investments with short-term turnovers that dominated in Serbia. In this respect, during the first transitional years only cafes, restaurants, clubs and several galleries chose to profit from the central location of the neighbourhood and its low-rent properties and they invested in setting up their small scale businesses there.   However,  the situation has recently changed as powerful international investors found a counterpart in Serbian authorities on various levels to jointly use their economic and political dominance to gain control over a highly profitable waterfront area of the capital city (\href{Zekovic}{\citealt{zekovic_spatial_2015}}; \href{Zekovic}{\citealt{zekovic_megaprojects_2016}}).
%(\cite{Zekovic et al., 2016 megaprojects}).
\\

Savamala is a typical East-European neighbourhood caught in post-communist processes of economic and political change in Balkan transitional countries.
In these circumstances, such cityscapes cannot resist copying urban models from the West, but meet extraordinary difficulties in doing so, because these cities lack the institutional and cultural infrastructure essential for the functional unity present in western cities (\href{Petrovic}{\citealt{petrovic_cities_2009}}).
Savamala is a unique area in Serbia with such a plausible collision between traditional and modern and past and present, rich in tradition, history and heritage. Therefore, Savamala, with its even more intensive top-down and bottom-up pressures, is a representative testing environment.

\textbf{"As a result, the area's old inhabitants were summarily evicted to Palilula  and their fragile homes flattened overnight by the Prince's men" (\href{Krusche}{\citealt{krusche_bureau_2015}}).}

\subsubsection{Contextual circumstances of Savamala's Urbanity}

When addressing the urbanity of Serbian cities, mainly its capital city and particularly Savamala, the goal is to affirm the deposition of temporal layers and to establish continuity in the process of urban system transitions (\href{Grozdanic}{\citealt{grozdanic_belgrade_2008}}).
%(\cite{(Grozdanic 2008)}).
In the course of history, it has been conspicuous that in this context the state has not hitherto managed to solidify the main pillars of a coherent pipeline of socio-economic progress and an adequate legal, institutional and educational framework in order to ensure the stability and sustainability of the whole system. Even during periods when the standard of living increased and the national economy seemed partly revived, the socio-political system showed mere traces of surface decentralization and democratization. In general, Serbian society remained heavily dependent on international relations, worldwide economic circumstances and regional political movements, usually only a passive recipient of what is happening on the global scale.
\\

The major characteristics of chaotic patterns of urban development in Serbia are:
\begin{itemize}
\item a multitude of actors,
\item various economic and political interests,
\item fragmented spatial transformations.
\end{itemize}

Therefore, the socio-spatial patterns of urban system transitions at the neighbourhood level are still only the product of the variety of top-down interventions from the regulatory framework, the exertion of local power poles and the interests and influences from international and foreign bodies. In order for these actions to harmonize the consequent urban system transitions, they should (a) be embedded in a particular social context, (b) be reactive to the shifts in local socio-economic, political and cultural settings and (c) be attentive to bottom-up tendencies and needs. Namely, the urbanity category referred to herein is the congregation of the past that has led to the particular state of tensions between urban life and the urban structure,
while the superposition of the current state of urban decision-making has brought about the prospects for the evolution of the system. 

\section{Stimulants and deterrents of the urban decision-making tradition at the neighbourhood level}

As was described in the previous section, Serbian society as a whole is undergoing a period of radical and swift shifts in terms of:  (a) the political system, (b) the economic order,
(c)	the social model, (d) ideological postulates, and (e) cultural patterns  (\href{Petrovic}{\citealt{petrovic_cities_2009}}).
Belgrade, its capital city, is thereafter a representative environment of the diversity and reciprocity in the nature of on-going transformations: political, economic, social and spatial. While some trends and directions within these transformations are clear and defined, uncertainty dominates decision-making and implementation in the turbulent environment of the post-socialist urban system transition (\href{Nedovic}{\citealt{nedovic-budic_adjustment_2001}}).
\\

In the course of these events, the Savamala neighbourhood has become a condensed example of overlappings, collisions and linkages of different layers of decision making.
In Savamala, the dynamic adjustments of the socio-spatial patterns depend on
(a) the relations between top-down impositions of urban frameworks, (b) national and supra national regulatory mechanisms,
(c) flows of international capital,
(d) local real estate arrangements,
(e) global cultural trends, and
(f) bottom-up civic arrangements.
\\

The purpose of the Savamala case study in this research project is to establish a context for the description and mapping of socio-spatial patterns and the corresponding urban system transitions. The post-socialist path of dependency and transitional prospects limited to the neighbourhood level is the chosen context for testing a new, MAS-ANT methodological approach (\href{Yin}{\citealt{yin_case_2009}}).
Therefore, Savamala in its emphasized and vulgarized complexity and dynamics is a viable choice for accurately describing and illustrating:
\begin{itemize}
\item top-down urban frameworks;
\item interest based real-estate transformations:
\item bottom-up participatory activities.
\end{itemize}

Thus, based on the conceptual framework of this research project and¬ empirical data from the context, the process of urban decision-making is broken down into these three levels of strategizing and operationalizations for directing urban system transitions. 

\subsection{Top-down management of urban issues}

Spatial planning is an interdisciplinary field and an activity that aspires to manage systematizations, implementations, monitoring and evaluations of a growing number of urban issues (\href{Fisher}{\citealt{fisher_building_2001}}). Its urban regulatory framework operationalizes the procedures of setting up strategies and action plans for development that achieve common viewpoints, goals and priorities within a bounded territory, an urban environment (i.e. a city) in the case of urban planning.
Urban planning is an essential part of the public domain of contemporary cities, it is deeply embedded in its concrete societal context. The societal context is a territorially bounded system with its unique identity formed from its social relations: the political (regime, bureaucracy, governance), the economic (market), and the cultural  (trends, values)  (\href{Vujosevic}{\citealt{vujosevic_planning_2006}}).
%(\cite{Vujosevic and Nedovic Budic 2006}).
Having said that an urban planning system reflects the identity of its immediate surroundings (\href{Stojkov}{\citealt{stojkov_prostorno_2012}}),
all the historically condensed negative effects and anomalies of post-socialist and transitional urbanity must be taken in consideration within the urban regulatory framework.
\\

A detailed analysis of the urban regulatory framework at work in Savamala requires addressing the scale of its influence and the nature of its agency. Based on the conceptual framework of this research and the empirical data, the agents of the urban regulatory framework are generally divided into (1) urban records and (2) urban institutions and they are all active on roughly four different levels: the international, national-state, regional-city, local- municipality. Urban records refer to either (1a) legal documents, or (1b) policy agendas or (1c) technical documentation, while urban institutions are either (2a) public administration or (2b) urban planning authorities. The hierarchical structure of the urban planning framework in Serbia determines the management of urban issues in a top-down manner first in Belgrade and then in Savamala. The structuralization of the collected data follow the scale supremacy within the hierarchic agency at work in Savamala, starting from the international level of urban records and finishing with the city level of urban planning authorities.
\\

As Serbia is not yet a part of the EU, the binding urban records that may address urban system transitions in a neighbourhood in Belgrade (Savamala) are reduced to the legal framework of EU candidacy negotiations
\footnote{In 2007 Serbia initiated a a Stabilisation and Association Agreement (SAA) with the European Union. Serbia officially applied for EU membership in 2009 and in 2011 it became an official candidate. In 2012 Serbia received full candidacy.}
and they are specifically focused on the policy agendas of European regional development.

\textbf{International Policy Agendas}
\\
As a part of European political space, Serbia-Belgrade-Savamala pertains to the regional policy of European spatial development. Belgrade is positioned at the crossing point of European corridor VII (Danube) and corridor X (international highroads E75 and E70). Its geostrategic location induces regional networking on three levels (\href{Stuper}{\citealt{stupar_aleksandra_recreating_2004}}):
%add in ANT diagram on the side from international recomendations

\begin{enumerate}
\item European:

\begin{itemize}
\item ESPON 2020 (European Spatial Planning Observatory Network) is a programme to support the reinforcement of EU Cohesion Policy in terms of national and regional development strategies;
\item Directive INSPIRE (2007) is a EU initiative to support sustainable development and infrastructure for spatial and geographical information; 
\end{itemize}

\item Macro-regional:

\begin{itemize}
\item South-East Europe (SEE) program for transnational spatial planning cooperation among the Adriatic and South-East European region countries (17 countries).
\item Danube 21 programme to fight criminal networks in the Balkan region.
\end{itemize}

\item Metropolitan:

\begin{itemize}
\item INTERREG IIIb network of cities for transnational cooperation in Europe (30 cities 13 countries).
\item EU Strategy for the Danubian Region.
\end{itemize}
\end{enumerate}

The Spatial Plan of the Republic of Serbia 2010-2020 acknowledges the Danube 21 programme and the continuation of ex-CADSES countries programmes (a regional spatial planning cooperation in the smaller territorial scope of South-East Europe (SEE)). The INTERREG IIIb network is included in the Regional spatial plan of the Belgrade administrative area (2004). Directive INSPIRE is incorporated in the Planning and Construction Act (PCA) in 2000s All these international documents are present in the Serbian urban regulatory framework in the form of recommendations and they abide by the local laws. 

\textbf{National Legal framework}
\\
According to the Constitution of the Republic of Serbia (2006), all spatial and building issues are incorporated in the Planning and Construction Act 2003 (revised in 2006, 2009, 2011, 2014). The positioning of a neighbourhood level at the national scale is precised with the Local Governance Act.
Finally, the battlefield of different interests in Savamala is under the jurisdiction of the laws that address urban land and the change of property for public urban land.

\begin{itemize}
\item The Local Governance Act 2002 (revised in 2007);

The Local Governance Act (2002) represented new democratic and decentralization trends after the regime change in 2000. The local authorities are of two levels: municipal and city authorities. Municipal authorities are: the municipal assembly, the president of the municipality, the municipal council and the municipal administration. The Act also aims at empowering cities in decision-making that concerns their territorial realms. In this respect, the first edition of the act (2002) instates the role of the city mayor, the city council and the city assembly. Moreover, the introduction of the City Manager, responsible for local economic development, and the City Architect, in charge of territorial development, in the city administration puts these role holders at the forefront at the local level (\href{Vujovic}{\citealt{vujovic_belgrades_2007}}).
%(\cite{Vujovic and Petrovic 2007}). 
\\

The major change in the revision of 2007 is the specification of the City and Municipality’s status according to the pertaining population. This decision deprived former cities of their status if their overall urban population is less than 100,000 inhabitants.
\footnote{Except in special cases defined by the law.}
\\
The issue of Belgrade is left to be regulated with the Capital City Act.
However, the Local Governance Act is not precise about the distribution of power in Belgrade - between the center and the districts and the city authorities and the city municipalities.

\item The Planning and Construction Act of the Republic of Serbia (PCA RS) 2003 (revised in 2006, 2009, 2011, 2014, 2016);

The 2003 Planning and Construction Act was pronounced a modern law, which had been drawn on contemporary international experiences. In practice, the contemporaneity of the act is mostly in its compliance with common terminology applied in EU legal urban documents (sustainability and sustainable development for example) (\href{Zakon}{PCA 2003}).
%ovo nemoj menjati (a sto ne stavis autora isto kao ime zakona?
Moreover, it was rather focused on technical disciplines and an engineering approach (\href{Nedovic}{\citealt{nedovicbudic_waves_2006}}) modeled after the French urban legal framework.
Possibly because of the influence of the French urban tradition, the Act was mainly drafted after the 1931 Planning and Construction Act. This reliance on an outdated law was harshly criticized as an old-fashioned recessive measure that neglects the demand for coordinated up- to-date development strategies and the importance of harmonization with EU norms.
%system transitions connection with 1931 Act
\\

The burning issue of the 2003 Act was putting together the issues that dealt with space and construction together. In fact, the 2003 Planning and Construction Act incorporated three previous laws: the Act on Planning and Arrangement of Space, the Act on Construction and the Act on Construction Land.
This Act also institutionalized the strategies for spatial and urban development on various levels. The part on space and planning regulates the organization of spatial and urban plans in an hierarchical order as technical urban planning documents. Conversely, the practical purpose of the part on construction was to fight thriving illegal construction. One of the issues was the introduction of a legalization procedure as well as setting a better suited procedure for building permits. These norms aimed to solve the problem of urban informality. The goal was to improve living conditions mostly in Belgrade, where there were around 140,000 illegally constructed structures at that time. The radical novelty was also the denationalization of urban land and the return to private land ownership. Therefore, the act distinguished three types of land and building ownership: national, local and private (\href{Vujosevic}{\citealt{vujosevic_planning_2006}}).
% (\cite{Vujosevic and Nedovic Budic 2006}).
\\
The overall goal was to set a legal framework in order to boost slow socio-economic transformation and the low rate of foreign investments (\href{Vujovic}{\citealt{vujovic_belgrades_2007}}).
The act has undergone significant revisions in the course of the past 14 years [revisions - in 2006, 2009, 2011 and 2015].
\href{ref}{\cite{nedovic-budic_serbian_2004}} suggested that the practical dysfunctionality of the act was caused by inaccurate assessments of the local condition as well as by ungrounded expectations of post-socialist transition at play locally.
\\
However, the latest revision of 2014 is of particular importance for the Savamala neighbourhood case study. Several revisions introduced in this act address the Spatial Plan for Special Purposes (SPSP). These revisions hide opportunities for exclusive and elite-driven urban interventions at play now in Savamala through a waterfront redevelopment project. 

\item The Act on Property Restitution and Conversion 2014;

\item The Act on the conversion of urban land leasehold rights into urban land property rights 2014;

\item Lex specialis 2015 (The Act on expropriation for private elite-housing and commercial purposes for the Belgrade Waterfront Project);
\end{itemize}

\textbf{Policy Agendas - National and City levels}

\begin{itemize}
\item National Strategy for Resolving the Problems of Refugees and Internally Displaced People 2002;

The harsh legacy of the wars from the 1990s was the number of Serbian refugees (mostly from Croatia and Bosnia and Herzegovina) from the war-affected Balkan regions. Most refugees settled down in cities, and moreover in the capital. After the 2000 regime change, this strategy aimed at solving the refugee problem. In doing so, the strategy was also a pioneer in dealing with the status of urban land (\href{Hirt}{\citealt{hirt_belgrade_2009}}).

\item The National Strategy of Sustainable Development 2008;

\item Implementation Act for the National Strategy of Sustainable Development 2009-2017;

\item The National Strategy of Spatial Development of the Republic of Serbia 2009-2013-2020 was never formally adopted;

\item The Strategy of Regional Development of the Republic of Serbia 2007-2012;

\item City of Belgrade Development Strategy 2008;

The strategic approach to territorial capital was also introduced with the transition processes after 2000.
The purpose of the strategic document for Belgrade is improving the status of the city at the European level
\footnote{Belgrade is now classified as a MEGA 4 level of Mega-City-Regions in Europe}
(\href{Nedovic}{\citealt{nedovic-budic_adjustment_2001}}).
The strategy was designed in the form of recommendations to strengthen local identity of the city. The main domain of strategic interest were: (a) decentralized local governance, (b) competitiveness and mobilization of capital, (c) citizen participation and (d)tourism and cultural activities  (\href{Vukmirovic}{\citealt{doytchinov_belgrade:_2015}}).
Following the liberal trends after 2000, the actions considered private property as a basic land ownership type. What is more, in the manner of decentralization, the key authorities are not the city but the municipal authorities. The strategy also recognized the role of land developers and investors as stakeholders in a transitional, post-socialist environment.
\end{itemize}

\textbf{Technical documentation - National, City and Municipality levels}
\\
Spatial and Urban plans are the core technical documents prescribed by several laws and bylaws to regulate territorial development in Serbia.
These documents set goals and strategies around environmental sustainability, economic competitiveness, territorial decentralization, social cohesion and strengthening cultural identity (\href{Hirt}{\citealt{hirt_belgrade_2009}}).
\\
Spatial plans cover distribution of territorial resources in the space of various, usually larger scales. Spatial plans deal with urban and agricultural land, as well as environmentally protected natural areas. They comprise the Spatial plan of the Republic of Serbia and various regional plans,
\footnote{Autonomous provinces are assumed to be territorial  entities  at  the  NUTS2  and NUTS3  levels,  and the Belgrade administrative  area (\href{Vujosevic}{\citealt{vujosevic_planning_2006}}). Nomenclature of Units for Territorial Statistics (NUTS) is a standard for referencing the subdivisions of countries for statistical purposes. The standard is developed and regulated by the European Union, and thus only covers the member states of the EU in detail.}
municipal plans,\footnote{Municipal  spatial plans are at the NUTS 4 level.}
and spatial plans for special purposes.
Even after significant decentralization efforts within the Serbian legal framework, the regions do not signify as real units of development and planning (\href{Vujosevic}{\citealt{vujosevic_regionalizam_2015}}). Consequently, the corresponding regional plans deal with protection, regulation and development, but they do not hold an official authority over spatial changes in their respective territories.
\\

On the other hand, urban plans consist of General Urban Development Plans, Plans of General Regulations and Plans of Detailed Regulations. They cover respectively smaller territory, incorporate all sorts of innovative, strategic and up-to-date methods, and in general offer detailed solutions for issues already conceptually covered by spatial plans such as land use and building zones, transportation, infrastructure, natural and cultural heritage, green, recreation, protected areas etc. For example, General Urban Development plans control development on a local level, so that they are prepared and adopted locally; but, because they are regarded as strategic documents with a certain influence at the national and/or regional level, the final consent upon their adaptation rests with the Ministry in competence. 
\\

Spatial plans of interest for this research are:

\begin{itemize}
\item The Spatial plan of the Republic 2010-2021;

This spatial plan is less a blueprint for the technical projection of changes, but rather a compendium of binding spatial and sectorial strategies for the autonomous provinces (\href{Stojkov}{\citealt{stojkov_prostorno_2012}}).

\item  The Regional Spatial plan of Belgrade administrative area 2011;

The Regional Spatial plan of Belgrade was adopted in 2004.
It positions the city among other metropolitan cities (MEGA 4 and MEGA 3 in perspective) and capitals in the region.
The Regional Plan addresses natural and urban heritage in the area, zoning and infrastructure. 

\item The "Belgrade waterfront Project" Spatial Plan for Special Purpose Area (BWP SPSP);
\end{itemize}

Concurrent urban plans are:

\begin{itemize}
\item General Plan GP 2003 (revised in 2005, 2007, 2009, 2014), and General Urban Plan GUP 2016;

The  General  Plan  of  Belgrade  2021 was adopted  by the Belgrade  City Assembly in 2003.
The plan was made by  the  Urban  Planning  Institute  of  Belgrade  in  2003 (\href{ref}{Sluzbeni glasnik Beograda 27/03}). The team leaders were Vladimir Macura PhD and Miodrag Ferencak MSc. The four actualizations and the new document adopted in 2016 reflect: (a) the turbulent and inconsistent trajectory of the urban regulatory framework in the Republic of Serbia and the city of Belgrade, (b) the changing nature of the legal urban framework, and (c) the swinging needs and projections of the structures in power 
(\href{Vukmirovic}{\citealt{doytchinov_belgrade:_2015}}).
%add in ANT analysis
The basis for the plan was the integration of Belgrade into the global network of socially-culturally-economically viable cities and the creation of a recognizable Belgrade identity.
\\

Following global and moreover European planning trends, this plan is only a  strategic base for setting spatial development in motion and bounding future urban planning activity as strategic processes of urban transformations (\href{Grozdanic}{\citealt{grozdanic_belgrade_2008}}).
The insistence on the flexibility and processual nature of the plan resulted in its missing several key points and legally abiding elements for directing its implementation (street regulations,  allotment plans, infrastructural systems etc.).
In the end, the results of fragmented implementations led to deregulation and chaos and endangered public interests.
\\
%technical issues
The major technical challenge was the regulation of population growth and urban sprawl. Compared to circumstances when the former plan was adopted during the SFRY period (1985), Belgrade had grown demographically 2.5 times its size.
The plan recognizes two developmental phases:

\begin{itemize}
\item until 2006
\item 2006-2021
\end{itemize}

The key urban transformations are identified as follows (\href{Stupar}{\citealt{stupar_aleksandra_recreating_2004}}):
\begin{itemize}
\item Brownfield, urban and natural heritage regeneration and preservation (ex-industrial  areas,  traffic  nodes, riverfronts,  suburban  and  rural neighborhoods);
\item renovation and extension of urban infrastructure networks;
\item introduction  of  modern, technologically  advanced  and  efficient  modes  of  urban management;
\item numerous  architectural  competitions (Belgrade marina, multifunctional business center 'Usce", numerous central squares, pedestrian streets, new office blocks, affordable housing etc.);
\end{itemize}

However, several contested issues within this plan were (\href{ref}{ibid.}):
\begin{itemize}
\item commercialization  of  urban  historical core;
\item completion and extension of residential  areas in central urban areas;
\item socio-spatial segregation as the result of the distribution of urban functions and unbalanced economic development;
\end{itemize}

\item The Plan of General Regulation (PGR);

Savamala related PGRs are: PGR for central Belgrade, Plan 4 phase 2 and PGR for fire station network 32/13.
\item The Plan of Detailed Regulation (PDR);
PDR Kosancicev venac 37/07, 52/12, 9/14
\end{itemize}

\textbf{Public Administration - National, City and Municipality levels}
\\
Even after the decentralization initiatives after 2000, centralized practices introduced during the 1990s produced an extended period of democratic deficiency and the lack of societal consensus on what is in the public interest in development-oriented planning (\href{Vujosevic}{\citealt{vujosevic_racionalnost_2004}}; \href{Vujosevic}{\citealt{vujosevic_collapse_2010}}).
However, the legal framework acknowledges that the Ministry of Civil Engineering, Transport and Infrastructure is in charge of spatial and urban planning at the national level.
Conversely, the National Government and the Parliament initiate their preparation and adopt the finalized versions.
Moreover, implementation plans and programs dealing with spatial development are also the responsibility of the Parliament and the Government of the Republic of Serbia.
The vertical coordination of activities incorporates: the Government with the Parliament and the Prime Minister, and Ministries with their Cabinets, Departments and Sections (\href{Stojkov}{\citealt{stojkov_prostorno_2012}}).
\\

Consequently, decentralized decision making enacted through the Local Governance Act stays mostly on paper. The authority of the city mayor, the city council and the city assembly have only local character, but with a direct link and subordination to the national authorities. This is even more the case for the municipal authorities, which consist of: the municipal assembly, the president of the municipality, the municipal council and the municipal administration. While all urban interventions in the city are officially under the jurisdiction of the city authorities, this does not happen to be the case for issues of political or economic importance.
%connect to intermediaries ANT

\textbf{Urban Planning Authorities - National and City levels}
\\
The institutional organization of urban planning authorities in Serbia corresponds to the administrative organization of the Republic. The Republic of Serbia is divided into 29 districts and 189 communes (including the 16 municipalities of Belgrade and the city municipalities of Novi Sad, Nis and Kragujevac). The districts act as political bodies, but they are not authorized to make their own decisions regarding spatial development. Therefore, in practice, the Spatial plan of the Republic, regional spatial plans and spatial plans for special purposes are under the jurisdiction of the National authorities.
\\
In this respect, the Ministry of Civil Engineering, Transport and Infrastructure is the key public actor at the national level in the domain.
Its field of influence extends from spatial and urban planning, construction and housing, infrastructural projects to the inspection and monitoring of activities.
In practice, this Ministry has the authority to (\href{Maksic}{\citealt{maksic_european_2012}}):

\begin{enumerate}
\item conduct administrative tasks;
\item govern strategic construction, site-development and infrastructure equipment works;
\item carry out monitoring;
\item perform inspection and supervise actions in the field;
\end{enumerate} 
%ANT diagram

On the other hand, cities and municipalities have the legal means and rights to make their own strategies, plans, and programs, as well as local regulations and rules in terms of urban development. In this respect, local authorities initiate and adopt all planning documents that control urban development and comprise the guidelines for administration of their respective municipalities/cities/communities. In parallel, the verification of plans functions on all three levels under the competence of corresponding Commissions for expert control of plans (national, city, municipality).
\\

At the national level, the supreme executive planning body had been the National Agency for Spatial Planning (NASP). The authority of this institution was defined by the Planning and Construction Act of 2003.
As a result of its constitution, all regional planning institutions were canceled and all regional planning tasks and issues were associated with the Agency. 
The Agency was discontinued in 2015 under the murky circumstances simultaneously with the adoption of BWP SPSP. The Agency was in charge of the preparation of this plan (BWP SPSP) just before its closure.
However, its sphere of competence had been (\href{Maksic}{\citealt{maksic_european_2012}}):

\begin{itemize}
\item development of spatial plans at all levels (national, regional, special uses, municipal);
\item technical assistance for plan preparations;
\item trainings in spatial planning ;
\end{itemize}

At the city level, the Secretariat for Urban Planning and Construction conducts regulatory control, consultation and implementation tasks. The work on urban plans is associated with the Urban Planning Institute of Belgrade, which in practice functions as a public-private enterprise. Finally, all urban interventions in the city should follow the GUP and they are performed in cooperation with the Belgrade Land Development Agency, professional associations (The Chamber of Engineers, The Association of Serbian Architects, The Association of Belgrade Architects), foreign embassies, international and national companies.
\\
%conclusion on top-down management - add to general conclusions
This is a brief overview of the regulatory framework at work in Savamala.  The body of urban records presented (the legal framework, policy agendas and technical documentation) show that territorial issues are givien inadequate attention in Serbia. They are based on ”borrowing” methods, not creating adequate operational frameworks, on managing the crisis but not solving problems 
(\href{Nedovic}{\citealt{nedovic-budic_adjustment_2001}}; \href{Vujosevic}{\citealt{vujosevic_conundrum_2012}}).
In the same manner, urban institutions are crammed with outdated and inert institutional arrangements and inefficient, incompetent management agencies (\href{Vujosevic}{\citealt{vujosevic_conundrum_2012}}).
% (\cite{Vujosevic and Maricic 2012}).
To sum up, the current urban regulatory framework lacks legitimacy and competence for supervise the social, ideological, political, economic, cultural and environmental transformation of the society.

\textbf{"Planning practices suffer from a generally inadequate information and research support." (\href{Vujosevic}{\citealt{vujosevic_novi_2012}})}

\subsection{Legitimacy of Interests in Transition}

As noted before, urban development relies on much more than strategic urban planning, in spite of the propensity of the planning community to regulate development processes to the greatest extent possible. Every urban issue relies directly on the economy and the mode of production and consumption in modern global cities. Namely, the capitalist economy needs urbanization to absorb surplus products, so that the deregulation of land use and property markets is the precondition for capitalist accumulation and thereafter proceedings to economic growth (\href{Harvey}{\citealt{harvey_rebel_2012}}). Following Harvey’s line of thought, the power extracted from the exclusive control over property or land is the source of capital produced by its locational, infrastructural, social or cultural capacity.
\\
In other words, the contextual resources of an urban environment in a developing country make it appealing for incoherent distribution of resources and responsibilities (\href{Bolay}{\citealt{bolay_urban_2005}}). Furthermore, within a diverse urban milieu, influential economic and political actors tend to abuse their powers and appropriate urban space when the regulatory framework is blurred and biased as it is in post-socialist cities (\href{Djokic}{\citealt{djokic_political_2007}}). 
Therefore, urban governance in these cities is more reactive to the interests of capital investments. These cities have been shown to be susceptible to tolerance of illegal practices more than is strategically proactive, which leads to organic rather than comprehensive entrepreneurial city development (\href{Petrovic}{\citealt{petrovic_cities_2009}}).
Moreover, a laissez-faire economy and global consumer culture  dissolves the democratic capacity of countries in transition (\href{Ellin}{\citealt{ellin_postmodern_1999}}). The main characteristics of urban transformations in post-socialist cities are marked by (\href{Section 2.3.2}{Section 2.3.2}):
(1) investor urbanism;
(2) political voluntarism;
(3)citizen passivity.
\\

In such circumstances, private interest and investment rationale dominate all decision-making levels. In practice, political corruption is responsible for the translation of interest into unjustified and biased planning decisions 
(\href{Nedovic}{\citealt{nedovic-budic_adjustment_2001}}).
Political corruption thrives within under-developed democratic institutions and the absence of the rule of law and germinates from the national to city and local levels. In times of transition, powerful economic actors, corporate bodies and the local oligarchy engage in tactical interest exchange with political actors in order to take advantage of empty or deteriorating areas. Their aim is to protect the socio-political circumstance of their success, promote their business models and extend their property ownership and individual wealth/capital (\href{Lazarevic}{\citealt{bajec_rational_2009}}).
\\

In case of Savamala’s developmental circumstances, powerful investors use their economic and political dominance to gain a good bargain when buying the highly profitable waterfront area of the Serbian capital and to ensure that its future development serves their needs. The most important of these hitherto active, interest-based urban transformations in the extended area of Savamala are:

\begin{itemize}
\item \textbf{The Beko Master Plan}

The former "Beko" textile factory is situated in the immediate vicinity of the city center and the Belgrade Kalemegdan Fortress. Accordingly, it is located on the cultural axis that connects several landmarks in Belgrade (\href{Vukmirovic}{\citealt{doytchinov_belgrade:_2015}}). 
This property was sold to Lambda development (LD), a Greek investment fund, in 2007. LD did not intend to renew the production activities of the textile factory, but  soon after the purchase it engaged in planning the property renovation and site redevelopment. 
\\

The former master plan for the Beko factory area dates back to 1969 and it does not allow building permanent structures on site. Instead, it proposes an extensive, open recreational area.
Conversely, the Belgrade Development Strategy of 2008 marked this area as brownfield and the most recent General plan for Belgrade from 2007 states that a detailed plan for the area is not mandatory (\href{Vukmirovic}{\citealt{doytchinov_belgrade:_2015}}).
Following the requests from the investor, in 2008 the Department of Urban Development of the Municipality of "Stari Grad" engaged the Center for Urban Development Planning ("Centar za planiranje urbanog razvoja" CEP), a local enterprise licensed for urban planning activities, to work on the new Regulation Plan.
\\

In 2009 the amendment to the City Regulation plan was officially established. The previous regulatory plan of 1969 ceased to be valid. The drafted plan conformed to the interest of the investor. Namely, the investor financed the drafting of the Plan, while several of those responsible had dual roles as both drafters of the plan and later employees of LD.
%double roles in LD
\\

The public was initially informed in 2009, but the draft plan was only made public in the summer of 2011. The plan was adopted in March 2012. It is still unclear if the investor ordered the Master plan before the adoption of the Regulation Plan or straight afterwards. However, at the Belgrade Design Week event in 2012, the new design of the Master plan for the Beko factory area was on display. The master plan contained the design for a multifunctional complex that would replace the existing depleted building of the factory. The plan was prepared by Zaha Hadid Architects. The design complex covered 94,000 square meters. It consisted of a state-of-art congress centre, retail spaces, a five-star hotel and a department store. There was a certain percentage planned for residential spaces, galleries, and offices. 
\\

The general public and moreover local experts and civil organizations contested the plan and the design. The criticisms were aimed at two issues. Concerning the urban regulatory framework, they were of opinion that swift and nontransparent change of technical documentation diminished the amenability of regulations and norms. Such a trend makes urban decision-making a deregulated and relative process of placing the interests of the powerful and wealthy within the building and development practices of the city (\href{Vukmirovic}{\citealt{doytchinov_belgrade:_2015}}). If such practices continue to dominate the urban realm of the city, they may de-strategize the exploitation of territorial capital, causing contextual resources in Belgrade melt down in the battle of economic and not at all public interests.
\\

Conversely, in terms of urban management the criticisms were summed up around the issue of engaging a "starchitect", such as Zaha Hadid, for deciding the future of one of the most important locations in the city without consultations and collaboration with local experts. The plan and the eventual project were marked as a product of political propaganda and the manifestation of the new neoliberal economic order (\href{Vukmirovic}{\citealt{doytchinov_belgrade:_2015}}).

\item \textbf{The City on Water - the Belgrade Port}

The area of the former Belgrade Port Company is also a strategic location that spreads from the Danube Riverfront almost to the city center. The territory covers an area of 470 hectares. The Planning and Construction Act of 2003 suggests that the urban design for a location of such importance should be the result of a professional competition  (\href{Zakon}{PCA 2003}).
In 2006 the Urban Institute of Belgrade announced a competition upon invitation for five public professional institutions: the Faculty of Architecture of the University of Belgrade (FAUB), the Association of Urban Planners of Belgrade (AUB), the Association of Belgrade Architects (ABA), the Academy of Architecture (AA) and the Architect Club (AC). The results were five different visions for an urban transformation of the area.
\\

The Belgrade Port Company was privatized in 2008. Since its privatization, the new owners started their own campaign to find the best solution for the area. The urban design of the Danube waterfront as well as upgrading the quality of the pertaining public spaces were commissioned to architects Daniel Libeskind and Jan Gehl. The official general urban design project was prepared with their cooperation during the period 2008-2010. The plan dealt with the renovation of the Dorcol Marina and surrounding area.
\\

Project preparation activities were followed by a series of public presentations about the project details at the beginning of 2009. Gehl Architects held the presentation of the project in the Belgrade Chamber of Commerce and Daniel Libeskind’s gave the lecture at the University of Belgrade on the same topic. A final presentation of the project was arranged by the Belgrade Land Development Agency (a public enterprise) at the Real Estate and Investment Fair in Cannes later on in 2009. The plan envisages the construction of residential and commercial buildings, structures for cultural facilities, a congress centre, a school, a nursery and a hotel. The main landmark of the area would be a 250 meter skyscraper marking the confluence of the two rivers, the Sava and the Danube.
\\

The  Detailed  Regulation Plan for the area was adopted by the City  Parliament in 2012. The plan propositions  were in accordance with the physical and functional program of the proposed design prepared again by star-architects.
Even though the project was in  line  with  the contemporary town planning principle that promotes the quality of urban spaces and diversity of urban practices, the governing rule in designing these urban spaces was to offer investment possibilities to powerful business and corporate actors  (\href{Vukmirovic}{\citealt{doytchinov_belgrade:_2015}}).
\\

Moreover, the question of the controversial privatization of the Port of Belgrade and the unresolved situation with the ownership of land put the project on hold. However, these murky business agreements were the focus of public attention. Interest in the project was significantly reduced after the change in the political party in power at the national level in 2012 and at the city level in 2014. 

\item \textbf{Beton Hala Waterfront}

"Beton Hala" (Concrete Hall) was built in 1937. It was a multi-functional port storage on the Sava river. Incorporating commercial, harbour and railway purposes, "Beton Hala" was very modern in its time. It served its purpose for decades. 
\\

During the isolation of the 1990s, its activities gradually declined and it became a depot and an industrial wasteland for the Yugoslav inland waterway shipping enterprise (JRB) under SRY jurisdiction or for the Kalemegdan fortress institution under the jurisdiction of the RS. Later, the management of the space was conferred to the city authorities and ever since it has swung between public enterprises - the City Greenery ("Gradsko zelenilo"), Agency for office spaces ("Agencija za poslovni prostor") and   Agency for property ("Direkcija za imovinu").
\\

After the regime change in 2000, "Beton Hala" became a catering and commercial hub first for tourists and then also for the local upper middle class. This practice encourages sparks of life along the Sava riverside, but the area remains without proper access points from the city center, which is in close vicinity (Kalemegdan Fortress and Knez Mihailova street).
\\

The "Beton Hala" (Concrete Hall) Waterfront Center was identified as a brownfield area according to the Belgrade Spatial Plan 2021  and  the Belgrade  Development  Strategy 2008. In 2011, the Association of Belgrade architects with the support of the City administration (Investment and Housing Agency and Urban Planning Institute) announced an International competition for architectural and urban solutions for the Sava riverside, the pertaining industrial heritage and the corresponding city  center waterfront  access lines. Even though two equal first prizes were delivered, the winning project "Cloud" designed by Sou Fujimoto  Architects  caught  attention because of  its  innovative and hyper modern character of the cloud hub that allows the continuity and totality of the  structures  and  history  to  co-exist  and  co-function.
\\

The detailed plan of "Beton Hala" and the area of the waterfront to the Kalemengdan fortress started in 2012. The plan proposes a new, ultra-modern identity for the area with an iconic ring structure and mainly commercial purposes. Bearing no detailed implementation strategy and an unclear urban development model, the public lost track of the plan and the projects for the area as soon as the political party in power changed at the national (2012) and city (2014) levels. Consequently, the condition and the status of "Beton Hala" remained the same and this urban transformation is remembered as an advertising campaign that presents a possible modern urban and architectural identity of Belgrade more than it investigates and proposes sustainable ways to accomplish it (\href{Vukmirovic}{\citealt{doytchinov_belgrade:_2015}}).

\item \textbf{The Belgrade Waterfront Project}

The waterfront area between Brankov and Gazela bridges is listed as a development area in all three strategic-planning documents for Belgrade: the Belgrade City Development Strategy 2008, the Regional Spatial Plan (RSP) for Belgrade 2009 and the General Plan for Belgrade 2021 (GP Belgrade 2021). In these documents, the area is addressed as the Sava Waterfront (Development Strategy), the Sava Amphitheatre (GP) or even a part of New Belgrade’s centre (SP). The area could also be addressed as a zone for brownfield regeneration of high regional and national importance  (\href{Peric}{\citealt{peric_evolution_2016}}).
The vicinity of the Kalemegdan historical site and the confluence of the two major rivers, the Sava and the Danube, makes it the prime location for construction in the Serbian capital.
\\

Ever since the announcement of the railway station relocation in 1923 GUP of Belgrade, the topic of the Sava amphitheatre and the distribution of the central urban zone of Belgrade down and along the Sava river have become a burning issue. As part of the waterfront area and Sava Amphitheatre, the area has been the subject of a series of competitions.
\\
As early as 1946, Nikola Dobrovic in his text on the urban development prospects of Belgrade offered a sketch of a park that connects the right bank of the Sava to the city center. Later, this riverside was joined to the New Belgrade center that covered two spatial, geomorphological and administrative units divided by the River. Respectively, the area was targeted by an international competition in the 1980s for the centre of New Belgrade while in the 1990s it was addressed in the competition for the Sava Amphitheatre and the project Europolis.
\\

Simultaneously, as a part of wider society modernization prospects, in the late 1980s the authorities promoted the ”Town on the Water” project. Consequently, the infamous Serbian leader Slobodan Milosevic supported the Europolis project as a part of his electoral campaign in 1995. The project was prepared by the Institute of Transportation CIP. Finally, in 2012, the Urban Planning Institute of Belgrade (UPI) collaborated with GIZ ("German Agency for international cooperation with the Republic of Serbia") to prepare an urban study of the integrated urban development prospects for Savamala (\href{Integrated UDP Savamala 2012}{Integrated UDP Savamala 2012}).
%ne menjaj u cite
The document aimed to announce a new round of international competitions for the Sava Amphitheatre.
\\

The Belgrade Waterfront Project targets the area within the Sava amphitheatre. It covers 90 hectares of land and introduces 1.5 million m\textsuperscript{2}  to be built up within the framework of the project. The project was first announced in 2012. Taking into account the scope and size of the project, it could not be financed by public funds and loans alone (\href{Vukmirovic}{\citealt{doytchinov_belgrade:_2015}}).
\\

The Belgrade  Waterfront Master Plan gained publicity before the national elections in March 2014 as a strategic partnership between the RS Government and an investor from United Arab Emirates (UAE).
It was presented in Dubai in March 2014 by  Mohamed Ali Alabbar, a potential investor
\footnote{He became famous for chairing the Emaar Properties that was engaged in the construction of the  skyscraper Burj Khalifa in Dubai.}
He is also the  director  of the  "Eagle Hills" investment company, the recently established investment company targeting the  flagship mega project in cities around the developing world.
\\

As the project developed, the strategic partnership grew into the "Belgrade Waterfront Company", a mutual enterprise of the Republic of Serbia  and "Eagle Hills". The initial value of the project was advertised as being 2.5 billion euros. However, by the time the official contract was made public in summer 2015, the amount had shrunk to 300 million euros with up to half of the money obtained from a loan at the expense of Serbian citizens.
\\

The purpose of the development project is to create a multi-functional complex with large commercial and business premises, exclusive apartments and luxury hotels overlooking the Sava River. The spatiality and height of the projected construction on the muddy terrain of the riverbank and at this exclusive urban location within the Belgrade metropolitan area required significant legal adjustment to comply with the urban regulatory framework in Serbia.
\\

Already in 2014, national and city authorities introduced speedy procedures for these legal adjustments. The amendments to the General Plan (GP) of Belgrade 2021, the new Planning and Construction Act and the Spatial plan for Special Purposes for the purposes of the "Belgrade Waterfront Project" were adopted. In the process, several national and city bodies were established (the temporary decision-making body of the city of Belgrade) or discontinued (the National Spatial Planning Agency) according to the sole interests of BWP. As the legal framework has been adapted to the needs of this project, the investor had in advance a Master plan that had already been prepared by foreign architectural firms.
\\

The adoption of the Lex specialis, a special law, in 2015 solved the on-site property issues and enabled the construction to start. The first 5-year phase consists of building two high-rise residential buildings, a huge shopping mall, a tower and the necessary transport infrastructure. Over the course of 2016, the clearance of the area continued, and the authorities even resorted to forceful removal of structures and inhabitants in order to speed up the process. The residential buildings passed all necessary legal checks. The construction is scheduled to be finished in 2017. On the other hand, the Sava promenade along the Sava riverbank was in its final stage at the end of 2016, even though the issue of coastal construction and the corresponding permits had not been dealt with 
The work on the iconic tower is due to start in 2017.
\\

The smooth path of legal adjustments and administrative procedures for the project have been enabled through the tight cooperation between the city and the national authorities, mainly through political party links\footnote{From 2014 onward, the prime minister and the city mayor have been from the same political party.}
%ANT analysis
(\href{Maruna}{\citealt{maruna_can_2015}}; \href{Peric}{\citealt{peric_evolution_2016}})
%\cite{Maruna 2015, Peric 2016}.
Adopting legislation that legitimizes investor-based urban decision-making is the result of centralized political power through political party mechanisms. In practice the decisions made on the national level were imposed on the regional/city/local levels. Having political bodies coordinate the creation of planning solutions causes the usurpation of both the formal planning procedure and the professional expertise of bureaucrats and the interests of private investors (\href{Maruna}{ibid.}; \href{Peric}{ibid.}).
\\

With no wider socio-spatial strategy, this nontransparent bureaucratic conduct of interest-based urban transformations seriously endanger the public interest (\href{Vukmirovic}{\citealt{doytchinov_belgrade:_2015}}). 
Generally speaking, the major criticisms apart from the complete lack of transparency and the exclusion of local experts and practitioners on the national level are:

\begin{itemize}
\item overcrowded space and an enormous amount of m\textsuperscript{2},
\item land given to the investor to build whatever he wants however he wants,
\item wasting such location (geometrical center of Belgrade and waterfront area) for housing and commerce with less than 1\% for cultural facilities,
\item collision with GP and planned National opera in the area,
\item difficulties for underground construction of parking space following the tenure examination,
\item the issue of cultural urban planning - what suits UAE does not necessarily correspond to the Serbian context)
\item eviction of marginalized local groups following the problematic special law (Lex specialis)
\end{itemize}

\href{Vukmirovic}{\cite{doytchinov_belgrade:_2015}}
%ne menjaj
well elaborated that the lack of context-related substance and a symbolic action makes Belgrade Waterfront Project rather a spin in its broad scope and intentions. However, its strong political backing and financial interest calculations paved the way at least its partial implementation.
\end{itemize}

%conclusion
These circumstances have lead to a multitude of interests, initiatives and projects of different scales with no effective and binding policies and institutionalized regulatory means for synchronization and coordination among them. Within such a provisory framework, the social, cultural and programmatic clashes that are happening between the present cultural and civic activities located in Savamala and the influential foreign investor backed by politicians of the highest rank are most likely to end up only in favour of the latter.

\subsection{Network of civic engagement}

Bottom-up, step-by-step urban transformations have been promoted as inclusive, gradual and effective in cities that are going through traumatic urban transitions, as do post- socialist ones. In theory, such planning offers an alternative to surpass current profit orientated
neo-liberal trends and benefit from social-spatial contradictions rising from the often blurred and twisted structure and puzzling development prospect at play. The recent boom of bottom-up spatial interventions and small-scale cultural projects in the Savamala neighbourhood have aimed at setting up just such a specific micro environment in Belgrade (\href{Urban Incubator}{\citealt{muller-wieferig_urban_2013}}).
\\

In general in Serbia and in especially Belgrade, the revival of civil sector activities started with the regime change in 2000. Meanwhile, taking advantage of the long gap in development, a number of local and international organisations and cultural entrepreneurs have focused their actions on Savamala. The neglected socialist and post-socialist status of this neighbourhood, rich in urban and architectural heritage, have attracted their attention. Their initiatives to transform abandoned places and to reactivate them through participatory, cultural, artistic and educational activities have been mainly supported by the local municipality  "Savski Venac" and international cultural institutions and programmes.
\\

What at first seemed like a sum of ephemeral local activities has become a driving force for the possible urban future of Savamala, at least the future preferred by most local urban actors who have taken an active role in it.
Without questioning the civic nature currently advertised as such, these pillars of Savamala urban reactivation are found in:
(1) Savamala cultural hubs,
(2) urban transformation programmes,
(3) individual urban projects, 
and
(4) NGOs addressing Savamala socio-spatial issues.
\\

\textbf{Savamala cultural hubs}: 

The first to settle in the neighbourhood and to promote this new image of Savamala was \textit{Cultural Center Grad (KC Grad)}. Establishing an alternative cultural institution in Belgrade was a joint initiative of the Cultural Front Belgrade and the Amsterdam Felix Meritis Foundation.
They obtained the old customs building from the Municipality "Savski Venac".
The building had been vacant for years and the municipality consigned the premises to KC Grad for undetermined temporary use. Consequently, its alternative artistic and cultural engagement set in motion the new spirit of Savamala, though in barely significant scope.
\\

Intensive aggregation of participatory activities truly started when the\textit{Mikser Festival} was organized on the streets of Savamala in 2011. The cultural activism of the Mikser team became branded when they rented the abandoned, recently privatized old warehouse building on Karadjordjeva Street 46. In 2013, the multifuctional space of Mikser House was opened to the public and has been an official hub for its cultural, educational and commercial activities ever since. What began as the Mikser Festival continues to serve as the common denominator for cultural and artistic crowdsourcing and the participatory engagement of activating Savamala’s public spaces.
\\
What is more, in 2012, on the border of Savamala on Gavrila Principa Street, a group of young cultural managers and entrepreneurs opened the door of its 350m\textsuperscript{2} co-working space, \textit{Design Incubator Nova Iskra}. Supporting young, emerging professionals of diverse freelance backgrounds by offering them the use of a full-capacity office and workspace at below-market-value prices is a well-known trend in  developed countries. 
Putting to work this combination of co-working and creative business, innovation platform, education, and design labs contributes to adding Belgrade to the map of the European creative economy sector. 
\\

\textit{KC Grad}, \textit{Mikser} and \textit{Nova Iskra} were the forerunners of this unconventional programming, functional and business model in Savamala, Belgrade and generally in Serbia (\href{Doytchinov}{\citealt{doytchinov_urban_2015}}). %{(ref Vanista Lazarevic in Doytchinov 2015)}). 
It also must be acknowledged that \textit{"Dom Omladine Beograd"}, a cultural and education city organization, has been using the old depository in Kraljevica Marka Street (MKM) for various activities since 2007.
The space (\textit{MKM}) is famous for hosting the programme of the independent cultural organization at the national and city level.
It can also be noted that \textit{MKM} established its relationship with the particularity and wider context of Savamala only when it agreed to accommodate several activities of the \textit{Urban Incubator Belgrade (UIB)}.
\\

This new cultural spirit that was spreading around Savamala prompted several gallery platforms and design collectives to house their activities in the ground-floor premises of Savamala buildings, which were built in the classical style before WWII (Gallery Kolektiv, Gallery G12HUB).
\\

As long as the management practices of these alternative places are closer to the social improvement and empowerment of local communities than to profit-oriented business, their particular and scattered interests still help spread a spree of participation in this post- socialist context.
\\	

\textbf{Urban transformation programmes}:

However, the participatory activities are best found in  the urban transformation programmes named the Savamala Civic District, in the \textit{Urban Incubator Belgrade} activities and their successors. Neither the wide variety of these actors, nor the exact steps and plans in this process were clearly articulated beforehand, --they were rather ”work in progress” projects.
\\

The \textit{Savamala Civic District} was originally envisioned as a set of participatory activities supported by the \textit{Mikser Festival}. In this manner, the Mikser Festival represented an umbrella for building a platform of all urban actors and stakeholders to engage in changing their immediate surroundings. However neither the wide variety of these actors, nor exact steps and end states in this process, were to a great extent clarified beforehand.
Its priority goal was to create a sort of civic district as a long-term participatory realm for taking the most of a range of opportunities for non-institutionalised, flexible and dynamic urban transformations through various levels of sharing: knowledge-, experience- and vision-sharing activities. In order to test this programme, an international group of experts, who worked on innovative models for bottom-up synergies coming from social, cultural, infrastructural, ecological and economical aspects of urban development, gathered in Savamala during the Mikser Festival in 2012 (\href{Cvetinovic}{\citealt{cvetinovic_engine_2013}}).
The event included a series of meetings, debates, collaborative works and public space installations taking place in the Savamala neighbourhood.\footnote{In order to reach this goal seven parallel workshops addressed the status and development of Savamala from different prospectives, such as (\href{Cvetinovic}{\citealt{cvetinovic_engine_2013}}):
\begin{itemize}

\item Unheard Stories of Savamala (SIMKA and Ana Ulfstrand, Stockholm) -

This artistic workshop promoted an ethnological approach towards research that could interpret the urban devastation through qualitative data. 

\item Urban Body (Alexander Vollebregt, Rotterdam) -
This workshop addressed the question of resilience and prosperity in the contemporary condition of Savamala. The participants had to learn to use the full potential of their minds and bodies to develop an enhanced urban comprehension.

\item 5 Obstructions for Urbanism (Todd Rouhe and Lars Fischer, Common Room, New York) -

Several on-site installations had been built according to the creative methodology approach. They were articulated by this collective, addressingthe  ordinary, interconnected, shared, different and the new in making and using architecture.

\item Butong Installation (Lars Hoglund, Stockholm and Benjamin Levy, Paris) - 

The Installation was made of Butong, material that was created in 2009. It served as a convertible eco barrier in order to protect the public space from the intrusive noise and pollution of heavy traffic.

\item Urban Bundle (James Stodgel, Santa Fe) -

The temporary public space installation provided an initial condition for citizens to gather, meet, debate, and collaborate in the process of creating and maintaining their social and urban environment.  

\item A Sense of Place (Ljubo Georgiev, Maja Popovic, Failed Architecture, Amsterdam) -

The workshop aimed to transform the courtyards of Karadjordjeva Street  into places where neighbours gather together to reinforce their sense of belonging to the place and to upgrade the space they are forced to share.

\item City COOP Web Platform (Ana Lalic, Vancouver) -
This workshop gathered students to conceptualize the website platform that could build a social network of citizens, experts, developers and institutions to exchange ideas and opinions related to the urban transformation of Savamala.  
\end{itemize}
}
\\

Among others, \textit{Urban Incubator Belgrade} - a Goethe Institute initiative - stands out as a certain operative and strategic gateway to influence the future development of this devastated but promising neighbourhood.
As a part of a broader international strategy of the German cultural institution (Goethe Institute) to focus on urban development, the programme and the organizational structure that it produced stemmed from Goethe Guerilla, an informal collective founded in 2010 under the umbrella of this cultural institute.
\\

The cluster of 10 site-specific projects targeted urban design and regeneration, art and culture in Savamala for the period of one year (2012-2013). All the actions within this project relied on communication among individuals, self-organised associations, public services and private enterprises as equal participants in the societal realm which would demonstrate its influence by performing spatial changes as social exchange. The aim was to eventually boost urban transformations in Savamala.
\\
These activities were divided into three groups according to what they address:
\begin{itemize}
\item Developing infrastructure for social change:

\begin{itemize}
\item \textit{Nextsavamala} - Crowdsourcing a City Vision
\footnote{This activity was based on its successful application in Hamburg, where these instruments had been developed and tested. It represented a web-based public forum and workshops for collecting and filtering citizens’ visions and discussing, developing and pitching ideas that could be later  implemented in actual planning processes for the Savamala neighbourhood.}
\end{itemize}

\item Systematic collaboration within the network of civic engagement:

\begin{itemize}
\item \textit{Bureau Savamala}
\footnote{This symbolic institution figured as the critical commentator of the whole Urban Incubator Belgrade project. It focuses on critically monitoring and analysing the contribution of artists, any creative projects and creative capital in general on the socially sustainable development of Savamala. Savamala at that moment was perceived as a venue that allowed alternative lifestyles very attractive to creative indi- viduals. The result of this activity was a sort of record on mapping how this neighbourhood changes and how the perception of locals and the broader public has changed accordingly (\ref{ref}{"Bureau Savamala" 2013}).}

\item \textit{We Also Love The Art of Others}
\footnote{This symbolic name denoted fostering a network among the artistic scene of Belgrade and beyond, with members from different origins. These artists entered into dialogue in this neighbourhood and were motivated as a group to engage in artistic research and interventions in a forgotten place, where all the artistic work would be adjusted to the current context.}

\item \textit{Model for Savamala}
\footnote{This component came from an architectural practice that had set itself the goal of building a 1:200 model of Savamala on the basis of collected local knowledge (\href{ref}{Model for Savamala 2013}).
This 3D representation of a physical structure incorporated soft data, namely the social structure of the neighbourhood, which was not based on aesthetics, but on information.
So, the information implied tracing all urban spaces and structures by creating a passport for each and every structure inside this area.}
\end{itemize}

\item Pop-up events and instant actions for the reconstruction of everyday life:

\begin{itemize}
\item \textit{Listen Savamala}\footnote{This sound-art project traced urban changes through sound recording in order to justify that it was not only a visual phenomenon.}

\item \textit{Camenzind}
\footnote{This was a Serbian edition of the Swiss magazine of the same name. It was an outcome of the exchange of Serbian-Swiss-German knowledge on the built environment through four issues of this new Serbian journal covering architectural issues and urban public events. It also reported on the collected local knowledge in and about Savamala.}
\end{itemize}

\item Upgrading the Urban Environment:
\begin{itemize}
\item \textit{Savamala Design Studio}
\footnote{The aim of the Savamala design studio was to establish a participatory design practice that could create new relationships among various urban actors and stakeholders with an emphasis on encouraging Savamala residents to join and find advice and active support for their own design and construction demands and requirements.}

\item \textit{School of Urban Practice}
\footnote{This project encouraged students in architecture and the arts to seek a way to improve the everyday environment of Savamala, whether through creating public policy, mediation, urban planning and architecture design, or any other form of design that involved citizens from the very beginning of the designing process.}

\item \textit{Micro factories}
\footnote{These small and new production facilities in Savamala, as the name says, strove to identify small suitable spaces in Savamala and to attract participants in order to define and tap into a creative production process that could transform local knowledge, capacities and ideas into innovative design products.
Participants collected local materials (usually from abandoned apartments or other places) and worked on the design of products that reflected what they had found in Savamala and its tradition of small craft workshops (i.e. carpentry).}
\end{itemize}
\end{itemize}

The project reactivated five abandoned physical locations in the neighbourhood:

\begin{enumerate}
\item The \textit{Spanish House}, a roofless structure at Brace Krsmanovic 2, where a temporary pavilion was built for the purposes of the programme;

\item Common space at Kraljevica Marka 8 (\textit{KM8});

\item The basement of the building at Crnogorska 5;

\item The municipality office space at Svetozara Radica 3;

\item The space at Gavrila Principa 2.
\end{enumerate}

The \textit{UIB} obtained these spaces from the Municipality "Savski venac" and created new local and international networks around them.
These new spaces for art and culture in Savamala formed the first network of symbolic capital identified therein.
\\
\textbf{Individual urban projects}:

The predecessor of all urban design investigations in Savamala may be the Master Class for urban students and young professionals that occurred in October 2011. The course was organized by Stadslab European Urban Design Laboratory and focused on the Sava’s right riverbank. Organizational and financial support was provided by local institutions (Belgrade international week of architecture BINA, Urban Planning Institute Belgrade and the Serbian Railways) and an international partner (Amsterdam Institute for Physical Planning).
\\

However, it was only the international initiative of the \textit{Goethe institute} which positioned Savamala on the mental map of urban culture in Europe. (\href{Kamenzid}{\citealt{kamenzid_urban_2013}}).
By the time UI was finished, most of the projects had found local counterparts and continued to exist in this new form. Consequently, the spatial capital in Savamala incited cultural entrepreneurial collectives and research groups to focus their activities there. The successors of the UIB programme, Mikser Festival and KC Grad have remained the gatekeepers and supporters of any kind of social, cultural or education activism in Savamala.
%for analyses
\\

\textit{Savamala, a place for making} was a participatory project, that proposed collective performative actions for re-vitalizing neglected community space in Savamala.
They worked with an Urban incubator studio space KM8 and ”Zupa” an old abandoned steamboat, situated at Savamala’s riverbank. The project took place during the spring and summer months of 2013 in conjunction with UIB and the design class from The University of Fine Arts of Hamburg (”The Hochschule fr bildende Knste” HFBK Hamburg).
\\

\textit{The game of Savamala} was a participatory urban planning workshop organized for foreign students and locals under the umbrella of the \textit{Mikser Festival} in 2015.
Its aim was capacity building in terms of producing an urban business model supported by the in-depth empirical research of the students. The result of the board’s planning game was the notion that participants got in terms of the role of architecture and design in the urban economy.
\\

Finally, another participatory project was organized in 2016 which addressed the on-going issue of urban transformations in Savamala under the pressure of the neighbouring Belgrade Waterfront Project.
\textit{My piece of Savamala}  was an urban design workshop organized by the \textit{School of Urban Practices}, \textit{City Guerilla} and \textit{Mikser} and monitored by city authorities and the City Architect himself.
Young designers, artists, architects and urban professionals from these organizations worked with citizens during 3 sessions in order to produce different urban solutions for the urban block at the crossings of five streets in Savamala: Karadjordjeva, Svetozara Radica, Kraljevica Marka, Hercegovacka and Travnicka. The resulting design solution as well as the report from the workshop were given to the city authorities and the Belgrade Waterfront company (BWC). According to the organizers, more than a year after they still remained without any response.
Other projects active in Savamala from 2013 onward are presented in the timeframe diagram (\href{FIgure Savamala PUD timeline}{Figure X}).
\\

\textbf{NGOs addressing Savamala socio-spatial issues}:

Bearing in mind all these initiatives, programmes, projects and events in Savamala, it is conspicuous that the main organizations dealing with its social and spatial resources and urban conflicts were those that stem from Goethe Guerilla and the Urban Incubator Programme. Urban Incubator was succeeded by the UIB Association, the organization that dealt with post production of the programme. However, most of the newly established organizations that participated in UIB gathered under the City Guerilla.
\\

Even though UIB managed to activate and collaborate with all important civil sector agencies in Savamala, there are still several organizations whose agendas diverges from that of UIB.
\textit{Streets for cyclists} is an NGO founded in 2011 and located in Savamala.
Even though their main activity is promoting biking culture in Belgrade, they played a crucial role in confronting local authorities and Belgrade Waterfront investors when they closed the principle cycling path along the river for construction purposes. 
\textit{Ministry of space} is an informal collective focusing on critical approaches to urban transformations in Belgrade.
The organization collaborates with national and international research and activist networks.  With a similar purpose, the collective participated in \textit{UIB} within the \textit{Bureau Savamala} framework.
As a response to investor urbanism embodied in the Belgrade Waterfront Project, \textit{Ministry of space} formed another NGO \textit{Ne da(vi)mo Beograd} initiative (NDVBGD) with the sole purpose of gathering human and material resources and proof in order to fight the project and negative effects that it has on overall urban development in Belgrade.

\subsection{Urban Decision Making in Savamala}

The elaborate state of urban decision-making in Savamala shows  crucial signs of failure within all three layers of influence. Establishing clear links between the process of strategic development, its institutional framework, the hierarchical structure of long-term and short-term objectives of all actors involved, and the real-time changes happening simultaneously in an urban environment has not ever been its goal.
\\

Post-socialist, transitional urban planning broke down through the lack of consensus on priority goals, action-oriented programmes of implementation and flawed coordination at different levels, sectors and areas. In practice these conditions ended by having the policy agendas and technical documents as an advisory vision on paper, but left the real actions and decision making to political and market forces. The actual carriers of urban interventions not only in Savamala, but in Serbia in general, are the representatives of big businesses. In these circumstances, the civic initiatives presented are also manifests of interests, and although different, they still might not be those of the disempowered and marginalized and ordinary.
\\

Since then, the urban transformations of Savamala have exceeded and diluted the common strategic framework defined with the public interest in mind.  The complexity of this interplay of urban decision making agents make Savamala a prime example of historical and contextual processes that could instigate such urban dynamics.

\section{Case Study Framework}

This chapter recognized, decomposed and restructured a historical and discursive overview of socio-spatial patterns of urban system transitions in the Savamala neighbourhood.
This chronological and causal interpretation of the complexity and dynamics of the urban system in Savamala exposes the factual and symbolic nature of all different elements at play at the neighbourhood level.
Respecting their origins and paths of evolution enables the estimation of the conditions and needs of today and the proposition of transition scenarios (\href{Grozdanic}{\citealt{grozdanic_belgrade_2008}}).

%visualization of the chapter structure

%%%%%%%%%%%%%%%%%%%%%%%%%%%%%%%%%%%%%%%%%%%%%%%%%%

\chapter{Urban Complexity through Actor-network Theory Lens}
%check where i use "urban development prospects" syntagme and change it and replace "agent" with actor, but keep "agency"

Apart from their physical structure, cities are a summary of all citizen behaviours, emotions and value systems of all previous times and the source of prospects for the future times of upcoming generations  (\href{Stojkov}{\citealt{stojkov_grad_2013}}).
Social and physical structures are perpetually interacting with one another, while the historical strata of these interactions accumulate one upon another. The city, and more generally the urban, is therefore ultimately a dynamic and immensely complex phenomenon. The term urban system transitions is used in this research to bound up the continuity of its fluctuations over time. Adding the time component in terms of discrete states of past, present and future enables the contextualizing of these processes. In this research, the study will concentrate on post-socialist cities and analogous contextual processes at the neighbourhood level.
\\

The current state of affairs in Savamala results from the deposition of the historical layers with their own explicit decision-making mechanisms of the time and its final blend with the current machinery of decision-making. Associating a spatial component to the structuralized historical deposits of data, procedures, and identities provides a background for space-time translation that have constituted the state of the elements/entities at play in Savamala (\href{ref}{Table X}). 
%visualize history - course of time - direction backwards only - represent in terms of vectors
The historical component elaborated in the previous chapter is just a one-way directional vector that reaches the present. However, all human, social and technical elements and networks assemble in their current incidence in Savamala. They are actually the active agents and the actors of the on-going transitions. The present of Savamala is a time- bounded picture of localized urban system transitions. 
\\

In this research project, Actor-network theory (ANT) serves for interpreting the present state of the local context in Savamala. The most prominent characteristic of ANT is flattening the social by symmetrical treatment of human, social and technical elements that all might be actors of urban system transitions (\href{Latour}{\citealt{latour_reassembling_2005}}). 
Based on Latours argument on the new research agenda for globalization and world cities, ANT is herein applied not as a theory but as a method. A figuration of human and non-human actors through networks
makes them the agents of urban dynamics in concrete space-time and produces a complex reality of urban system transitions. The actors existence is its status in a connection or connections. According to ANT, actors do not exist if their networks are not labeled. In this way they become agents. Therefore ANT serves for structuring the data on human and non-human agents and urban assemblage networks at the neighbourhood level.
\\

This chapter brings the first round of data analysis with Actor-network theory. First of all, it tentatively reinterprets the specificities of a post-socialist neighbourhood according to ANT logical framework and terminology. The boundaries of Savamala for the purposes of this analysis are established according to the investigation among experts and young professionals on the issue (\href{Questionnaire Experts Savamala}{Questionnaire X}, \href{Questionnaire PhD Savamala}{Questionnaire X}, \href{Questionnaire Students Savamala}{Questionnaire X}).

%(\href{ref}{Savamala questionnaires - experts Q18, phds, strudents}).
The Savamala neighbourhood corresponds to the area between the Brankov and Old Sava bridges; from the Sava riverside to Brankova street, from Zeleni Venac (the "Green Wreath" market to the park in front of the Faculty of Economics. An urban assemblage map, which is laid out further in the chapter, summarizes these ANT interpretations in terms of the relational networks between urban key agents and the identified social aspects at work in the local and wider context of Savamala. Finally, the conclusion discusses the results, risks and opportunities of extending ANT in order to enable research to go beyond descriptions toward its operationalization in a particular urban setting. In Savamala, the results are summarized to address the current state of the neighbourhood and the course of its possible future developments.
%visualization: deposition of historical layers and merging of decision-making on each layer

\section{An ANT Overview of Urban Agency in Savamala}

Bearing in mind that actor-network explanations give real results only in strongly defined situations (\href{Farias}{\citealt{farias_urban_2011}}), the neighbourhood level is confined but yet significant enough for the analyses to work and for the results to matter. The study applied a flattening composition of all heterogeneous (1) human/non-human actors (ANT) in Savamala. These actors (1) were identified from qualitative data collected on two different tracks: as key urban actors and within the layers of urban decision-making. Further on, the collected data are structured on 4 more levels in relation to ANT, in terms of (2) intermediaries/mediators, (3) free associations, (4) stabilizing \& destabilizing agencies and (5) urban assemblages (\href{ref}{Table 3 ANT paper}).
The congregation of these categories serves to visually describe the urban reality of a post-socialist neighbourhood - Savamala.
\\

Following the circumstances found through the in-depth case study research design, the empirical and theoretical data are structured according to five dimensions of actor-networks in the following way:

\begin{enumerate}
\item all human and non-human actors;
\item intermediaries and mediators;
\item free associations;
\item stabilizing and destabilizing agencies;
\item urban assemblages.
\end{enumerate}

\subsection{All human and non-human actors}

The rough scheme of human and non-human actors is formed according to case study description
(\href{ref}{Chapter 3}). It is further completed with the data from the questionnaires (experts, young professionals and students) and the interviews
(\href{ref}{Annex A}).
\\

In ANT terms, the Savamala neighbourhood is represented as a venue (urban territory/ space) with material constitutional elements (built environment - urban structures), wherein a variety of urban actors and stakeholders (individuals and groups) - interrelated to social factors (political, economic, cultural and social components of urbanity) and within a specific regulatory framework (institutional relations and records) - engage in actions
\\

Since it has been argued here (\href{ref}{Chapter 2}) that the rapid flow of people and information in the modern globalised world has profoundly transformed the perception of space and time, lifestyles and our sense of community and self (\href{Ellin}{\citealt{ellin_postmodern_1999}}), it must be then acknowledged that the vital cohesive force of the modern city also incorporates technical solutions (urban infrastructures) and technologies (communication and media).
Graham and Marvin (2001) address these with an all-embracing term ”the infrastructure scapes (electropolis, hydropolis, cybercity, autocity)” stating that these elements invigorate urban life, fuse urban spaces and serve as mediators of transitions.
\\

Complex infrastructural systems are nowadays also the core of human actions and institutional relations that enables its extension in space and time and produces new conditions of urban reality (\href{Graham}{\citealt{graham_splintering_2001}}).
The ampleness of these non-human actors as well as their intrinsic geography of places and connecting flows (\href{Swyngedouw}{\citealt{swyngedouw_communication_1993}}) from splintering urbanism) bring in a new layer of actors and networks. Its specificity and volume allows for this research to exclude these actors to a certain extent, mainly for the reasons of their limited resources. However, the structure of the methodology accounts for their inclusion in future works and studies.
\\

The main sources of human and non-human agency taken into consideration in this analysis are: (a) urban actors and stakeholders, (b) urban space and the built environment, and (c) the urban regulatory framework. Social components (the political, economic and cultural aspects) are rather considered in an integral way as contextual, post-socialist or transitional circumstances traversing these different space-time layers. Also taken into account as the bearers of non-human agency, they are figurated into urban assemblage networks as active elements linking urban actor-networks, space-time layers and the levels of decision making with urban development prospects.
\\

The morphology of urban decision-making tend to catalyze and hold sway over urban complexity. The layers of urban decision-making function throughout a network. They embody the relationality of urban elements and reveal the sources of urban agency (\href{Section 4.2}{Section 4.2}).
Even more relevant, the reason for this limitation of the scope of analysis lies in Friedmann's (1992) thesis of four key determinants of urban agency:
(1) governance (executive, legislative and civil),
(2) polity (political organizations, social movements),
(3) economy (markets, corporations, financial institutions) and
(4) society (individuals, groups and associations).
%(Lazarevic Bajec 2009, Friedmann, 1992, 27)}
In this manner, the initial identification of human and non-human actors is summarized within:
(o) top-down urban planning structure,
(o) interest-based transformations,
and
(o) bottom-up participatory and urban design activities, in the urban realm of Savamala.

\begin{itemize}
\item \textbf{Top-down urban planning actors}:

The analysis of top-down decision-making in Savamala retains the structure of this layer identified through the data collection.
Human and non-human actors include:
(a) the regulatory framework,
(b) urban actors
and
(c) space.
\\

\textbf{Regulatory framework}: 
These actors are divided into institutions (public administration and urban planning authorities) and records (legal framework, policy agendas and technical documentation).
They are assorted on scale levels: international, national-state, regional-city and local-municipality ((\href{Section 4.2}{Section 4.2})). 

Adapted to the European administrative framework (\href{Sluzbeni glasnik}{Sluzbeni glasnik RS 09/2014}), the urban regulatory framework in Serbia has hitherto followed the recommendations of the Council of Europe 
(\href{ref}{\citealt{ministarstvo_prostora_urbani_2014}}). The set of European strategies and programmes influence the rather general organizational flow of Serbian urban institutions and assign major directives for the adaptions of urban records. 
According to the local experts
(\href{InterviewX}{Interview X}),
%(\href{ref}{IAUS interview})
the international regulatory levels do not hold direct relations to the Savamala urban environment.
\\

On the other hand, the institutional framework in Serbia corresponds to the administrative organization of the Republic. The Republic of Serbia is divided into 29 districts and 189 communes (opstine) [including the 16 municipalities of Belgrade and the  municipalities of Novi Sad, Nis and Kragujevac]. The districts act as political bodies, but they are not authorized to make their own decisions regarding spatial development. Therefore, in practice, The Spatial Plan of the Republic, Regional Spatial Plans and Spatial Plans of Special Purposes are under the jurisdiction of the National authorities. The Ministry of Construction, Transportation and Infrastructure (CTI) is the key public actor at the national level in the domain which it: (1) conducts administrative tasks, (2) governs strategic construction, site-development and infrastructure equipment works, (3) carries out survey jobs, and (4) performs inspection and supervision actions in the field (\href{Maksic}{\citealt{maksic_european_2012}}).
\\

Conversely, cities and municipalities have the legal means and rights to establish their own strategies, plans, and programs, as well as local regulations and rules in terms of urban development. Urban plans consist of General Urban Plans (GUP),
\footnote{GUDP present long-term strategic commitments and land use proposition at the city level.} Plans of General Regulations (PGR) and Plans of Detailed Regulations (PDR).
\footnote{PGR and PDR are operational documents and they are prepared where applicable.}
They cover respectively smaller territory, incorporate all sorts of innovative, strategic and up-to-date methods, and in general offer detailed solutions for issues already conceptually covered by the spatial plans. For example, General Urban Plans control development at the local level, so that they are prepared and adopted locally. Yet, as they are regarded as strategic documents (GUPs) with a certain influence at the national and/or regional level, final consent for their adaptation rests with the Ministry in competence. Local authorities adopt all urban plans and strategic documents that control urban development and comprise guidelines for the administration of their respective territories. These plans are ratified by the City Assembly.

National and city authorities, planning bodies and policy agendas are subjected to continuous pressure to solve an old issue of Belgrade’s peak waterfront area. This area is rather adjacent to or even part of Savamala, depending on the interpretations of the neighbourhood.
\footnote{According to the data from questionnaires (\href{ref}{Annex B}), there are several interpretations of what the borders of Savamala are: either a large area from Gazela bridge to behind Concrete Hall (Beton Hala) and to the Gavrila Principa in the north and along the Waterfront in the south, or several different smaller territories including the one adopted in this study. These varying ideas of the Savamala borders are also widespread among experts, professionals and citizens.}
These initiatives date back to the 1920s. The 1923 GUP announced the relocation of the railway station. Milos Somborski’s 1950 GUP formalizes the new spatial organization of Belgrade as an integrated unity of Zemun, New Belgrade and old Belgrade, positioning the Sava river and Savamala’s coast as the central urban area. Later, the Sava subway project, which was to cut through Savamala, was included in the 1972 GUP. The 1985 GUP focused its attention on the Sava Amphiteatre on the both sides of the Sava river as the prime location for urban redevelopment with central urban functions. Finally, the 2003 GP and the Belgrade Development Strategy 2008 (BDS 2008) confirmed the importance of the revitalisation of Kosancicev Venac and the rehabilitation of Savamala in the domain of Belgrade brownfields. BDS 2008 was even more specific, marking the period for the interventions (January - June 2011)) (\href{ref}{BDS 2008}).
\\

\textbf{Urban space}:
According to the technical documentation and policy agendas, the Savamala urban space is treated unilaterally or it may be eventually separated from its coastal area. Urban heritage in Savamala is also covered by the legal framework, policy agendas and present in technical documentation. Moreover, the pre-socialist past is still present in Savamala with reference to the architectural and cultural heritage of the period after the liberation of Belgrade from the Ottoman Empire and before WWII. 
Serbian rulers persistently attempted to fight substandard living circumstances in this neighbourhood and to develop a commercial and artisan town quarter and administrative centre there.
Various traces of these initiatives are still present in Savamala public spaces and the built environment’s infrastructure. For urban plans and strategies addressing the Savamala neighbourhood that are currently in force, consult the chart (\href{Chart Planovi CH5}{Chart X}) %ref ANT paper 17022016. 
\\

\textbf{Social aspects}:
Documents on urban development (development strategies, spatial and urban plans) serve to define  \underline{public interest} in cities.
Yet, the singular initiative for technical urban documentation (plans) usually comes from the investor (private or public) and is drafted based on the investor's interests and guidelines by a public or private enterprise certified for urban planning practice (\href{ref}{\citealt{ministarstvo_prostora_urbani_2014}}). %(\cite{Urbani razvoj u Srbiji Ministry of Space 2014}). 
The initiative is submitted to local authorities (e.g. the Secretariat for Urbanism, Urban Planning Institute, Municipality Planning Departments) for further procedures.
\\

The design of spatial and urban plans is under compulsory supervision of the Planning commission on the corresponding level (national, city, local). The commission validates the subjugation of the plan to urban legislation and to the planning documents of higher authority, as well as to the feasibility study of the plan and its accordance with the results of the public review (\href{ref}{Interview X})
%interview data Sekretarijat
\underline{Public review} is a filing objections process during 30 days with an open-to-public session where objections are discussed by the Planning commission. However, the report over the review of the general public and the final decision is made in a closed session by the Commission. 
When the procedure is over, the decision is published in the local gazette.
\\

Based on local experience (citizens, civil sector, urban planning professionals), political interests have hegemony over urban planning decisions in Serbia (\href{InterviewX}{Interview X}).
Civil and public interests are being neglected and diminished by the \underline{non-transparency} of the planning process and pertaining \underline{corruption} - in terms of supervisions, inspections and public hearings (\href{InterviewX}{Interview X})
(\href{ref}{\citealt{ministarstvo_prostora_urbani_2014}}).
%\href{ref}{Urbani razvoj u Srbiji Ministry of Space 2014}
Local authorities also emphasize that the lack of \underline{financial institutional capacity} (means and resources) contributes to poor public participation (\href{ref}{ibid.}).

\underline{Centralized decision-making} is even more often conducted through \underline{political dominance} than by the legal framework itself.
This is the case because the central decision-making body in practice is not the national authority, but the \underline{political party} in power;
more precisely, the  \underline{Prime minister} or the President, depending on how important and influential they are in the political realm in Serbia (\href{Questionnaire Experts Post-socialist}{Questionnaire X})(\href{InterviewX}{Interview X}).
%(\href{ref}{expert questionnaire, interviews}).
Such \underline{administrative hierarchy} extensively subdues any legal, professional and financial jurisdiction of local authorities to the National authorities, making the scope and the management of their activities unstable and difficult to strategize (\href{Vujosevic}{\citealt{vujosevic_novi_2012}}).
\\
The high politicization of institutional relations contributes to the growing imbalance between the social goal of public spending (public social services) and its developmental role (market-oriented) (\href{Questionnaire Experts Post-socialist}{Questionnaire X}).

%(\href{ref}{questionnaire data: experts post-socialist framework q13}).
In the case where political interests are often tied to individual interests of economically powerful actors and to investors' financial power, the issues of \underline{the budget} (all levels), \underline{public procurement} and \underline{public tenders} become the means of an unregulated urban economy (\href{Questionnaire Experts Post-socialist}{Questionnaire X}).%(\href{ref}{questionnaire data: experts post-socialist framework q13}).
Coupled with a flawed \underline{taxation system}, broken \underline{property structure} and inadequate \underline{land use indemnity} and \underline{land rent}, the prolonged regulatory gap in terms of investments has led to inadequate and even illegal construction practices, an overload of the infrastructural systems and an overall lowering of overall urban conditions in Serbian cities (\href{InterviewX}{Interview X}).
%(\href{ref}{interview IAUS}).

The majority of these social issues are linked to the multifaceted circumstances of post-socialist urban development and the prospects of transition toward a capitalistic social order. Post-socialist backtracking refers primarily to the present of Savamala, but in reference to the past and the characteristics of the socialist regime which are fading away (but not yet completely and not without leaving traces):

\begin{itemize}
\item \underline{state control};
\item \underline{public ownership};
\item \underline{hybrid market circumstances}.
\end{itemize}

Transitional prospects refer more to what the future brings. Transition actually means marking the processes of change towards:

\begin{itemize}
\item \underline{a democratic political system};
\item \underline{privatization} and the dominance of private ownership;
\item a market led economy and \underline{market mechanisms};
\item clear class division and \underline{uneven distribution of resources}.
\end{itemize}

The confusing overlap of these conditions has especially aggravated economic order, producing:
(1) low \underline{economic growth},
(2) high \underline{public debt},
(3) a high \underline{unemployment} rate
and
(4) \underline{poverty} germination (\href{Questionnaire Experts Post-socialist}{Questionnaire X})
%(\href{ref}{questionnaire experts post-soc framework q13}).
Such as the case is, the overall social situation is represented by rather spontaneous urbanization, ad hoc definition (or merely formalization) of public interest and the institutional adjustments to prevalent illegal construction and occupation of space (\href{ref}{\citealt{ministarstvo_prostora_urbani_2014}}).
%table?
\\

\textbf{Urban actors and stakeholders}:
Strategic behaviours and political power relations are identified as the pillars of top-down urban planning practice.
They aim at enabling dialogue between: investors, citizens, local authorities, and planning bodies.
(\href{Vujosevic}{\citealt{vujosevic_problem_2006}}).

Even though the sole purpose of the political structure, authorities, and political bodies is to define and protect public interest, this usually is not the case. More often than not, political actors act on behalf of political parties, movements and leaders, lobbysts or in the worst case of real estate investors  (\href{ref}{\citealt{ministarstvo_prostora_urbani_2014}}).
Consequently, in Serbia urban planning professionals are those who handle and balance various interests. Unfortunately, as a result of biased institutional relations and an emphasized institutional hierarchy, they usually serve only as a technical body, the staff to pursue investors’ wishes (\href{InterviewX}{Interview X}).
%(\href{ref}{interview data}).
Therefore, it is important to establish a two-way, meaningfull dialogue between: investors, citizens, local authorities, and planning bodies.

In this respect, the key top-down actor and stakeholder groups active in Savamala are
(\href{Questionnaire Experts Post-socialist}{Questionnaire X})
 %(\href{ref}{questionnaire experts post-soc framework q7}):

\begin{itemize}
\item Municipal authorities;
\item City authorities;
\item National authorities;
\item City planning departments (architects, town planners, engineers, public administrators);
\item Ministry of Construction, Transportation \& Infrastructure;
\item Professional association (architects, town planners, engineers, artists);
\item Universities and educational institutions;
\item Public enterprises;
\item Public-private enterprises;
\item Private enterprises;
\item Citizens.
\end{itemize}

\item \textbf{Interest-based real transformations}
\\

\textbf{Social aspects:}
As it was already elaborated above, the transitional reality is a thriving
ground where planning very often serves as a supportive mechanism for 
\underline{uncontrolled privatisations} and \underline{wild marketisation}.
These circumstances bring to the fore large (mega) projects instead of strategic programs.
Furthermore, influential business stakeholders and corporate bodies profit from
\textit{ungrounded institutional formalizations}, \textit{inconsistent institutional procedures} and \textit{flawed institutional processes} and above all from \textit{vertical clientelism} in the institutional framework to cater for their profit-oriented interests.
\\

\textbf{Urban actors and stakeholders}
As practiced today, the public interest is defined by the most powerful social class (enterprises, services, corporate business, politicians and ruling elites, landlords, banks, Trade Negotiations Committees) - in the brace between political and economic elites (\href{ref}{\citealt{ministarstvo_prostora_urbani_2014}}).
In the local context, these private interest are also promoted through national and local media (publicly and privately owned newspapers, TV and radio stations) and some of the mainstream intellectuals (\href{Nikolic}{\citealt{nikolic_medijska_2015}}).

The human engine of interest-based real estate transformations consists of (\href{Questionnaire Experts Post-socialist}{Questionnaire X}):
%(\href{ref}{questionnaire expert post-soc framework Q7}):

\begin{itemize}
\item Private investors
\item Private investment funds
\item The media
\end{itemize}

\textbf{Urban space}:
Powerful investors use their economic and political dominance to gain a good bargain in buying the highly profitable waterfront area of the Serbian capital and to ensure that its future development serves their needs. This battle for land started even before the official fall of socialism in SFRY.

In the late 80s, national authorities promoted the "Town on the water" project, which addressed the entire area of the Sava Amphiteatre integrally with its counterpart on the New Belgrade side (\href{Savski}{\citealt{urbanisticki_zavod_beograda_program_2008}}).
Later, the infamous Serbian leader Slobodan Milosevic supported the CIP Europolis project as part of the electoral campaign for local elections (gradski izbori) in 1995. The project was based on an international competition for the urban design of the Sava amphiteatre (\href{Savski}{\citealt{urbanisticki_zavod_beograda_program_2008}}).

After 2000, the most important projects hitherto active in the extended area of Savamala are:

\begin{itemize}
\item Lamda Development investment for Beko factory renovation;
\item The "City on water" project by the Belgrade Port Company (\textit{Luka Belgrade});
\item An international competitive bidding for architectural design of the Beton Hala Waterfront;
\item The Eagle Hills and Belgrade Waterfront Project (BWP).
\end{itemize}

According to the current state of affairs, the crucial interest-based intervention in Savamala’s space occurred in 2013 when the investor from the United Arab Emirates [\textit{Eagle Hills} Company] bought the National Shipping Company and all its land. Slowly but surely afterwards, the company set in motion a range of legislation and planning document changes to accommodate the investor’s interests within the \textit{Belgrade Waterfront Project} (BWP).
\\

%map
The exact area of intervention in these project phases varies in scope, starting from Gazela Bridge and extending to the far end of the Dorcol marina or even includes the New Belgrade riverbank. The common denominator for most projects is the coastal area on the right Sava riverbank. The majority of the projects have also advocated for the relocation of the bus and railway station. 
\footnote{The bus terminal and railway station are actually outside the area which is referred to as Savamala in this
research.  But as they are very close, as well as in a busy urban hub, they are still important generators of urban functions, activities and urban actors in Savamala.}
Furthermore, the recent Belgrade Waterfront project also targets several architectural heritage buildings. According to the agreement signed between:
(1) The Republic of Serbia (The Minister of Construction, Transportation and Infrastructure), 
(2) Belgrade Waterfront Capital Investment LLC (Mohamed Ali Alabar),
and
(3) Belgrade Waterfront d.o.o. (acting director) and
Al Maabar International Investment LLC (Mohamed Ali Alabar).
Serbian national authorities have committed to concede several protected buildings in Savamala for the investor’s purposes without any financial compensation. The agreed upon the investor’s right to choose buildings for reconstruction and lease without a fee from the list of non-contributed   ones in the area. The listed  buildings are: the Belgrade Cooperative, Bristol Hotel and the "Simpo" building inside Savamala and Railway station headquarters, the Paper Mill, Train Turn Table and Post Office outside Savamala. The buildings of the Belgrade Cooperative, Hotel Bristol, and "Simpo" are examples of architectural  heritage  of  national  and  city  importance.
\\

\item \textbf{Bottom-up participatory and urban design activities}
\\
\textbf{Social aspects}:
A most important particularity of Savamala is the rise of the civil sector and non-formal organizations, rather more typical of western European cities than of post-socialist ones. Namely, their arrival rests on the idea that, with the lack of efficient institutions and official strategies, the protection of public interest occurs through non-institutional, non-governmental organizations (\href{ref}{\citealt{ministarstvo_prostora_urbani_2014}}).

In play in the Serbian context are the cultural and behavioural patterns inherited from 40 years of socialism: (1) a predominantly  \underline{middle class society},
(2) \textit{suspicion, lethargy and ignorance} toward social participation,
(3) a top-down approach to spatial and social development and
(4) \textit{ideologically-framed civil rights} (\href{Bolay}{\citealt{bolay_instrumental_2014}}).
These circumstances entail significant resources (time, knowledge, human capital, networking) which are required  for those activities to work (\href{ref}{\citealt{ministarstvo_prostora_urbani_2014}}).
What is more, it is also essential for the organizers and participants (citizens) to have at least a vague notion of whom they might confront.

\textbf{Urban actors and stakeholders}:
Having identified the transitional capital of Savamala in the local context, a number of small-scale public initiatives and creative services have found their place in Savamala from 2008 onward (\href{Cvetinovic}{\citealt{cvetinovic_engine_2013}}).
In absence of an overall urban development strategy, these independent cultural entrepreneurs, supported by the municipality Savski Venac, have started the  transformation  of unused warehouses and craft shops into spaces open for public participation and social production. These associations and private initiatives have finally introduced an opportunity for an alternative strategic path to influence the urban development of the neighbourhood (\href{Mikser}{\citealt{mikser_festival_mikser_2012}}) and have made it well known on a global scale as one of creative clusters in European metropolises  (\href{Monocle}{\citealt{monocle_are_2016}}; \href{MikserHouse}{\citealt{mikser_house_guardian:_2016}}).

The bottom-up actors and stakeholders identified as active in Savamala are confirmed through the questionnaires and interviews in the qualitative data collection process (\href{Questionnaire Experts Post-socialist}{Questionnaire X},\href{Questionnaire Experts Savamala}{Questionnaire X},\href{Questionnaire Students Savamala}{Questionnaire X})

%(\href{ref}{questionnaire experts post-soc framework q7; questionnaire experts Savamala; questionnaire students Savamala}):
%Spatium paper (Figure 4)
\begin{itemize}
\item Citizens;
\item Local NGOs;
\item International NGOs;
\item Local community;
\item Artists and cultural workers;
\item National cultural institutions;
\item International cultural institutions;
\item National and international educational institutions.
\end{itemize}

A detailed description of the bottom-up participatory activities in Savamala is provided (\href{Section 4.2.3}{Section 4.2.3}).
Taking into account their number, but also the similitude of their activities, those that signify as poles of urban agency within this analysis are (\href{Figure X}{Figure X}): 
(1) Cultural Centre Grad (\textit{KC Grad});
(2) the Old Depository in Kraljevica Marka Street (MKM);
(3) \textit{Mikser} multidisciplinary platform;
(4) \textit{Nova Iskra} design incubator;
(5) the \textit{Urban Incubator Belgrade} project (UIB);
(6) \textit{Ministry of Space} Collective;
(7) \textit{Ne da(vi)mo Beograd} initiative (NDVBGD);
(8) \textit{My piece of Savamala} - a participatory urban design workshop;
(9) \textit{The game of Savamala} - a participatory urban planning workshop;
(10) \textit{Savamala, a place for making} a participatory project;
(11) \textit{Streets for Cyclists} NGO;
(12) Common space in Kraljevica Marka 8 street (\textit{KM8}).
\\

\textbf{Urban space}:
These civic activities have also aimed to profit from abandoned and empty spaces in Savamala. In this respect, most of the spaces were obtained for a temporal use or for use over a non-determined period by the Municipality of Savski Venac. However, some are also rented from the private owners under market conditions. The very first established place was the Gallery Magacin (MKM), a space on \underline{Kraljevica Marka 1}.
It was under the supervision of \textit{Dom Omladine} (a Belgrade Cultural Institution) and it hosted the activities of the independent cultural platform in Belgrade.
However, the crucial step in making Savamala a home for the non-formal civic and cultural scene was when KC Grad moved into the old building at  \underline{Brace Krsmanovic 5}.
This was followed by setting the Mikser Festival on the streets of Savamala and renting the old storage at \underline{Karadjordjeva 46} to accomodate Mikser House. Later on, Urban Incubator Belgrade (UIB) activated several spaces for their programme in the broader area of Savamala:
(o) the Spanish house (\underline{Brace Krsmanovica 2}),
(o) KM8 (\underline{Kraljevica Marka 8}),
(o) C5 (\underline{Crnogorska 5}),
(o) Bureau Savamala (\underline{Svetozara Radica 3},
and
(o) Nextsavamala (\underline{Gavrila Principa 2}). 
It is also important to mention Nova Iskra, a private co-working space and platform for cultural activities. It is located at \underline{Gavrila Principa 43}, a building at the very edge of what is in this research assumed as the Savamala neighbourhood.
\end{itemize}
 
\subsection{Intermediaries and mediators}

Starting with ANT and its open approach to comprise whatever may be an element of a complex urban system and the loose definition of actors relationality, the actors’actual influence on site was rather straight-forward and indicative. It has been realized that their human/material nature should be acknowledged as it unmistakably designates their roles in urban development processes. Therefore, this wide conceptual field has been shrunk to the category of an actor nature. This tells us whether the human/non-human element serves as intermediary or mediator. In this respect, the researcher differentiated its figuration as human, entity, artifact, or event, and indicated whether it is an individual or group element (set of elements). So to speak, the nature of an element defines whether it actually bears or merely changes meaning - in one manifestation they do, in the other not (\href{Figure 5 ANT paper}{Figure X}).

\paragraph{Individuals}
The distribution of roles in the Serbian regulatory framework is substantially dependent on the individual human and group actors who are assigned certain position (\href{Questionnaire Experts Post-socialist}{Questionnaire X})
%(\href{ref}{expert questionnaire post-socialist})
Namely, the institutional structure is more often than not encumbered by tripled representation of functions for certain positions [i.e. Prime Minister (\href{Questionnaire Experts Post-socialist}{Questionnaire X}; \href{Interview}{Interview X})]:
(1) the official administrative scope of the institutional role assigned to an individual,
(2) the symbolic significance assigned to the role based on the historical dominance of the individual approach over the institutional one,
(3) the individual human agency of the person who holds the position.
\\

This is especially evident when the official engagement of the institution in local networking depends on the role of its executive officer. For example, the activities of the Urban Planning Institute are often molded according to the predominantly managerial or professional approach of its lead, in which sense the entire duty of the institution varies from consultative to managerial tasks at the city level (\href{Annex}{Interview X}).
%UZ interview
\footnote{
According to an interviewee, based on the attitude of its executive officer, the Urban Planning Institute may take an active role in directing interventions and advising the city authorities on strategies and plans (\href{InterviewX}{Interview X}).}
%\href{ref}{UZ interview}.}
\\

However, another important figuration is the parallel decision-making structure installed in the Serbian institutional framework through the individuals of a political party who perform certain institutional duties.  The political party usually sets its own party staff at high public positions, so that they, as public officers, make important decisions in the public domain. But, on the other hand, they are subordinate to the party interests through the party hierarchy and they introduce their political party reasoning into public-interest decision-making. 
\footnote
{An interviewee gave the example of the public money spent to accommodate private interests,
while the financial structure of political parties in Serbia is still non-transparent and there is no legal means to investigate it (\href{Annex}
{Interview X})
%Kucina interview 2013}.
Another gave the example how a professional in high education had slowly changed his approach to the job during his term as a high officer of the political party in power(\href{Annex}
{Interview X})}
%Association of architects interview}}
%The role of a politician in Serbia - after only 1 year, they do not perceive the reality as it is, blinded by power and benefices.%
\\

Another layer of importance for human agency is added by an overlap of jurisdictions from local and international professionals.
New market conditions have brought in international corporate capital with their own business conduct and rules, and installed them into the local market in Serbia.  Namely, there is a substantial subordination of tasks from international to local personnel.
While foreign architects, engineers, project teams perform all designs, calculations and decision-making, Serbian enterprises and professionals primarily work to adjust those to local requirements and standards with significant restrictions even in this domain
\footnote{ The valid BWP agreement contains an article that obliges RS to adjust its laws and regulations if they prevent smooth implementation of the project.}
(\href{InterviewX}{Interview X}).
%\href{ref}{BWP interview}.
Even more important, it appears that different business model traditions [the socialist "to have the work done" among local professionals compared to profit-oriented approaches from the foreign professionals] influence their respective financial and managerial arrangements with the investor and in this manner direct the course and the implementation of projects.
% architects: Arabs, English, USA; SOM architects design of the tower, Energoprojekt lokalni projektant; questionable adjustment with local regulations; different business model, don't know how to charge changes and adjustments and additional tasks

\textbf{"The politicians in Serbia become blinded by power and benefits after less   than a year of working as party staff in the public domain" 
(\href{InterviewX}{Interview X}).}
%\href{ref}{Association of architects interview}.}

\paragraph{Documents}

An additional interventionist role in terms of "localizing the global" (\href{Latour}{\citealt{latour_reassembling_2005}}) is put to work through different approaches to the implementation of agreements and projects in the Serbian setting compared to that of others. For example, the Serbian urban framework does not recognize the legally binding role of a Master plan. Without discussing the political and economic articulation in the local context, the Belgrade Water Project obviously features as a source of new urban regulations. And as such, it extensively influences land use and property management in Serbia.
\footnote
{The BWP is also included in the Spatial plan for the Special Purpose Area of the Sava Waterfront. Namely, the project is part of its title - The Spatial Plan for Special Purpose Area "for the Belgrade Waterfront Project". As local experts have argued, this is an official benchmark for high urban planning authorities to accommodate the needs of a single project.}
\\

The issue of a new Spatial plan for the Special Purpose Area, Lex specialis and changes in the General Urban Plan of Belgrade set up a new order of priorities, allowances and restrictions in the local planning ecosystem. An interviewee with a background in architecture pointed out that construction indexes, once raised, will never be lowered again and that will surely change the recognizable Belgrade veduta (\href{InterviewX}{Interview X}).
%\href{ref}{interview DAB}.

\paragraph{Spaces}

Furthermore, the widespread network of civic engagement brings to the fore the mediatory function of the newly established spaces for culture and arts in Savamala. Dozen of local sites in the neighbourhood (Spanish house, KM8, Magazin, KC Grad, Mikser House, Galley Stab, HUBG12, Nova Iskra, C5, Svetozara Radica 3, Miksaliste) lodge various local, national and international organizations and actors,  tourists  and clients in overlapping time-frames (\href{ref}{Figure X}),
%see figure: time diagramme of civic engagement activities 2007-2015)
while being extensively covered by local and international media
(\cite{add references from media sources archive}).
\\

In this manner, this continuous communication of entities from different backgrounds promotes an active role of the local towards the global, compared to the passive role of construction industry professionals in BWP. An engaging example is the quick reaction of Mikser house to establish a centre for the middle-east refugees passing through Belgrade on their way to Western Europe
The promptness and efficiency of their [Mikser House] response placed them quickly into the international humanitarian aid network and their work acknowledged by the international actors in the domain (\href{InterviewX}{Interview X}).
%(\href{ref}{Mikser interview}).

\paragraph{Events}

Finally, the multifaceted nature of events organized in Savamala and their multi-scale character practically intertwine local interactions and global structures in order to:
(1) regionalize culture and artistic production in the Balkans,
(2) set a local life cycle of design,
(3) promote cooperation strategies for multidisciplinary activities and international projects (\href{InterviewX}{Interview X}).
%(\href{ref}{Mikser interview}).
Some of these include: the Mikser festival and Mikser house [established the Balkan Design Network,  , Miksaliste refugee assistance and Info Park aid - humanitarian work]; City guerilla association and Urban incubator (UI) association [brought international artists and organizations to temporarily work in Savamala]; Hub, Stab and other galleries [exibit international artworks as well]; KC GRAD [international funding and collaborations]; Magacin gallery [incentives for the national cultural scene and international collaborations], Bike kitchen, Streets for cyclists, ”Beograd Vlograd” festival [international visibility] etc.
(\href{Questionnaire Students Savamala}{Questionnaire X})
%(\href{ref}{Q15 students questionnaire Savamala}).
\\

The process of associating agency to human and non-human actors without leaving to social forces to endow it with meaning is under constant threat of a reductionist approach to uncertainties and controversies about who and what is the actual source of action (\href{Latour}{\citealt{latour_reassembling_2005}}).
In this, the interpretation of intermediary/mediator role depends on "localizing the global", "redistributing the local" and "connecting" within a zero-value map of "local interactions to other places, times and agencies" (\href{Latour}{ibid.}).
Based on the empirical data, the crucial distinction between the source of action of an individual, documentary or event actually determines the local or de-localized
\footnote{In Latour's (2005) terms, de-localization does not serve to de-spatialize the action, but to indicate that it has been disconnected and re-connected to some other place. Namely, under certain circumstances, it has been globalized.}
%ne diraj Latoura
nature of interventions in Savamala.
What is more, this is also the way to place into networks what may in other prevailing conditions be determined as uncertain, controversial or, in general, the social.

\textbf{"Even in the legal framework, in Serbia everything is left to an individual" (\href{InterviewX}{Interview X})}

\subsection{Free associations}

From the qualitative data (expert questionnaires, workshops, interviews and documentation), it has been realized that classical urban categories (the social, structure and scale) cannot be fully undermined, though they are used not as explanations, but as associations of performativity and enactment (the network of influence and socially functional categories)  (\href{ref}{Figure 6}).
Thus pertaining artifacts are also converted into actors.
In other words, these association-based actors actually operationalize urban concepts and categorize actual forces and actions.

\paragraph{Structure networks}

To answer the how question of actors' activation in networks also involves the character of their agency within local boundaries.
Institutionalized urban planning structure is under the top-down, supreme jurisdiction of the Ministry of Construction, Transportation and Infrastructure, which makes it the supreme regulatory, planning, administration, control and verification body in the urban domain.
\\

However, the city of Belgrade planning institutions manage to gain certain authority in the national discourse.
First of all, they produce massive amounts of general and detailed regulation plans which equals the production of the rest of Serbia together.
Then, they hold a special role in the national scope as the city legally has a special status
\footnote
{For example, for central city municipalities there is no need for a Regional spatial plan and Detailed Urban Plans as they are regulated with the General Urban Plan; even though GUP 2009 is only a strategic document, while 2003 GP included the articles on implementation as well (\href{InterviewX}{Interview X}).}
%(\href{ref}{Sekretarijat interview}).}
(\href{ref}{Zakon o Glavnom Gradu 2007 [Law on Capital City]}).
Finally, various regulations were once pioneered in Belgrade, like that of the Planning commission instated in Belgrade through the City Statute since 1974 and legally introduced at the national level with the 2003 Planning and Construction Act (\href{InterviewX}{Interview X}).
%(\href{ref}{Sekretarijat interview}).
%Secreatariat - article 46 Law on planning - the domain of Secretariat for urbanism
%projects of public interest and big investments - not clear how it is estimated
%Direkcija za gradjevinsko zemljiste administer these projects in public interest
\\

On the other hand, private initiatives are actually the pillars of transformations in Savamala. The case of the Belgrade Waterfront Project exemplifies that even though the Master plan is not a legally binding document in Serbia, but a simple statement of the investor’s wishes, its rules and decisions are implicitly assigned as an obligation to be incorporated in the Detailed Urban Plan for the area. The master plans in this manner become legalized and legitimized (\href{InterviewX}{Interview X}).
%(\href{ref}{Sekretarijat interview}).
\\

Similarly, even though the activities of the first Mikser Festival and later of Mikser House, aimed at culture, art and design at first, the actual implementation went into a more fair-like   and consumerist direction. In this respect, the other cultural workers in the neighbourhood have said that Mikser attracted attention to Savamala as a neighbourhood for partying and easy money and, in the long run, paved the way for night clubs, cafes and restaurants to local there
(\href{InterviewX}{Interview X}).
%(\href{ref}{KC Grad interview}).
\\

Finally, the wide range of activities instigated by the civil sector in Savamala provide evidence of an informal collaborative network that involves local and international actors and addresses the spaces in Savamala and other places. Their engagement revolves around local implementation for:  (o) promotion of urban culture, arts, design, architecture and urban design; (o) support for strategic project management, education and practice-based research; (o) humanitarian and fund-raising actions; (o) empowerment of citizen participation and local community bonds; and (o) the incorporation of a certain number of commercial, entertainment and leisure activities 
(\href{Questionnaire Students Savamala}{Questionnaire X}).
Their potential to move around the city
\footnote{Several interviewees from the cultural sector mentioned the possibility of Mikser House, KC Grad and the Galleries (Stab and HUB) deciding to move from Savamala in the very near future.}
under the unsupportable threat of the local megaproject (BWP) indicates the resilience of the constituted network.  

\paragraph{Network of influence}

According to the detailed analysis of the decision-making structure in Belgrade and in Savamala (\href{Section 4.2}{Section 4.2}), it is conspicuous that there are several scales of the distribution of agency in the local context throughout different layers of decision-making.
In urban planning discourse, the issue of internationalization is present to a small extent in the adjustment to the European Union urban legislative. As the pace of the joining process is slow, slow becomes the change of the system as well. 
However, a certain confusion in the local context is produced by continual shifts of jurisdiction on certain issues from top-down and then from ground-up through the hierarchy of the regulatory framework (\href{InterviewX}{Interview X}).
%association of architects}).
With a lack of insight into the judiciary structure, citizens, stakeholders and investors usually resort to individual sources of authority in public institutions (\href{InterviewX}{Interview X}).
On the other hand, in the historical overview of real-estate transformations after 2000, the influence of international corporate actors and investors has been indisputable (e.g. Beko Factory, BWP etc.).

Finally, in the so-called bottom-up network of engagement, international actors usually serve as source, support and manager of local actions.
Even though their role is usually described as empowering and/or leading, the management strategies of international organizations in the civil sector are also indicated as manipulative in how they formulate the actions and adjust them to their own goals rather than inquire about and investigate what the [goals] of the local  population are (\href{Questionnaire Experts Savamala}{Questionnaire X}, \href{Questionnaire PhD Savamala}{Questionnaire X}, \href{Questionnaire Student Savamala}{Questionnaire X})
%(\href{ref}{expert and phd student questionnaire and workshop}).
The local distributors of international actions are: the Goethe Institute Belgrade, KC Grad, Mikser House and Urban Incubator.
While citizens, young professionals and students become mere participants/clients of these also top-down built agency networks (\href{Questionnaire Experts Savamala}{Questionnaire X}, \href{Questionnaire Students Savamala}{Questionnaire X})(\href{InterviewX}{Interview X}).
%(\href{ref}{expert, student questionnaire Savamala, citizen interviews}).
The only real bottom-up actions may then be small-scale and sporadic initiatives of several citizens to either help minorities in their neighbourhood, or renovate parts of common spaces, or to contribute  something that the neighbourhood is in need of (a tap with fresh water in the waterfront area) at their own expense. (\href{InterviewX}{Interview X}).
%(\href{ref}{citizens interviews})

This further deconstruction of the actor roles serves to reveal rather an internal networking than an external one. Namely, the structure and scale internalized in the social manifestation of the analysed actors offer a possible perspective on how they engage in networks and bring up certain social constellations (urban development prospects).

\subsection{Stabilizing and destabilizing agencies}

In general, the actors were identified according to their social function/action in the urban realm, and accordingly "flattened" and re-addressed from an ANT standpoint. According to the qualitative data collected, mainly through non-structured interviews, it is apparent that analyzing functional and supportive agency brings an additional layer of explanations of urban reality. Apart from intermediary/mediator roles and associations, this interpretation brings in another type of internal networking (\href{Figure 7}{Figure X}).
\\

In fact, the differentiation of functional and supportive networks indicate the possibility that actors change their roles by altering their internal network engagement. Socially functional networks indicate the social category of actors in reference to general categories of the politico-economic social order (political role and the system of finance). The notion of supportive/secondary networks is laid out more as a significant subset of the actors’ agency already figuring in any of structural networks. However, secondary networks explicate their bipolar character and their presence in more than one internal network simultaneously. As a primary role is associative within structural networks, a secondary role might be stabilizing/destabilizing and it figures within supportive networks. 

\paragraph{Socially functional networks}

The most important issue of these networks is to (re)- distribute actions across the local social realm. Of special importance is the fine tuning of a range of public actors: polity, public institutions, urban authorities, public enterprises and public utility companies. The data on these socially functional networks are extracted from the questionnaires among senior and young experts and young professionals in urban planning and architecture. 
\\

An interesting example is the case of Urban Planning Institute. Even though it is the most important consultant of the city authorities upon urban development and the major planning body in the city, it is no longer financed directly from the city budget. According to the informant and publicly available data, the Urban Planning Institute is financed from public procurement at the city level and from the financial means of the Belgrade Land Development agency for public sector engagement at the city level. Moreover, the institute also acts as a private company, engaged by the public sector through public procurement at the national level, as well as competing for other privately financed jobs on the market (\href{InterviewX}{Interview X}).
%(\href{ref}{UZ interview}).
The procedure is the same, with the client either a private company or a public institution, while the conduct may vary depending on the clientele relationships within the institutions.
\\

A special place is devoted to the rising agency of public-private partnership: BWP, public transportation in Belgrade, ”the Dom Omladine” cultural institution; as well as those in prospect: Sava Center congress hall and Airport Belgrade. While private enterprise is a sole metaphor of private interest, the agency tracking within networks indicate that it is very common that public-oriented actors actually pursue private interests through their activities.
\footnote{This is especially evident in the agreements and activities of the authorities around the Belgrade Waterfront Project. (\href{InterviewX}{Interview X})}
%(\href{ref}{Ministarstvo prostora collective interview})}
\\

The representatives of the civil sector include both formal and informal organizations. The informants indicated a wider range of categories [i.e. national authorities, cauthorities, municipal authorities, the Ministry of CTI, city planning departments, research institutions, universities and education, professional associations, public enterprises, international NGOs, the local community, local NGOs, the media, political parties, public-private enterprises, private investment funds, private investors and citizens) than had been aggregated to the distinct public, private or civil agency of actors.
\\

An example is the importance of public enterprise and public utility companies for local planning, namely on the scope of physical and practical constraints they set on spatial interventions and building (\href{Questionnaire Experts Post-socialist}{Questionnaire X}).
%(\href{ref}{expert questionnaire post-socialist}).
An interviewee from the public sector stated that, during the plan preparation procedures, the Urban Planning Institute usually has to consult between 50 to 100 public institutions regarding their requirements and constraints for planning and construction (\href{InterviewX}{Interview X}).
%(\href{ref}{UZ interview}).
In the case of BWP, the right bank of the Sava in Belgrade is of substantial interest to several public enterprises (Javna Preduzeca) and public utility companies (Javno-komunalna Preduzeca), such as (\href{InterviewX}{Interview X}):
%(\href{ref}{IAUS interview}): 

\begin{itemize}
\item Coastal services: Serbia Water-management company and the Directorate for Inland Waterways;
\item Railway transportation company;
\item Belgrade waterworks and sewage;
\item Belgrade Land Development agency;
\footnote{As the financial management office for real estate in the city}
\end{itemize}

The very important question is their articulation in local networks and the discrepancy between the real and formal role they take in the cycles of urban planning and implementation.

\paragraph{Supportive/secondary networks}

Another very important issue in terms of stabilizing and destabilizing agency in urban space is the actual relationship with space which is in itself incorporated in the actor's nature.
The extended list of urban functions was identified in Savamala through the qualitative inquiry
\footnote{These function are: culture, transport, commercial, abandoned areas, leisure, residential, educational, public services and industrial}
(\href{Questionnaire Students Savamala}{Questionnaire X}).
%(\href{ref}{Q11 students questionnaire Savamala}).
Apart from primary functions, which in this case may serve as secondary,  the secondary networks are summarized separately and also involve: urban related, space related, data related, non-governmental, infrastructural, services and transportation related issues (\href{Questionnaire Students Savamala}{Questionnaire X}).
%(\href{ref}{Q11 students questionnaire Savamala}).
In addition, these added categories are a significant stabilizing/destabilizing  source  of  agency  in  Savamala.
\\

In these circumstances, it is crucial to mention how the rising global trend of practice- based research and education, which is coming from outside the formal institutions, have also entered Savamala through the initiatives of the Goethe Institute (international actor), Mikser House and KC Grad (national actors). These actors gather cultural workers and associations, young academics, architects, designers, and young people in general around methodological (School of urban practices), educative (The game of Savamala), participatory (My piece of Savamala), practice-based (Urbego), and urban related activities (\href{InterviewX}{Interview X}).
%(\href{ref}{Kucina interview 2013+2015, Mikser interview}).
\\

Moreover, the relationality to space in Savamala was kept above all as a question of resources - either for alternative culture, artists, social organizations, service-oriented entrepreneurs
\footnote{The founder of a bike tour company indicated that Savamala was the perfect place for his business, because of its location and the density of tourist-oriented urban activities in the neighbourhood. (\href{InterviewX}{Interview X})}
%(\href{ref}{private sector interview})}
or even for local residents.
They have all been in need of space for their rather diverse interests (\href{InterviewX}{Interview X}).
%(\href{ref}{intervies: KC Grad, Citizens, private sector}).
In this respect, the actions of the non-governmental sector has been marked as limited as it was not  participatory enough nor enough bottom-up - the activities were not coming from the local community and did not represent local needs (\href{InterviewX}{Interview X}).
%\href{ref}{interview citizen, private}.
The framework of what were initially announced as bottom-up actions were actually rather imposed from the top- down by international actors and local action distributors.
\\

Finally, the issue of infrastructure (connectivity) and transportation (accessibility) in technical terms is marked as a materialization of the level of urbanity of a location (\href{InterviewX}{Interview X}).
%(\href{ref}{Association of architects interview}). 
In this manner, Savamala has high importance and high rank in the city, but according to the artificial market conditions
\footnote{
The approach to urban land regulation in Belgrade and Serbia is more administrative than market-oriented, yet construction land management takes place according to real-estate market rules (\href{Zekovic}{\citealt{zekovic_spatial_2015}}).
%(\cite{Zekovic and Maricic, 2016 land market}). 
The series of instruments work in favour of the real, functional real estate market rules: conversion of land-use rights, leaseholds on urban (construction) land, no taxation of land rent etc. (\href{Zekovic}{ibid}.).}
in the real-estate market in Belgrade, the actual value of land and building stock does not always correspond to the real, material value (\href{InterviewX}{Interview X}).
%ZEkovic interview

\textbf{Stabilizing and destabilizing agency of functional and supportive networks}
\\
While tracing interactions and interconnections among actors collected through participatory action research methods (\href{Table 3 Savamala PUD}{Table X}), it was revealed that various social manifestations of these actors have a double effect. They can either work stabilizing (practice-based urban related research) or (de)stabilizing (public utility  companies in planning, non-governmental actions in Savamala). In both ways they offer another  "reading" of the social world in Savamala.
\\

Research activities taken up by the non-governmental sector and international actors provided an elaborated picture of what can be found on the ground and how it can be put into action with minimal financial means. An example is The Model of Savamala project within the Urban Incubator which provided detailed data on the physical and social structure of five streets in Savamala. The physical model presents the area and saturates the represented built structures with parallel data on social structure and local information and knowledge about the quarter. The Model was exhibited to the public for about two years with the aim to expose local knowledge and provide a time-space-data vision of the neighbourhood. With such activities, Savamala’s vivid cultural present was supported with a layer of verified data and elaborated knowledge.
%pictures from the model of Savamala
\\

On the other hand, public utility companies directly influenced by the BWP chose to stay quiet about the actions that threatens their  property  and  activities  and, in general,  endanger the public interest. The voice of the coastal services has not been publicly heard in the case of illegal coast fortification. Similarly, the railway transportation company did not react over Lex specialis and the land offered to the foreign investor, a large part of which is the property of the railway company. While this is not the usual behaviour of these actors, in this case they contributed to destabilizing the procedures of how planning is administered and managed in Belgrade.

\subsection{Urban assemblages}

After illustrating the Savamala urban environment through the actors, their figuration and agency, the complexity of its urban development was interpreted through node-link reality (\href{Figure 8}{Figure X}).
Taking into account the post-socialist context, significant pressure from private investors and the articulation of civic initiatives and participation in Savamala, the network of translations were identified to refer to different layers of decision-making.
\\

These translations consider the centrality of actors and the nature of the links among them. In this sense they represent an ”assemblage” process of agency dissemination. These overarching urban assemblage networks are: management, verification, consultation, administration, planning, construction, regulation, control, finances, implementation and social aspect networks (\href{ref}{Figure 4}) (\href{Questionnaire Experts Post-socialist}{Questionnaire X}).
%(\href{ref}{expert questionnaire post-socialist}).
They encompass a significant number of humans and non-humans, their actions, agencies and forces. They all have a figuration in Savamala, which allows the outlining and tracing of the distribution of any political, economic and cultural repercussions among them (\href{Table 5 and 6 Mira charts}{Chart X, Y}).
\\ 

As a result, the full congregation of urban assemblage networks in Savamala reveals different orders of things/actors. From Latour’s perspective on how the social may be reassembled, there is a certain redistribution of the decision-making layers in this case study (\href{Latour}{\citealt{latour_reassembling_2005}}):

\begin{enumerate}
\item "localize the global" (governance)
\\
Interestingly, the main agents in setting up the real environment of the new transitional circumstances (the neoliberal market, democratization of the social realm) in Savamala, do not originate locally, but come from either powerful international investors/investment funds intervening in the real-estate or from international formal/informal organizations and NGOs engaged in what is popularly known as bottom-up activities. The engagement of these actors, even though different in its actualization (real-estate and bottom-up) is in fact effectuated through the same networks of conduct: financial, managerial and implementation. In reality, these actors also act as supreme decision-making bodies, as a type of top-down authority instating the issue of the network of influence as the important question in the local context. 

\item "re-distribute the local" (operationalization)
\\
Recognized top-down urban planning actors are active in the planning, regulatory, and consulting urban assemblage networks. Instead of instating urban strategies and distributing tactical operations and interventions in urban space, the pillars of urban regulatory framework in Serbia, in the case of Savamala and the Sava waterfront, took a completely subordinate position and acted as an executive body of private interests defined elsewhere.

\item "connecting sites" (actualization)
\\
Finally, what happens on site in Savamala is the fragmentation of spaces at different levels. This far-end decision-making is actualized in terms of Belgrade Waterfront construction activities (assemblage network) or it is the local administration of projects, events, and activities prepared away from and unaware of citizen opinions and needs (\href{InterviewX}{Interview X}).
%(\href{ref}{citizen interview}).
In both cases (real-estate transformations and bottom-up activities), there are certain controversies raised from the amenability of urban assemblage networks through the networks of influences (international-state/national-city/regional-local), or, in other words, directional subordination from top down.
\end{enumerate}

Urban assemblage networks, when approached in their totality, represent a processual construction of the Savamala neighbourhood through the ANT lens. Namely, the combination of external networks (assemblages) and internal ones (the nature of the actors) holds in itself the actualization of the social relations of power, influence, class and capital, rather than having that as a starting point. From this perspective, ANT analysis of the Savamala neighbourhood contains answers to how and why certain urban development prospects of this post-socialist neighbourhood have come into being.

%Mikser cooperation with: artists, architects, designers, organizations, collectives, gallery kolektiv, city guerilla, Goethe institute and with all international cultural centers in Belgrade, embassies, international and local NGOs, faculty for management (in Savamala), faub, landscape and forestry faculty, municipality savski venac, local community of the citizens of Savamala

\section{Urban Assemblage Map}

At the final stage of ANT analysis, the contribution of this research project is the translation of what has been perceived through the 5-step ANT framework of Savamala urban complexity onto its visual map of ANT relations (\href{Figure 9 ANT diagram}{Figure X}). 
\\

Data are collected from context-based information and knowledge and also traced from relevant influences, interests and interpretations of Savamala. The agency and relationships of the human/non-human actors chosen here are tracked by their associations within different levels of decision-making (top-down urban planning, interest-based transformations and bottom-up participatory and urban design activities) in a visual manner. The resulting actor-network map is a node-link illustration of the present day urban complexity in Savamala.  The visualization strategy in terms of categories comes from the adopted ANT elements:  each node is a human/non-human entity (category:  nature of actors) visually interpreted through mediator, association and agency properties (nature of networks and networks of influences) while the number and quality of links (nature of assemblages) represent the type and number of urban assemblage networks they contribute to.
\\

First of all, the potential of such illustration of actor-networks at the local level is in its strong relationship to ”the global”. Moreover, this type of  visual map of actors and the relations they build contain information loaded associations (nature, type, primary \& secondary function, scale of influence).
While actors are nodes whose form depends on their intermediary/mediatory role, their size indicates influence, the fill and outline represent their primary and secondary function, their location in the cycle corresponds to their social function, and their proximity to the center is their network of influence.
The connections between them are assemblages
\\

Yet, such an interpretation could not bias the potential reader, as it is without any notion of value or meaning initially inscribed in it. Namely, the networks might be interpreted differently according to the interpreter’s background and interest, but still keeping the minimal amount of information already inscribed in how the networks are visualized. Another quality may be its data saturation and contingency and its capacity to contain the complexity of the social world.
\\

The introduction of qualitative tuning for nodes-actors gives this diagram a self-containing character. Namely, its advantage to other ANT diagrams is that this one embodies internal networks (the nature of elements) as well external ones (assemblages). Moreover, this diagram aims at keeping the relation to spaces as well by making concrete spatial references of the social distributions (actor-networks) to the exact places on the map (\href{ref}{Figure XX}).
%add another diagram with a map
\\

In sum, the combination of such traits facilitates the digitalization of the diagram and keeps it strongly related to reality.  Digitalizing the diagram may enable adding a 3rd dimension to it and visualising the time component through the stack of parallel layers. Even though piled, these time-space realms of the social are also interconnected. Therefore, it is also essential to overcome such intersecting and represent the social as a continuity, what it actually is in reality. 
  
\subsection{Mapping actor-network distributions in urban decision-making}

Bearing in mind that the initial actor/actor groups are identified through the morphology of urban decision-making, this extensive ANT analysis has argued in favour of representing its amenability/conduct in terms of the overlap and collision of different urban assemblage networks.
\\

Most of the preexisting methodological approaches in urban development studies consider certain socially bounded explanations such as the dichotomies of importance-influence, impact-priority, power-interest, support-opposition, and constructive-destructive attitude as self-containing explanatory categories for mapping actor and stakeholder engagement in the social realm (\href{Mathur}{\citealt{mathur_defining_2007}}).
%(\cite{Mathur}).
However, the ANT approach starts from the other end, flattening the social unity of all human and non-human actors. Only afterwards, do the generating networks in themselves contain information on the social world. In this manner, the ground-up ANT analysis performed herein provides the answers on how urban decision-making is processed in Savamala and enables an argumentation for why these social dichotomies are still at stake in post-socialist neighbourhoods.
\\

Chart X %ANT paper first table aspects
incorporates the listing of many actors, in the reference to space in Savamala and their distribution through the morphology of urban decision-making.
Through the chart, the main agency of action is associated with the actors of the urban regulatory framework.
Moreover, their interconnections  with other actors through urban decision-making layers are also characterized by the social effects they produce within these actor-networks (political, economic and cultural aspects: 1-22).
\\

\textbf{Biased Regulatory framework}
\\
The most obvious and even self-evident factor of local urban planning is contained within the agency of regulatory framework actors.
However, a significant space fragmentation in their approach must be admitted. Namely, their interest and action are almost exclusively oriented towards the highly-attractive waterfront area, without taking into consideration the potential and development status of the already established civil and cultural agency in the upper Savamala.
\\

As \href{VUjosevic}{Vujosevic (2012)} states - urban decision-making in Serbia in general is rather the combination of crisis-management, supporting privatization, the market-oriented and project- led conduct of technical issues than a critical overview of local factors and global methodological shifts in planning and the acknowledgment of stakeholder collaboration and strategic governance. While the 2003 Planning and Construction Act, as well as the 2009 General Plan of Belgrade 2021 show an improvement in terms of a strategic approach to urban development, their loose connection to implementation networks produce certain regulatory gaps when it comes to public administration and city planning authorities.
\\

As more than one informant explained the situation - city planning authorities are used to approaching urban planning as a procedure embedded in the legal framework, so that any less deterministic attempt ends up either in perpetual adaptations of legal documents and technical documentation or in arbitrary decisions on priorities and projects. In this case, most urban planners and public representatives have not shown enough vigor, interest and professional necessity to expand their knowledge over new global trends in planning and the radically changed circumstances of transition.
\\

From the institutional perspective the sole solution is seen in the hyper production of policy agendas and technical documentation and their constant revisions without concrete and operational implementation mechanisms.\footnote{
Substantial changes to the once adopted 2003 Planning and Construction Act happened in 2006, 2009, 2011, 2014. As for general plans, the 2003 GP Belgrade was updated in 2005, 2007, 2009, 2014 and 2016. The most important changes were in 2009, when it officially does not comprise any prescriptions on implementation, and in 2016, when it was redefined as the General Urban Plan of Belgrade.}
According to the elaboration of decision-making amenability through urban assemblages networks (\href{Section 5.1.5}{Section 5.1.5}), the urban regulatory framework in Serbia does not hold any effective means of control and verification.
By confronting internal and external networks of the actors engaged in control and verification assemblages, it may be concluded that they do not go beyond mere institutional formalizations, which are either not applied in reality or their application is rather bogus/phony/artificial.
\\

This trend is even more at play for numerous policy agendas.
The conditionality of rules and strategies does not only depend on political and economic influences, but also on inadequate policies and instruments for its conduct and management.
An example is the loss of conducting agency through the networks of implementation even though all actors of conduct were defined. The detailed Implementation programme for the Spatial Plan of Serbia 2010-2020, prescribed who (The Ministry of Finance), what (urban rehabilitation, environmental protect and tourist strategy) and how (budget) a small rehabilitation program of Belgrade waterfront areas was to be conducted. A year later, in the report the issue is marked as ”data not available”, and by the end of the first phase of the implementation plan (2015), significant demolitions occurred on the Sava waterfront, and the first megastructure arose although it was not mentioned in the strategic priority number 51 of the implementation programme just described.
\\

These and similar practices were made possible usually by the politically biased roles of individuals within the institutional framework (\href{Section 5.1.2}{Section 5.1.2}).
The interest-based pluralist political life also sneaked into the urban planning domain. The political background of actors in the urban regulatory framework has made planning networks at some points coincide with either administrative or financial networks. However, in both cases urban planning institutions are deprived of their professional, strategic and public-interest role, and planners consequently end up deprived of binding authority and professional dignity in carrying out their public functions.
\\

Another position of the conflict of interest may be that of the institutions when they have overlapping roles in the public domain (and consequently overlapping assemblage networks), so that all their institutional activity becomes flawed. This was the case of the Ministry of Construction, Transportation and Infrastructure after the discontinuation of the Republic Spatial Planning Agency. The Ministry became the responsible body for both regulatory and executive tasks in the public domain of urban planning.
\\

On the other hand, the Cadastre has held the position as the supreme body for defining criteria, means and methodologies for land and building assessments and the executive body for performing assessment tasks for a long time. The position of the conflict of interest on various levels is also a point of departure for introducing individual interests into the institutional framework. The still centralized structure of urban decision-making (with the Ministry of construction as the supreme decision-making body in planning, implementation, control and verification networks) and the concentration and cooperation of political and financial powers under the new demand driven economic model instate vertical clientelism and powerful economic actors as the know-how of doing business in the post-socialist cities in transitional countries.
\\

\textbf{Powerful Private Investors}
\\
The transition from a planned to a market-based economy created a certain void in political and social practices and in the aspects and solutions of the legal and economical frameworks. Rudimentary market-based regulatory instruments enable powerful financial actors to be individually involved in the building process. Not only do building codes and regulations become defined by investor interests, but they also profit from unregulated urban economy incentives and measures and gain valuable urban land in public to private ownership transition processes.
\\

In these circumstances, financial engagement of the public institutions (Budget of the Republic and its decision-making bodies - the Government and the Parliament) also becomes problematic, such as the growing imbalance between the social role of the budget (public social services) and its developmental role (market-oriented). The prime example of the kind is the adaptation of the urban regulative and public-private partnership built around the Belgrade Waterfront Project.
\\

Most of the regulatory, planning and implementation processes around the project are conducted behind closed doors and have only become introduced to the public through the interventions of the Transparency Serbia NGO, the National Anti-Corruption Agency and the NDVBGD collective. A set of official decisions has been made in order to enable smooth conduct of the project: (1) The Government Resolution (Decision) introducing BWP as a project of special importance for the Republic, (2) changes of the General Plan to enable construction of high rise buildings on the Sava waterfront, (3) Spatial Plan of Special Uses explicitly formulated to accommodate the interests of the BWP investment group, and (4) Lex specialis on the property issues in the area.
\footnote{Apart from biased legal reasons and incentives for most of these legal documents, very important is the posteriority of some of them. Namely, several decisions targeted the project, Belgrade Waterfront Master plan, but in the manner of its formulation it was obvious that these documents were also the source and the cause of the decision (Government Resolution, BWP SPSP) (\href{Izvestaj}{\citealt{pravni_skener_alternativni_2016}}).}
%(\cite{Alternativni izvestaj grey lit}).}
\\

Most of these decisions were followed by inconsistent institutional procedures and mainly happened without public insight into the procedure and documentation (\href{Izvestaj}{\citealt{pravni_skener_alternativni_2016}}).
%(\cite{alternativni izvestaj}).
The approach of public institutions and urban planning bodies in this case acted without any major concern regarding the public interest in the case and what the wider space-time concerns of such projects are.
\\

Urban planning is, in fact, deeply embedded in the context of the transition towards a service-based society, where planning law and planning practice have not yet managed to integrate physical planning, economic factors and market mechanisms into urban interventions that comply with public interest and outweigh mere growth without development actions  (\href{Vujosevic}{\citealt{vujosevic_collapse_2010}}).
An example of a biased, potentially financially dangerous binding document for the city of Belgrade and the Serbian society
\footnote{
According to the agreement, a part of the obligations of the project are transferred from the city to the national level (\href{ref}{JVA 2015}).}
is the agreement that was signed in April 2016 by:
(a) The Republic of Serbia (The Minister for Construction, Transportation and Infrastructure),
(b) Belgrade Waterfront Capital Investment LLC (Mohamed Ali Alabar),
(c) Belgrade Waterfront d.o.o. (acting director),
Al Maabar International Investment LLC (Mohamed Ali Alabar).
\\

The  agreement  was  made  public  only  three  months  after  the  agreement  was  signed  and after terrain  clearance  and  the  construction  had  already been  launched.
\footnote{The document  was published on the website of the Government in September 2015- a full version in English of 259 pages and the version in Serbian of only 69 pages. Taking into account the language barrier, the exact details and consequences of the contract are still not available to the general public in Serbia.}
The Republic of Serbia obliged itself in the construction, financial, regulatory and administrative conduct of the project (assemblage networks of the project).
\footnote{The Republic of Serbia is obliged to: (o) perform infrastructural works at the location; (o) exempt the investor of infrastructural equipment fees; (o) confer the property rights for architectural heritage buildings to the foreign investor Bristol Hotel, Railway station headquarters, Paper mill, Train Turn Table, Post Office; and the first in line to be contributed are Belgrade Cooperative (already in possession), Simpo and Iskra (from the beginning of 2017); (o) guarantee for additional loans not predicted by the contract or the feasibility study provided by the foreign investor but guaranteed by RS; (o) take loans for infrastructural works; (o) enable future conversion of property rights to the investor without compensation; (o) adjust the legal framework to ensure the rights stated in the agreement. On the other hand, the official financial binds of the foreign investor are not 3.5 billion euros as was advertised in the media  (\href{Politika}{\citealt{politika_zemljiste_2015}}), but 150 million euros, as a loan with no obligation or any guarantees for the project’s implementation. Moreover, the agreement gives the rights to the foreign investor to request infinite project changes and the adjustments of the legal framework accordingly. With all these contributions, Serbia remains
the minority owner of the BWP company and the future profit of the project.}
Above all, the agreement constrains the Serbian institutional framework to prevent any verification or control activities addressing the project or the area of its interest.
\\

In the end, the most disturbing fact might be that national and city institutions have consequently no influence on the technical urban planning documentations for the location. The Republic obliged itself to establish a state agency for Belgrade Waterfront legal adjustment tasks at its own expense and that, once having the use permit, the investor obtains the full property right of this most valuable land in Belgrade. 
\footnote{This also means the right to sell it without any influence from the local or national authorities. Even during the construction phase, Belgrade Waterfront d.o.o. was conferred the right of land use and the collection of all the profit from temporary structures and advertisement on the territory. During 2016, the company started sub-renting spaces under non-transparent conditions (Restaurant 1905, Eagle Hills and construction subcontractor offices, Savanova restaurant).}
It is not only a case of unequal distribution of resources, but a certain practice of abolishing the country’s sovereignty and territorial integrity over the Belgrade Waterfront plots of land is also at play.
\\

The strong space relation of the foreign investor's activities speaks of the local incapacity to address and solve the issue of the prolonged regulatory gap in terms of investments (\href{Questionnaire Experts Post-socialist}{Questionnaire X};\href{Questionnaire Experts Savamala}{Questionnaire X}).
% (\href{ref}{expert questionnaire}).
In this manner, the secondary network involvement of Eagle Hills (the initial company of the foreign investor) indicates that the representatives of corporate and international capital manage to find the weakest and most profitable point of entrance into the local market - the unregulated and still centrally governed land market in Serbia.
\\

\textbf{Un-institutionalized culture}
\\
%\href{ref}{ref Spatium article})

The core of the analyses also include the publicly present civic and private organizations.  Even though  they most often pertain to either  urban or NGO secondary networks, several of them have an unclear and non-transparent funding structure - while they receive some public funding, they are also partly profit-oriented (KC Grad, Mikser). Even though some of their activities are publicly funded, KC Grad and Mikser also incorporate profitable services (café-bars, shopping areas, concerts, exhibitions and other lucrative events/activities). While for KC Grad, the sponsors and partners are  publicly presented on their website, this is not the case for Mikser. Nova Iskra is the only explicit privately-based organization.
\\

The actors’ social function is strongly connected to their level of influence in this case. All these bottom-up actors are active at the local level, less often the city level, and usually on the international level. Though their international role is rather passive and their international visibility is more in the domain of funding - several are recipients of international financial support (foreign embassies and foundations, European cultural and art organizations and programmes) or under direct supervision of international entities (Urban Incubator Belgrade was the initiative of the Goethe Institute). This character of civic activities is also the result of cuts in national and city budget spending for culture (\href{InterviewX}{Interview X}).
%(\href{ref}{expert interviews}).
However, there are others with a transparent financial scheme (Ne da(vi)mo Beograde initiative). 
\\

This concentration of culture, creativity and innovation in Savamala also results from transition toward neoliberal markets and the country's opening to global funding, trends and guidelines (\href{Questionnaire Experts Savamala}{Questionnaire X}).
%(\href{ref}{expert questionnaire}).
This limited scope of intervention and radical change of urban vision from "big is beautiful" to small and private is the prototype of the non-intrusive commercialization of arts and cultures as well (\href{InterviewX}{Interview X})
%(\href{ref}{expert interview}).
Namely, the creative cluster in Savamala was not anything unique, but was rather a typical example of the current European wave of hype urban culture (\href{B92}{\citealt{b92_savamala_2015}}).
Thus, the local cultural, artistic and civic scene shows signs of total dependency on global trends and guidelines rather than an independent and bottom-up movement. 

\subsection{Mapping urban agency and social aspects}

In Savamala, the identified dynamic, interactive actor-networks were articulated through decision-making mechanisms of (1) top down planning, (2) interest-based real estate transformation, and (3) co-design and creative participation actions. In this case, global and local political, economic, and cultural factors, placed in a particular spatially and socially constrained context (Serbia, Belgrade, Savamala), are the main forces of urban development and they constitute social artifacts (actors) and social aspect networks (urban assemblages) (\href{Table 3}{Chart X}).
The detailed mapping and visualization of these actor-networks also accounts for contextual, post-socialist and transitional circumstances, but by avoiding explanations coming from the reproduction of the social order, power and class.
\\

In other words, the collision of these grand narratives is present in the current Serbian context through:
\begin{enumerate}
\item  the crisis of common social values and civic society standards, 
\item the lack of healthy investment interest and fair competition, 
\item the absence of concern for the public interest and public good,
\item a battlefield of significant power pressures and interference of interests from authorities, business actors and civil actions.
\end{enumerate}

Based on the performed ANT analysis on Savamala, urban actor-networks and their distribution within urban decision-making layers, a general summary of the urban development prospects in the neighbourhood includes the following: 
(1) a lack of elaborated, strategic policies in urban development and investment;
(2) a cumbersome institutional structure;
(3) the distribution of publicly owned empty plots and spaces in Savamala to private investors/owners;
(4) vertical clientelism in the institutional framework (\href{Vujovic}{\citealt{vujovic_belgrades_2007}});
(5) up-to-date legal documents and policy agendas which do not correspond to urban reality;
(6) an overpowered and personalized Nation State as a key actor on citywide scale (BWP example);
(7) semi-legal institutionalizations become official practice and a pool of opportunities for future exploitation;
(8) provision of instruments for powerful actors to realize their interests through controversial institutionalizations;
(9) unregulated economic incentives and measures;
(10) economic aspects strongly influence political aspects and actors in the post-socialist context;
(11) institutionalization of the private interests of powerful economic actors and marginalization of civic initiatives and public interest;
(12) "growth without development" (\cite{Vujosevic and Maricic, 2012}) rooted in the top-down approach to regulatory, managerial and financial networks;
(13) privileged foreign and domestic developers in the Waterfront/ Sava amphitheatre/ Dorcol Marina Redevelopment (\href{Djordjevic}{\citealt{djordjevic_system_2009}}), (\href{Expert Workshop}{Workshop 1});
(14) political actors in Serbia have support for the replication of extreme neoliberal practices, following the Thatcher-Reagan model  (\href{Expert Workshop}{Workshop 1});
%(\href{ref}{workshop data});
(15) housing and commercial purposes for 80\% of BW spaces (\href{Zekovic}{\citealt{zekovic_megaprojects_2016}}); 
(16) spatial fragmentation and unequal distribution of resources in the Savamala - Waterfront and Upper (Urban) Savamala;
(17) the lack of participatory and communication culture;
(18) the biased role of the media in advertising urban projects (BWP);
(19) the apathy of the population concerning semi-legal, anti-constitutional, neglected public interest issues in BWP;
(20) the lack of long-term strategic approach to cultural institutions and agendas - activities and initiatives (such as those in Savamala) are short lasting with no certain future (\href{InterviewX}{Interview X})
; 
%(\href{ref}{expert interview Petovar});
(21) civil initiatives in Savamala have neither social nor political power, nor sufficient public support and funding (\href{InterviewX}{Interview X});
%(\href{ref}{expert interview Petovar});
(\href{Table 5, 6}{Chart X, Y}). 

The dissemination of these important factors through the distinguished urban assemblage networks offer an overview of how the pre-socialist, socialist, post-socialist and transitional in the Savamala neighbourhood have merged into its current urban reality.
Based on the visual material (\href{ANT diagram}{Figure X}; \href{aspect tables 5,6}{Chart X, Y}), the effects of these urban development prospects (1-21) within urban assemblage networks are such that:
\\

\begin{itemize}
\item \textbf{Managerial}
\\
The pillars of the managerial amenability of tasks at the neighbourhood level in Belgrade are the functions of the City Mayor, City Architect and City Manager as introduced by the 2002 Law on Local Governance (\href{Vujovic}{\citealt{vujovic_belgrades_2007}}).
More generally, it is the role of the Minister of Construction, Transportation and Infrastructure at the national level.
In practice, these functions have shown to be political party figures and crucial links in distributing central decisions (government, prime minister) at the city level.

For example, in the case of BWP, although the Prime Minister was the leading figure in the negotiations and deals with the Arabian investor, the agreements were signed by the Minister and the City Mayor in the name of the Republic of Serbia
While taking all the credit in the media for the announced success of the project, the Prime Minister has enough power to distribute duties and avoid direct responsibility in this obviously disputable case (\href{InterviewX}{Interview X}).
%(\href{ref}{Association of architects interview}).

Another example is the instant discontinuation of the Republic Agency for Spatial Planning, the chief national executive body of spatial planning, after its director refused to sign the questionable Spatial Plan for the Purpose Area of the Belgrade Waterfront (BWPSPSP). It was not a change of the management structure in the agency but its complete removal that, above all, figures as a manifestation of political decisionism in national institutional structures and contributes to a complete disruption of its spatial planning system.
\footnote{After the discontinuation of the agency, the Ministry of Construction, Transportation and Infrastructure holds both the regulatory and executive role for all spatial planning tasks in the national domain.}

Therefore, in the mediatory manner, these individual roles (\href{Section 5.1.2}{Section 5.1.2}) are also the core bearers of the set of political issues instigated by political voluntarism  that very present in Serbian urban planning discourse even from pre-socialist times (\href{Section 4.1.1}{Section 4.1.1}) (\href{Table 6 social aspects}{Chart Y}).
It may also be said that the urban planning framework and practice are deeply embedded in their societal context. While people are aware that there are troublesome laws, corrupt institutions and complicated local circumstances, they usually avoid these issues or get use to them without battling against them (\href{ref}{Table 6 cultural aspects}).
Following the thesis of \href{Stojanovic}{Dubravka Stojanovic (2010)},
%ne menjaj u ref
one informant also suggested that this is the result of the Ottoman period and the Ottoman corruption model that also thrived during the pre-socialist period (\href{InterviewX}{Interview X}).
%(\href{ref}{Association of architects  interview}).
\\

\item \textbf{Control and Verification}
\\
The issue of control and verification networks is an example of the formalized and provisory legal framework and its conditional and performative implementation. While public hearings and planning commissions are legally assigned bodies of verification and control of spatial and urban planning, they are either dismissed or performed without any real authority.
\footnote{Concerning the public review and hearing of BWPSPSP and the changes of GUP Belgrade 2021, numerous remarks were artificially summarized (98 remarks of various institutions, private and public entities were reduced to 48 examples), and then easily rejected usually with superficial and evasive explanations (\cite{Alternativni izvestaj}).}
Such application of legal procedures was enabled  not only by the inconsistency of legal framework formulations, but also by the irregularity of its implementation (\href{InterviewX}{Interview X})
%(\href{ref}{Association of architects interview})

This inconsistency of yet formalized institutional procedures may also be interpreted as a legacy of the centralized state from socialism which is still at play, although under a different political and economic regime praising neoliberal transition.
The artificial decentralization and democratization is now enriched with a layer of powerful economic actors,
\footnote{Not only the state as it was  the most powerful economic actor during socialism.}
who profit from top-down decision-making.
They couple with political powers in order to reduce control and verification procedures and troublesome professional actors to the minimum.
\\

\item \textbf{Consulting}
\\
The core element of consultation networks are research and professional organizations and international organizations through European and international capacity building programmes and funding instruments.

Citing their research-oriented colleagues, urban planning professionals in Belgrade usually approach the city as a procedure combined with a technocratic view on urban development so that their role in incorporating opportunities and possibilities to improve the regulatory and implementation phases of planning becomes rather a repetition of what has been standardized or imposed from the top-down (\href{Questionnaire Experts Savamala}{Questionnaire X},\href{InterviewX}{Interview X}
).
%(hl{Experts - Savamala - questionnaire; Association of architects interview}).
However, in the case of regulation changes for BWP, multiple professional organizations (Academy of Architecture, Association of Architects, Serbian Academy of Sciences and Arts) raised their voices against the irregularities and the endangering of the public interest and presented elaborated reports and statements, but without enough media coverage and any consideration from the side of the authorities (\href{Bibliography}{Bibliography - Media Sources}).
In this manner, it is possible to say that a part of these networks is invisible and without any real influence in the public and regulatory domain.

On the other hand, the not up-to-date practice of urban planning in Serbia also contributes to  the marketization of planning domains and the expansion of foreign influences that do not fall under any evaluating procedures. This was the case with the urban design and construction solutions for BWP. The practice of copying European documents and experiences without a critical perspective and important adjustments to local traditions and context, as well as introducing international experts directly into the local field for interventions, are very present in Serbia. More often than not, political powers directly interfere in planning and decide and communicate with foreign professionals without consulting local scientific and professional communities. Such practices contribute to controversial rather than progressive foreign influences. They are not properly translated to the situation in Serbia and, in this manner, create more room for misconduct than for its prevention (\href{Vujosevic}{\citealt{vujosevic_collapse_2010}}).
\\

\item \textbf{Administration}
\\
The administrative body of urban planning at the city level in Belgrade is the City Secretariat for Urbanism with its departments and sectors that aim to proceduralize any transformation and change introduced through urban planning instruments
\footnote{However, any spatial and social change happening on sites and not from top-down stays somehow invisible for this body. For a long time, this was also the issue with illegal construction (\href{InterviewX}{Interview X}).
%(\href{ref}{Sekretarijat interview}).
Fortunately, there is now a separate City Secretariat focusing on this issue, the Secretariat for Legalization.}
(\href{InterviewX}{Interview X}).
%(\href{ref}{Sekretarijat interview}).

Continuous changes of regulations, conditions and authorities in charge of decision-making are reported as the cause of practical problems such as (\href{Questionnaire Experts Post-socialist}{Questionnaire X}):
%(\href{ref}{experts questionnaire post-socialist}):

\begin{itemize}
\item relevant institutions are in conflict of interest %(\href{ref}{ref Chapter 5, pp. XX}); 
\item institutions have a monopoly in certain domains %(\href{ref}{ref Chapter 5, pp. XX});
\item institutions in practice exceed the jurisdiction they are legally assigned to %(\href{ref}{ref Chapter 5, pp. XX});
\end{itemize}

Apart from the authoritarian hierarchy of institutional power that empowers certain institutions to exceed or bias their jurisdictions, the issue of individual responsibility is seriously taken into account in the old socialist manner. Namely, most public officers avoid taking responsibility and therefore split it among themselves. In practice, having many people signing a document usually means slowing down and encumbering the process and postponing implementation
(\href{InterviewX}{Interview X}).
%(\href{ref}{Association of architects interview}).
In such circumstances, manipulation, clientelism and paternalism became the most successful strategy to navigate through the existing system nurturing multiple institutional zombies from previous socialist times (\href{Vujosevic}{\citealt{vujosevic_collapse_2010}}). 
\\

\item \textbf{Implementation and Construction}
\\
Construction and implementation networks are associated together as construction is a practice of spatial interventions, while implementation involves both social and spatial practices. Moreover, both issues have confronted a certain de-institutionalization of their practices in the recent transitional context in Serbia, and both suffer from the over- presence of international actors at the local level - either from international formal and informal organizations or from private and corporate investors.

While Urban Incubator Belgrade and its successors demonstrate the capacity to implement at least small-scale socio-spatial projects, they also indicate significant lack of strategic development goals for cultural institutions and agendas and a limited extent of networking and collaboration at the local level in cases where it is not bonded within a larger institutional framework and financial model (\href{Questionnaire Experts post-socialist}{Questionnaire X}).
%(\href{ref}{expert questionnaire post-socialist}).
This also testifies to the disappearing middle class, disempowered and impoverished by transition, as well as the marginality and incapacity of cultural and educational institutions in these new circumstances of transition (\href{Doytchinov}{\citealt{doytchinov_belgrade_2015}}).
\\
Even though certain experts state that Savamala established its identity through these civic activities (\href{InterviewX}{Interview X}),
%(\href{ref}{Kucina interview 2}), 
others claim that these programmes were too professional, typical for either local architects or the ”genius loci” of other places - artistic and foreign trends which were not adequately translated into the local context (\href{InterviewX}{Interview X}).
%(\href{ref}{Association of architects interview}).
A similar opinion is shared by some citizens, who explain that it seems that they [foreign and local organizations] come up with already prepared solutions and visions to be implemented, but with no concern for the real needs and ideas of the local population (\href{InterviewX}{Interview X}).
%(\href{ref}{citizens interview}).

A similar dependence on global trends and circumstances is obvious in the growing involvement of foreign investors and investments funds in real estate in Serbia. The consequences of the political treatment of property and the discrepancy between planning and implementation during socialism take its toll and are still in play through the fast-moving, profit-oriented practices under neoliberalism and transition - once built, the structures are more difficult to change, which has been the logic of BWP
\footnote{Despite the double rejection for the building permit from the Ministry of Construction, Transportation and Infrastructure, the coastal fortification was finalized during 2016. It is difficult to estimate, but, having two residential towers under construction nearby, it is very unlikely that the coastal fortification will be dismantled and removed, or even adjusted to local technical requirements.}
(\href{InterviewX}{Interview X}).
%(\href{ref}{Association of architects interview}).
\\

\item \textbf{Regulatory and Planning}
\\
The overlapping of the regulatory and planning networks is multiple and overwhelming, even more so as most of the responsible institutions behave like management agencies, rather than taking strategic approaches  (\href{Vujosevic}{\citealt{vujosevic_collapse_2010}}).

Generally speaking, in Serbian urban planning discourse the built environment is the product of the regulatory framework rather than any strategic and professional engagement that surpasses it (\href{InterviewX}{Interview X}).
%(\href{ref}{Association of architects interview}).
In this manner, the troika of the Prime Minister, the Minister of Construction, Transportation and Infrastructure and the City Mayor are the actual power poles in urban decision- making in Serbia. Namely, inconsistency in the legal framework and the overlapping of jurisdiction (municipality, city, republic) support parallel structures of power and parallel roles (\href{ref}{Intermediaries/Mediators, pp. XX}) (\href{InterviewX}{Interview X}).
%(\href{ref}{Kucina interview 2}).
In these circumstances, it is not the quality of legal and planning documentation frameworks, but the reliance on an individual sense of responsibility and public interest that causes problems.
\footnote{Knowing that the Prime Minister and City Mayor are political functions and that they usually are not experts in the domain of urban planning and architecture, the influence on the most important decisions comes from either those who advise them or those with economic means and a clear and rationally-defined criteria of their interests 
(\href{InterviewX}{Interview X}).}
%(\href{ref}{Association of architects interview}).}

The extended influence in the various domains of the individuals who have supreme political power follows the historical heritage in Serbia from pre-socialist (Prince Milos and Mihailo, King Alexander), socialist (Josip Broz Tito) and post-socialist and transitional times (Slobodan Milosevic and the Prime Minister today).  
Disregarding the importance of institutions resulted in the lack of operational and efficient feasibility studies, provisory reports and strategies and corrupted plans and regulations.
In these new transitional circumstances, authorities address the issue of the economic revival by focusing their capacities and attention on investors and adapting the regulatory framework to serve their needs,
\footnote{Various informants and multiple reports and analysis indicate the existence of the Investor’s Master Plan
for BWP which was the source of numerous regulation changes and even more so it was used in the construction phase of the coastal fortification and towers, but it has been kept secret to this day ((\href{InterviewX}{Interview X, Y})
%BWP interview, Association of architects, experts,
(\href{Alternativni}{\citealt{pravni_skener_alternativni_2016}}; \href{NDVBGD}{\citealt{inicijativa_ne_davimo_beograd_analiza_2016}})
% grey lit ref
Several experts also emphasize that the design for the BWP urban structures was created under different circumstances and was rather an urban structure for the seaside as river-currents have different dynamics from sea tides (\href{InterviewX}{Interview X}).}
%(\href{ref}{Association of architects interview}).}
while citizens are excluded from the decision-making (\href{ref}{\citealt{ministarstvo_prostora_urbani_2014}}).

While planning has been identified as an implementation tool for investors' requirements to be effectuated in the transitional discourse ["fast line for investors" as it was explained by an informant (\href{Questionnaire Experts Post-socialist}{Questionnaire X})],
%(\href{ref}{expert questionnaire post-socialist})],
several experts suggest the possibility that "investor urbanism" may be traced back to the socialist period after the constitutional change of 1974 and the introduction of self-managed public enterprises that dominated supply-demand chains in the local real estate market (\href{InterviewX}{Interview X}).
%(\href{ref}{Kucina interview 2}).
Whether public (socialism) or private (transition), the instruments to exercise power might be similar, and the instruments that once served to reinvigorate housing construction and nation state economy in the public interest might also become dangerous weapons if used for individual interests and particular purposes.

Another problematic issue of the cumbersome institutional structure inherited from socialism is the lack of any official procedures to assign a regulation as outdated.
With strong authoritarianism and hierarchy in urban institutions, it is very common that obsolete and inefficient structures, documents and procedures are replicated, while the public interest is usually not served and very often it is not sufficient as an excuse for regulation changes  (\href{InterviewX}{Interview X}).
%(\href{ref}{Association of architects interview}).
In this manner, the once thriving cultural and civic activities in Savamala (2012-2013) have to date been left unregulated and uninstitutionalized %(\href{ref}{KC Grad}), 
even though politicians often officially use them as examples of good, local practice (\href{N1}{\citealt{n1_princ_2016}}).
\\

\item \textbf{Financial}
\\
The distribution of financial networks coincide with the impact of globalization and this new context of transition from the international isolation of Serbia in the 1990s to its integrated position in Europe and in the world after 2000.

Global capital and finances have affected all layers of decision-making in Serbia: top-down, real estate and the civic sector, either by international investment banks, private and corporate investors and international organizations, embassies, NGOs and European and international funding bodies.
However, the amenability at the local level still maintains the trends from the 1990s when pressure, money and connections were the means of local tycoons, who also were the major economic actors at that time (\href{ETHZ}{ETH Studio Basel 2012}).

Moreover, access to public funding is also an issue in Serbia.
While the public debt of Serbia is growing (\href{MF}{\citealt{ministarstvo_finansija_republike_srbije_javni_2016}}),
%grey lit
the budget spending of public money favours the questionable business models of several public-private companies (Air Serbia, Belgrade Waterfront), while civic initiatives in Savamala receive minor local funding and support and usually only at the municipal level (\href{InterviewX}{Interview X}).
%zekovic, kucina, vesna cagic
\end{itemize}

The ANT approach to actor identification and their distribution through networks facilitates
(o) logical argumentation for urban dynamics,
(o) enables mapping urban complexity, and
(o) visualizing actors and networks through diagrams.
In order to interpret the urban development of Savamala, specific political, economic and cultural aspects are also treated as actors (social artifacts).
The distribution of these networks is traced within the map through the identification of:

\begin{enumerate}
\item the key actors involved,
\item the levels of decision-making which it stems from,
\item the sets of social aspects aggregated together.
\end{enumerate}

The key findings are articulated through a comprehensive description of on-site complexity, which these conflictive political, economic and cultural aspects produce. In this approach, the researcher maintained certain traditional concepts from urban theory and practice, but reinterpreted them in the ANT logical framework. In this manner, it was clarified what type of networks (urban assemblages) these conflictive aspects address. However, the ANT framework’s quality of offering a playground for explanations from within and below makes the task of relating and distributing agency through networks a never-ending story that also depends on the participant/researcher who conducts it. Therefore, even historical trans- positions/interlacements, roles and links through networks become ephemeral and should be put to inquiry through multiple perspectives, if possible  (\href{Latour}{\citealt{latour_reassembling_2005}}: 256-257).

\textbf{"The  urbicide  in  Belgrade  is  fed  by  the  mentalities and  the  logic  of  incompleteness:  unfulfilled  urban development  plans,  vane  political  promises  and abandoned  projects." (\href{Doytchinov}{\citealt{doytchinov_belgrade_2015}})}

\section{Conclusion}

The ANT data analysis presented in this chapter addresses the contemporary urban reality in Savamala.
Most of the pre-existing methodological approaches in urban development studies consider certain socially bounded explanations (like the dichotomies of importance-influence and power-interest) as self-containing explanatory categories for mapping actor and stakeholder engagement in the social realm (\href{Mathur}{\citealt{mathur_defining_2007}}).
\\

First of all, the ANT approach starts from the other end.
Its point of departure is the identification of human and non-human actors from the ground-up, within the historical deposits of data, procedures, and identities contextualized through the morphology of urban decision-making in Savamala.
Thereafter, the analysis is directed toward pinning down, describing and tracing  their agency and relations at the neighbourhood level.
\\

In short, the catalyzation of urban agency in Savamala has been two-folded: spatial and social. In spatial terms, intensive real-estate transformation created an invisible division of Savamala. With complete disregard for the cultural agency active and the life and activity there in general, recent radical, profit-oriented construction enterprises at the waterfront are directing the development of the area toward what is known as a "gated community" where upper Savamala will be only an unpleasant pass-way.
\\

Socially speaking, the role of individuals is at the core of urban interventions. While in the regulatory framework this practice is obvious and dominant, with politicians making decisions in favour of their political parties, not their respective public whose public servants they are. On the real-estate side, it is very often said that individual international investors or domestic tycoons are the ones pulling the strings. Moreover, for bottom-up engagement, the informants usually testify that for getting a job done, powerful and persistent individuals are usually needed behind it. Taking into account the historical background, Serbian society may be described as fundamentally authoritarian. In such circumstances, both the functionality and reliability of institutions and the empowerment of bottom-up sectors can hardly happen until the approach is changed for a more egalitarian and horizontal one. 
\\

Secondly, in terms of methodology, the chosen 5-step ANT analytical framework served to fine-tune the internal and external actor-networks and reveal the nature of urban agency. The final diagram illustrates the totality of circumscribed urban assemblage networks through the ANT lens (\href{ANT diagram}{Figure X}).
Moreover, the visualized overlaps and the collisions of various actor-network and social aspect distributions provide extensive explanations about how urban decision-making is processed in Savamala and why these social dichotomies are still at stake in a post-socialist neighbourhood.
\\

Enriching it with the space-time component, this ANT analysis aims at decoding the urban complexity and dynamics of Savamala from the past to the present moment. In this respect, a very important point of disruption and radical change in Savamala is the officialized procedure for the adjustment of the regulatory framework to investor needs. At this point, the future of spatial interventions in the whole city might be directed according to what here and now there a single investor prefers and requires. 
\\

Furthermore, the methodological asset of ANT is its tendency to enable explanations from within and below.
Such a descriptive research practice makes the task of relating and distributing agency through networks a never-ending, researcher-based story telling.
Therefore, even historical transpositions, roles and links through networks become ephemeral and with no means to address and question what will come next.
\\
 
In Savamala, and even more so in Belgrade, an important spatial issue is the transformation of the city landscape according to the new regulations set to satisfy BWP requirements. The city will rise in height and most probably, as there are no zoning restrictions, this may happen in the center. While the number of people living and visiting the area will rise, the question of efficient and sustainable transport may well come up, even though it seems that professionals are not strategically addressing such a future. On the social side, the role of international actors and global trends at all levels is unavoidable and overwhelming. The question of positioning local experts, professionals, authorities and citizens therefore might be crucial. Between international influence and an intervention in the particular context, there must be a meso-layer of local urban actors and professional and regulatory frameworks.
\\

Finally, the capacity of ANT to practically address the future from its conclusions is lame. In general, the practice-based approach in urban studies has had hardly any benefits from ANT. Earlier applications of ANT did not address operational diagnosis worthwhile for tracing urban system transitions. Therefore, the next stage of the analysis in this research focuses on a framework for a constant extension of agency and relations when the actors collide, overlap and interfere in networks. 

\chapter{Re-assembling Urban Dynamics within the Multi-Agent System}

%%%%%%%%%%%%%%%%%%%%%%%%%%%%%%%%%%%%%%%%%%%%%%%%%%

Within an urban system, all elements are interdependent. The ANT analysis provides an extensive overview of complex actor-network relations. These assemblages also bound space-time dynamics, linking past-to-present translations of urban decision-making practices and processes. Actor-networks, while influenced by the others, also simultaneously influence them, as do all the agents within the Multi-agent system (MAS) methodological approach (\href{Bousquet}{\citealt{bousquet_multi-agent_2004}}).
\\

Similarly to Actor-network Theory (ANT), the Multi-agent system (MAS) further deconstructs urban complexity and dynamics. It traces agent profiles (assembly of urban agency) and their inter-relations (assembly of operations). The bearers of urban agency are key urban actors operating under the continuous negotiations within the morphology of urban decision-making.
\\

However, the main contribution of MAS lies in tracking down the character of agents’ links. These qualitative categories are reinterpreted by MAS through passive and active contextual elements (objects and relations). They serve for connecting the present to the future based on the past-to-present explanations of agencies and relations. In this manner, MAS also manages to operate the concept of urbanity through the categories of: social practices, urban conflicts and contextual resources (spatial capacities and social potential). While spatial capacities and social potential are passive, social practices and urban conflicts are active contextual elements which are continuously operated by the key actors and within this initial networking.
\\

In Savamala, these contextual elements are entrenched within the Serbian urban regulatory framework, current urban and architectural projects on site, and civic initiatives that activate the social fabric of the neighbourhood. Savamala has been marked by all major transformations of Serbian society over time, the current hype of arts and culture in line with the worldwide spread of hipster neighbourhoods. In addition, it has also fallen under the massive, but rather disputable waterfront mega-project, that aims to remodel Belgrade’s landscape according to modern high-rise metropolis patterns (\href{Figure Belgrade and Savamala}{Figure X}). %MAS-ANT Figure 2
These overlaps of past, present and future processes and mechanisms are the core framework for socio-spatial patterns in Savamala.
\\

This chapter provides the second stage analysis with MAS. First of all, a narrative of resources, conflicts and practices indicates the links between the current state of affairs in Savamala and future oriented urban system transitions. Further on, associating these socio-spatial patterns with urban agents reveals agent preferences. Within MAS, agent preferences are defined to have bounded up the agent’s capacity to influence the future of the neighbourhood. Finally, this chapter concludes with notes on how these capacities might be operationalized within urban transitions by the hybrid method that combines MAS and ANT.

\section{Socio-spatial Patterns in MAS: Dynamism of Objects and Relations in Savamala}

Any urban environment is in a constant state of flux.
In this respect, a kaleidoscope of collected data on Savamala and the first round of analysis on urban assemblages have revealed sets of relationships between urban agents, the morphology of urban decision-making and urban complexity.
As discussed above, these relationships contain only the reference to past and present urban processes.
Conversely, "glocalized"
\footnote{global and local simultaneously}
socio-spatial patterns cover up the space-time relations of urbanity and address the future prospects of development.
\\

To gain a fuller appreciation of these future-oriented processes, close vivisection of socio-spatial patterns in Savamala relied on the qualitative data obtained in four rounds of consecutive data collection:
documents review, questionnaires, workshops, and interviews.
Based on these data and in reference to the MAS methodological framework, socio-spatial patterns are divided into:

\begin{enumerate}
\item Passive contextual elements or objects in MAS terminology
\item Active contextual processes or the assembly of relations in reference to MAS.
\end{enumerate}

\subsection{Passive contextual elements}

As already mentioned, passive objects (contextual elements) are a MAS unit of analysis. In its methodological scope, they are assumed passive because agents (key urban agents) are the core activators/exploiters of their agency. More precisely, this category is not about the individual objects, but rather brings up sets of elements, factors and circumstances that, once united, become the basis for action. Thus, this interpretation already reveals its future oriented span. 
\\

Theoretically speaking, passive contextual elements refer to the concept of urbanity and its potential to be measured (\href{ref}{\citealt{marcus_spatial_2007}}; \href{ref}{\citealt{vujosevic_postsocijalisticka_2010}}).
%post-soc tranzicija i ter kapit 2010; Marcus 2007}
Based on the conceptual framework of this thesis (\href{Section 2.1.5}{Section 2.1.5}),
the measurability of the urbanity concept stems from the issue of territorial capital (\href{ref}{\citealt{camagni_regional_2013}}), and more in detail: spatial (\href{ref}{\citealt{marcus_spatial_2007}}) and social capital (\href{ref}{\citealt{golubovic_mrezna_2009}}).
Moreover, there are several different categorizations of capital, to mention but a few: hard and soft, material, social intellectual, human capital  (\href{ref}{\citealt{healey_collaborative_1997}}). %collaborative planning
\\

However, for the purposes of the MAS methodology the issue of capital is simply addressed in the sense of resources.
In this manner, all possible variations of the term are reduced to its reference to either space (spatial, material, hard capital) or society (social, human, intellectual, soft capital), and the context altogether (territorial capital).
\\

Thus, passive contextual elements are considered as:
\begin{itemize}
\item spatial capacities
\item social potentials
\end{itemize}

\subsubsection{Spatial capacities}

Rivers are very important city landmarks of modern cities. %examples maps
As \href{Mann1988}{\cite{mann_ten_1988}} puts it, there are various trends in waterfront redevelopment,
\footnote{\href{Mann1988}{\cite{mann_ten_1988}} identifies the ten most influential ones:
(1) large-scale mixed-use development;
(2) an open and accessible riverside;
(3) reducing the railroad and highway capacity along the riverbanks;
(4) the development of commercial areas along small bankside waterways; 
(5) historic restoration of the river corridor;
(6) blossoming of market places;
(7) world exposition development on the waterfront;
(8) environmentally adjusted art;
(9) ephemeral events and structures;
(10) growing local urban regulations of waterfront sites.}
but they all indicate refocusing the attention of planners, investors and citizens on "the central urban waterfronts with their transitions away from the inefficiencies and clutter of 19th and early 20th century industrial and commercial patterns" (\href{Mann1988}{ibid.}).
\\

In Serbia and Belgrade, this transition of trends coincided with the moment when the river Sava was not the border anymore.3 Even though Savamala’s rise started much earlier, it referred more to the hinterland while the only sparks of urban life on its waterfront were the docks. According to experts, Savamala was considered as the flooding zone (the whole area below Crnogorska) in those times (\href{InterviewX}{Interviewee X}).
%Association of architects interview
It was only when Belgrade and Zemun became cities of the same country that the Sava waterfront gained importance. Soon afterwards the bridge [Brankov bridge] was built to emphasize this connection and the Savamala hinterland was fortified against flooding.
\\

Henceforth, the development of Belgrade was led in such a way that the connection with Zemun and the left Sava riverbank (New Belgrade side) were ever intensified. Therefore, even though Belgrade is physically placed at the confluence of two rivers, it is nowadays perceived as a city on the Sava river with the Danube nearby, rather than as a city on the Sava and the Danube (\href{InterviewX}{Interviewee X}).
%Association of architects interview
\\

The long, accessible, central waterfront zone largely determined the position of Savamala in the growing Belgrade metropolitan area with not yet efficiently resolved public transport and a heavy traffic infrastructure  (\href{Questionnaire Experts Savamala}{Questionnaire X}).
The combination of the proximity of the river and the city center, but even more so its position in the geometrical center of the vast area of the city of Belgrade was its major capital; "the middle of the three cities [Belgrade, New Belgrade, Zemun]" as one informant said (\href{InterviewX}{Interviewee X}).
%Biking tours - interview
\\

However, all these favourable factors did not prevent the neglected attitude of the socialist authorities after the Second World War.
Actually, it was the high representation of urban, architectural and cultural heritage from previous (prewar) capitalist times
\footnote{A helpiful circumstance for the socialist authorities was the almost complete destruction of Savamala during the bombardments in WWII  (\href{Section 4.1.3}{Section 4.1.3}).}
in Savamala that determined the viewpoint of the socialist regime (\href{Section 4.1.3}{Section 4.1.3}).
\\

Making its transversal Karadjordjeva street the main road for heavy transport as a way to bypass the city center predominately marked the nature of Savamala urban neighbourhood during the socialist period.
What is more, this transit roadway became increasingly surrounded by poor warehouses and manufactories that replaced the bombed palaces for a while (\href{Section 4.1.3}{Section 4.1.3}). 
With the extensive construction of transport infrastructure in the recently industrialized country (SFRY), the growing  industry of transportation means, and consequent rising mobility of people; the urbanity of Savamala then was also aggravated by the proximity of the busy bus and train terminals.
\footnote{"Savamala hosted the enlargement of the state’s major traffic infrastructure, including the nearby main
train station, the bus terminal, the river terminal and two of the city’s main bridges connecting the city centre to New Belgrade, the area constructed during the socialist Yugoslavia that projected its high collective ideals onto urban development by appropriating the concept of modernist urban development." (\href{ref}{\citealt{cvetinovic_engine_2013})}}
The socialist Savamala was a crowded, polluted, noisy area and the home for marginalized groups, outcasts and prostitutes.
Unfortunately, this atmosphere and image had long surpassed the socialist regime and has followed Savamala to the present day.
\\

The first post-socialist period in the 1990s continued to contribute to its deterioration and devastation. The abandoned industrial plots and derelict and ruined buildings, its problematic social structure and the gloomy circumstances of the  civil wars and national turmoil in the Balkan region flagged Savamala as economically underdeveloped, socially disadvantaged, and unsafe with a reputation as a home to outcasts, prostitution and criminality (\href{ref}{\citealt{cvetinovic_engine_2013}}).
\\

Conversely, the urban and architectural richness of the pre-socialist times was also preserved to date. The architectural masterpieces from the beginning of the 20th century, the Belgrade Cooperative [Geozavod], Bristol Hotel, Vuca’s House) still adorn the neighbourhood and the city. Their accessibility (to the bus and train terminals and city center) and even more so its promotion by the recent hubs of hype culture, which have been placed in the neighbourhood, put these buildings on the tourist maps and make them the "must-see" of the post-socialist south-eastern European capital (\href{Questionnaire Experts Savamala}{Questionnaire X}).
\\

The amorphous urban form and small, irregular street matrix below Karadjordjeva street (\href{Questionnaire PhD students}{Questionnaire X}) has been the recognizable urban structure of Savamala until quite recently, when it was forcefully and illegally destroyed for the purposes of the Belgrade Waterfront Project (BWP).
\footnote{Under the pressure of the BWP agreement obligations and under murky circumstances on election night between the 24th and 25th of April, the shabby structures in Hercegovacka street were razed to the
ground by a group of masked men armed with baseball bats (\href{ref}{\citealt{popovic_porusili_2016}}).} %refmedia
\\

The recent terrain clearance is actually the most radical, but also most consequential step of national and city authorities, the legal framework and planning bodies, resulting from the continuous political and economic pressures to solve an old issue of Belgrade’s peak waterfront area. 
\footnote{These initiatives date back to the 1920s (\href{Section 5.1.1}{Section 5.1.1}).
The exact area of intervention in these planning phases has varied from Gazela Bridge to beyond the Dorcol marina, but their common denominator is the relocation of the bus and railway station.}
After multiple competitions and projects for the Sava amphitheatre  (\href{Section 5.1.1}{Section 5.1.1}), the Urban Planning Institute came out with a circumstantial urban analysis and a programme for the area. Referring to much of what is herein said and with extensive multidisciplinary expert analysis, they specify the spatial circumstances and developmental capacity of Savamala as such  
(\href{Program}{Analiza razvojnih mogucnosti Savskog amfiteatra I}; \href{Program}{Analiza razvojnih mogucnosti Savskog amfiteatra II}):
%ne diraj ovo

\begin{itemize}
\item the bounding network of major city roads directly connected to the Savamala street map and limited by surrounding private property structures;
\item the importance of the wider city context and utmost priority of linking its urban development to that of the corresponding area on the left bank of the Sava  (the New Belgrade side);
\item the outstanding quality of the location and its urban vizuras on the city;
\item the priority of waterfront revitalization as a significant spatial resource of the city and its common good;
\end{itemize}

These positive traits as well as being in the vicinity of the major transportation hubs of Belgrade
\footnote{"having visitors/tourists coming by chance on their way from the bus terminal or the train station" (\href{InterviewX}{Interviewee X})}
%Mikser interview
were the main linchpins for the first civil and cultural initiatives to settle in the neighbourhood. According to the informants, the choice of Savamala was made after the preliminary draft mapping of its contextual resources (spatial and social ones) (\href{InterviewX}{Interviewee X}).
%Mikser interview
Another important quality emphasized by the civil sector is the disposition of plots and buildings to be reactivated and offered to citizenry to use for recreation, cultural and community activities (\href{Questionnaire Students Savamala}{Questionnaire X}).
%students - Savamala - questionnaire
\\

In terms of spatial capacities, it is also crucial to stress that, even though they may sound permanent, spatial capacities (and conversely spatial drawbacks) are highly dependent on the distribution of urban activities and the indirect moderating of the situated urban practices and urban conflicts. In Savamala, even though there has not been significant urban reconstruction of buildings and spaces at play before the BWP was instated, the devastated and deteriorating state of the neighbourhood was reduced by the cultural, artistic and civic activities happening there, giving it rather a hype than ruined appearance. Even more importantly, with the increasing number of users from different social backgrounds and accompanying services, Savamala became a posh and safe area, but conversely without obvious signs of gentrification and change in  social status as there was not any significant increase in rents and apartment costs
(\href{InterviewX}{Interviewee X}).
%Bureau Savamala
%visualization of different importance orders and how it changed over time - based on questionnaires - experts-phds-students.
\\

It has been conspicuous that Savamala’s position and provision of empty spaces and buildings, brown- and greenfield areas was a ready-made trigger. Once activated, it becomes an almost unstoppable drive of urban transformation or radical change, even though its urban essence within the image of the whole city of Belgrade stays the same for a while with its run-down buildings, derelict empty plots and open spaces.
\footnote{Even with the large waterfront intervention of the BWP, from a certain perspective and in certain areas (i.e. the Savamala hinterland, known as Upper Savamala), Savamala looks exactly the same as it did a few years ago when it was only a cultural and creative hub of the city.}
However, with the lack of a sustainable, participatory, transparent, feasible, and long-term development model, these spatial capacities may easily turn into social impediments and setbacks.
%add new photo from BWP from the air and the old one, photos around KC Grad

\subsubsection{Social potentials}

Urban planning and urban design are the offsprings of architecture and spatial development that mold urban transformations in contemporary cities (\href{ref}{\citealt{rode_city_2006}}).
While planning anticipates urban future through the debates and decisions on the playground of different interests, urban design focuses on urban form aiming to capture its dynamics in compliance with human objectives (\href{ref}{\citealt{lynch_theory_1958}}: 201).
\\

However, while urban forms represent designers’ intentions at a particular time, its life cycle outgrows these intentions and forms the contextual urban processes (\href{ref}{\citealt{tonkiss_cities_2014}}).
Physical structures are formative for what goes on in an urban setting, but urban reality is in constant flux through its merging with particular urban practices at play. The on- going everyday transformations and future urban design solutions thereafter are set by social capital, rational action that embodies social relations (\href{ref}{\citealt{coleman_social_1988}}).
These are social potentials bounded in any urban context.
In Savamala, the morphology of urban decision-making sets forth and catalyses transformative social forces.
\\

Top-down urban planning interventions bound the planning mechanisms emerging from the new political context, the universal and regime-based forces and trends, society-specific and culturally unique features of the context and the applicability of experiences and tools in the local system (\href{ref}{\citealt{nedovic-budic_adjustment_2001}}).
%Adjustments of Planning practice Nedovic budic 2001.
Participatory, transparent mechanisms and public interest aims have already been defined in the regulatory and strategic documents in Serbia  (\href{ref}{\citealt{vujosevic_conundrum_2012}}).
However, the particular practice and patterns show that institutional and organisational frameworks at play are rather the clients of developers and private actors (\href{ref}{\citealt{mrdjenovic_tatjana_urban_2015}}).
In this manner, certain individuals in institutions, while they can be the executors of such interest, also can contribute to promoting and supporting participatory and transparent processes.
For example, at some point, the municipal authorities officially supported the civil and international initiatives to extend the urban life of Belgrade down to its riverbanks and to reactivate Savamala spaces (\href{ref}{\citealt{doytchinov_urban_2015}}).
\footnote{For example, Nemanja Petrovic, the Assistent Mayor of the Municipality "Savski venac" and Nina Mitranic, the chief architect of the Municipal building inspection (\href{ref}{\citealt{doytchinov_urban_2015}}). Their positive role was also emphasized by the civil sector informants (\href{InterviewX}{Interview X, Y}).}
%interview Kucina, Dobrica 
\\

%timeline of civic initiatives
On the other hand, the small-scale cultural and civic initiatives that were popping up around Savamala from 2008 onward signified as the actual congregators of social potential. This conglomerate of artistic practices, crowdsourcing activities, creative industries, urban manufactories, and cooperative economies was growing, gradually infusing the sparks of new urban life in Savamala. Conversely, these activities also aimed at returning Savamala to its former pre-socialist fame through the attempts to revive old arts and crafts still present but almost disappearing in Savamala.
\\

The first driver of this trend was Urban Incubator Belgrade (UIB) with two projects focusing on the recognizable cultural identity of Savamala and its richness of tradition and crafts. "Savamala Design Studio" aimed to produce a manual of local knowledge, practices and cultural values through participatory action research (\href{Cvetinovic}{\citealt{cvetinovic_engine_2013}}).
The idea was to work with residents and other urban actors, learn from them and map their everyday practices (i.e. preserving food, recycling waste, barter economy, illegal building construction) (\href{Cvetinovic}{ibid.}).
\\

Similarly, "Micro factories" was another participatory activity targeting collecting local materials (usually from abandoned apartments or other places) and working on the design of products that reflects firstly what they had found in Savamala and secondly its tradition of small craft workshops (i.e. carpentry) (\href{Cvetinovic}{ibid.}). 
These new small production facilities are intended to be microeconomic structures establishing the relationship between urban space and industrial production and boosting the local pride and identity of a once prestigious trade and artisanal area (\href{ref}{Micro factories 2013}).
%practice-based wrong citation
This initiative was nurtured and extended with UIB successors, such as the projects: "Cooking together", "Healthy living in Savamala", "Heritage of the ship graveyard" and "Zupa Zupa".
All these activities were situated on the old abandoned steamboat Zupa (\href{ref}{\citealt{belic_parobrod_2014}}).
%ref media
\\


In the long-run, they strove to physically transform the neighbourhood and to influence urban transformations that are based on social interest rather than on real estate speculations. However, the choice of Savamala was not accidental. The availability of abandoned buildings, abundance of architectural heritage might have been the major instigators of that choice 
(\href{Questionnaire Experts Savamala}{Questionnaire X})
%Experts - Savamala - questionnaire
Several organisations testify to mapping Savamala before installing there and except for its spatial capital, the sparks of night life (cafes, clubs) were the main reasons for their decision 
(\href{InterviewX}{Interview X}).
%Mikser interview}
\\

The goal was to change Savamala's urban image by converting  abandoned warehouses into socially productive facilities, activating riverfront usage, encouraging local community participation, attracting new visitors to the neighbourhood (professionals and the general public, both local and global), and finally revalorizing and repositioning the Savamala neighbourhood within the physical and functional scheme of the whole city of Belgrade (\href{Cvetinovic}{\citealt{cvetinovic_engine_2013}}).
\\

Yet, through qualitative investigation of these activities, it was revealed that this wave of action was only possible with the high influence and strong presence of national NGOs and, even more, international NGOs and formal organisations 
(\href{Questionnaire Experts Savamala}{Questionnaire X}) 
%Experts - Savamala - questionnaire}.
"Urban Incubator Belgrade" was based on Berlin experience
(\href{InterviewX}{Interview X})
%Kucina interview 2013}.
The rationale of this project was also embedded in Goethe Institute strategy, whose focus has been urban development, urban transformations and creative industry. As creative industry is an important sector of the German economy
(\href{Lanz}{\citealt{lanz_cities_2012}}),
%practice-based
such financially and logistically supported activities may also be interpreted as an economic strategy for the export of services (\href{Questionnaire Experts Savamala}{Questionnaire X})
%Experts - Savamala - questionnaire
In any case, the outburst of these and similar projects, events and activities (\href{Section 4.2.3}{Section 4.2.3}) made Savamala the nexus of creative capital in Belgrade (\href{B92}{\citealt{b92_savamala_2015}}). %% PROBLEM WITH REF
\\

In sociological terms, Savamala developed a multiple identity recognizable on the national and international scene. For the limited artistic scene in Belgrade, Savamala with its  several important galleries (\href{Figure X}{Figure X})
%list, map} 
is an national, urban, artistic hub.
For cultural and civil workers, it is a cultural cluster.
Finally, for numerous tourists and visitors, it is a place to party all night long (\href{InterviewX}{Interview X}).
%KC Grad interview}
\\

While Savamala was very peaceful area during communism,
\footnote{As an informant, a Savamala inhabitant for many years, recalled: during the SFRY era there were a lot of local police in the streets and the social discipline was different
(\href{InterviewX}{Interview X}).
%citizen Retired person
Such circumstance might have created a different safety net back then.}
it was only during the turbulent wartime of the 1990s that it became unsafe.
The neighbourhood kept that image until these new fusions of urban life settled in Savamala (\href{InterviewX}{Interview X}).
%citizen Retired person
\\

This new spirit, accompanied by the many night clubs
\footnote{Over several years, the famous clubs and cafes in Savamala included: Disko Klub Mladost, Brankow, Apartman, Corba, Magacin, Ben Akiba, Dvoristance, Prohibicija etc.}
settling in the neighbourhood,
put Savamala on the city map as a  party area with a vivid night life
%Savamala clubs map
(\href{Bureau Savamala}{Bureau Savamala 2013}).
%Bureau Savamala data report !!!!
The gradual reduction of heavy traffic, which was caused partly by the construction of the Pupin bridge over the Danube, was replaced with intensive and loud music from the clubs that created severe noise all night long in Savamala
(\href{InterviewX}{Interview X}).
%citizen Gezovic interview
Consequently in 2012, Belgrade and indirectly Savamala were internationally presented as the major party capital in the world by Lonely planet (\href{Lonely Planet}{\citealt{planet_ultimate_2012}}).
\\

Similar to the wasted potential of this uninstitutionalized culture that has never been included in any official cultural strategy of the city and the state (\href{Section 5.2.1}{Section 5.2.1}), this potential of making Belgrade internationally famous was abruptly stopped by the regulation of 2011 when the commerce of alcohol was banned after 11 PM and the bars were to close at that time as well (\href{Vreme}{Vreme 2011}). %% PROBLEM WITH REF
On one hand, this had a bad influence on tourism; on the other, the ban changed Savamala, too. That is to say that even more bars moved there as its location offered a possibility to extend the party hours. Less populated and close to the riverside recreation area and abandoned plots, in Savamala people were still able to party the whole night (\href{InterviewX}{Interview X}).
%Biking tours - private}
Even though the ban on alcohol is no longer in effect, the party places stayed in Savamala and have been influencing its image as a cultural and civic hub.
\\

This recent sequence of events testifies that the city authorities did not apply any sustainable strategy for either the spatial capacities of Savamala (empty spaces and plots) or for cultural and tourist activities  (\href{InterviewX}{Interview X}).
%KC Grad interview
This rather sporadic interest in envisioning developmental options was even more obvious in terms of the Belgrade Waterfront Project (BWP). Taking into consideration the circumstances around BWP deals, the attitude of the city and national authorities towards the exploitation of the spatial capacity of the Sava waterfront contained little, if any, concern for public interest and relinquished this outstanding resource to the whims of the interest of the private investor.
\\

Fortunately, the initiative NDVBGD took this real estate transformation very seriously and critically addressed the issue from the very beginning (\href{NDVBGD}{NeDa(vi)moBeograd WORDPRESS}). %PROBLEM WITH REF
%NDVBGD wordpress practice-based
The initiative assembled local organisations and individuals around the issue of urban and cultural policy and urban development in Belgrade. As soon as the first irregularities appeared, accompanying the collaboration between the RS and the company from the UAE ("Eagle Hills"), the activists from the initiative raised their voices and organized public performances against, what they call "investor urbanism" (\cite{SlobodnaEvropa2014}).%PROBLEM
 At the beginning, they were mainly active through social media (Facebook, Twitter) and with sporadic public performances that made them visible beyond the city scale.
\\

As the irregularities rose, so did the activities from the initiative. The social role they took in Savamala, also contributed to raising the consciousness and involvement of citizens in urban decision-making. Following this path, the initiative has been continuously organizing critical events in parallel with the BWP project and its planning proceedings. The adoption of the Belgrade Waterfront Spatial Plan for Special Purpose Area (BWPSPSP), GUP Belgrade 2021, Lex specialis, as well as the signing of the contract with the foreign investor, the start of construction work, and the gloomy demolitions in Savamala have all been marked by citizen protests (\href{Mitic}{\citealt{mitic_ekskluzivno:_2016}}), symbolic public performances (\href{vice}{Vice}) and a range of critical texts from local (\href{Petovar}{\citealt{petovar_princip_2014}}) and international experts (\href{Spectacle}{\citealt{the_spectacle_blog_inura_2014}}; \href{ref}{\citealt{krusche_bureau_2015}}).
\\

However, to  understand how they set the right actions at the right place and time, it is crucial to mention the genesis of their involvement in Savamala. Their parental organisation "Ministry of Space" was present in Savamala within Urban Incubator Belgrade (UIB), through the Bureau Savamala as the critical commentator of the project. Moreover, the collective "Ministry of Space" has been persistently pointing out the problems in urban policy, management and civil rights in the Serbian Capital.
\\

After the scandalous demolition in Savamala in April 2016, NDVBGD organized several protests with up to 20000 participants at one point (\href{Balkan Insight}{Balkan Insight 2016}).
Not only did these actions make this flawed fast-lane approach to investors internationally visible (\href{ref}{\citealt{eror_belgrades_2015}}; \href{ref}{\citealt{wright_belgrade_2015}}), but these manifestations of civil disobedience also made the local population less afraid to protest and expresses their concerns for the public interest and public good
(\href{InterviewX}{Interview X}).
%Dobrica interview
The slogan "Whose city? Our city" (\href{ref}{\citealt{tulimirovic_protest_2016}}), even though a political phrase, express the rise of consciousness about civic rights and the capacity to understand the importance of participation and transparency in urban management after the decades long citizen apathy and authoritarian governance model.
\\

According to the herein presented state of the affairs, it is obvious that social potentials are highly variable and easily shifted from the positive ability for transformation into a retrograde and dangerous urban conflicts. Yet, if anything, they are an essential force for transforming the negative effects of the conventional urbanisation model, accelerating globalization influences, neoliberal trends and unsuitable urban patterns into a contextually appropriate developmental  impetus. 

\subsection{The Assembly of Relations}

Based on the convergence of urban agency toward  the context presented above, several issues seem to be stumbling blocks throughout all the layers of urban decision-making
(\href{Questionnaire Experts Post-socialist}{Questionnaire X}):
%expert questionnaire post-socialist Q12

\begin{itemize}
\item the lack of transparency;
\item correlation between political and economic interest groups;
\item neglected estimation of real circumstances and feasibility;
\item social polarization and unfair wealth redistribution;
\item total dependency on global trends/guidelines.
\end{itemize}

These obstacles surpass the local context of Savamala and further characterize the urban development circumstances in Belgrade and in Serbia in general.
\\

The  urban  development  of  Belgrade  is  revealing  routines, challenges  and  traumas both from the historical perspective and the traits of culture and mentality
\footnote{Under cultural traits and mentality, \href{Samardzic}{\cite{doytchinov_belgrade_2015}} gives these examples: poverty, sharp social, cultural and ideological differences, inheritance and influence of nationalism, socialism and political religion, undeveloped, inappropriately or poorly maintained developed infrastructure.}
and the analyses of the current condition (\href{ref}{\citealt{doytchinov_belgrade_2015}}). 
The dynamics of these antagonistic relations set forth the complex system dynamics.
While it is essential to incorporate the transformative capacity of passive contextual elements, congregating the relations of system maintenance and system collisions sets a comprehensive overlay of procedural resilience of the system and urban scenarios for radical interventions.
\\

In this respect, active contextual relations in Savamala are categorized as:

\begin{itemize}
\item urban practices
\item urban conflicts
\end{itemize}

\subsubsection{Urban practices}

Social practices are embedded in the local context. Therefore, situated urban practices generate space production and its usage (\href{ref}{\citealt{de_holanda_exceptional_2011}}), so that they are the engines of urban system evolution. %where urban practices stem from
\\

In Serbia what happens in the local context very often deviates from what is planned. So, this tendency may rather be spoken of as a practice of abandoning urban plans and the discrepancy between plans and realizations than of serious strategies with parameters of economic development, a comprehensive socio-spatial analysis and feasibility studies. Therefore, the approach to urban planning documentation is technocratic with loose links to implementation and actions in actual space.
\\

Usually attributed to the socialist inheritance in planning (\href{Vujosevic}{\citealt{vujosevic_novi_2012}}; \href{vujosevic}{\citealt{vujosevic_regionalizam_2015}}),
such praxis is aggravated by the dominance of political party or private investor interests
(\href{Expert Workshop}{Workshop 1}).
%expert workshop}
In post-socialist Serbia, the pluralist political life made the dominance of political and powerful economic actors highly dependent on everyday politics and the shifts of political parties in power. Consequently, the activities of the high planning and decision-making authorities rarely produce sustainable urban practices when brought down to the neighbourhood level.
\\

Yet, the social coherence of urban neighbourhoods in Serbia is also inherited from the socialist period. Still present in public land ownership and the significant amount of accessible open spaces bind civic activities to urban space 
(\href{Student Workshop}{Workshop 3}).
%workshop experts post-socialist}. 
Numerous empty plots and abandoned buildings in Savamala have allowed an urban related framework for a rise in civic activities (\href{Section 4.2.3}{Section 4.2.3}).
\\

Moreover, the distribution of these places and the compactness of urban forms and spaces spatialized these activities even more than they intended to.
Mikser festival, for example, even though its initial relationship to Savamala was only as a site specific event, in its later festival years incorporated more and more urban and Savamala related activities
(\href{Mikser}{Mikser Festival 2012}; \href{IAAU}{IAAU 2015}).
Furthermore, seen as an important local actor, Mikser House staff testify that locals often come there for advice or to express their doubts and fears about the current spatial transformations coming from the BWP interventions 
(\href{InterviewX}{Interview X})
%Mikser interview}.
\\

On the other hand, during the period 2008-2015 the design, communication and creative industry were the main drivers of urban revitalization of Savamala.
This global trend created local impact through micro interventions and private ventures in urban space.
Even though informal liaisons among urban actors and stakeholders were set off locally, the supreme  authority were international and national formal and informal organisations (NGOs and collectives) (\href{Section 4.2.3}{Section 4.2.3}). %student work bottom up networks diagram
\\

These new circumstances arose as a result of the transitional marketization of society, increased interest in the service economy and the spur of consumerism. The dominance of international actors, new creative class, young cultural entrepreneurs and local hype design activities created a strong, yet uninstitutionalized and informal cultural and artistic base in Belgrade (\href{PHD Workshop}{Workshop 2}).
%Phd students questionnaire - Savamala}.
However, although it is widely known that such circumstances are key triggers for gentrification  (\href{ref}{\citealt{krusche_gentrification_2015}}),
%practice-based
this generally was not the case with Savamala (\href{InterviewX}{Interview X}).
%Kucina interview 2014, 2015
Namely, the residence structure in Savamala has not significantly changed during the years of intensive cultural activities, neither do the rents, which are are still in the same price register as before (\href{InterviewX}{Interview X}).
%Kucina interview 2
An important circumstance in this favour is that 90\% of the building stock in Savamala is owned by its inhabitants, so that they are to decide on selling the apartments (\href{ref}{\citealt{krusche_role_2015}}).
Furthermore, the real-estate market and the policy on urban services in the city is not yet regulated so that it is difficult to expect that the prices in Savamala are susceptible to current social trends.
Finally, the legal process for obtaining construction permits is cumbersome and complex, so that even small changes and adaptations as well as initiatives coming outside the circle of authorities and those close to them is unlikely to happen (\href{Klaus}{ibid.}).
\\

Yet the change in residential structure actually happened later,  with the advancement of the BWP, their relocation programmes and forceful evictions of the local population (\href{Princip}{Princip BLOG 2015}; \href{Beoland}{Beoland 2016}).
The case of the Belgrade Waterfront mega-project and its blatant focus on residential and commercial purposes almost exclusively
\footnote{According to the Belgrade Waterfront Spatial Plan for Special Purpose Area (BWSPSP) less than 2\% of the total area is allocated to public and non-commercial urban functions (\href{PPPPN}{BWPSPSP 2015}).}
%LEGAL, FALI
also results from low suburbanization trends in Belgrade as there are plenty of attractive areas in the vicinity of the city center that have empty or deteriorating land available for such development (\href{ref}{\citealt{hirt_belgrade_2009}}).
\\

Even more so as the transport system in Belgrade seems old-fashioned, inefficient and centralized (\href{ref}{\citealt{grozdanic_belgrade_2008}}) and the best accessibility mainly characterizes central urban locations (the historical center and New Belgrade). Therefore, any significant change in land use is grounded upon re-conceptualized transportation patterns, which is the core of the new urban development strategy for Belgrade 2016-2020 (\href{InterviewX}{Interview X}).
%Damjanovic palgo interview}.
\\

In this regard, the development of sustainable transport, walking paths and especially cycling, came up as an overall priority and consistent urban practice that is being raised, regardless of the circumstances. While the rising civic protests against the BWP are widely ignored by the authorities, similar activities coming from the biking community were taken into account and their voices were heard in official meetings  (\href{UZB}{UZB 2016}).
Their requirements were met regarding the new temporary bike lines while the waterfront is occupied with the BWP construction site (\href{uzb}{ibid.}).
\\

Generally speaking, urban policy agendas on one side and localized civic engagement on the other have produced a certain consistency in socio-spatial interventions in Savamala. However, the lack of transparency of all decision-making procedures, inaccessibility of all the following documentation and biased and provisory participatory mechanisms and instruments destroy any such positive effects and transform practices into conflicts (\href{ref}{Ministarstvo Prostora 2014}).

\subsubsection{Urban conflicts}

Cities entail urban conflicts. The dynamics of spatial distributions and social relations in cities unavoidably produce conflictive urban issues (\href{InterviewX}{Interview X}).
%Kucina interview 2013}
Though the task of good urban management is to reduce them to the minimum and ensure a harmonious urban environment for all citizens, in reality, the tensions between antagonistic interests and power poles collide within the morphology of urban decision-making. The escalated urban conflicts lead to the shifting points in urban system development.
\\

The socio-spatial patterns of post-socialist development and transition entail the majority of current urban conflicts in Serbia (\href{Section 2.3.1}{Section 2.3.1}).
The socialist system instated a certain vision of the urban and institutionalized a range of practices toward its achievement. So far, institutional and legislative reforms and ad-hoc interventions in recent years have contributed to the dissolution of the old system, but with poor legitimacy for the proposed reforms (\href{ref}{\citealt{world_bank_cities_2000}}; \href{ref}{\citealt{vujosevic_postsocijalisticka_2010}}).
%practice-based cities in transition
%post-soc tranzicija vujosevic
Therefore, it may be concluded that the hyperproduction of current conflicts stems from keeping "the worst of both worlds".
Savamala with its accentuated antagonism of interests undoubtedly reflects these circumstances.
\\

The recent course of events in Savamala show the dominance of an  \textbf{anti-planning concept and unsustainable development patterns} (\href{ref}{\citealt{vujosevic_postsocijalisticka_2010}}) in dealing with the highly profitable land in the capital. The non-institutionally managed cultural and civic activities in Savamala informally built the base for what was suggested in the General Plan 2021 (2009), insisting on the importance of culture to promote the attractiveness of urban areas with planning and organisational solutions (\href{BDS}{BDS 2008}; \href{GP 2009}{GUP 2009}).
%ne menjaj ove ref
The obvious lack of data on the social and physical structure was recognized and partly dealt with through the Urban Incubator Belgrade (UIB) and the Model of Savamala project.
\\

The 3D representation of the Savamala neighbourhood incorporated soft data (\href{ref}{\citealt{cvetinovic_engine_2013}}).
All urban spaces and urban structures in Savamala were tracked through "passports" that comprised objective facts (i.e. height of building, type of roof, façade, number of units, age of the building) and subjective references (the historical layers, social structure and general impression of the structures) (\href{ref}{\citealt{lee_yaniya_model_2013}}).
All these non-governmental initiatives revived the image of forgotten architectural and urban heritage in Savamala and produced its current recognizable cultural identity
(\href{Questionnaire PhD Savamala}{Questionnaire X})
%Phd students questionnaire Savamala}.
\\

The reactivation of Savamala spaces and filling them in with new functions also reflected the domination of free market ideology for generating socio-spatial configurations (\href{ref}{\citealt{world_bank_cities_2000}}). 
%Cities in Transition 2013
At the beginning having the bulk of the country’s creative human capital coming and staying in Savamala was seen as a social potential for revival and urban revitalization. In reality, Savamala soon turned into the heart of mainstream world culture and urban trends in Serbia. Spaces of alternative culture were neither further extended nor diversified, but insistently more and more surrounded by hipster hubs, services and gathering places, and even more with fancy clubs, cafés, restaurants. In general, \textbf{marketization} and \textbf{globalization} patterns in Savamala also became rendered from the "quasi" ground up.
\\

The international and particular interest-based (popular art and culture) network generated in this way held the bonds that surpassed the spatialized identity of these activities represented in the image of Savamala. Recent unfoldings of the situation show that even though they are weak in confrontation with powerful economic interests and local authorities, they are able to keep their audience and actors and move them around the city  (\href{city}{\citealt{jovanovic_dorcol_2016}}).
On the other hand, the actual cultural policy in Serbia and Belgrade was rather reduced to \textbf{city marketing} - without an overall cultural policy or a law on culture and cultural activities (\href{ref}{\citealt{volic_belgrade_2012}}).
\\

Nevertheless, there are still some traces of professional initiatives to improve and up- date urban planning approaches in Belgrade, even though usually with gloomy prospects to be realized. The Urban Planning Institute in 2012 established a project to create an integrated urban development plan for Savamala, which would rely on previous studies of the Waterfront area (2007)
and of the Sava amphiteatre
(\href{Urban1}{\citealt{urbanisticki_zavod_beograda_program_2008}}; \href{Urban2}{\citealt{urbanisticki_zavod_beograda_program_2008-1}}).
The project relied on an up-to-date account of the activities in Savamala, its brownfield capacities, and addressed global sustainable development, creative industry, and climate change trends 
(\href{UZ}{\citealt{urbanisticki_zavod_beograda_integrated_2012}}).
This forward-thinking project was the result of professionals and experts in the Urban Planning Institute and their non-governmental partners "Ambero Serbia" (Service for Strengthening of local land management in Serbia) and "GIZ Serbia", the Serbian branch of GIZ (Deutsche Gesellschaft für Internationale Zusammenarbeit), which tried to keep track of what is going on in the city. Unfortunately, this is not the regular practice, as the procedures, strategies and plans in Serbia are more often than not detached from actual space 
(\href{InterviewX}{Interview X}).
%Zaklina, Sekretarijat}.
\\

Generally speaking, the regulatory framework in Serbia supports an \textbf{administrative approach to urban-land management} that becomes inefficient with market rules at work in post-socialist urban practice
(\href{InterviewX}{Interview X}).
%Zekovic et al. 2015}.
The initiatives for improvements have mainly ended with rushed-in decentralization and problematic horizontal coordination
(\href{World Bank}{World Bank 2000}).
%cities in transition 2000
Top-down regulatory structures and individuals in these institutions tend to keep power in their hands (\href{ref}{\citealt{vujovic_belgrades_2007}}), while there is a general lack of political will and institutional and expert capacity to solve the issue of missing or inefficient regulatory mechanisms and institutions (\href{ref}{\citealt{zekovic_planning_2015}}).
In this manner, even cultural ventures, interventions, and policies when they have repercussions in the urban economy become politically biased.
\\

The prolonged regulatory gap in terms of the  instruments  of  the  urban economy, real estate investments, urban development mechanisms were even further deepened in the local contexts of SEE (South Eastern Europe) with the global economic crisis 
(\href{Zekovic}{ibid.}).
No wonder that such contextual conditions put forth the appetites of private investors as the fast-lane, short-term approaches to dealing with the disastrous consequences of transition: low economic growth, high public debt, high unemployment, and the rise of poverty (\href{ref}{\citealt{world_bank_cities_2000}}).
Megaproject development for the very attractive location at the Sava waterfront that relied on government interventions and was dependent on external investments more  closely reflected the peak of the  \textbf{commercialization of the urban fabric} that actually started in 1989 (\href{ref}{\citealt{hirt_belgrade_2009}}) than an effort to establish a systematic investment policy in the country's construction industry (\href{World Bank}{World Bank 2000}).
\\

While the choice of the Sava waterfront, which partly pertains to Savamala, was made because of its spatial capacity, the biased planning process, the disputable change of legislation and the institutional framework and unprecedented top-down political pressures that followed, turned decision-making on the BWP into a highly conflictive issue at local, city and national levels. The fragmented approach to the evaluations of such a project (non-defined priorities, no feasibility studies, poor if any cost estimations) and a generous attitude toward the economic interests of the investor (the contract between the RS and the investor with detrimental consequences for the Serbian taxpayers, regulated the ability of the investor to privatize urban land and convert the leasehold on urban land into property without a fee) contributed to thriving  \textbf{commercialization, commodification and privatization mechanisms in the urban economy}
(\href{Questionnaire Experts Savamala}{Questionnaire X}).
%Experts - Savamala - questionnaire
The inadequate and unstable regulatory framework and its selective and solely administrative application make these mechanisms far from strategic spatial and urban development instruments and contribute to an unregulated, proto-capitalist market with limited legal and regulatory independence and reliance for the protection of the public interest 
(\href{Questionnaire Experts Savamala}{Questionnaire X})
%{Experts - Savamala - questionnaire
\\

Moreover, the planned urban functions and aesthetization of the Sava waterfront proposed by the BWP reflect major conflicts rooted in post-socialist and transitional spatial patterns.
\textbf{Socio-spatial fragmentation of urban land}, the segregation of the rich and poor, and de-industrialization of the urban economy may further result from luxury residential and over-sized commercial areas planned for the area.
Reliance on automobile transport and commercial functions for mass use
\footnote{"the biggest shopping mall in the Balkans" (\href{ref}{\citealt{ekapija_beogradska_2014}}; \href{ref}{\citealt{mucibabic_soping_2016}}; \href{ref}{\citealt{n1_beograd_2016}})}
in the central urban area testify to the territorialization approach in the urban economy (\href{Hirt}{\citealt{hirt_belgrade_2009}}), but without elaborated strategies for long-term turnovers and urban development in the public interest. Promoted as the local driver of economic growth in the media (\href{Novosti}{\citealt{novosti_vucic_2016}}), the roll-out of the BWP actually does not touch upon the current situation at the wider city scale.
\\

The issues of traffic bottle neck, the poor population and marginalized people living and gathering in the area, illegal building, the informal economy and petty crimes are not fought against, but only moved to other areas in the city. Namely, the violent measures of terrain clearance and unjust circumstances of citizen relocation have even worsened the situation. And what should have been a project of urban regeneration, in reality has created the threat of disintegration of the urban and architectural heritage and tradition. Offering the most important historical heritage in the neighbourhood (e.g. Hotel Bristol, Belgrade Railway station headquarters, Belgrade Cooperative, Post office, Paper Mill) as contributions in kind for BWP represent a high disregard by the decision-making bodies (the Government) for what should be cultural heritage of national importance  (\href{JVA}{JVA 2015}).
%Agreement between RS and UAE ne diraj ref
\\

All measures, incentives, instruments and mechanisms put in place to enable a smooth path for the BWP epitomize \textbf{the lack of political dialogue and consensus} on what the public interest is and what the modalities to achieve it are 
(\href{Vujosevic}{\citealt{vujosevic_post-socialist_2010}}).
%vujosevic emphirical evaluation
%timeline planning and legal
As an expert explained, people in Serbia and Belgrade do not trust community actions and bottom-up engagement, because, as an impulse inherited from socialism, they expect the authorities to organize them
(\href{InterviewX}{Interview X}).
%Kucina interview 2013}.
Jumping from socialism into neoliberalism and wild capitalism, which highly value the conflict of interests and competition, the citizens are introduced into democratic procedures with nominal equity but with no measures and means of participation and cooperation, the expert further explained (\href{InterviewX}{Interview X}).
%Kucina interview 2013}.
In such circumstances citizens are resigned and passive, so that voting has become the only act of participation in Serbia, he concluded (\href{InterviewX}{Interview X}).
\\

\textbf{Non-transparency and low communication} in and around the BWP have led to social exclusion and misinformation of the general public
(\href{Questionnaire Experts Savamala}{Questionnaire X})
%Experts - Savamala - questionnaire}.
Citizens, independent and local experts, media and civil society, namely the broad public and the representatives of the public interest in general, were not informed adequately and on time about what was happening,
What is more, they were not explicitly warned about the actions and consequences and generally were excluded from all planning phases
(\href{ref}{\citealt{ministarstvo_prostora_urbani_2014}}). 
% LOSA REF
%Urbani razvoj u Srbiji Ministry of Space 2014}.
It is not surprising at all that several actions in particular (signing the contract between the RS and the investor; the unclear circumstance of the night terrain clearance in Savamala, setting up legal processes against the media workers who reported on the phantom demolitions for the BWP) were followed by protests by the civil sector and thousands of citizens in Belgrade (\href{NDVBGD}{NDVBGD Facebook profile}).
\\

The most disastrous effects came from \textbf{the systematic exclusion of local experts}.
Being bluntly skipped in the strategic and planning phases of the project, the educated public, professional and expert organisations and individuals loudly advocated against various aspects of the projects and the legal and planning adaptations and changes that accompanied it 
(\href{ref}{\citealt{stojkov_crne_2015}}; \href{ref}{\citealt{stojkov_zaboravljene_2015}}; \href{ref}{\citealt{stojkov_sahrana_2015}}; \href{ref}{\citealt{stojkov_djindjuve_2015}}).
%media articles
Even though there were multiple competitions, analysis and studies that aimed to optimize the urban design solution for the Sava Amphiteatre and the pertaining waterfront, none of them were taken into account in the preparatory phase of the BWP (\href{SANU}{\citealt{sanu_nacrt_2014}}).
%visualize all different sources
Not only was that vast amount of effort and solution framing wasted, but the BWP is also under threat to entail the exact conflicts and construction deadlocks already predicted in the that documentation (i.e. difficulty of construction on the wetland, threats of flooding because of the low ground level of the Sava waterfront etc.).
\\

The National Anti-Corruption Agency in its report on the 2013-2018 period presented detailed analysis of the BWP and the corresponding regulatory framework adaptations. All the ambiguities, biases. flaws and deviations in official procedures and documents and in the work of institutions were analysed by the multidisciplinary expert team and elaborated and precisely referenced in 70 pages of text (\href{ref}{\citealt{pravni_skener_alternativni_2016}}).
\\

Apart from the various documents analysing and critisizing the whereabouts of the project by the civic sector (with the NDVBG in front) (\href{NBG}{\citealt{inicijativa_ne_davimo_beograd_analiza_2016}}), others raised their voices against the project, including the independent media [NIN, Vreme, N1, Kontrapress, Pescanik (\href{Bibliography - Media sournces}{Bibliography - Media sournces})] and international media (\href{Wright}{\citealt{wright_belgrade_2015}}; \href{Eror}{\citealt{eror_belgrades_2015}}; \href{BBC}{\citealt{ahmed_controversy_2016}}), and the professional organisations of architects.
The Association of Serbian Architects (ASA) officially published three declarations against the BWP (\href{Deklaracija1}{\citealt{akademija_arhitekture_srbije_deklaracija_2016}}; \href{Deklaracija2}{\citealt{akademija_arhitekture_srbije_deklaracija_2016-1}};\href{Deklaracija3}{\citealt{akademija_arhitekture_srbije_deklaracija_2015}}). 
The Association of Belgrade Architects (DAB) also reacted and officially filed complains against the BWPSPSP and against the changes and adaptations of GUP 2021 in favour of the BWP (\href{UAS}{\citealt{udruzenje_arhitekata_srbije_odgovor_2014}}; \href{Primedbe}{\citealt{drustvo_arhitekata_srbije_primedbe_2014}}).
\\

Finally, the Serbian Academy of Science and Arts (SANU) published voluminous documents with complaints and suggestions concerning this new Spatial Plan for Special Purpose Area. The document maintained a neutral tone and the authors tried to substantiate their arguments by elaborating on the issues of: (o) the adequacy of the area for special purposes plan and the planning and legal basis for such attribution; (o) the reliability of the 3D model; (o) the subordinate function of local institutions; (o) the methods, measures and instruments of planning at play; (o) capital investment and infrastructural design solutions; (o) the distribution and choice of urban functions for the area; and (o) the choice of the implementation instruments, among others (\href{ref}{\citealt{sanu_nacrt_2014}}).
%primedbe i sugestije 
However, according to several informants, this harsh criticism was diluted in the individual engagement of professionals and experts when they were directly, openly confronted by the political authorities or when they were offered official engagements
(\href{InterviewX}{Interview X,Y})
%palgo damjanovic and dobrica interviews}.
%visualize all documents and arguments against
\\

Away from its broad consequences at the national and city level, the BWP substantially \textbf{changed the atmosphere} in Savamala and the social structure of people now present there
(\href{InterviewX}{Interview X}).
%KC Grad interview}
Bicycle paths are gone or re-routed and the way from the city center to Ada Ciganlija is not straight forward anymore. The urban poor are evicted and their shabby, partly illegal structures are replaced with a big, muddy construction site. Even tourists, when they inquire and are informed of all these circumstances become affected by the corrupt project and instinctively build a hostile attitude towards it 
(\href{InterviewX}{Interview X}).
%biking tours interview
No information about what is going on and what is going to happen is reported as a problem especially for local entrepreneurs
(\href{InterviewX}{Interview X}).
%citizen biking tour interview
Hitherto, the toll of the BWP is that it ruined a lot, created insecurity, covered the project with a veil of obscurity and secrecy, made things hard for local businesses and forcing them to decide whether to move or to stay
(\href{InterviewX}{Interview X}).
%Biking tours - private
In sum, the BWP is not only the creator of conflicts, but a sort of inherit conflict in itself. Consequently, the greatest fear of all is that the project will never finish, and the situation may entail enduring aggravated urban conflict at a city scale and seal the fate of Savamala for many years.

\textbf{"However, the biggest urban conflict of all is such that actually certain people ("those close to the authorities and the investor", speculated an informant)  take advantage of the situation (renovating façades, renting places, even the City architect has a bar in Savamala) while the others are under constant threat of eviction and their business plans and projects are consequently doomed to short-term and at risk."} (\href{InterviewX}{Interview X})
%biking tours interview

\section{Agent Preferences: Scope and Operationality}

Contextual analysis of the social circumstances in Savamala has shown that the contextual capital, which was identified therein, has always been a driver of top-down propositions and solutions for the Sava amphiteatre and Sava waterfront. However, in recent years, it has also been gradually attracting a number of small-scale civil initiatives and creative services to settle in Savamala (\href{ref}{\citealt{cvetinovic_engine_2013}}).
Only later, and independently, did the attractiveness of the waterfront bring a very powerful international actor to the neighbourhood. The links to high political structures enabled tremendous changes to the regulatory framework. The negligible and violent attitude of the dominant and powerful actor new to the context in Serbia and Savamala was later the main source of local conflicts.
\\

Historically speaking several important characteristics have been continually developed and finally escalated in recent transition years in Serbia (\href{Section 2.3.1}{Section 2.3.1}; \href{Section 4.1.1}{Section 4.1.1}). They have influenced the distribution of urban conflicts and management of contextual resources during the different periods through (\href{Figure 1}{Figure 1}) (\href{ref}{\citealt{simmie_self-management_1989}}; \href{ref}{\citealt{vujovic_belgrades_2007}}; \href{ref}{\citealt{petrovic_cities_2009}}; \href{ref}{\citealt{vujosevic_postsocijalisticka_2010}}): 
%Spatium article
(a) restricted and ideologically-framed civil rights, 
(b) state control over capital areas, resources and infrastructure,
(c) a top-down approach to spatial and social development and renovation and revitalization,
(d) public ownership of land and building stock,
(e) hybrid market circumstances,
(f) and societal self-management planning
\\

The antagonistic societal values hidden within the political background of the issues in practice heavily endangered the legitimacy, contents, and procedures regarding the space management (\href{ref}{\citealt{vujosevic_planning_2006}}).
The authority of law and traditional social rules are left in limbo through newly generated distortions in the power - knowledge - action triangle (\href{ref}{\citealt{friedmann_planning_1987}}).
In the broader social and spatial context of Belgrade, this meant a contradictory and inconsistent manner of city branding and urban management
(\href{ref}{\citealt{doytchinov_belgrade:_2015}}).
%Vukmirovic in Doytchinov et al 2015
\\

Reduced to the neighbourhood level, these circumstances made Savamala a scaled example of the pre-socialist material legacy, the socialist cultural and societal matrix, a transitional reality and a condensed case of multi-faceted circumstances of post-socialist urban development.
%ANT article
The extreme variations in societal circumstances to which the regulatory framework and planning practice should respond offer rich opportunities to observe the distribution of decisions from top-down and from outside-in (\href{Peric}{\citealt{peric_evolution_2016}}).
Bearing this in mind, Savamala is a good illustration of the changing political and socio-economic circumstances, including both the challenges and traumas from the recent turbulent history (\href{Chapter 4}{Chapter 4}) and the long term social values and mentality (\href{ref}{\citealt{doytchinov_belgrade_2015}}).
\footnote{As mentioned previously in this chapter, under mentality, the author alludes to poverty, sharp social, cultural and ideological differences, inheritance and the influence of nationalism, socialism and political religion, an undeveloped or inappropriately developed attitude towards capital investments and infrastructure development  (\href{Samardzic}{ibid.}).}
\\

Based on the previous analyses \{historical/contextual (\href{Chapter 4}{Chapter 4}), actor-network (\href{Chapter 5}{Chapter 5}) and object-relations (\href{Section 6.1}{Section 6.1})\}, the therein aggregated urban agency is retained in the layers of urban decision-making and disseminated through the networks of influence (national, city, local).
In this respect, it has been noted that urban agency is distributed among various and dispersed elements, but several agents exceed and overtake the direction of contextual elements and relations.
\\

The most influential top-down agents are: 
(1) The Ministry of Construction, Urbanism and Infrastructure (MCUI),
(2) Belgrade General Urban Plan 2021 (GUP BGD 2021),
(3) Belgrade Waterfront Spatial Plan for Special Purpose Area (BWPSPSP), 
(4) the City architect (CA),
(5) Urban Planning Institute (UPI),
(6) the City mayor (CM),
(7) Lex specialis (LS),
(8) The Republic Agency for Spatial Planning (RASP),
(9) The Prime Minister (PM),
(10) Planning and Construction Act (PCA),
(11) Belgrade Urban Development Strategy (BUDS).
\\

As follows, the agents of investor-based real estate transformations are human actors assembled around
(12) the Belgrade Waterfront Project (BWP). Moreover, Savamala’s recently established  hotbed of creativity and participation of national importance and the main protagonist of bottom-up participatory activities are also protagonists at the local scene: (13) the cultural centre "Kulturni Centar Grad" (KC Grad),
(14) the Old Depository in Kraljevica Marka Street (MKM), (15) Mikser multidisciplinary platform, (16) Nova Iskra Design Incubator, (17) the Urban Incubator Belgrade project (UIB), (18) the Ministry of Space Collective (MSC),
(19) "Ne da(vi)mo Beograd" initiative [Don't drown Belgrade] (NDVBGD),
(20) "My Piece of Savamala" participatory urban design workshop,
(21) "The Game of Savamala" - participatory urban planning workshop,
(22) "Savamala, a Place for Making" participatory project,
(23) "Streets for Cclists" NGO,
(24) Common space at Kraljevica Marka 8 (KM8).
%importance of actors diagram
\\

Agent preferences are defined in connection to their relationality towards the contextual resources, social practices and urban conflicts figuring in Savamala and the social artifacts they are influenced by or they have influence on.
In this manner, we become aware of their field of maneuvers in Savamala.
In order to identify and elaborate how urban planning, real estate interests and participatory activities influence urban development in Savamala, it is essential to translate these qualitative categories into factors which might denote a positive impetus.
\\

Accordingly, with regards to the definition of urbanity in this research (\href{Section 2.1.5}{Section 2.1.5}), it may be concluded that contextual resources, either spatial or social, are the attraction factors that make Savamala a neighbourhood saturated with different actors and interests.
Based on our qualitative research on Savamala, the most prominent aspects in direct correlation with agent functioning at the local level are:
(o) political (participation, transparency, and the institutionalization of culture),
(o) economic (public funding),
and
(o) cultural (global flows of ideas, trends, information and knowledge).
Consequently, the following clusters of resources, conflicts and practices have recognized (\href{Figure 4}{Figure 4}):

\begin{itemize}

\item \textbf{Spatial capacities (SC)}: 
(1.1) accessibility;
(1.2) central position in the city;
(1.3) brownfield area;
(1.4) architectural diversity;
(1.5) proximity of the river;
(1.6) deteriorating area;
(1.7) green area;
(1.8) waterfront area;
(1.9) recreation area;
(1.10) empty plots.

\item \textbf{Social potentials (SP)}:
(2.1) lack of private interests before 2012;
(2.2) cultural heritage;
(2.3) social diversity;
(2.4) aroused interest for this neighbourhood from cultural and artistic groups, individuals and organisations;
(2.5) historical trade and artisanal area - traditional crafts;
(2.6) creative cluster;
(2.7) participative and self-organisational initiatives;
(2.8) small commercial area;
(2.9) underdeveloped area;
(2.10) diversity of interests and power poles in the area;
(2.11) educative initiatives.

\item \textbf{Urban conflicts (UC)}:
(3.1) disintegration of heritage;
(3.2) lack of systematic investments in construction industry (debt crisis 2008-2012);
(3.3) lack of data about the state of physical structures;
(3.4) lack of data on the social structure in the neighbourhood;
(3.5) attractive location for private investments;
(3.6) poor population, squatters and marginalized groups in the area;
(3.7) dominance of profit-oriented activities from 2014 onward;
(3.8) property issues;

\item \textbf{Social practices (UP)}:
(4.1) support of urban related activities (urban design and public participation);
(4.2) support design activities (interior, fashion, graphic), art, culture, education on the city level;
(4.3) translation of global cultural trends into local and regional practices;
(4.4) design, communication and creative industry activities in Belgrade;
(4.5) local and global economy trends in the area;
(4.6) develop waterfront recreation area;
(4.7) sustainable transport (cycling);
(4.8) technical planning activity;
(4.9) land management.
\end{itemize}

The data in (\href{Table 1}{Table X}) show how different agents opt for these contextual resources, urban conflicts and social practices in Savamala and what the relation is between the decision-making layer (\href{Section 4.2}{Section 4.2}) they represent, their actor nature(\href{Section 5.1}{Section 5.1}) and these preferences. 
%relation with decision-making visualized

\subsection{Civil and Creative Articulation of Agents}

Speaking chronologically about this aggregation of urban agency in Savamala, the so-called "bottom-up activities" are the first to revive the image and the role of this neighbourhood at the city and national scale and beyond (\href{Section 4.2.3}{Section 4.2.3}).
\\

The very first bottom-up activity in Savamala was the establishment of the MKM cultural space in 2007. However, the intensive aggregation of participatory activities started when KC Grad gained an abandoned building in Brace Krsmanovic street for their cultural activities in 2009. Mikser House (MH) was officially opened in April 2013, while the Mikser Festival moved to Savamala in 2012. Before that, the MH building served the festival in 2012 and was occasionally used for exhibitions and markets immediately prior to the opening 
(\href{InterviewX}{Interview X}).
%Mikser interview
\\

Beforehand, the premises of MH had been a warehouse and a garage used by the current owner, who bought it during the mass privatization initiatives after the 2000 regime change (\href{InterviewX}{Interview X}).
%Mikser interview
With other cultural activities coming from the Goethe institute initiative, moving in to Savamala exploded at the end of 2013. This condensed interaction between urban spaces and civic life lasted for around two years (\href{Figure 2}{Figure X}).
%timeline participatory activities
\\

The instigator for the choice of Savamala may therefore primarily be its location and the availability of affordable places for these activities. However, the further concentration of similar activities was based on the social potential and appeal generated with these first cultural forerunners. Finally, the settling of a significant number of commercial activities (clubs, cafés, restaurants) was incited by both spatial and the recently added social quality of the neighbourhood 
(\href{InterviewX}{Interview X}).
%Biking tours - private}.
\\

As the cultural, artistic and educational activities continued to settle there, the neighbourhood started to gain the attention of the other top-down (municipality) and international actors (foreign embassies, cultural centers, formal and informal institutions), but with the very limited actual involvement of local citizens 
(\href{InterviewX}{Interview X};
%Ksenija Petovar
\href{Expert Workshop}{Workshop 1}).
%Expert workshop - Ksenija Petovar prez}.
Therefore, for further explanations, the focus for the most part shifted to influential public and private organisations involved in these civil initiatives.
\\

With the entrance of top-down and outside-in actors to the local scene, the issue of funding appears as an important distributive factor.
Namely, several of these agents have an unclear and non-transparent funding structure - while they receive some public funding (usually from the municipality and foreign funds), they are also partly profit-oriented (KC Grad, MH) and incorporate profitable services (café-bars, shopping areas, concerts, exhibitions and other lucrative events/activities).
Nova Iskra is the only explicit privately-based organisation.
However, there are others with transparent financial schemes and crowdsourcing attempts (NDVBGD).
\footnote{The data on gains and spendings are presented in detail on their website (\href{NeDavimoBG}{\citealt{ne_davimo_beograd_ne_2015}}).}
\\

Above all, the exploitation of Savamala’s contextual capital at the international level is based on the recently attractive issue of creative industry and its economic potential (\href{ref}{\citealt{landry_creative_2012}}).
The Goethe Institute officially emphasizes the focus on urban development (especially the fields of architecture, culture, urban planning, public space, public participation, urban art) in their programmed activities (\href{Goethe}{\citealt{architecture_-_goethe-institut_architecture_2016}}).
%(\cite{Architecture-Goethe institute website}).
In collaboration with motherland universities, international stakeholders (i.e. other cultural centers) and local actors, they promote this creativity as an asset for city branding across the transitional and developing countries (Hamburg, the Baltic region, Malaysia, Indonesia, China, Vietnam) (\href{ref}{\citealt{waibel_creativity_2014}}).
\\

In the local context, the targeted audiences are at all levels of decision-making: policy makers, professionals, scientists, artists, NGOs, and the general public (\href{British}{\citealt{british_council_creativity_2014}}).
In this respect, they also touch upon delicate social issues and address urban planning and research, by asking questions (such as who owns the city, who creates the city, what is a good city) and producing the documentation on what is achievable based on their urban interventions in developed cities (\href{ref}{Goethe-Institut 2016}).
By promoting practice-led urban research, these institutions also foster their influence on various local contexts around the world.
While creative economy is an important source of income in Germany (\href{MFG}{MFG WEBSITE}; \href{BMWi}{BMWi 2015}), it is also a product for exportation through such initiatives.
\footnote{In different countries around the world, through similar activities, the Goethe Institute also introduces German scientific and creative actors (universities, collectives, NGOs, freelance individuals and small enterprises) into new markets (\href{ref}{\citealt{waibel_creativity_2014}}).}
\\

Contextual resources in Savamala were recognized and used by these civil and creative agents, but it must be clear that the ideas and organisational initiative  did not come from the ground up.
The torrent of cultural and participatory activities was instigated from the outside, even though they were seen as an asset for economic upgrading and local recovery.
\\

The consequences of civil-oriented catalysis of the Savamala contextual capital under the watchful eye of international cultural organisations and actors, formed in Savamala what everywhere else in the world is a creative sector and urban creative hub: co-working spaces, initiatives of local cooperation (crowdsourcing of activities, open sourcing for vacant self-organized spaces and industrial lots, creative commons, vacant industrial lots), and visibility at the city, national and international scenes.
\\

According to the experiences of the developed cities, this might be the long awaited impulse for the diversification of the social power structure and cultural development 
(\href{Expert Workshop}{Workshop 1}).
%Expert workshop - Ksenija Petovar prez
Furthermore, following the nature of these agents, the researcher apprehended that the cultural and artistic activities in Savamala do not belong to the national cultural and artistic frameworks and programmes.
Having said that, most of them relate to the NGO sector or they acquire or occupy publicly owned spaces which they use for these activities.
MKM and KM8 are municipal spaces shared with different NGOs and offered for multiple projects/activities/events by different actors.
\\

The majority of these agents aspire to have a consulting role on a wide range of urban issues, culture, art and education or to implement a range of ideas/solutions/interventions at an urban or social level.
In the Serbian context, they aim to provide an alternative body for catalysing available human resources and translating global knowledge into the local context of Savamala and Belgrade.
All bottom-up agents that have an active approach to the urban environment (through projects, activities and events) also direct their initiatives toward solving urban conflicts. However, those that include profit converge more to social practices that maintain the current urban order.
Consequently, these agents refer to their contextual preferences, and they organise and engage in networks at local or superior levels, in this way influencing the state of the urban environment in Savamala.
    
\subsection{Top-down Technocratic Approach to Space}

The socialist tradition of planning as a technical activity (\href{ref}{\citealt{vujosevic_planning_2006}}), results in having documents often detached from the actual space, users, architects and designers who actually intervene in reality.
On the other hand, laws and bylaws give orders not recommendations, and 
that is also the case with laws on spatial and urban planning and construction.
\\

In the context of Belgrade, the Urban Planning institute (UPI), according to law, receives the strategic tasks concerning urban development (\href{ref}{Zakon o planiranju i izgradnji 2003 [PCA 2003 and Amendments]}).
Therefore, UPI represents the city authorities in the planning process through drafting bodies, inspections and commissions (\href{InterviewX}{Interview X}).
%UZ Zaklina interview}.
While the research and professional education are not requested or supported by the authorities, as a professional institution, UPI collaborates with European planning bodies and local educational institutions, but these happen rather on a personal basis than as an official, institutional activity
(\href{InterviewX}{Interview X}).
%UZ Zaklina interview
\\

However, the underdeveloped legal framework and political party power over the social domain allow for private and group interests to rule the actions of planners (\href{ref}{\citealt{vujovic_belgrades_2007}}).
In this manner, planning bodies do not actually relate to the real context and contextual resources, but engage only in reproduction of the current social order (social practices) by the application of minimum urban standards (\href{Vujovic}{ibid.}) or engender new urban conflicts by succumbing to the investors' dictate and political pressure
(\href{Expert Workshop}{Workshop 1}).
%Expert workshop - Bata Stojkov prez}.
\\

Planners are  actually deprived of dignity and professional authority, and the low capacity of their actions is characterized also by the lack of overall planning principles (\href{ref}{\citealt{petrovic_cities_2009}}).
In the case of public participation and public hearing, the fear of defending the public interest marks the behaviour of planners (\href{Expert Workshop}{Workshop 1}).
%Expert workshop - Bata Stojkov prez}.
\\

The Planning and Construction Act (PCA) prescribes that planners have to consult everybody: public power holders (nosioce javnih ovlascenja) and the general public through public hearings (one in an early planning phase and the other for the draft of the plan) (\cite{PCA}).
The public hearing is the moment when conflicts come out, should be dealt with and, in the end, all the solutions must be argumented (\href{InterviewX}{Interview X}).
%UZ Zaklina interview
Taking into account recent experience with the Amendments to the PCA and the adoption of BWPSPSP (\cite{ref media all different}), public participation was rather a farce and public hearings were a pure formality with obvious and maybe even illegitimate disregard for negative comments (\href{Expert Workshop}{Workshop 1}).
%Expert workshop - Bata Stojkov prez
\\ 

This extensive weakness of the urban planning regulatory framework was evident from the blurred transition of old, socialist institutions into neoliberal ones, with the conversion of political capital into economic ones (\href{ref}{\citealt{vujovic_belgrades_2007}}).
In Serbia, wealth and power are concentrated in the hands of political and economic actors. While the politicians are focused on the image of the Serbian capital in the world and their short-lived success in the elections (\href{InterviewX}{Interview X}),
%palgo damjanovic center interview}
private investors dominate Belgrade's urban development and have the sole interest for construction land, especially that of the Capital. As the most valuable part of the territorial capital of the country, the land market in Belgrade has not yet been regulated according to market principles and therefore has had the land on offer below its market value (\href{SKGO2013}{\citealt{skgo_finansiranje_2013}}; \href{Zekovic}{\citealt{zekovic_planning_2015}};\href{Zekovic}{\citealt{zekovic_megaprojects_2016}}).
%Financing of komunalne izgradnje 2013 SKGO
%Zekovic sve ref za zemljiste 
    
\textbf{"The most dramatic problems of Belgrade’s urban development during the last 15 years, in the planners’ views, were linked to the pervasiveness of uncontrolled and illegal development. In the city, such patterns of development led to the deconstruction of its urbanity and the abuse of its public spaces."} \href{Vujovic}{\cite{vujovic_belgrades_2007}}
%ostavi cite

\subsection{Belgrade Waterfront Whereabouts}

Current large-scale redevelopment of the Sava Waterfront is commissioned to privileged foreign and domestic developers
(\href{Expert Workshop}{Workshop 1}),
%Expert workshop - Bata Stojkov prez}
according to the schemes mentioned above (\href{Section 6.1.2}{Section 6.1.2}).
The simple calculation is such that they received a good bargain from the Serbian authorities for the highly profitable urban land in the capital city (\href{Politika}{\citealt{politika_zemljiste_2015}}).
Even though Serbian tycoons had managed before to secure similar successful investments (\href{InterviewX}{Interview X}),
%slavka zekovic
experts say that the limited financial capacity of the state and  the  current  financial  crisis  have  contributed  to urgent needs for real estate investments after its sudden halt in 2010 (\href{ref}{\citealt{doytchinov_urban_2015}}).
\\

The Belgrade Waterfront Project (BWP) is a typical megaproject (MP) and a flagship of the dominance of the neoliberal doctrine in the Serbian context  (\href{ref}{\citealt{harvey_rebel_2012}}).
A widely known hypothesis states that the success of a MP depends on its appreciation of the local historical and current context, its ability to respond to the contrasting aims and objectives and the integration of communities and localities involved (\href{ref}{\citealt{hoyle_global_2000}}).
\\

However, what happened in the case of the BWP was quite the opposite, and it more closely corresponds to a Machiavellian formula for a MP’s roll-out with under-estimated costs, over-estimated revenues, undervalued environmental impacts and exaggerated developmental effects  (\href{ref}{\citealt{flyvbjerg_megaprojects_2003}}).
%find ref HK paper
\\

The situation in Belgrade (and Savamala) can be rather explained as a classic case of investor urbanism with powerful political figures having a fetish for investments, a powerful financial investor and a shiny, faulty 3D model - as an interviewed expert explained 
(\href{InterviewX}{Interview X}).
%Zekovic interview
Interestingly, another explained that having the actual model made all the socio-spatial faults of the project visible
\footnote{Contrary to this, usually urban plans in Serbia have no spatial interpretations so that their flaws and conflicts stayed hidden, especially for design and construction professionals and citizens (\href{InterviewX}{Interview X}).}
%Association of architects interview
(\href{InterviewX}{Interview X})
%Association of architects interview
And, without a doubt, the BWP is the instigator of a plethora of urban conflicts within Savamala (\href{InterviewX}{Interview X}).
%Dobrica interview
\\

A very important legal precondition for the realization and distribution of benefits within the BWP was a Joint Venture Agreement (JVA)
\footnote{While the shares of profit are: 32\% for Serbia and 68\% for the UAE partner (\cite{JVA BWP, 2015}) %legal
, the actual division of costs according to the agreement is estimated as fully reversed (there are unofficial estimations that 88\% to even 98\% of invested funds are to be borne by the Republic of Serbia).}
between the Republic of Serbia, Belgrade Waterfront Capital Investment LLC, Beograd na vodi, d.o.o. and Al Maabar International Investment LLC
\footnote{The data on this company owned by the investor are officially proclaimed secret by the Commission
for Protection of Competition. According to online data, the company was licensed in 2015 with 0 money share and with a limited license for 1 year  (\href{Analiza}{\citealt{inicijativa_ne_davimo_beograd_analiza_2016}}).} 
%Analiza ugovora "Beograd na vodi"
(UAE).
The agreement was signed in April 2015 and publicly announced on the website of the Serbian Government 5 months later (September 2015).
The consequences for the Serbian society are also as following 
(\href{Expert Workshop}{Workshop 1})(\href{Zekovic}{\citealt{zekovic_megaprojects_2016}}):
%Expert workshop - Bata Stojkov prez, Zekovic interview
(a) prompt law and urban regulation changes (Lex specialis, GUP 2021);
(b) cultural patrimony and architectural heritage contributed without financial benefits,;
(c) exemption from any legal duty (Law on Applying Agreement on Cooperation Between Serbia and the Emirates);
(d) protected assets provided for the private developer (BWPSPSP);
(e) state aid and expropriation for elite-housing and commercial spaces (Lex specialis),
(f) and lease of public land without a fee.
Conversely, the investor does not have to wait for construction to end and wait for the future gentrification of the entire area to make a profit; he gains that profit easily with minimum investment and with the minimal realization of the plan/agreement 
(\href{InterviewX}{Interview X}).
%Zekovic interview
\\

As worldwide examples show, urban decision-making on MPs is often exclusive with an ex post integration into planning documents (\href{Kennedy}{\citealt{kennedy_large_2013}})
%add from HK paper
and with possible serious legal and ethical issues (\href{Flyvbjerg2009}{\citealt{Flyvbjerg_survival_2009}}; \href{Flyvbjerg}{\citealt{flyvbjerg_megaprojects_2003}}).%(\cite{Flyvbjerg2009}; \cite{Flyvbjerg et al. 2003}).
%add from HK paper 
In this case, the BWPSPSP also allows the Serbian Government currently in power to gain the exclusive right to act in the center of Belgrade  (\href{InterviewX}{Interview X}).
%Zekovic interview
Such legal adaptations weaken the local authorities by imposing orders from the national level and the top-down pressure to realize projects with neither interference nor objection
(\href{Expert Workshop}{Workshop 1}).
%Expert workshop - Bata Stojkov prez
\\

In sum, there are no local power poles or real decision makers in Savamala, and everything is grounded upon the powerful, centralized nation state
(\href{Questionnaire Experts Savamala}{Questionnaire X}).
%experts questionnaire - Savamala
Moreover, the BWP exemplifies the closed wheel of decision-making procedure circumscribing political and economic actors into an interconnected and interdependent system.

\textbf{"Belgrade stepping into neoliberal trends, disregarding the relevant planning documents and causing them to change, lacking long term vision."} \cite{doytchinov_belgrade:_2015}

\section{Conclusion}

This chapter displays the MAS analysis of  the current urban reality of Savamala. It presents the analysis of urban agency in terms of elements and relations. The terminology of the MAS method is contextualized through the story of local contextual resources, social practices and urban conflicts. Such a structuralization of elements and events in Savamala sheds new light on the  analysed actors and their networks (\href{Chapter 5}{Chapter 5}). The aggregation of agent preferences summarizes the socio-spatial patterns identified in Savamala. Finally, setting up the relations of urban agency and the identified resources, practices and conflicts offers an explanation of its dynamics within the scope of urban decision-making layers.
\\

In theoretical terms, the interaction of urban agency with contextual elements and in contextual relations defines urbanity (\href{Section 2.1.5}{Section 2.1.5}) - the level of urbanity in Savamala is created from the fluctuating relationship between the agency and the context. While the initial imbalance produced in 2011-2014 worked pro social potentials and the activation of spatial capacities toward a vivid and diverse urban life, the intensive aggregation of urban conflicts from 2014 onward proved to be the other way around. There is a growing unification of functions and activities in Savamala, while civil life has been slowly retreating and moving away. Conversely, the waterfront area is growing at a stable pace with architectural mastodons bringing in a new, modern architectural front to the historical hinterland. The level of urbanity in Savamala seems to trot up and down various scales while its contextual elements and relations are exposed to the battlefield of multiple human and non-human agents.
\\

Following the operationalization of agents and their preferences presented in this chapter, several causal relations on the current circumstances in Savamala become conspicuous.
The social and spatial capital of Savamala were accentuated and activated through these abundant civic and creative initiatives.
These initiatives responded to global trends and the urgent need for social and cultural spaces in the city.
However, this alternative status of Savamala has insurmountable limitations - the obstruction and neglect from the State.
Thus, this relationship between alternative culture and institutions, instead of becoming a harmonized practice, has germinated into an urban conflict generator.
\\

Moreover, this theme park narrative of the BWP's projected future, the disastrous estimations of the obligations for Serbia and the secretive and murky circumstances of its current realization is an immense and even bigger source of urban conflict and a kind of exploitative mechanism for local capacities and potentials.
In these circumstances, urban planning institutions and documentations appear passive and figure only as an instrument of power and interest, replicating thusly either the practices or the conflicts at play.
\\

The identification of agent preferences through the links between the agents/actors and the context work out the current state of urban dynamics in Savamala. However,
as urban dynamics allude simultaneously to the continuous changes and constitutions of new realities and urban development prospects, the aggregation of agents’ preferences and the structure of their agents’ nature evolve into a dynamic category forming agent behaviour. The consequences of this agent behaviour are actually the forces of maintenance, transformation and/or change processes of an urban system. The proposed hybrid method that combines MAS and ANT could serve for setting up an iterative procedure that refers to urban development in terms of urban system transitions. Such a narrative may provide a framework to reframe urban complexity and to capture urban dynamics at the neighbourhood level.

\chapter{Urban Processes and Cross-Pollination of Data}

%MAS-ANT Data and Results Display}
The narrative of urban system transitions turns new light on urban development and its processual nature. It relates back to the totality of an urban system addressing harmonization, contestations and collisions of a variety of structural elements, social factors and vested interests existing in an urban environment.
While the theoretical scope of the levels of urbanity and the multi-layered morphology of urban decision-making circumscribe urban complexity and dynamics, there appears a gap between fluid interpretations of contemporary cities and operational urban development prospects.
\\

Due to these conditions, weak urban systems, such as post-socialist cities surely are, tend to poorly understand and anticipate trends, risks and changes (\href{UN}{\citealt{un_habitat_state_2012}}). 
The far-reaching aim is to capture urban complexity and the dynamics of post-socialist cities in order to avoid urban development and related mistakes based on the western planning paradigm. But the more general question remains as follows: how to systematically approach analysing a vibrant and fluid context in order to adjust and balance urban development processes (\href{Bolay}{\citealt{bolay_technology_2011}})?
\\

This thesis research argues in favour of methodological revisions, adaptations and complements for urban research. It applies a methodological hybrid that combines the Multi-agent system (MAS) and the Actor-network theory (ANT) to demystify contextual processes of system resilience, flexibility and contradiction (transformation, maintenance, and/or change) within the complex urban agency map.
\\

To begin with, these contextual processes will be bounded into the narrative of urban system transitions explaining how this MAS-ANT theoretical construct is to be applied on Savamala.
Bearing in mind the identified local actors and networks they engage in (\href{Chapter 5}{Chapter 5}), and having indicated the links between the actors/agents and the context (\href{Chapter 6}{Chapter 6}), the logics of agent behaviour will provide the explanations on interconnections between urban decision-making layers, localized socio-spatial patterns and the dynamics of contextual processes.
Later, this chapter also displays the collected and analysed data in a new MAS-ANT cross-pollinated set of agent profiles.
Tracking agent profiles indicates the paths of maintenance, transformation and/or change in Savamala as an urban development indicator. Furthermore, the MAS-ANT diagram displays the results in such manner that the multitude of urban elements (complexity) are incorporated, the variety of their links are mapped (integration), and the past-present-future processes of long duration (dynamics) are marked. As the final result, this chapter concludes with the elaboration of the flexibility trait of that diagram based on its potential for both a social and algorithmic approach toward urban development.

\section{The Body of Urban Transitions}

Focusing on how cities are built, how they function and what will happen in the future unavoidably sheds light on urban planning, a research-led and practice-based field that revolves around the continuous process of decision-making on future actions for achieving objectives  (\href{Stojkov}{\citealt{stojkov_teorijska_2012}}).
%Stojkov and Dobricic 2012 05
However, what is actually going on in cities is usually far from what is planned.
Moreover, urban system transitions are processes of long duration (\href{Braudel}{\citealt{braudel_history_1970}}).
\\

Examining the urban by overriding current space-time boundaries captures the pace of change and the multi-layered nature of transformation, with the focus on the process of change in the city’s economy, society, system of governance and the spaces of production and consumption. Thus, for analysing urban system evolution, multiple factors must be included: (1) political - internal political process and political regime, ongoing international relations; (2) economic - market forces and government influences; (3)  the cultural - social milieu, professional culture, educational expertise and civil society power (\href{NedovicBudic}{\citealt{nedovicbudic_waves_2006}}).
%waves of planning 2006
\\

The high dynamics, interdependence and overlaps of these and similar factors endanger the efficiency of urban system planning, which does not mean that the system was prevented from evolving. While most Western capitals gradually assumed their urban coherence and cultural identity, the urbanity of post-socialist cities has either been diluted during its turbulent history or happens immediately at an extraordinary pace  (\href{Doytchinov}{\citealt{doytchinov_modernization_2015}}).
The contextual processes in post-socialist cities that cover the actual on-site system evolution usually are not planned or directed, though they are also far from chaotic - in short time lapse they show their intrinsic logic, functionality and time-coherence which usually becomes complicated, neglected and blurred from a wider perspective. Even though the unrolling patterns for such processes are not obvious, their influence on the system evolution is quite deterministic - either maintaining, or transforming or changing it (\href{Section 2.3.1}{Section 2.3.1}). 
\\

To gain a fuller appreciation of the different types of contextual processes, already during the initial research phase, the dynamic, historical, contextual and on-site data were critically addressed and filtered through the rounds of data collection going from an abstract document-based version through group analysis in professional, educational workshops and elaborated interpretations from local experts and coming back to the theoretical stances.
\\

As discussed above, a kaleidoscope of collected and analyzed data on the Savamala neighbourhood have revealed sets of relationships between the identified actors, the morphology of urban decision-making and the level of urbanity. ANT analysis highlights individual features of actors and constitutes agent structures, while MAS elaboration explains their contextual preferences and involvement in urban affairs. Relating human and non-human actor-networks to urban decision-making mechanisms and urban agency to contextual elements and relations enables pointing out the key actors of urban system transitions - maintenance, transformation and change of urban systems.
\\

A multitude of human and non-human actors shape top-down, interest-based and bottom-up developmental action and influence the multi-layered decision-making structure in terms of decisions for maintenance, transformation and/or change of the system. 
Therefore, the character - profile - of these agents might provide the explanations of their references to the system evolution.
Agent profile is a combination of agent structure, agent preferences, and agent behaviour (\href{Section 3.3.2}{Section 3.3.2}).
In this research project, the agent behaviour combines agent structure and preference and refers to system development (maintenance, transformation or change). 
Agent preferences have a crucial role for tracing urban transitions:
(a) resources instigate transformations,
(b) practices identify system maintenance,
and
(c) conflicts boost potential changes.
On the contrary, agent structure in combination with these preferences determines the agent's role within the morphology of urban decision-making.
Moreover, the decision-making networks also define what is desirable, possible or what is actually happening.
\\

The analysis of the identified agents according to these principles gives us the opportunity to determine urban development prospects - their influence on the system evolution, their capacity to intervene and the biases that cause eventual negative effects.

\section{Profiling the agents}

The first step in displaying the body of urban transformations and developmental prospects in Savamala is the definition of agent profiles.
The core of agent profiles are their structure and preferences, based on ANT and MAS analyses of contextual, qualitative data. Agent structure and preferences are treated as dynamic features of urban agency and the level of urbanity in Savamala.
They are the basis for tracking the behaviour of these agents and their influence on the state of urban environment and urban system evolution in Savamala.
Finally, this has allowed the summing up their capacities and limitations and demonstrated how they influence urban development prospects.
\\

Post-socialist urban development induced radical political, economic and cultural shifts in neighbourhoods in Belgrade. Savamala has been marked by all major transformations of Serbian society over time and therefore is a representative case for intensive collision of top-down and bottom-up pressures.
Endowed with a prime location in the Serbian capital, Savamala has been directly or indirectly targeted by most of General Urban Plans (GUPs) and/or General Plans (GPs) since the beginning of the 20th century (GUP 1923, GUP 1950, GUP 1972, GP 2003 (revised in 2005, 2007, 2009, 2014), and GUP 2016) (\href{Section 4.2.1}{Section 4.2.1}).
It has also captured the attention of national and international capital through glorious architectural projects ("Town on the Water", CIP Europolis, Beko Masterplan, Belgrade Waterfront Project etc.) (\href{Section 4.2.2}{Section 4.2.2}).
Following the trends, it has also fallen under a recent, massive, but rather disputable waterfront mega-project, that aims to remodel the Belgrade landscape according to modern high-rise metropolis patterns (\href{FigureX}{Figure X})(\href{FigureX}{Figure X})
%Figure 2 - MAS-ANT paper scale Savamala and BWP
In Savamala, a complement of post-socialist urban development is found in small-scale cultural practices, crowdsourcing civil and  activities, artistic projects and educational programmes which, slowly but surely, spread from the upper Savamala to the riverbanks (\href{Section 4.2.3}{Section 4.2.3}).
\\

The hybrid field of overlapping MAS (Multi-agent system) and ANT (Actor-network theory) methodological approaches proposed an innovative concept to define causal relationships among all the various urban elements and developmental prospects of their inter- relations and interconnections.

\subsection{Agent structure}

For the analysis of agents' structure, their basic characteristics are already identified within the key categories from the ANT methodology (\href{FigureX}{Figure X}).
%Figure 3 - Spatium and MAS-ANT
These ANT categories indicate the figuration of the chosen agents in their environment. They are adapted according to the interpretation of the ANT methodological approach applied in this research.
Agent structures are circumscribed based on the roles these agents play in Savamala, i.e. (\href{Cvetinovic}{\citealt{Cvetinovic_participatory_2016}}): %Spatium and ANT

\begin{enumerate}
\item agent nature - its operational manifestation;
\item level of influence - the boundaries of the activities and target groups;
\item structural networks - the agent's primary activity;
\item socially functional networks - the social function they are assigned;
\item secondary networks - the subordinate function(s) they take up.
\end{enumerate}

Agent structure is the product of thorough ANT analysis (\href{Section 5.1}{Section 5.1}). 
(\href{FigureX}{Figure X})
%Figure 2 Spatium and MAS-ANT 
and summarizes the structure of the set of key urban  agents based on the data provided from the key informants. Various layers of decision-making reflect the difference in agent nature, while structural networks interpret the functions of these agents at the local level.
\\

The social function of the agent is strongly connected to their level of influence in the case of bottom-up actors. They are all active at the local, but less often at the city and international levels, though their international visibility is also more in the domain of funding - several are the recipients of international financial support from foreign embassies (Swiss and Netherlands embassies) and European cultural and art foundations, organisations and programmes (Felix Meritis, Balkan Design Network, EU funding instruments) or under direct supervision of international entities. Urban Incubator Belgrade (UIB) was the initiative of  the Goethe-Institute.
\\

Top-down agents are activated in the form of documents (BW SPSP
\footnote{Belgrade Waterfront Spatial Plan for Special Purpose Area}
and GUP BGD 2021),
\footnote{General Urban Plan for Belgrade - different versions}
institutions (MCTI
\footnote{Ministry of Construction, Transportation and Infrastructure}
and UPI)
\footnote{Urban Planning Institute}
or the assigned roles in the public domain (Prime Minister, Cty Mayor, City Architect).
\\

The dual position of the Ministry of Construction, Urbanism and Infrastructure (MCTI) as both normative and executive planning body after the ambiguous discontinuation of the Republic Agency for Spatial Planning (RASP) opens the floor for the twisted institutional practice reflected further on in agent behaviour. The choice of policy agendas is based on their problematic engagement in the enactment and changes of the regulatory mechanisms for the wide area of Belgrade waterfront. BW SPSP and GUP BGD 2021 received harsh professional criticism. The first relied on the multiple misinterpretations of the Planning and Construction Act (PCA), while the modifications of the second enable construction of profitable high-rise commercial and residential buildings in the coastal area. Furthermore, in the process of BW SPSP adaption, the City Architect and its cabinet exert power and influence in the expert control and public inspection process (\href{Pravni skener}{Pravni skener 2016};\href{NDVBGD}{NDVBGD 2016)}).
The responsibility and the competitiveness of the local authorities is insignificant in this case and is being controlled mainly through political mechanisms (political party structures). 
\\

The key inspiration for these actions has been the private investment of the Belgrade waterfront project (BWP). Similar ambiguities have happened to the amendments to the City Regulation Plan (CRP) concerning the Beko factory location nearby. The foreign investor figurated by Lambda Development financed the drafting of the CRP and directly influenced the makers of the plan by employing them afterwards. Conversely, the initiative NDBGD is the key entity performing control of urban governance outside the top-town regulatory framework. Through its actions (events, media and publishing) NDBGD has constantly advocated for more transparency and participation in the Serbian urban planning framework.
\\

The engagement of various individual human and human actors surrounding the BWP investment is mainly assembled around the financial agreements and benefits (structural networks). Following contradictory information in the media, the exact details on the profit distribution is not yet clear (\href{ref}{\citealt{malenovic_ko_2016}}).
%profit distribution - media
These circumstances result in the conflictive relations on land use and property issues (secondary networks in reference to space and the built environment):
(o) the harmful, illegitimate, unconstitutional decision on expropriation of land for residential and commercial purposes based on the special law (Lex specialis) (\href{Danas}{Danas 30/03/2015});
%Danas advokati
(o) the misleading information on the land property agreement, the land is not rented for 99 years, but granted to the investor as soon as the buildings are constructed (\href{Politika}{Politika 20/03/2015});
%media Politika otudjenje zemljista
(o) the obscure circumstances of the initial apartment sale before the construction even began - who actually governed this sale managed by the BWC (Belgrade Waterfront Company) and how (\href{ref}{\citealt{lazovic_kako_2015}}; \href{ref}{\citealt{bukvic_ko_2015}});
% pescanik stan
%media Danas Macka u dzaku Danas};
and
(o) the unrealistic real estate market circumstances in Belgrade and the irrationality of residential construction projects such as the BWP while a significant amount of recently finished residential buildings are empty in the extended central area of Belgrade (i.e. Belville in New Belgrade) (\href{Bukvic}{ibid.})(\href{Expert Workshop}{Workshop 1}).
%media Danas Macka Busatlija, expert workshop 
\\

Conversely, another important question rose around the infrastructure and infrastructural works in the BW area (secondary networks).
As the public utility services and municipal infrastructure are the main sources of costs and profits on land, neither of the economic regulatory rules have been applied in the case of BWP, such as the operational separation of the executive and regulatory role, stability control and transparent mechanisms(\href{SKGO}{SKGO 2013});
%Finansiranje komunalnog opremanja gradj zemljista 2013 SKGO
The costs for the infrastructural works are: underestimated, large in scale and unrealistically planned under the agreement bonds,
\footnote{Among others, the infrastructural works incorporate: constructing the vehicle and railway bridge near Vinca, relocating the railway station and bus terminal, finishing the railway station "Prokop" and building from scratch another single or multiple bus terminals (\href{Politike}{Politika 2015}).}
%Akademija Arhitekture Politika march 2015
and in practice marked by murky financial deals with investors\footnote{According to informal sources from the BWP construction site, the initial infrastructural works are funded by the foreign investor and will be reduced from the land indemnity payment obligation. The circumstances of the value calculation and obligations for both sides are not known (\href{InterviewX}{Interview X}).} 
%BWP interview
(\href{InterviewX}{Interview X}).
%BWP interview}
This means placing an additional burden on Serbian tax payers in the future and endangering the economic prosperity of the country rather than boosting it.
Moreover, the role of public utility companies (socially functional networks) is side-lined, devaluated and reduced to vertical, political obedience, e.g. Belgrade Waters [Beograd vode], Serbian Railways [Zeleznice Srbije] etc.).
\\

Conversely, in terms of agent nature, the chosen bottom-up agents figure as sets of horizontal entities of events/projects/activities. In this respect, the scope of secondary network characteristics is focused on urban or NGO sectors or small-scale services.
Moreover, socially functional networks formed from the ground up are mainly formal/informal collectives with non-transparent or unclear internal organizational or foundational procedures.
\\

KC Grad, as the first informal collective settled in Savamala, was also known as the "Dutch center" as the Dutch embassy provided the initial funding support to establish such alternative cultural center in Belgrade at the moment when the Student Cultural Center (SCC) [Studentski kulturni centar (SKC)] was falling apart and the "Dom omladine" Cultural Center was under reconstruction (\href{InterviewX}{Interview X}).
%KC Grad interview}.
Mikser took over the leading role as the local stakeholder and has kept it to this day, with significant efforts made to gain an overview of
what is going in Savamala.
\footnote{In this sense, Mikser also took part in or actively supported several projects such as UIB, The Game of Savamala, My Piece of Savamala etc.}
However, the two of them are the only organizations/ collectives/ non-formal actors to be included in the new plans for Savamala (BWP SPSP, GUP 2021, Investor's Master Plan for BWP).
\\

The analysis of agents structure and qualitative data on the Savamala neighbourhood indicates urban assemblage networks formed and contributed from the ground up. Namely, implementation and management of participatory activities is the focal point of urban interventions in Savamala. These networks involve a range of local and city NGOs as well as several IOs, initiatives and collectives. In a few cases (Urban Incubator Belgrade, Mikser festival etc.), municipal authorities provide support in these managerial networks. However, local, municipal and city authorities as well as international funding organizations take part in financial networks and of several projects in the implementation networks (i.e. Savamala, a Place for Making; The Gthe ame of Savamala; Camenzind, NextSavamala and Savamala design studio projects within UIB).
\\

In the case of UIB, the activities in Savamala also comply with the campaign of the Goethe Institute to focus part of their activities on ”Cities and Urban space” in their branches worldwide. Namely, there are several different Urban Incubator projects around the world. The Goethe Institute assembles local and international actors around participatory and educational projects to rething the city and practice citizenry. In this respect, they are endeavors to involve local educational institutions (Universities and most often Faculties of Architecture and Arts) and to foster collaborations among them. Setting the extensive and diverse network in the local context, UI and the Goethe Institute are not the primary agents on the ground.
\\

On a limited level, a few agents (My Piece of Savamala, KC Grad, Mikser) engage in consulting networks with municipal and city authorities (City mayor, City architect, Municipality of Savski Venac) and real-estate actors (Eagle Hills, the Belgrade Waterfront Project). In these relations, critical connotations against the authorities are aptly avoided and the projects and participants become side-lined. 
\footnote{This happened to the "My Piece of Savamala" project.  As soon as they submitted the results, which did not comply with the official solution for the park in Savamala, to the authorities and to the BWP officers, the communication was interrupted and they have not received any reply to date
(\href{InterviewX}{Interview X})}
%School of Urban practices interview november 2016
\\

Finally, the sole interest of  the  Ministry of Space [Ministarstvo Prostora] collective and Ne da(vi)mo Beograde initiative (NDVBGD) is the activation of the control and verification networks for all urban questions, problems and solutions. The agency of these entities overlap in the course of individual participants and actions as NDVBGD is the initiative led by this collective.
In general, their civil and urban activism refers back to the city as the place for all its citizens and to city and national urban and political authorities and experts respectively.
\\

In general, various structural networks clarify agents’ roles and indicate the paths of their behaviour and networking capacities at the local level.

\subsection{Agent preferences}

The second level of data analysis is MAS analysis of agent contents, fields of interest and influences.
Based on the MAS-ANT methodological pollination, agent preferences are defined in accord to their relationality towards the contextual resources, social practices and urban conflicts figuring in Savamala (\href{Section 6.2}{Section 6.2}).
In this manner, we become aware of their field of manoeuvres in Savamala. 
\\

The visualization of agent preferences confirms that the spatial and locational capital of Savamala is the main attraction for human and institutional actors at various levels.
%diagram
Interestingly, locals have not exclusively perceived it like that.
According to some, the BWP is the recent spark that has brought back economic and aesthetic value in the neighbourhood
(\href{InterviewX}{Interview X}).
%citizen Retired 
Conversely, the recent wave of refugees from the Middle East is perceived as a danger and a problem 
(\href{InterviewX}{Interview X}).
%citizen Retired
\\

Their short-term visions suggest that refurbished facades and cleared waterfront are seen as a step forward, while the Savamala inhabitants actually show little interest in the illegal and corrupt actions surrounding the project. In the city context, this shows that shallow actions and obvious cosmetic changes have wider social effects than any disastrous decision that is limited to policy agendas or documentation in general.
\\

While Savamala's contextual resources are of special importance for Belgradians and Savamala inhabitants, the attitude of the investor shows no difference between the Belgrade waterfront or any other waterfront in the world, and especially in the Arab/Persian Gulf
(\href{InterviewX}{Interview X, Y, Z}).
%ref Dobrica, Slavka, KC Grad}
The investor plans to build: the coastal fortification, walking paths, urban furniture, two residential towers (20 storeys each), the central tower (160m) with a network of streets leading toward it, a shopping mall behind this tower, and a hotel next to Brankov bridge (\href{InterviewX}{Interview X}).
%BWP interview
Numerous expert assert that the design of the area and buildings better correspond to the environment between the dessert and the sea, as in Dubai, than to that of Belgrade 
(\href{InterviewX}{Interview X, Y}).
% interviews Zekovic, Cagic, Kucina
The BWP spatial plan and functional programme show that it does not take into account the contextual resources and urban conflicts of the whole city, as a project of such scale should. Its only purpose is access - people come, park, settle down, and consume - as an informant close to BWC explained 
(\href{InterviewX}{Interview X}).
%BWP interview
\\

In contrast, the professionals advocated for an integral development strategy for Savamala, scientifically and practically elaborated that is actually based on intrinsic qualities and that addresses the well-known urban conflicts at the city level, such as: (1) linking the design solution to a sustatainable urban mobility plan, the project was prepared within the United Nations Development Programme (UNDP) with the support of city authorities through the engagement of the Land Development Agency (or Directorate for Construction Land - direkcija za gradsko gradjevinsko zemljiste); (2) promoting the policy of land management and sustainable restitution plan; (3) supporting the financially liable project to be financed from City budget; (4) reducing design costs through viable competitions, (5) the participation of representatives of the city authority and city management offices in the cut-off workshops and decision-making 
(\href{InterviewX}{Interview X}).
%UZ Zaklina interview}.
\\

The first, local source of problem was the position of wider area of Savamala on the border between two city municipalities - Savski Venac and Stari Grad, with different management structures and priorities. Morever, an insurmontable obstacle for such a project at the city level was unresolved property issue and illegal construction in the area
(\href{InterviewX}{Interview X, Y}).
%UZ Zaklina interview, Kucina interview 2015}
Under any other circumstances and under different times, the action is usually halted.
However, these problems were solved in a blink of the eye as soon as it was promoted from the national level and by the political party in power. 
Large scale urban project and big investment were seen as a source of finance and popularity for the new elite coming to power with the political change that followed the national elections in 2012
(\href{InterviewX}{Interview X}).
%Kucina interview 2013
After all, it showed that the centralized state is the main power pole of urban development in Serbia.
\\

When complemented with information from expert and participant interviews, the conclusion is such that the urban regulatory framework has been purposely distorted to accommodate private (investors’) interests.
In so doing, city authorities and city planning departments are the pillars of such biased governance mechanisms, while failing to provide adequate expert control of the planning processes, development plans and implementations and to enable a smooth transformation of post-socialist urban systems.
\\

Finally, BWP SPSP is a flagship among regulatory documents that legitimizes blurred, non-transparent, unfeasible, interest-based urban planning procedures. In addition, the BWP lead in itself manifests an official disregard for expert opinion, induces citizen revolt and causes a lack of faith in public authorities. On the city scale, investment-based initiatives like this one reduce the potential of democratization and decentralization of urban processes.
\\

Even though social potentials are nominally addressed by policy agendas, the leader of gradual, small-scale, participatory urban changes and possible brownfield regeneration has been the cultural initiatives and NGOs settled in Savamala in the last three to five years. 
As they are part of the NGO sector, provision of spaces has been 
a crucial question, not for their existence, but for their visibility and activity at the city level.
In this respect, "Magacin u Kraljevica Marka" (MKM) and "Kraljevica Marka 8" (KM8) are perceived as anchor places.
Obtained for use from the Municipality "Savski Venac", they are  shared among different NGOs and offered for multiple projects/activities/events from different actors, even outside these organisations.
\\

Although collaboration is not a strong point among these bottom-up initiatives, these shared spaces incite the sparks of mutual projects and actions usually among those organizations based on the voluntary work of young activists and professionals (around Goethe Guerilla) (\href{InterviewX}{Interview X}).
%Olja interview
In general, most of the participants and organisers are architects, artists or cultural workers. They are actually people with different backgrounds, but they all share the same interest for continual personal development and education and critical approach towards their profession and their city (\href{InterviewX}{Interview X}).
%Olja interview
\\

These organisations are found in the middle of real estate interests around the BWP and of cultural institutions in the city (”Dom Omladine”, the Goethe institute). As a consequence, they were eventually dislodged from the KM8 when the place was officially returned to the city authorities at the end of 2016. Even though in limbo while trying to acquire new place, they still work on strategies for future projects on participation and better internal organisations that will help popularize and widen the scope of their actions (\href{InterviewX}{Interview X}).
%Olja interview
In terms of space, they usually meet informally or are contributed space from their partner and friend organisations.
\\

Another way of profiting from Savamala's contextual resources is the non-governmental promotion of biking culture, an organised action to fight urban "pollution and stink"
(\href{InterviewX}{Interview X}).
%Biking tours - private}
Savamala's location (waterfront, proximity to the marina, half-way to Ada and Novi Beograd) led to positioning the headquarters of the biking community there. 
Following the international boost and support for city biking, the biking community "Streets for Cyclists" became an important actor and one of the few listened to and consulted by the city authorities (City Mayor and City Architect) (\href{Kontrapress protest zbog blokade}{Kontrapress 16/07/2015}}; \href{vice}{Vice 16/07/2015}; \href{danubeograd}{Dan u Begradu WEBSITE 27/11/2016}) (\href{Section 6.1.2}{Section 6.1.2}).
\\

Namely, the majority of these bottom-up agents actually aspire for a consulting role within the wide field of urban issues, culture, art and education or to implement a range of ideas/solutions/interventions at the urban or social level. In the Serbian context, they aim to provide an alternative body for catalysing available human resources and translating global knowledge into the local context of Savamala and Belgrade.

\subsection{Agent Behaviour}

The final stage of the analysis is MAS-ANT cross-pollination on the level of agent behaviour. The broad domains of the agent profile answer the question of who and what acts in the network of complicated relations among human and non-human urban elements, and how.
\\

In Savamala, dynamic, interactive urban assemblages are articulated through decision-making mechanisms of urban planning, real estate and co-design and creative participation actions.
The behaviour of agents enables tracing the interrelations of contextual resources, urban conflicts and social practices (MAS) (\href{Chapter 6}{Chapter 6}) in these urban assemblage networks from ANT (\href{Chapter 5}{Chapter 5}).
\\

Another layer of ANT-MAS cross-pollination is the illustration of agent behaviours in terms of active-passive relations. In this way, they personify the links between key urban agents and recognized political, economic and cultural factors of post-socialist urbanity. Namely, ANT assemblage networks altogether describe Savamala’s urban complexity, while MAS interpretation adds active-passive roles to key urban agents and their contextual preferences. All these behavioural paths are also confirmed within the qualitative data.
\\

%table 1
[\href{TableX}{Table X}] explains how political, economic and cultural tenets of post-socialist urbanity are distributed among urban key agents and in which way they influence Savamala’s con- textual elements. The correlation between spatial capacities and urban conflicts, which are dominant preferential elements of the chosen urban key agents, are of special interest for such analysis. The level of post-socialist urbanity directly depends on the balance between how the society/local community as a whole profit from contextual resources and reduce the negative effects of urban conflicts.
\\

The analysis reveals that centralized decision-making and provisory rules, among other political aspects, are the front runners of biased exploitation of the spatial capital. The inconsistency of expert opinions and lack of coordinated actions for supporting certain professional values and defending the position of experts in decision-making on important social subjects also poses a threat on the regulation of planning practice and implementation of planning documents (\href{Vukmirovic}{\citealt{vukmirovic_city_2013}}).
%Vukmirovic et al 2013
\\

What is more, the private investors, who easily satisfy their interest regarding Serbian spatial capital, do not have any concern or interest for the social benefits of the society. Instead of enabling the social potential of current local cultural and civil initiatives, they contribute to their extermination from the neighbourhood. This new picture of a trendy and rather safe Savamala brings with it threats of the expulsion of the local population, marginalized groups and alternative culture
(\href{Krusche}{\cite{krusche_bureau_2015}}).
%Krusche and Klaus, 2015).
In addition, the disjunction between expert and practical knowledge and regulatory and implementation actions contribute to the deterioration of the quality of urban life.
\\

These ambiguous relations induce collisions in financial, implementation and regulatory assemblage networks. In terms of spatial capacities, they particularly threaten highly profitable urban land (\href{FigureX}{Figure X}).
%media tag cloud
Institutional actors (authorities and urban planning) cherish and defend corruption and opportunism in urban transactions and land development instead of fighting and curbing it (\href{NedovicBudic}{\citealt{nedovic-budic_mornings_2011}})
%Mornings after Nedovic Budic
Even when the instruments to fight it are available (a tax on extra profit, reform of the security sector, reform of the tribunal, reform of the prosecution, mechanisms of restitution and denationalisation) (\href{Vujosevic}{\citealt{vujosevic_postsocijalisticka_2010}}),
%Vujosevic et al. postsoc tranzicija i teritorijalni kapital
in day-to-day administration of urban
development, there isn’t the political will and institutional firmness to implement them.
\\

The coordination of actions, resistance to political pressures and endurance of regulatory processes among institutions at different levels is essential for setting up adequate and efficient arrangements in the urban land market, urban land use management and land taxation systems in compliance with urban development goals. This may also be extremely difficult in a political context where elections occur very often and the institutional structures and personnel have been changing accordingly.
\footnote{In the last 6 years, three rounds of elections have been held (1) Early Presidential Elections (2012), (2) Early Parliamentary Elections (2014) , and (3) Early Parliamentary Elections (2016).
The 2012 elections proved to be a turning point as the Serbian Progressive Party (SPP) took power, led by Tomislav Nikolic and Aleksandar Vucic. The next elections did not bring any change, thus further strengthening the SPP’s power and especially that of its leader, the current Prime Minister (2017).}
\\

On the other hand, the cultural cluster, which was in a full swing 2012-2015, was so neglected by the official authorities that it has not managed to produce real, long-lasting positive effects in terms of place-making, the promotion of arts and the integration of culture in the city, and to positively influence local tourism. These tenets of creativity for urban development in Savamala were wasted in vain when the whole establishment turned to the BWP.
While national and international experts, cultural and civil workers and several independent media tried to protect this new spirit of Savamala, the Ministry for Culture, Official and Popular Media in Serbia did not work in favour of cultural purposes
(\href{InterviewX}{Interview X}).
%KC Grad interview}.
The destiny of these activities and places depend on private investors and their ”buddies” in the city and the national authorities and their whims, interests and decisions. As they are not interested in volunteering or any socially beneficial work or cultural and artistic practices, even if the purposes of city branding are at play, the only left option for these organisations is to leave Savamala
(\href{InterviewX}{Interview X})
%Mikser interview}.
\\

In terms of urban conflicts, these are already entrenched in the complex relations surrounding contextual resources in Savamala as has  already been mentioned. Apart from the institutional incapacity, the major source of conflict is the lack of participation, transparency and respect for the general public interest. In Savamala, this issue is of special importance as all three layers of urban decision-making express a certain interest and have a certain power within the Savamala boundaries. That said, it is important to acknowledge that the clash of the authoritarian and civil sector in Savamala over the controversial megaproject (MP) is the shifting point of agents’ behaviours in Savamala when comparing it to other neighbourhoods in Belgrade and Serbia.
\\

Aside from regulatory adjustments to accommodate private investor interests that have paved the way as an official planning practice and instrument in urban decision-making in Serbia, this narrative at the neighbourhood level has another actualization. The cultural and civil activities in Savamala are on the verge of accommodating, and even catering to the interests and tastes of the new audience coming to the neighbourhood, with whom they share few values 
(\href{InterviewX}{Interview X}).
%KC Grad interview}.
Other activities are on the verge of expulsion with no real means to fight it. 
\\

This was the case for the refugee center "Miksaliste", whose demolition was announced only 48 hours beforehand, which is on no account sufficient for moving infrastructure and people
(\href{InterviewX}{Interview X}).
%Mikser interview
Moreover, disobedience was also out of question, because  the destiny of hundreds of refugees pouring in Serbia and Belgrade was at stake (\href{InterviewX}{Interview X}).
%Mikser interview
\\

On the other hand, there are others (NDVBGD) who do not want to communicate and collaborate with the BWP staff and supporters as long as certain illegitimate issues that threaten public interest are not solved (\cite{ref media}).
Even though their actual achievements to stop the project are minor, their public actions, elaborated documentation, sharp analysis and even legal moves have raised awareness about the social values and rights in the city.
They have also made the illegitimate circumstances of this project visible at an international level.
\\

Following their civil protests in the capital with dozens of participants and the official, high level criticisms that they provided from numerous experts, the unresolved issue of the Savamala night demolition and terrain clearance became an issue of high citizen priority and a burning political issue. 
\footnote{David McAllister, European Parliament rapporteur for Serbia, mentioned the "case of Savamala" in a draft of his latest report before the Foreign Affairs Committee of the European Parliament (EP). He was asking for quick and efficient legale resolution of controversial circumstances around Savamala demolition (\href{stojanovic}{\citealt{stojanovic_fajon:_2017}}).}
%Danas 09-10 januar 2017
From the point of view of the tourists coming to Belgrade, after their positive suprise at the scale of construction at work in Savamala, they soon became disappointed when they heard the whole story about the demolitions, illegality and above all about the "Dubaization" of the Belgrade waterfront, which they were not eager to come to see in the Balkans (\href{InterviewX}{Interview X}).
%Biking tours - private
\\

Finally, however problematic, the amenability of urban plans and its implementation in the Serbian planning discourse was historically built and consolidated during the socialist period. In the socialist discourse, with enough political will, any urban project can be realized, at least it could be provided with the regulations and documentation needed. For example, the introduction of new building/construction indexes that make possible the construction of high-rise buildings at the Belgrade Waterfront actually means that it will be the new reality on the Belgrade skyline (veduta). As soon as the indexes are raised in official documents (legal and technical), they will not ever be lowered again 
(\href{InterviewX}{Interview X})
%Association of architects Cagic Milosevic interview
In terms of urban practices, the system reproduces itself and is slowly coordinating with European legislation and trends; the city is functioning, but what is happening is rather ”spontaneous occurrence” than a strategic pathway.
\\

The second step of agent behaviour analysis is estimating its influence on the distribution of urban agency within urban assemblage networks - the differentiation between proactive and passive urban key actors.
%table active-passive Spatium and MAS-ANT
[\href{TableX}{Table X}] therefore displays the preliminary division of agents according to the ANT and MAS analyses performed previously (\href{Chapter 5}{Chapter 5}; \href{Chapter 6}{Chapter 6}).
\\

Browsing through different networks, the distortion of post-socialist institutional framework is conspicuous, because policy agendas usually feature as passive elements despite their essence as a regulatory framework for strategies and actions. What is more, personalized institutional relations, figuring in particracy dominance over political power, make public authorities more vulnerable to a variety of political and economic interests. Horizontal, vertical and cross coordination among individual decisions and public policies loses its professional feature and falls into the domain of marketing and political campaign. Leaving decisions on the public interest in the hands of individuals eases the abuse of power. Furthermore, the weakness of the institutional framework leads to the lack of control and flawed implementation of sanctions.
\\

Another important question is the responsibility for the design solutions of the prime urban location in the Serbian capital. Namely, the Belgrade Waterfront Capital Investment LLC (the company of the foreign investor) is the sole provider of the design, construction, management, marketing and trade activities according to the Joint Venture Agreement for the BWP  (\href{NDVBGD}{NDVBGD 2016}).
The foreign investor defines the Master plan and all levels of planning documentation, while the Republic of Serbia is responsible for its legalization, terrain clearance and infrastructural works (\href{JVA}{JVA2015}).
\\

Finally, the citizens (inhabitants of Belgrade and Savamala) are the most neglected party, deprived of any role in the public domain and even of that to express their preferences and interests regarding urban changes that affect their own property. The authorities are not actively providing any help and support in defending citizen rights and interests. Moreover, when citizens officially require it, the procedures are so complicated and time consuming that people do not dare to engage in such possibly vain efforts 
(\href{InterviewX}{Interview X}).
%Citizen Gezovic interview}. 
\\

The two levels of agent behaviour explain the proactive or passive role of agent structure and connect agent preferences with urban assemblage networks. 
%Figure 4 MAS-ANT
[\href{FigureX}{Figure X}] visualises the MAS-ANT analysis in relation to agent structure, preferences and behaviour of urban key agents in Savamala.
In this diagram the links between contextual preferences, the social aspects addressed and the agents with their explicit structure (level, nature, and functions) and roles (proactive-passive) are explicited.
\\

In this respect, we can acknowledge urban system references - the indicators of maintenance, transformation and/or change.
The agency and relationships of the urban key agents are the cornerstone of urban assemblage networks constituted at the local level.
As they primarily depend on the contextual preferences which the agents attribute to their activity and relations, tracking these associations is also a crucial factor of urban development processes, if there is any.
Therefore, with the MAS-ANT method the aim is to describe the resilient components that maintain the urban system, suggest the flexible factors of constant adaptations and point out the threats and opportunities for dynamic shifts in urban system evolution.

%media tag cloud

%preklapanje assemblage networks i preferences in intensity diagrams (Olja)

\section{Decision Making Layers' Scenarios as an Indicator of Urban Development Prospects}

Multiple maintenance, transformation and change actions that influence the state of the urban environment in Savamala are identified according to agent profiles and networks.
Contributing to the body of local social practices, and benefiting from social potentials and spatial capacities (contextual resources), as well as addressing urban conflicts involve the continuous reviewing of how the collision of these positive and negative influences actually produce a variety of opportunities for maintenance, transformation and change.
In this case, the conceived social aspects (political, economic and cultural) of Savamala are those that contribute to local resources, conflicts or practices and thereafter aspire to generate system evolution (\href{Cvetinovic}{\citealt{cvetinovic_engine_2013}}). 
\\

The current situation in Savamala is a prime example of the overall state of Serbian society and its attitudes and propensity towards evolutionary processes.
Namely, in Serbia transformations usually go with the flow, especially those that engage a broad civil audience, like the cultural initiatives in Savamala did
(\href{InterviewX}{Interview X}).
%KC Grad interview
On the contrary, having something strategized, planned and implemented necessarily involves political power poles and financial actors (\href{InterviewX}{Interview X}).
%KC Grad interview
The structure is replicated and multiplied from the local to the national levels emphasizing an authoritarian state of mind and values for the society as a whole.
\\

Bearing this in mind, the case of Savamala is also an example of what is possible in Serbia. In this reference, the future prospects of different decision-making layers will be described in chronological order - based on their occurrence in Savamala and their engagement into its developmental interventions - from bottom-up actions, a mega-project to the urban regulatory framework and state institutions who were mere providers of the framework and the framework of legitimacy.

\subsection{Operationality of bottom-up, on-site interventions}

In order to identify and elaborate how participatory activities influence urban development in Savamala, it is essential to translate these qualitative categories into factors which could denote a positive impetus.
\\

First of all, the settlement of civil organisations has supported service and commercial activities and recreation zones already present there. Several traditional craft shops have been in Savamala for decades and now. Following the hype of its low-profile, bars and restaurants, as well as art and culture initiatives fostering cooperation, globalisation and modern business trends have positioned themselves there.
Their significance, not only in the city but also at the international level, promotes Savamala among architects, artists and all young creative workers of the region and Europe.
\\

Visible spatial transformations are (1) the activation of the waterfront area (for a while activities and events were organised on the abandoned ships on the Savamala coast before they were removed), and (2) the preservation and improvement of cycling paths (the initiative of the Streets for Cyclists NGO). (3) The preservation of skills and traditional crafts (Savamala, a Place for Making project); (4) fostering the sense of community and sharing (UIB was the pioneer in this participation, followed by the Goethe guerrilla collective, which organises and supports civil, participatory and design activities and operates in the KM8 community space); and (5) informing and educating the public (The Game of Savamala, My Piece of Savamala and other projects and programmes) are the major social transformations which have been directly induced by this pioneer bottom-up agency. Moreover, the local population emphasises that these participatory programmes, with reference to their organisational preferences and capacities, take into account the needs of the locals (Cooking together, Roasting Peppers), youngsters (UIB) (\href{Muller}{\citealt{muller-wieferig_urban_2013}})
and marginalised groups (Ministry of Space and NDVBG)
(\href{Mitic}{\citealt{mitic_ekskluzivno:_2016}}).
%nedeljnik
Conversely, the development of Savamala’s creative cluster, small-scale hype brownfield regeneration and public place design are the major smooth transformations that have made Savamala visible on an international scale.
\\

Having followed the aims and results of the activities in Savamala analysed herein, the following capacities worked for Savamala to transform a crisis of aggregated urban conflicts into an opportunity for urban development:
mobilisation of available local human resources,
complying with current global trends in participatory urbanism, low-budget revitalisations and creative economy initiatives, educating the apathetic local population on the importance of active participation in urban planning and development,
and
having a critical and "learning by doing" attitude  towards urban planning.
\\

It is also important to acknowledge that, even so, local citizens are not the main actors in these interventions.
In this manner, the bottom-up nature of the agency in Savamala is rather limited to the activation of the alternative and non-institutionalised cultural scene with the focus on the whole city, as well as the aggregation and multiplication of similar NGOs in Savamala.
However, negative changes have taken place as well - the first intrinsically bottom-up organization in Savamala (the Club of Savamala fans and friends) having been placed in the middle of different agendas and interests, has ended up as a type of informal political body in service of the party in power.
\\

To this extent, the livelihood of Savamala is still assumed to be at least disseminated from the ground up through the social bonds between different social groups (artists, youngsters, students, senior citizens) and among neighbours and locals.
The vibrant atmosphere in Savamala was achieved through the mutual efforts from participation and dialogue among these urban actors with different backgrounds.
At some point, these internal relationships surpassed all their campaigned and institutionalised initiators (UIB, Mikser festival), being followed by informal events such as:
meetings of the locals in the "Spanish house" space, the co-action of roasting peppers, and open access to spaces for artistic and educational purposes (KM8).
Moreover, through several of the activities, a variety of urban actors have become engaged in using these open public spaces (UIB, KM8, Mikser festival) and they have been actively thinking and imagining what the positive future of these places might be. In this light, the major benefit that could transform the socio-urban landscape of Serbian cities is the strong expression and statement of cultural and artistic interests within the agendas of these activities and raising the awareness and promotion of participation in the urban domain.
\\

Though it may also sound pretentious, the intensive UIB media campaign  and the role of the Goethe-Institute have certainly paved the way for Savamala and have ensured its place among the European neighbourhood symbols of creative clusters and urban upgrade potentials.
In response, it should be attentive to the possible negative effects of such a trendy image that could lead to gentrification and the expulsion of the current population.
The growing presence of Savamala in the media has also led to the exposure of its contextual resources to several powerful and uncompromising actors.
In addition, instead of exploring the potential of bottom-up approaches, actions and actors, certain decision makers have contributed instead to the commodification of culture and space and resorted to transnational companies  to support their activities.
In sum, the lack of strategic development goals, public funding and institutionalised approaches for cultural institutions and agendas have certainly made these bottom-up activities seem ephemeral and sporadic. Consequently, they could easily be wiped away by the whim of more powerful interests and political influences focused on Savamala’s spatial capital.
\\

Urban change induced by these bottom-up activities is limited in its scope, but it shows significant potential if these activities encounter understanding and support from city authorities.
Forming the Savamala civic district, as well as participatory urban upgrade, brownfield and urban heritage regeneration were also their ultimate goals.
\\

It is also important to mention that the combination of Savamala’s spatial capacity (its central urban position and its proximity to the bus and train terminals) and the primary activity of these bottom-up agents (inclined to boost knowledge and vision building as well as experience sharing potentials) has led to prompt and adequate reactions to the current refugee crisis that has hit Europe, and with it, Belgrade. The activities for helping refugees/migrants were coordinated by Mikser and financially supported by many national and international organisations - the United Nations High Commissioner for Refugees (UNHCR), CARE International (Cooperative for Assistance and Relief Everywhere), the Red Cross etc., as well as by supplies and care from the locals. These efficient actions also speak of the competence and alertness of bottom-up agents to respond to the dynamics of the modern urban context.
\\

Finally, in an authoritarian society like Serbia, gaining power means being close to the political parties or the state nomenclature 
(\href{Expert Workshop}{Workshop 1}).
%Expert workshop - Ksenija Petovar prez
The cultural scene, following the tradition of free and independent actions, while being densified with these activities in Savamala, is continually obstructed by the state
(\href{InterviewX}{Interview X}).
%KC grad interview
As these initiatives have neither socio-political power, nor public support outside the municipal authorities and the citizens of Belgrade, their future seems uncertain and their effects short lived (\href{Expert Workshop}{Workshop 1}).
%Expert workshop - Ksenija Petovar prez
\\

In light of recent changes happening after the night demolition in Savamala during election night of 2016, the physical structure of the neighbourhood has altered, citizens feel unsafe, 
\footnote{..as the police did not react when 30 masked man armed with poles were destroying the property in Savamala on the plots commissioned to BWP by the BW SPSP.}
and the function of rented spaces is changing. The value of these spaces is degrading with new, insignificantly different activities. These overall circumstances generally reflect the temporariness of the new situation and suspense over what will come next. In these circumstances, cultural activities are moving out. 
\footnote{KM8 was closed in November 2016, Mikser House makes plans to move and KC Grad most likely will do the same (\href{InterviewX}{Interview X})}
%KC Grad and Mikser interviews
\\

The end of Savamala cultural cluster does not mean the end for alternative culture in Belgrade, actually quite the opposite. The experience from the Savamala period and the local bonds created through it are certain signs that the cultural scene will pop-up somewhere else, and there are already other places in the city indicating this, such as Dorcol Platz (\href{jovanovic}{\citealt{jovanovic_dorcol_2016}}).
\\

Even though limited in scope, these cultural and civil activities, with special emphasis on the activities of NDVBGD, represent an important step towards a critical society.
Subtly and slowly but surely, they train the populace to recognize, show and advocate their needs to be directly applied in the city.
They also mould these needs according to what the city can actually offer and how they altogether have to act or interact with the world around them, which is in constant change (\href{Harvey}{\citealt{harvey_condition_2003}}). 

\subsection{Interest-based Waterfront Future}

In practice, the small-scale vision of the cultural cluster in Savamala was replaced with a waterfront megaproject assigned as a national priority of strategic importance (\href{Decision}{Government Decision 2014}; \href{Ordinance}{Ordinance 2015}). 
%find ref legal
The basis for the decision at the national level was not a strategy, a vision, a long-term plan, but an anonymous model of BWP presented in Dubai and in Cannes in 2014 (\href{Politika}{Politika 13/03/2014)}).
The idea for this project began in 2012 as a part of the political campaign for the national elections. It later transcended into city gossip, explained as a testing strategy for the Belgrade city authorities governed by the opposition party (\href{Vreme}{\citealt{georgijev_hod_2014}}). 
%find ref media
\\

Apart of a profit-oriented strategy typical for megaprojects, the flawed circumstances of the Serbian regulatory framework contributed well to the feasibility of such a project from the investor’s point of view. No efficient market-oriented taxation on land and urban land, the regulation on the conversion of the right to use the land to the private property ownership and the practice of buying deteriorating state assets due to its underestimated land value were the main sources that impoverished the country in the transitional years after 2000 
(\href{InterviewX}{Interview X})
%zekovic rad 2016 Spatium, Zekovic ostali radovi
\\

In the case of BWP, within the agreement between the RS and the investor, these legal instruments were improved  to provide maximal financial benefits for the investor.
Firstly, as soon as the buildings are to be constructed the investor has right to gain ownership of the most expensive urban land without fees (\href{JVA}{JVA 2015}).
Then there is a list of non-contributed buildings with significant cultural and architectural value offered for reconstruction to the investor with the right to lease without a fee (\href{JVA}{ibid.}).
\\

What is more, during less than three years, several legal and planning documents have been enacted that will forever change the disposition of power between the political and financial poles on one side and the citizenry and general public on the other; i.e.:

\begin{itemize}

\item A Study of high-rise buildings of Belgrade ceased to be valid in April 2014. It means that buildings higher than 10 floors are allowed to be built without any zoning restriction or recommendations;

\item The changes in 2015 of the Amendments of GUP 2021 excluded the obligation of international competition and changed the land-use rules. This action contributes to the centralization of power and and privilege of the Government (and political powers within it) to directly influence the decision on the projects;

\item The secretive "Belgrade Waterfront Master Plan"was an official base for the compilation of the Belgrade Waterfront Spatial Plan for Special Purpose Area (BWPSPSP) and the sole source of the on-site design solutions;

\item  The Amendments of the PCA without real constitutional background introduced the project of national importance as a source of "protected status" for projects;

\item Adoption of BWSPSP offers a special project status and gives the exclusive decision-making role to the Government over the central area of the capital city; 

\item The discontinuation of the Republic Agency for Spatial Planning (RASP) is a direct intervention of the high national authorities in the urban planning cycle and another act of power centralization.
Reassigning the RASP tasks to the Ministry of Construction, Transportation and Infrastructure (MCTI) puts this institution in an obvious position of conflict of interest and degrades  planning as an organised and regulated profession in Serbia (\href{Stojkov}{\citealt{stojkov_sahrana_2015}});
%vreme ziva profesija

\item Fast-lane enactment of Lex specialis, the special law regulation for the expropriation of land for BWP. Such a legal instrument for private residential and commercial premises is a dangerous, first-hand tool for manipulation and it threatens the property rights guaranteed by the Serbian constitution (\href{Figure Timeline BWP}{Figure X}).

\item Finally, the Joint Venture Agreement (JVA) is formulated in such a way that it guarantees the profit to the foreign investor whatever may happen with the implementation of the project.
It puts a financial burden on the RS budget, whose actual value is not yet exactly prescribed or estimated, and makes the case for the investor in front of any international tribunal.
\footnote{The time-frame for the completion of the BWP is 30 years. The first evaluation is set for after 20 years, whereas the positive limit of implementation is 50\% of all the works by both partners, taking into account the preparatory infrastructural works assigned to the RS. In case of a negative evaluation, empty plots will be sold according to the foreign investor’s preferences and the profit  will be split based on the investment percentage - 68\% for the investor and 32\% for the RS (\href{Pravni skener}{Pravni skener 2016}; \href{NDVBGD}{NDVBGD 2016}).}

\end{itemize}

These interventions within the regulatory framework opens a new field of not only architectural, but economic deals, while some are already taking place: privatization of the Belgrade airport, the issue of rural land in Voivodina etc. 
(\href{InterviewX}{Interview X})
%Association of architects interview}
\\

All these issues make the actual future of the Belgrade waterfront murky and uncertain.
Taking a closer look at the neighbourhood, only the investor of the BWP and his closed circle within the national political elite know what is going to happen and can make strategies for their own gains in Savamala
(\href{InterviewX}{Interview X}).
%Biking tours - private}. 
In this respect, certain people are opening businesses and renting spaces in Savamala, while local entrepreneurs feel frustrated and scared about what is going to happen (\href{InterviewX}{Interview X}).
%Biking tours - private}.
\\

Moreover, the locals in Savamala testify that local investors who collaborate within the BWP (without clear knowledge of the exact initiators) are offering the renovation of façades in Savamala (but only the front façades)
(\href{InterviewX}{Interview X}).
%KC GRAD interview}.
According to these on-site signs and expert opinion, in the long run Savamala will be destroyed as part of the "Belgrade Waterfront" programme
(\href{Expert Workshop}{Workshop 1}).
%Experts - Savamala - questionnaire}.
\\

Taking into account the privileged position of the investor according to the JVA, the suspense actually follows the question of what is actually going to be built except for the two residential towers whose construction  already started in 2016. Even more, the question arises of whatlife will be like in the central area of Belgrade if it is going to be a permanent construction site for the next 30 years.\\

Bearing in mind that rail-tracks serving the transportation of dangerous materials through Serbia cross the BWP area close to the residential towers under construction and that these tracks cannot be removed until new ones are built,
\footnote{These new rail-tracks incorporate the railway bridge near Vinca whose construction has not yet even started.}
these materials will be passing by luxurious housing twice a day for some time 
(\href{InterviewX}{Interview X}).
%BWP interview
In addition, the  preservation of the current rail-tracks in the area requires that the lowest terrain level remains at 75 m above sea level, which is at the limit of the level of century-old water (\href{InterviewX}{Interview X}).
%BWP interview
\\

In these circumstances, citizens and the civil sector have no efficient civil society tools and mechanisms at hand to claim their rights (\href{InterviewX}{Interview X}).
%Kucina interview 2
The people and the public interest are seen as the victim in this case - the fight between David (Savamala and the people) and Goliath (BWP) is a picturesque metaphor used by an informant 
(\href{InterviewX}{Interview X}).
%Biking tours - private}.
Moreover, people are scared to express their discontent and protest, because in the Serbian patriarchal and nepotistic context their personal and professional lives depend on the whims of political actors (\href{InterviewX}{Interview X}).
%Dobrica, Damjanovic Palgo interviews}.
\\

Consequently, the engagement of the NDVBGD initiative and their partner organisations (the Ministry of space being one) is seen as a brave upswing toward the overused and abused concepts of democracy and civil society. At every stage of the BWP, NDVBG continually reacted in opposition to them  (\href{NDVBGD}{NDVBGD 2016}):
%hronologija nacionalnog znacaja

\begin{enumerate}

\item official complaint to urban, national and city authorities regarding the project and consecutive regulatory framework changes;
\item organized actions for filing complaints against BWPSPSP and Amendments of GUP 2021 and protest performances against irregularities around public hearing events;
\item urban protest against signing the JVA contract, construction works and opening events, destruction of bicycle paths etc.;
\item letters and declarations, printed media issues, web publishing, press conferences, critical and expert documents;
\item a set of 6 massive urban protests against the irregularities of the night demolition in Savamala, requesting the identities of those responsible for the obvious criminal offense.

\end{enumerate}

Even though these actions did not really endanger the implementation of the BWP, they actually had more effect on the mindset of Belgradians through the slow incremental transformation of behaviour, at least among the young  toward a more participatory approach and the practical application of the civil right to disobey and intervene when the public interest is threatened, and, above all, to apply an educative approach to the urban regulatory framework and the rights to the city
(\href{PHD Workshop}{Workshop 2}).
%Phd students - Savamala - questionnaire}.

\subsection{Strategies and Tactics of Urban Institutions}

In all these circumstances, urban planning institutions have been either side-lined or so continually directly subordinated to political powers that their sense of their profession and the city in general has become biased.
\\

A public servant’s testimony speaks to this by explaining how the urban transformations happen: first of all, the terrain is cleared by any means at hand, legal or illegal, but even illegally cleared terrain is a step closer to transformation. The building, however, happens according to the laws and the procedures in force at that moment (\href{InterviewX}{Interview X}).
%Sekretarijat interview
A very important issue is that the procedures are abiding, so for large scale interventions it is more likely that procedures will be changed than that anything will be built against the law - were the further explanations  (\href{InterviewX}{Interview X}).
%Sekretarijat interview
This interpretation explains the urgency of providing legal instruments. With the cumbersome institutional infrastructure inherited from socialism, these newly established procedures are the only real means that secure the interest-based construction and land acquisition around the BWP.
\\

On the other hand, the attitude of urban planners in Serbia are extremely passive. They are not reacting to what is happening in the city, butwait to be invited by the authorities or investors before they take an active role in urban projects and urban development issues in general 
(\href{InterviewX}{Interview X}).
%UZ Zaklina interview}.
Therefore, planning solutions are usually technical and involve engineering tasks mainly concerned with infrastructural equipment (\href{Stojkov}{\citealt{stojkov_teorijska_2012}}).
%Stojkov and Dobricic 2012 05
\\

While it has lost the monopolistic position held for so long during socialism (\href{Chapter 4}{Chapter 4}), the planners’ profession is paying a price for its lack of knowledge and appreciation of the humanities and the missing social approach to planning  (\href{Peric}{\citealt{peric_evolution_2016}}).
Accordingly, they have become passive observers in their professional environment, testifying to the incompetence of political figures who take over decision-making and even planning, which last happened during the rule of Prince Milos (\href{Section 4.1.2}{Section 4.1.2}).
These circumstances explain the double roles they often play concerning the planning of  political projects  (and in this case the BWP) - in the professional domain they work in favour of it and in their private lives they publicly advocate against it (\href{InterviewX}{Interview X, Y}).
%dobrica, palgo interviews}.
\\

Due to these conditions, any particular education on horizontal coordination, the importance of transparency, risk management and evaluation methods for administrative stuff 
(\href{Expert Workshop}{Workshop 1}),
%Experts - Savamala - questionnaire
or further policy agendas on the integration of planning, design and cost-benefit and feasibility analyses, may be insufficient and ineffective ”busy-work” in practice. Urban transformations and changes at play in Belgrade, and in Serbia in general, are rather a result of luck or spontaneity, but certainly not of an operational action within a framework of rules and regulations, and not at all in reference to any strategy or long-term elaborated programme in the public interest (\href{Vanista}{\citealt{doytchinov_urban_2015}}).

\section{Resilience, Flexibility and Contradiction in Urban Processes}

According to the current circumstances, Savamala is the playing field of visible and invisible power structures, group interests and on-site activities, all of which have their own intentionalities towards its future. Following the arguments provided here, the prosperity of certain interest groups does not imply prosperity for the whole neighbourhood. Radical changes themselves do not mean qualitative improvement.
\\

The multi-level MAS-ANT analysis performed in this research offers a double-layered picture of the system evolution processes at play in Savamala.
In the previous section, the results of the MAS-ANT analysis were presented in the narrative of short-term evolutionary paths according to each layer of urban decision making.
In this respect, certain future prospects for Savamala are formulated from the illustrated agent profiles.
\\

Following that, a more general overview on the agency of long-term urban processes will be introduced. The elaboration of these processes consists of pointing out agents and networks of urban resilience, flexibility and contradictions in the neighbourhood. Such an outline addresses the question of urban development in a manner that  does not speak of constant growth, but of the positive and negative impulses of maintenance, transformation and change of the urban system.

\subsection{The Practice of Maintenance}

In terms of urban development, the role of system maintenance processes is usually disregarded as a self-evident proof of the reproduction of the social order and cultural values (\href{Section 2.1.1}{Section 2.1.1}). 
Moreover, when it is mentioned, it is its negative effects which are emphasized and presented as threats to developmental preferences.
\\

It is evident that, in the Serbian context, there is a multitude of problematic practices, procedures and values passed down from generation to generation within the regulatory framework and social relations. 
First of all, the continual reproduction of the authoritarian decision-making structure and vertical institutional communication have marked the Serbian regulatory framework since the early years of the Serbian State (\href{Section 4.1.1}{Section 4.1.1}). 
In such circumstances, regulatory networks are run by power, fear and authority rather than by legitimate and knowledgeable subjects. This is further jeopardized by cumbersome institutional organisation and a divergence of accountability inherited from socialism. These circumstances have gravely affected administrative networks, which function in a self-sufficient, impractical manner, without taking into account real users and the ephemerality of procedures. The users and the space are actually defined as a set of regulations, devoid of any real attachment to the people or the city (\href{InterviewX}{Interview X}).
%Association of architects Cagic Milosevic interview}.
\\

Another backward practice is the tradition of favouritism and nepotism. This is not a distinctive feature of Serbia and is present in every culture, though the practices, symbols and rules differ from context to context (\href{Carikci}{\citealt{carikci_favoritism_2009}}).
In Serbia, favouritism and nepotism are followed by corruption and stem from the Ottoman cultural discourse (\href{Çarikçi}{ibid.}), which was adopted in Serbia during the five centuries of their rule. Even though this method of assigning roles was less present during socialism, it was far from eradicated. However, although the socialist system that was established in Yugoslavia 
\footnote{During SFRY period, even though the republics had significant autonomy and denoted particular Slav nations, the idea of one Yugoslav nationality for all was strongly supported and promoted.}
put forth egalitarian values, its influence, even though effective for the time being, was  easily dismantled and abandoned with the rise of capitalist values.
Namely, \href{Çarikçi}{\cite{carikci_favoritism_2009}} %PROVERI REFERENCU;DARKO DODAO
state that the applications of nepotism result in the regulatory framework altogether losing any real power over the economy and politics. Apart from the obvious negative influence on the social circumstances of everyday life for ordinary people and the overall public interest, the issues of civil rights and control networks are especially altered while being ruled by the principle "justice does not serve everybody equally".
\\

The socialist period has left another, though nominally positive practice. The regulatory discourse of policy agendas, rules and documentation ruled over urban space during socialism. Even then, the implementation and binding role of these regulations were problematic, but the situation has escalated ever since. 
Namely, regulatory framework has already  been established and set to work during socialism  (compared to some other developing countries); regulations are multiple and multiplying; and they are necessary, but they are not binding actions in space. More precisely, they might be binding, but not necessarily, which refers back to corruption, favouritism and nepotism as the rule of choice in the institutional framework.
\\

Another highlight might be the tradition of international competitions, already introduced during the Serbian state and the first Yugoslavia and welcomed and cherished in SFRY (\href{Section 4.1.1}{Section 4.1.1}).
The harsh pressure for easy money and profit have been threatening  to disband this good practice. Bearing in mind what happened with the BWP and BWPSPSP, this issue becomes even more discouraging.
\\

Regarding Savamala within its physical boundaries, certain issues have been maintained in more-or-less positive manner throughout its long history. The distribution of its urban functions have been maintained throughout the different periods - cultural, leisure, residential, commercial, port and public services might have varied in their emphasis during the different periods, but they have all been there 
(\href{Student Workshop}{Workshop 3}).
%students questionnaire Savamala
Above all, all its activities and places were always accessible to all social groups and classes, which might not be the case after the construction of the BWP.
\\

Moreover, its spirit of urban culture from the early periods of the Serbian state spread its sparks to the civic district which found its place in Savamala for some time (2008-2012), before they were brushed away by the BWP. It may sound strange, but having the cultural scene on the run is rather its permanent status
(\href{InterviewX}{Interview X})
%KC grad interview
\\

Taking into account all these different examples, this research justifies the hypothesis raised about the Serbian version of history, where usually the coming periods run down the inventions from the previous ones, but more often than not keep its flawed mechanisms and biased practices (\href{Peric}{\citealt{peric_evolution_2016}}).
Even though this might be a cultural trait, there is a certain level of perseverance (with internal consistency, avoidance of fragmentation and excessive complexity, a holistic view and the small-steps approach) which could only be achieved over time. This might be a problem in Serbia as the periods of continual evolution in a Serbian case are reduced to up to 50 years.
  
\subsection{System Transformation}

Urban planning in its essence is the professional practice for effective and rational urban transformations through measures, instruments and services of strategizing, regulating, programming, financing etc.
(\href{Stojkov}{\citealt{stojkov_teorijska_2012}}).
%Stojkov and Dobricic 2012 05
Therefore, planned urban transformations are ever present phenomenon, though their effects vary.
The causes for urban transformations to fail are found in the imperfections  of  urban  legislation  and regulations, the personal biases of those involved in decision-making processes and low quality urban designs and implementations (\href{Grozdanic}{\citealt{grozdanic_belgrade_2008}}).
On the other hand, coherency, sequency, deliberateness, careful implementation, a gradual general course of actions, targeted propositions, concrete statements about the content, time and space are at the core of good planning practices of space transformation (\href{ref}{Boisiere 1981}).
In this respect, spatial and urban strategies are the core documents that should ensure their quality, supported through planning and implementation networks.
\\

As for the strategic level, the National Strategy of Sustainable Development 2009- 2021 and the Belgrade Urban Development Strategy 2011-2016, which are currently in force, back up the low capacity and provisory role of such a document. Both strategies raise the issue of the Sava Amphitheater and the Belgrade Waterfront, but more in terms of integrative development plan, which was abruptly and suddenly turned into a massive Belgrade Waterfront megaproject (\href{NSDS}{NSDS 2008}; \href{BUDS}{BUDS 2008}).
\\

Even more curious is the case of Belgrade Urban Development Strategy 2016-2021.
According to the informants currently working on its draft version, it deals with infrastructural and transportation problems and solutions for the central zone of Belgrade, but it hardly even mentions the BWP, and treats it as a insignificant side project (\href{InterviewX}{Interview X}).
%Damjanovic Palgo}
Having such a large-scale, publicly promoted project side-lined within the new strategy is evidence of either the future of the project or that of the strategy. In any case, the regulatory framework in Belgrade is at loss.
\\

Furthermore, on the level of urban documentation, the urban regulatory framework is still maintaining certain systemic solutions from socialism. During socialism, the rules ruled, and the constraints were the essence of urban transformations (\href{InterviewX}{Interview X}).
%Association of architects Cagic Milosevic interview}.
Moreover, urban transformations were questions for professionals, and the opinions and needs of locals, the  actual physical context with its past and present social frameworks (genius loci) were hardly taken into account at all (\href{InterviewX}{Interview X}).
%Association of architects Cagic Milosevic interview
This type of professional and institutional practice that disregards what is possible and desirable (contextual resources) posed a threat to the city when it was confronted with financial capital and private interests that came in with transition.
\\

Under these circumstances, urban and architectural heritage is the first victim. While the active protection of the cultural heritage might fit it in with the life of the city, an  attitude of neglect contributes to the production of abandoned and deteriorating places  
(\href{Vanista}{\cite{doytchinov_urban_2015}}).
This was the practice of socialist city authorities towards Savamala that led to the deterioration of its pre-socialist architectural heritage (\href{Section 4.1.3}{Section 4.1.3}).
Further abandonment during the post-socialist period put that heritage at the disposal of public usage when transitional and capitalist values appeared on the stage
(\href{InterviewX}{Interview X}).
%Biking tours - private
\\

The cultural transformations in Savamala were the product of space, time and activities. These gradual civil and cultural improvements were the only intrinsic engines of transformation at the local level, without significant support from institutions and no help from the state.
\\

Comparing these civil initiatives to planners’ activities demystifies the status of urban transformations in the Serbian context that are hardly at all happening at an institutional level. Institutionalization encircles sustainable investment programmes and management structures, aspirations towards community involvement and global competitiveness, involvement of experts and the sense of identity and public interest 
(\href{Volic}{\citealt{volic_belgrade_2012}}).
Obviously, what was happening in Savamala was an informal, but effective action that is now over.
    
\subsection{Urban Change}

Speaking in historical terms, Belgrade is in a constant struggle between traditionalism and modernism, the conservative and the progressive (\href{Roter}{\cite{doytchinov_modernization_2015}}).
The crisis might be also treated as an opportunity for building an evolutionary resilience within the urban system.
However, the permanent state of conflict at play in the Serbian capital speaks otherwise.
This situation stems from the lack of consensus on the most important issues: poor political legitimacy of transitional reforms, no clear political will for institutional regularizations,  and no societal consensus on the priorities and actions toward the public interest 
(\href{Vujoseivc}{\citealt{vujosevic_postsocijalisticka_2010}}).
%Vujosevic et al. postsoc tranzicija i teritorijalni kapital
\\

In such circumstance, the everlasting vision of abrupt, radical change is seen and usually promoted as a solution.   
An informant put it nicely - instead of focusing attention on what is going on, the authoritarian regimes, who hold power over urban space, usually turn to building a new city
(\href{InterviewX}{Interview X})
%Biking tours - private
And the historical narrative tells the same (\href{Section 4.1.1}{Section 4.1.1}).
The notion of erasure and the condition of tabula rasa urbanism is associated by Serbian urban planners as the inheritance of Emilijan Josimovic and his plan for Belgrade (\href{Maksimovic}{\citealt{maksimovic_idejni_1978}}; \href{Perovic}{\citealt{perovic_iskustva_2008}}; \href{Blagojevic}{\citealt{blagojevic_urban_2009}}).
It might also be the result of the technical approach to the city still at play in Serbia, in which case every change is seen as a good change and the more radical it is the better, as an expert explained
(\href{InterviewX}{Interview X}).
%Kucina interview 2
\\

With architectural and urban planning professions in transitional crisis as they have neither been regulated nor adapted to new market-oriented rules and EU regulations
(\href{InterviewX}{Interview X}),
%Association of architects Cagic Milosevic interview
as professionals, they become easily eliminated from the decision-making processes.
In Serbia, urban change, if there is any, can be and must be brought up by economic or political actors, or most often by both groups working in mutual interests.
\\

Economic actors are led by profit.
Urban land, as a scarce resource, is an unmistakable source of profit, especially in countries with still hybrid land-market circumstances as Serbia is.
The entry of global capital into Serbia, the general trends of residential construction and the availability of urban land in central areas of the capital city is a good combination for  
"building large, flashy and often gated communities, targeting expatriates, employees of foreign firms and embassies, and Belgrade’s top business echelon" (\href{Hirt}{\citealt{hirt_belgrade_2009}}).
In this sense, the BWP has already been announced with great precision by the experts, but the overall disregard for research and scientific discourse in Serbia set it out of sight.
\\

In fact, political actors are power mongers. Moreover, there is also a statement that Serbs (though not exclusively them) have a strongly spatialized identity through the discourses of a small, but extraordinary country/nation 
(\href{Savic}{\citealt{savic_where_2014}}).
Therefore, radical and large-scale interventions in space are the manifestation of power, a mark set in the historical space-time, a long-lasting "spatialization" of the people (elite) in power.
\\

Conversely, those who oppose the project (led by NDVBGD) have no power or capacity to intervene. 
They do not even pretend to offer a systematic solution or to open a dialogue.
They only aim to address the individual sense of righteousness and to educate and empower people for future actions as this is only the beginning of the new morality, set by new legal, institutional and media means.
\\

The strategy of these powerful structure is such that their interventions for urban change either disrupt or distort urban assemblage networks; such actions twist the system setting as practices something that might have been hitherto illegal/illegitimate.
\\

\textbf{"In the current context  of  the  Serbian  polity,  the ‘binding’  of  all stakeholders  and  moving  towards  a  common interest  might  seem   a  difficult  endeavor."} \href{Volic}{\cite{volic_belgrade_2012}}

%international architectural competitions during SHS, now no competition for Savamala

%diagram koji su akteri u kojim procesima; koje su mreze u kojim procesima; koji su aspekti u kojim procesima

\section{The MAS-ANT Diagram}

The MAS-ANT illustration of agent profiles formulates an exhaustive classification of elements, relations and processes that concern urban development.
Its goal is also to enable the necessary relation of all social phenomena to the physical space through the categories of contextual resources, urban conflicts and urban practices.
Their spatial projections (maps) are also a strategy to surpass the time perspective.
\\

Following \href{Braudel}{\cite{braudel_history_1970}}, events and actions are only instantaneous interventions in space, but urban processes are long-term processes - a reality of foundations and obstacles that survives through time and is only slowly eroded.
In this respect, the MAS-ANT diagram aims to apply mathematical  techniques to represent the  relation of social phenomena  to  geographical space and to introduce a  long term historical perspective.
Such visualized methodology, characteristic for architectural interpretations of space and society, responds to \href{Braudel}{\cite{braudel_history_1970}} challenge posed upon social studies, as he criticized their approach for either event-based research or omitting the time dimension altogether.
\\

The MAS-ANT methodological approach, modest in its initial conclusions but reach in simple illustrations and clarifications, builds a framework for actions outside biased expert and institutional dimensions (\href{Diagram}{Figure X}).
It describes urban dynamics, positions the level of urbanity of the chosen environment and indicates the field of action, but without clarifying the single steps in that direction.
The historical perspective or maintenance, transformation and change processes expose the weakest links and the contingent points of intervention and conceivable actions. 
\\

In the case of Savamala, the Serbian institutional discourse lacks mechanisms serving to translate destructive conflicts into constructive and productive elements (\href{Vujoseivc}{\citealt{vujosevic_regionalizam_2015}}).
%Vujosevicet al. 2015 Regionalizam 2
An elaborated indication of the weakest links as well as unleashed socially harmful relations may be a driver for interventions from either top-down (public expert institutions) or ground up (civil organizations, NGOs, independent professionals, artists and cultural workers).
Even though these conclusions may sound familiar and obvious, the lack of methodological and evidence-based explanations may have hitherto led to dissolution and manipulation of the information.
\\

The current state of the urban environment in Savamala indicates that what is going to happen is not yet beyond the point of no return. The chance has surely been missed for making Savamala a mixed neighbourhood where the bottom-up and top-down meet to build a vibrant cluster for leisure and a tourist destination with the combination of heritage, old crafts, a balance of public and private services, and a mixture of day-night usage. On the other hand, this prime city location on the riverside is cleared and infrastructurally prepared for the first time in 200 years. However, according to the elaborated vision on urban processes in Savamala, both utopian and dystopian visions for Savamala are still in the air. Therefore, it is essential to recognize the discrepancy between what is agreed behind closed doors, what is planned and incorporated in policy agendas and urban documentation and what is being built.
\\

First of all, urban planning professionals and researchers in Belgrade have to work out all the different urban scenarios that will result from the BWP and continuously adapt the measures to assess impact and reduce risks of such a large-scale project in the city centre. Such a project, if not adequately managed, can ruin the overall international image and urban potential of the city, which once was a cosmopolitan metropolis of the Balkans. It is also important to work on an interactive model for planning implementation (\href{Alexander}{\citealt{alexander_planning_1989}}) and educate professionals to surpass the static and linear, administrative and technical approach toward the city (\href{Stojkov}{\citealt{stojkov_teorijska_2012}}).
However, the biggest problem still remains the regulatory changes that re-mold urban reality, the profession and the limits of right and wrong in Belgrade. Instead, the civic and cultural initiatives should continue the strategy of the watchful eye, waiting on every wrong move and examining the pulse of the city in order to profit and maintain its cultural, educational and artistic programmes and likewise make them more participatory and better adjusted to citizen needs.
\\

In order for these actions to better correspond to current post-socialist urban reality, diversity and reciprocity in the nature of current social circumstances must be acknowledged and taken into account. Economic (transformation of production and consumption in relation to space, income polarization and poverty), political (urban governance, participation and decentralization), spatial (demographic trends and distribution of functions) and cultural (social exclusion, civil activism and informality) issues all have their signifiers in the behaviours of urban key agents and should be traced accordingly. In sum, the MAS-ANT diagram gives an overview of what constitutes the level of urbanity of an urban neighbourhood - the core of the maintenance, transformation and change processes spread among key urban agents and within urban assemblage networks throughout space and time.
%diagram historical networks - depth; mentioned networks here (economic, political, spatial, social)

\section{Conclusion}

This is the final chapter of analysis that blends the results achieved through separate examinations of ANT (\href{Chapter 5}{Chapter 5}) and MAS (\href{Chapter 6}{Chapter 6}) within the category of agent profiles.
These methodological achievements are later applied for theoretical discussion on: context-related developmental prospects from the decision-making perspective and on more general issues of the level of urbanity interpreted through the exhaustive set of urban transitions.
Based on these data, the MAS-ANT diagram provides a robust illustration of urban complexity and dynamics.
\\

From the historical perspective, such interpretation offers a playing field for estimations of what contributes to an improvement of the life and functionality of urban systems through the example of this case study.
\\

From the point of view of urban theory, the MAS-ANT methodological hybrid brings about a  vision of a multidimensional environment that incorporates space and time, but exceeds the Euclidian interpretations of the same. This environment is in constant movement relative to everything around it; and as such, acknowledges the importance of system maintenance for the urban system evolution.
\\

In terms of urban practice, MAS-ANT methodological approach suggests the flexibility of reference frames and a posteriori reasoning that enable differentiation of what should be done from what is possible to be done.
\\

In terms of urban research, the this analysis has aimed to address the issue of up- grading the level of post-socialist urbanity through the deconstruction of the complex planning logistics and the exposure of links between and among top-down, real estate and bottom-up forces, interests and actions at the neighbourhood level.
\\
 
In terms of Savamala, it is difficult to not refer to Serbia and the urbanity of Serbian cities in general, when considering Savamala. Even more so, as it has recently become the symbol of citizen revolt against the system and against the intrusion into the citizen’s comfort zone, symbolized by the recent illegal demolitions. With that event, the short span of the tradition of system maintenance rather brought to the fore the issue of history repeating itself. Belgrade is a city with a long history of citizen revolts and revolutions (the liberation of Serbian cities from the Ottomans, the beginning of WWII, revolts against Milosevic’s regime, against NATO bombing). Such a  state of affairs has made Belgrade teeter between the reproduction of the repressive order and the radical change of its system, while transformations have also been easily wasted in these heroic and turbulent times. The current situation in Savamala includes all these elements. When referring to the future, the destiny of the Belgrade Waterfront Project will demarcate a new transitional urbanity in Serbian cities.
\\

Finally, this visualization technique is a deployment model for the representation of urban system evolution that takes into account urban complexity and dynamics, leaves value judgments to the interpreters, and has the potential to be digitalized and loaded with the layers of information in different dimensions.

%%%%%%%%%%%%%%%%%%%%%%%%%%%%%%%%%%%%%%%%%%%%%%%%%%

\chapter{Conclusions}

%%%%%%%%%%%%%%%%%%%%%%%%%%%%%%%%%%%%%%%%%%%%%%%%%%

The scope of this thesis was to grasp the actual urban development  in cities.
In this reference, the researcher's initial position was not to solve a problem, but to establish an exploratory framework for disassembling urban complexity and tracing urban dynamics as constitutive elements of urban development processes.
The starting point was the methodological question of how to investigate these issues in an inclusive and flexible manner.
\\

Taking an interdisciplinary course through an architectural approach in urbanism, the determined field of research in terms of post-socialist cities and the neighbourhood unit of analysis helped bound together urban agency, urban decision-making and urbanity for an operational and procedural study. The general research assumption was that the hybrid method of Actor-Network Theory (ANT) and the Multi-agent System (MAS) is the tool for mapping the body of urban system transition. While the scientific background of this thesis was in urban theory and research methodologies, at the research level it argues for a double-layered approach to data, both qualitative and visual.  
\\

This chapter is divided into four parts. The first part summarizes the research findings from a three-part analysis of Savamala (the actors, socio-spatial patterns and processes of urban system transitions). The second part represents the exact scientific results of the study in relation to the research framework. The third part of the chapter returns to theory. Even though methodological explorations and practical application are the core of the thesis, the concepts that supported the research in the beginning were relevant againg as a binding force for the generalizations and attributions. Bearing in mind that urban theory was mainly off-focus for this research project, the key conclusions in this regard are pointed out in the limited scope for the central concepts and partly for those that were controversial. The critical review consists of an evaluation of the conducted research project. Finally, the future- oriented reasoning addresses the practical sides of the work and the "to-be-developed" (TBD) potentials for future research.
\\

In sum, the chapter and the overall work is concluded in three directions - starting with the summary of research findings, the resulting framework is provided 
in terms of what has been done and to what extend (Conclusions and Limitations), and what can be done and how it could be done in the future (Practical Implications and Future prospects).

\textbf{"To live is to leave traces" Walter Benjamin}

\section{Actors and Processes in Savamala}

The cross-pollinated method of analysis containing the Multi-agent System and Actor-network Theory set an actor-network-process scheme for interpreting the urban complexity and dynamics of the Savamala neighbourhood. The application of Actor-network theory onto the qualitative data from Savamala provided an overview of urban complexity in post-socialist circumstances. On the other hand, Multi-agent system put in motion the localized contextual elements and processes for simulating urban dynamics.
\\

While the articulation of urban agency through actor figuration and networks revealed the complex relationality of urban elements (\href{ANT diagram}{Figure X}), in terms of real factors of influences, the scheme has become much simpler. While the production of urban space in general might be treated as a conglomerate of political, economic, professional and civil sectors and elements, in Savamala, the picture is exceedingly disbalanced. Nominally, the regulatory framework dominates space interventions in Serbia. The advanced system of legal and planning institutions and documents, as well as the institutional basis for participation in urban decision-making, are inherited from the socialist period. Yet, the recent transition towards capitalism has fostered the bonds between political and economic actors and instated parallel structures for navigating through the system. Political voluntarism and favouritism rule the relations and procedures of the legal, institutional and planning frameworks in Serbia. Furthermore, corruptive mechanisms open the floor for external interests and influences to enter the system. In reality, power-mongers and profit-seekers actually dominate the Serbian regulatory framework. And what has been happening in Savamala is not an exception, but the rule of functionality of post-socialist systems.
\\

Middle-class society, a stronghold of socialism, is disappearing in Serbian society today. Meso-layers of professional and civil sectors and citizens, even though present on the ground, are invisible in urban decision-making. Serbia and Belgrade are an inseparable part of the European cultural discourse, especially boosted during socialism when it was a cultural space that overpassed the iron curtain separating the East and the West. The recent rise of culture on offer in Belgrade and Savamala has showed that this image is to come back. However, the culture in Serbia is moving out of the institutions. Without any official assistance provided at the national level, this sector has been handed over to international support systems and funding instruments to set the framework and agenda for their activities. Being slowly but surely dislodged from Savamala, this sector is time and again on the run at the city level.
\\

On the other hand, professional discourse has become narrowed down to the clique of technicians and apparatchiks for constructing the legitimacy of decisions within the bu- reaucratic system. The planning framework in Serbia replicates either practices or conflicts, but there are very few mechanisms and instruments for finding solutions and adaptations. Any expert activity, research and critiques have been destructed and obstructed from public institutions and political party officers (partocracy). Serbian society is fundamentally authoritarian, and decision-making is extensively marked by hierarchical conformation of institutional roles and individual political figures within the institutions.
\\

In this setting, locals are the underprivileged group. Even though agreat part of Savamala citizens identify themselves with the neighbourhood, their needs have not been a priority of the civil initiatives present there and even more so they are victims of the Belgrade Waterfront Project and its investor’s whims. The recent situation in Savamala is an example of the transitional blend of the global and local that is happening now above all in the civil and real estate domains in Serbia. The art and cultural activities in Savamala have been an exercise in democracy, while the BWP deal is a school of free market mechanisms at play in class societies. Apart from judging its positive and negative influences, the case of Savamala brings this new mindset inherent in the capitalist order to the local stage.
\\

The future of Savamala is now tied to the destiny of the Belgrade Waterfront Project. The urbanity of Savamala is now labelled by the social and spatial segregation of the upper Savamala and the waterfront areas. The historically mixed neighbourhood with visible heritage from previous urban times is undoubtedly on the path of unification towards the envisioned future for the Waterfront. The tradition of functional diversity and accessibility is unavoidably threatened by the BWP requirements that are spreading extensively to the Savamala hinterland. The circumstances in Savamala indicate the point of discontinuation with its present of civil engagement and current regulatory framework in Belgrade, as well as its urban and architectural inheritance from the past.
\\

In terms of urban development in Savamala, the MAS-ANT methodological hybrid presents the interpretation of urban transitions as complex and dynamic processes of imperceptible regularities in the dynamic fluctuations of urban complexity over time (\href{Lee}{\citealt{lee_fernand_2012}}).
Generally speaking, in Serbia, there is little historical continuation in the development of Serbian cities; while there is a certain repetition in urban processes over time, they are usually short-lived. In Serbia, urban evolution either suffers from backward system replication or is faced with abrupt changes and "tabula rasa" solutions. In any case, there is hardly any tradition of transformation, reform or systematic and gradual adaptation. Whichever scenario is to be followed in Savamala, the most important question of any time is the legitimacy of the decisions made and a clear distinction between the right and public interest and the wrong, something that endangers it; which may well be the case in Savamala and its neighbouring future megaproject.

\section{Conclusions to the research framework}

The research conclusions first of all address the goals, inquires and assumptions set in the research framework.
An initial deductive line of reasoning in constructing research objectives, questions, and hypotheses of research are herein taken into account in the opposite, inductive order.
Having the results presented in this manner, also enabled the reflection upon the capacities and limits of the study.
Moreover, the special position of the mixed method used for the analysis and the display of data and the originality of this methodological pairing brought to the fore several important issues that were not anticipated in advance.

\subsection{[..] to the research hypotheses}

An overall assumption of this research was that the mapping of urban complexity and dynamics is the necessary and sufficient condition for tracing urban development processes [RH].
\footnote{Based on the central research hypotheses [RH] defined in \href{Section 3.1.2}{Section 3.1.2}.}
This viewpoint was further supported by the interpretation of urban development in terms of neutral urban system transitions (\href{Section 2.1.1}{Section 2.1.1}) and the transparency and adaptability of the MAS-ANT methodological hybrid (\href{Section 2.2}{Section 2.2}).
The turning point of such an approach was seeing urban development as not a goal, but as a set of processes. In addition, the validity of such a conceptual connection was found in the openness and extensiveness of this method and the overarching nature of the process to surpass and to incorporate a goal or a prospect, if necessary [\href{Chapter 3}{Chapter 3}].
This was documented in the overview of different scenarios for Savamala [\href{Section 7.3}{Section 7.3}].
\\

In view of this, urban system transitions were extracted from the vast image of urban complexity and dynamics by the mixed method [RH3].
\footnote{According to the assumption from the third research hypothesis [RH3].}
The source of agency of the maintenance, transformation and change processes was identified within the morphology of urban decision making [\href{Section 2.1.4}{Section 2.1.4}]. On the other hand, the level of urbanity embodied the track of the evolution of these processes in the context.
Namely, urban system transitions extended the current image of Savamala with traces of historical processes of long duration, but limited the view of the current state of the system only to those elements that have active agency and to socio-spatial patterns that are attributed the role    [\href{Section 7.4}{Section 7.4}].
\\

In this manner, the resulting urban development processes in Savamala were deduced from the complex map of urban agency disseminated within the morphology of urban decision making [RH1]
\footnote{Answering in this way to the first research hypothesis [RH1].}
(\href{Chapter 5}{Chapter 5})
and the dynamic system of the fluctuating level of urbanity that bounds relations of urban agency and contextual socio-spatial patterns at the neighbourhood level
(\href{Chapter 6}{Chapter 6}) [RH2].
\footnote{As it was nominated in the second research hypothesis [RH2].}
\\

In terms of urban complexity, the morphology of urban decision-making served to identify human and nonhuman, material and non-material actors. However, the distribution of complex actors and networks was illustrated in the 5-dimensional map of urban agency (\href{Figure ANT diagram}{Figure X}).
This abstraction, which was based on ANT principles, is an extensive categorization that exceeds the concrete situation in a post-socialist neighbourhood and bounds urban complexity based on roles [the nature of actors (individuals, sets, hierarchies of human, material, relational and spatial), the nature of networks (structural, supportive and functional), the nature of influences (international, national, city, local)] and relations [the nature of assemblages and exposure] (\href{Section 5.2}{Section 5.2}).
\\

Conversely, the level of urbanity contextualized urban dynamics by linking actors to local socio-spatial elements within the Savamala boundaries  (\href{Chapter 6}{Chapter 6}). 
While urbanity is a static concept and refers to a state in a particular time, the level of urbanity has  the capacity to interpret a fluctuating state that depend on the balance among urban conflicts, contextual resources and urban practices and the agency that attributes them (\href{Section 2.1.5}{Section 2.1.5}).
The contextualized MAS map of the level of urbanity therefore spatialized the points of harmonization, contestations and collisions at the neighbourhood level (\href{Figure MAS diagram}{Figure X}).
\\

The logic of the distribution (ANT), the congregation (MAS) and the reduction (MAS-ANT) were based on the researcher's analysis of the qualitative data. While the course of interpretations might vary depending on the researcher's standpoint, the logic of the process and the categorizations of elements were persistent, as was suggested within the hypotheses [RH-1-2-3].

\subsection{[..] to the research questions}

The question of an inclusive and flexible approach to urban development [RQ]
\footnote{Proposed as an overall research question [RQ] in this thesis.}
was dealt with on 2 levels:

\begin{itemize}
\item scientifically - in order to achieve comprehensiveness;
\item socially - in order to make it comprehensible and accessible for broader audience.
\end{itemize}

An inclusive approach to urban complexity was constructed from all human and non-human elements who were attributed agency in urban networks (\href{Section 5.1.1}{Section 5.1.1}).
Extending the figuration of non-humans from its symbolic role in urban networks to an active urban agency shed new light on how decision-making happens. While human agency produces non-human artifacts, once these non-human agents are established and instated in urban networks, they become generators of activities and transitions. So, in urban agency networks, non-humans then interact and interconnect with human and other non-human agency, producing the effects for both humans and non-humans. Non-human agency then is disseminated and regenerated through the morphology of urban decision making. Therefore, while decision-making processes and networks are unavoidable in the urban realm, they are not necessarily the sources of explanations per se, but the generators of urban agency (\href{Section 5.2.1}{Section 5.2.1}). The decision is the precedent of any action, but it does not necessarily imply a predetermined political, power and interest association in it. Thus, this three-party fusion (non-human, agency, decision-making) encapsulates urban complexity in an ordinary city. while the roles might be passed around differently, the non-humans, urban agency networks and decision-making distributions are its core concepts [RQ1].
\footnote{As an answer to the first research question [RQ1], the core of urban decision-making, urban agency and non-humans are recognized as the constituents of an inclusive approach to urban development.}
\\

On the other hand, flexibility was achieved in addressing dynamics, not according to the qualitative character of the processes, but in reference to the system as a whole. Whatever happens at the qualitative level must result in the maintenance, transformation or radical change of the whole and is reflected in the level of urbanity of the neighbourhood (\href{Section 6.2}{Section 6.2}). In a dynamic system there exists no status quo [RQ2].
\footnote{Association between urban agency, contextual elements and system evolution within the level of urbanity provides the detailed relations between the urbanity concept and urban dynamics, as asked in the second question [RQ2].}
\\

Another point of view on the MAS-ANT approach was resorting to terms that explain system dynamics in general (used for systems in engineering and the natural sciences) and comprehensive categories in order to operationalize the methodology for practical application in different contexts, for different users and readers of the diagrams  (\href{Section 7.5}{Section 7.5}).
The information loaded visual material might be more appealing and understandable for uses in the field and might improve participatory practices [RQ3].
\footnote{Binding together urban complexity and dynamics within the MAS-ANT diagram fields the third research question [RQ3].}

\subsection{[..] to the research objectives}

Envisioning an inclusive and flexible approach to  urban development processes, as outlined beforehand, constituted a challenge with regard to redefining the theoretical framework to meet practical needs [RO].
\footnote{The MAS-ANT methodological hybrid strives to encompass the systemic complexity and dynamics inscribed in urban development processes, as set as an overall objective of this research [RO].}
Re-evaluating theoretical concepts and setting operational definitions for the terms used in this thesis [urban development, urban agency, urban decision-making and urbanity] addressed the growing urge set upon the humanistic sciences, especially in a fragile context, to at least reflect upon bringing theory to the ground (\href{Section 2.4}{Section 2.4}).
\\

If applied in practice, the localized attitude to  development, sources of agency, paths of decision-making, conflicts, practices and resources might be applicable for non-professional. 
Bringing means of analysis outside the professional and decision-making clusters make them better articulate their needs and visions and elaborate their position.
While other sources of power might be difficult to seize, knowledge could be the way to intervene and insist on a new vision of cities that is best suited to the local context, in this research post-socialist cities and neighbourhoods.
Re-formulating urban development to encompass the dynamics on the ground, yet staying tuned to the local and ordinary, is a solid background for envisioning the empirical reality at the neighbourhood level [RO1].
\footnote{In reference to the first research objective [RO1].}
Moreover, contextualized concept of urbanity through urban conflicts, contextual resources and urban practices directly links spatial capital with social and urban agency and localizes the actions through the entities/materials/places of common interest at the neighbourhood level [RO2].
\footnote{These links of qualitative data and urban theory put forward the relationality between urbanity and urban dynamics.}
\\

The proposition of neighbourhood as a source of  useful empirical data and a playground of action was the way to defy the global. It was an important level of strategic organization in cities throughout history and recently it is coming back on stage (\href{Section 2.3.3}{Section 2.3.3}).
While cities are spreading and diluting, even more so with new technological means of transport and communication, the neighbourhood might foster the social bonds as a level of data collection and practical interventions in an ordinary city [RO-RO1-RO2-RO3].
\\

Finally, the diagrammatic vision of the social and the spatial  in cities was a means to display data without directly assigning meanings.
Besides, the overlapping of methods and visualizations through diagrams was a system of data triangulation and presentation of shifting points - from urban agency (complex actor roles and synthesized networks), through urbanity (contextualization of interests and interventions) to urban system transitions (contextualized relations between urban networks and time-frames between past, present and future). The systemic correlation between theoretical, methodological and empirical frameworks was presented as a tool for addressing urban complexity and dynamics [RO3].
\footnote{Set as the third research objective [RO3].}

\subsection{[..] to the methodological approach}

This thesis answered the question how ANT and MAS methods complement each other and how they address urban reality at the neighbourhood level.
While ANT describes complex state of an urban environment, MAS describes the dynamics of a system evolution.
\\

The hybrid method was constructed on the tendency to operationalize the theoretical framework and to visually analyze and display the data.
This approach is very common for architectural research.
Diagrams are actually very useful for communicating research-based data and scientific reasoning to nonspecialists.
The task of the researcher was to follow actors and processes in all their heterogeneity - track the relation types, get rid of inscribed symbolic meanings by the assignment of actual agency (\href{Nimmo}{\citealt{nimmo_actor-network_2011}}).
However, like many other methods in social sciences, 
the researcher brought in their point of view on facts and data.
Therefore, even non intentionally, the researcher introduced their subtle critical engagements and providing legitimization for certain attitudes, points of view and actions, which, if disseminated, may influence the system, processes and actors (\href{Baiocchi}{\citealt{baiocchi_actor-network_2013}}).
\\

This illustration of Savamala urban context is but only one version of visualizing urban agency, urbanity and urban system transitions.
However, the contribution of the hybrid method lies in the rise of explanatory framework from rigid and static to relative and dynamic and susceptible to adaptations through continual iterations.
At the professional level, MAS-ANT addresses diagnosis and action tools for practice-based research.
Moreover, the users of this map (professionals) could be able to indicate gaps (actors, networks, contextual elements, processes) for possible operational interventions in their respective domains.
At a local level, visual methods with scientific stronghold/background could be very useful for activist to distribute knowledge and substantiated facts among powerful media campaigns, top-down dissemination of information and an overall trend of "alternative facts".

\section{Conclusions related to the theoretical framework}

In the course of this thesis, theoretical background was used for building internal validity of the methodological hybrid and data analysis.
Even though the analytical strategy was not extended for use across cases, the results from applying theory over qualitative data and the other way round contributed to sharpen the definitions of theoretical constructs and to re-frame their theoretical scopes for the operationalization in practice.
\\

While theory is a neutral, explanatory tool (\href{Sears}{\citealt{sears_good_2005}}), it also supply meanings and interpretations that enter everyday life and influence processes and practices.
In this respect, I have returned to the initial conceptual, epistemological and local frameworks in order to re-place the results of this research within its initial theoretical boundaries.

\subsection{Urban Development Taxonomy}

Urban development is a predominant concept in urban theory and practice and even more, in everyday life.
The issue of urban development was extensively addressed and used in this thesis research.
First of all, the concept was traced back in scientific literature and the view on it as a goal, prospect or process was dealt with respectively.
According to these interpretations, its articulation in practices, I argue, subtly molds the behaviours of professionals and authorities and decision makers in general.
Furthermore, urban development as a positive vision of the future has also been put to question.
Not only concerning whose positive vision it is, but also in terms of the intrinsic evaluation of what positive might be and the authority to decide upon it.
\\

Taking all this into account, the issues raised about urban development are summarized as such:

\begin{itemize}

\item \textbf{Urban development involves a vision, yet not a single one, but multiple individual visions of the future.}
When we speak about urban development as a goal, it is a positive vision directed by particular interest group who see it so.
The accorded positive values of contemporary global society already, long ago, implicitly encroached upon the connotation of development giving it a value of growth, evolution, maturation, modernization and created a wild-goose chase of, actually, an illusory expectation
(\href{Esteva}{\citealt{esteva_development_2010}}).
Instead of futile pursuits disconnected from the concrete space and historical moment, urban development should be, accorded among local populations, civic sector, professionals, experts in their local spaces, while the implementation rests with the negotiations among different layers of decision-making from international to local level.

\item \textbf{Urban development addresses the present as much as the future - it is a process, yet not a single one, but a range of processes.}
Human actions and behaviours are inherently temporal.
They involve an intention, a decision (conscious or unconscious) and implementation, so that they necessarily stretch from the present moment into the future. In this respect, on the way to achieve one goal, multiple other goals are crossing human paths and pulling off course.
While urban reality is inherently populated by humans, urban development should stand and work for multiple actors and various interests at play.
Regardless of their particular nature which might greatly vary depending on actors and time-frames, the reference of these processes to the current state of the system could be referenced as maintenance, transformation or radical change of the same system.
Finally, processual nature of urban development comprise whatever may occur in the future, without explicating its interest/value notion, but only its influence on the current state of the system.

\item \textbf{It is important to be clear what urban development means and how it is articulated in between a goal, a prospect and a process.}
The symbolic meaning of urban development is deeply rooted in human historical and social discourse. It is unrealistic to expect it to be change and abandoned, though the narrative of urban development has cautiously been questioned in scientific and professional domains. 
Consequently, it is also important to reference the results to different interpretations of urban development (a goal, prospect, process).
In comparison to a goal- and process-oriented idea, urban development prospects keep track of reality and probability among what may the scenarios be and what particular interests of different actors and stakeholders are.
Within modern urban framework, the implementation of urban development intervention is be dispatched through:
policies [broad field of possible actions],
programmes [prioritized actions set as procedures or protocols],
and
projects [a course of actions over time with clear goals and detailed plans and budgets: clearly stated goal - a period of time - defined budget] (\href{Vujosevic}{\citealt{vujosevic_regionalizam_2015}}).
Accordingly, after being clear about what urban development is, it is important to tackle also how it is set in motion within policies, programmes, and projects.

\item \textbf{Urban development addresses future in relation to past and present - it might be everything one can make of it.}
Apart from the crisis of capitalism and the critics of growth, the
urges for diversity right (\href{Amin}{\citealt{amin_good_2006}}) and participatory approach speak also about urban development and the importance of the redefinition of the term. 
In this reference, one step forward might be acknowledging the significance of of the system maintenance processes (which might be wrongly referred to as only stagnant), not only those that imply improvements and growth.
\end{itemize}

In sum, urban development is rather an inappropriate concept for describing and guiding urban processes in cities outside the Global North.
Even more so, because the term is being used for tracing and directing urban system transitions in the way that levies the same, globally accorded as universally beneficial objectives to particular inhabitants and a particular urban environment.
Generally speaking, it is a widespread and rarely criticized concept that is extensively used in practice and practice-based research.
Without delving into hidden motives and circumstances (economic, political, colonial), this research addressed the core meaning of this concept, while its scope and aim were kept in order to produce an operational methodological framework for possible practical investigations of the palette of different cities around the world.

\subsection{Urbanity}

Urbanity was another crucial concept undertaken in this research and framed to denote:

\begin{itemize}
\item \textbf{Measurable concept:}
\\
Urbanity includes state and a transition/evolution and therefore it can be socially and spatially evaluated through the level of urbanity.

\item \textbf{The level of urbanity:}
\\
In neutral value state, an overarching definition of the level of urbanity enables grasping urban dynamics.

\item \textbf{Contextualization of urbanity:}
Urbanity is inseparable from the context as it is based not only on processes (density and diversity), but also  on local socio-spatial capital.

\item \textbf{Urbanity results from the combination of planning and design:}
The level of urbanity represents the balance of social and economic planning and socially and culturally responsible urban design.
Otherwise, "planning can slide into blind procedure and design can deteriorate into blind aesthetic
(\href{Van}{\citealt{van_assche_co-evolutions_2013}}) and both cases affects urbanity.
\end{itemize}

\subsection{An Ordinary City}

Having this research grounded in the domain of ordinary cities theory, an important question must be raised about the relevance of the redefined urban development concept for this theory and these cities in terms of their urban complexity and dynamics.
\\

A dominant practical suggestion within the ordinary cities theory is putting all cities together so they learn from their contrasting contexts and experiences. Doing research on ordinary cities is, more or less, all about comparisons and theorizing the South at the South   and theorizing the North from the South (\href{Chaplin}{\citealt{chaplin_architecture_2015}}).
In Jenifer Robinson's words (2016), comparing and developing frameworks, presumptions and dynamic factors that govern urban systems help in constructing validity and cut across the empirical to reach the general (categorizations, concrete and practical abstractions) (\href{Robinson}{\citealt{robinson_thinking_2016}}).
\\

This research is based on a single case study, testifying the complexity and dynamics of a single urban system and its space-time realm.
Yet, it tends to propose several observations on how urban development processes are bound together in an ordinary city.

\begin{itemize}
\item
Local urban culture, the civic order and the value systems enclosed in the historical processes of long duration cast localized urban development processes, contribute to make the city ordinary and entail their intrinsic originality.
Building cities as clones from the visions and models of other places and other cities
deprive them of their particular historical path.
\textbf{An ordinary city cannot be observed in present and in present only.}

\item 
In order to make the urban system evolution contextually appropriate  and resistant to biased power relations and individual interests , it is important to continually keep track of wider social repercussions and assess risks of a range of swift or biased interventions or
transits in the timeline of the historical processes of long duration.
\textbf{Embracing uncertainty, the trial and error iterations and locally adjusted pace are essential for catalyzing local and global developmental interventions.}

\item
When speaking about cities as ordinary, it means that they are navigating its system evolution in an original manner.
In historical terms, the processes of maintenance, maturity, reproduction and resilience (\href{Galtung}{\citealt{galtung_peace_1996}}) indicate historical course of urban development.
The processes of maintenance, transformation and radical change are thereafter localized versions of historical time.
Accepted as such, there is no room for imposed vision of what the future should look like, as the way from maturity to resilience is strongly personalized path. 
In this light, most of top-down, outside-in actions seem like a tactics of imposition. \textbf{An ordinary city should be assembled, not govern from top-down.}

\end{itemize}

\subsection{A Post-socialist City}

The scope of post-socialist cities has directed and confined this research in methodological and empirical sense.
The deficiencies and incompatibilities between the reality of post-socialist cities and the existing western paradigm of the urban directed the theoretical account on the said cities.
Post-socialist cities, in general, are struggling to balance post-socialist path-dependent traits with transitional trends and influences.
In this references and according to the analyses conducted in this thesis, several issues should be highlighted, such as:

\begin{itemize}
\item \textbf{A reflective, yet less presumably critical attitude towards socialist past should be embraced in urban practice;}
\\
Neither future-oriented nor path-dependent directive is solely adequate for approaching urban complexity and dynamics of  post-socialist cities. Namely, a synthetic and objective attitude towards space-time flux is necessary for understanding post-socialist urban evolution (\href{Thomas}{\citealt{thomas_thinking_1998}}; \href{Nedovic}{\citealt{nedovicbudic_waves_2006}}).
There are numerous examples that socialist practices, traditions and values were disregarded and dismissed without a grounded reason and replaced without real criticism with those imposed by transition.
As already mentioned, the evolution of urban systems is not a directional trajectory towards a single goal.
The state of post-socialist cities nowadays resulted from a convergent path from past to present that should be continued toward the most contingent future, not just interrupted and replaced with something trendy and brand  new.

\item \textbf{Balance borrowing, imposition and replication practices with finding locally grounded solutions;}
\\
Urban development outside the Western world was undoubtedly a practice of borrowing, imposition and replication (\href{Nedovic}{ibid.}). So, it could not be avoided nor neglected.
Yet, it is important to base and argument choices and plans in what is found on the ground.
In developing countries regulating and guiding urban development usually starts from exogenous references.
On the other hand,  in post-socialist cities regulatory framework exists and often it does not lack quality nor local reliance, but institutional processes and difficult societal circumstances make them fail in implementation (\href{Nedovic}{\citealt{nedovic-budic_mornings_2011}}). In this circumstances, elaborated and objective evaluations, professional approach, expert and local knowledge are necessary for making the right decision.

\item \textbf{Control, verification and consulting networks should be extensively distributed among local actors, in planning processes and within institutional procedures;}
\\
A burning issue in post-socialist cities is the issue of coordination and cooperation within urban framework.
Horizontal and vertical distribution of roles and decision-making urges for revisions.
While vertical relationality should be fostered in legislative coordination and rules of conduct, horizontal connections are necessary for internal rationalization, de-politicization, continual improvements through education and code of ethics toward work and in interpersonal relations (\href{Vujosevic}{\citealt{vujosevic_regionalizam_2015}}).
     
\item \textbf{Culture, tradition and values that affect urban framework should be mainly local, not global;}
\\
Culture has been recently growing in importance as a resource for urban  and  community development (\href{Bianchini}{\citealt{Bianchini 1999}}; \href{Mercer}{\citealt{mercer_cultural_2006}}; \href{Volic} {\citealt{volic_belgrade_2012}}).
Culture is now ever present and global.
Following the example of Savamala, opening to the international markets, democratic social values and EU joining procedures introduces global cultural practices in the local system with traditional cultural values. 
In this blending, local population with their contextually rooted priorities and needs become side-lined in this quest for internationally recognized innovation, creativity and diversity.
Yet, as Manuel Castells has
argued:
'...local societies...must preserve their identities, and build upon their historical roots,
regardless of their economic and functional dependence on the space of flows. The
symbolic marking of places, the preservation of symbols of recognition, the expression
of collective memory in actual practices of communication, are fundamental means by
which places may continue to exist as such...' (\href{Castells}{\citealt{castells_informational_1991}: 350-351})
As post-socialist cities are more and more embracing this international cultural trends, it is also important to appreciate local and historical in their immediate surroundings.

\item \textbf{Post-socialist cities are in between the North and the South and should profit from the position;}
\\
There are many examples of processes, procedures and practices proving that what is happening in post-socialist cities is neither that featuring developing world, nor it pertains to the developed one, but it is rather a link between the two (\href{ref}{ETHZ 2012 Belgrade Formal Informal}).
Post-socialist cities very often suffer from biased political order, flawed economy and  dysfunctional social organization, but they have usually sustained its developed regulatory framework from socialism and high level, public educational practices and social services.
Having engendered problems from both developed and developing countries, post-socialist cities many a time also have the mixture of solutions from both worlds. 
Taking into account the recommendations based on ordinary cities theory, what could be the knowledge transfer from the South to the North, might actually begin in post-socialist cities.

\item {\textbf{Local professionals should increase their knowledge of risk and vulnerability and accept to deal with uncertainty;}}
\\
Finally, in relation to the concrete context in Serbia, which is not the unique case (at least not in the Balkans), sharp social decline and system dissolution is an example of anti/de-development.
Mistakes and misconducts should be exposed in order not to be repeated and a critical and introspective attitude may provide a potential knowledge on what makes the system maintain itself, reproduce, mature and finally become resilient.
\end{itemize}

\subsection{And in local context... [Serbia - Belgrade - Savamala]}

The integrity of this research is also supported with the significance of the case study selection for general public interest, contextually relevant recommendations and practical implications (\href{ref}{\citealt{yin_case_2009}}).
\\

The case study choice was based on the recognized potential of the incomplete and spontaneous urban system evolution happening in transition.
An unclear social, economic and political situation in post-socialist cities is a fertile ground for a fragmented, small-scale approach to urban conflicts, which could eventually produce more long-term and far-reaching results.
A myriad of strategies from above, interventions instigated by different interests and small changes from the ground up are the constitutive forces of the aforementioned incompleteness of post-socialist cities.
However, research into these underlying forces has been limited notwithstanding the importance of solving urban conflicts and making cities durable, flexible and visible on the global scale.
\\

However, the researched local context is rather chaotic and difficult to handle.
Serbia is still identified as a post-socialist melting pot where representative democracy, civil society and market economy principles collide and merge with authoritarianism, vertical decision making and the practices of populism.
Serbia is experiencing the reversed tendency from what urban development should stand for - it is under strong pressure of negative natural increase (\href{ref}{UNICEF 2013}), declining economic capacity and deteriorating living standard (\href{ref}{Vujosevic}), growing ignorance for democratic political practice (\href{ref}{\citealt{peric_evolution_2016}}), and disregard for research, education and  knowledge \href{ref}{\citealt{vujosevic_conundrum_2012}}.
In these circumnstances, most important issue to deal with is the harmonization of the disbalanced morphology of post-socialist urban decision-making towards solving urban conflicts, in the way that:

\begin{itemize}
\item \textbf{Local experts and professionals} - return dignity and influence to experts and professionals, who are often side-lined;

\item \textbf{Professional domains} - move away from authoritarian and partocratic approach in professional domains;

\item \textbf{Urban planning} - re-establish the role of urban planning as an instrument of knowledgable not powerful, and as a practice of socio-spatial management of urban environments, not a technical task;

\item \textbf{Bottom-up actions and initiatives}  - provide means for prioritizing and fostering positive practices, which are usually happening on the ground with no effective  and  binding policies  and institutionalized  regulatory  means  for synchronization  and  coordination  among  them.
\end{itemize}

Finally, the study of Savamala neighbourhood enabled exposing the game of power relations and influences among urban key agents by indicating
(1)	interrelated processes and procedures generated by local urban agency;
(2)	assemblage network of relations between regulatory framework, urban actors,  and spatial issues generated through a bottom-up logical argumentation;
and
(3)	urban  patterns  and  social  impact that are produced therein.  
\\

Unfortunately, based on what is happening on site, Savamala could not be saved from the destiny prescribed to it from the top-down.
Yet the scientific results taken from this case study might be a strategic basis for more effective reactions in the future and didactic material for further education which address the on-going generation of urban conflicts and bottom-up interventions and offer a summary of small movements and partial approaches which surpasses the model of a post-socialist city on which it has been built and target an integrated system of urban development processes.
 
\subsection{Epistemological framework - When ANT opens doors for revisions..}
 
This research recognized the quality of ANT scientific approach as an explanatory construct that studies associations and symmetrical relationality (\href{ref}{\citealt{farias_urban_2011}}).
From the results presented herein, usefulness of ANT interpretation was a mere entry point on an operational agenda for further research.
Accordingly, ANT appeared to have limited capacity for going any further and it seemed unable to receive practical recognition, influence the reality and go beyond descriptions.
\\

Bearing in mind complexity and dynamics of urban development phenomenon, the vitality of ANT approach lies in: (1) encompassing the active role of non-humans, (2) seeing the totality of the world as process, and (3) overreaching radical categories of time and space by representing horizontal links and associations.
Although these premises grasp the core concept of urban dynamics, this methodology does not imply the capacity to deconstruct and interpret such complex aggregation of all real-life urban processes.
In urban terms, ANT is, therefore, still perceived as a conceptual methodology whose integral approach works out only in confined urban environments, where it could comprise a dynamic, interactive process of interdependences and connections among all active urban actors and the formation of urban assemblages through roles, associations and agencies and their calibration within the chain of decision-making.
\\
In this respect I would like to recapitulate ANT setbacks to stand out as an overarching methodological approach for urban research:

\begin{itemize}
\item ANT in its insistence on general symmetry \textbf{fails to go beyond the description of the empirical reality of urban processes}. Although it succeeds to include causal relations between actors, the future state of the system based on these relations stays undiscovered (\href{ref}{\citealt{elder-vass_searching_2008}}). For example, the engagement of political, economic and cultural aspect in networks was signified, but without any insight on the valence of these aspects to engage in new networks, while this relation between what is and what will/would/could be is actually the essence of urban development concept.

\item Though ANT inaugurate flat ontology of the social (\href{ref}{\citealt{latour_reassembling_2005}}), \textbf{ANT networks are "narrated" by human constituents} (\href{ref}{\citealt{pickering_epistemological_1992}};\href{ref}{\citealt{czarniawska_narrating_1997}}; \href{ref}{\citealt{whittle_is_2008}};  \href{ref}{\citealt{marshall_mapping_2015}}) and interpretations and translations are chiefly the product of the researcher's positionality  (\href{ref}{\citealt{rose_situating_1997}}; \href{ref}{\citealt{ruming_following_2009}}). It is acknowledged that the ANT diagram reorder and multiply if transferred from an actor’s viewpoint to other’s or if it is re-iterated by other researcher. Therefore, the question of producing the same results regardless of iterations or agency could be raised.

\item Perceiving ANT results only as detailed empirical descriptions means discrediting "how and why" questions, leads to thinking that \textbf{its sole aim is maintenance of the system} (\href{ref}{Amsterdamska 1990}; \cite{Lee and Brown 1994}; \href{ref}{\citealt{lee_otherness_1994}}), with no regard to prospects for its change or transformation (\href{ref}{\citealt{gabriel_post-social_2008}}).
From ANT examinations of Savamala, the results were the developmental flows of human collisions and coalitions, finances, practices, information and knowledge, yet without any tools at hand to point out where maintenance, transformation and change of the system happens.
With ANT results the researcher was incapable to intervene in an urban system - to articulate social practices, anticipate conflictive urban issues, and provide an overview of actions, solutions and changes - which were identified as the pillars of urban development within this research project.
\end{itemize}
 
In sum, while ANT theoretical perspective has aspired to explain the totality of the world without relying on "other" frameworks (\href{ref}{\citealt{lee_otherness_1994}},  \href{ref}{\citealt{gad_consequences_2010}}), it actually remained on the level of description that may appear insufficient and ineffective for practical application (\href{ref}{\citealt{gabriel_post-social_2008}}).
The ANT diagram assembled in this research addressed complexity and provided framework for future extension of actors and new relations when they collide, overlap and interfere in networks.
These networks represented a base for system dynamics of cooperative, discontinuous, contradictory or even mutually exclusive relations among constantly changing actors and consequent continuous production of new urban realities.
In this research, ANT scheme neither could have told us anything about this, nor could have it indicated how the urban system maintain, transform or change itself. In this respect, even though ANT categorizations and interpretations have successfully dealt with urban complexity, it still fell short to meet the expectations as a potential interpretive tool for urban dynamics.
\\

Bearing all this in mind, this research undertook the task of complementing ANT approach in order to facilitate an understanding of undercover processes and mechanisms or to provide explanations, recommendations or operational diagnosis on how to cope with urban development as a process.

\section{Limitations of the research}

The limitations of the research project mainly result from the restrictions of the research framework and of the methods used.
\\

To begin with, \textbf{a trade-off between knowledge and time} is very often at play in the domain of thesis research (\href{Harrison}{\citealt{Harrison 2002}}).
The study was an individual research project and processes of data collection and analyses were adapted to the resources at hand.
Firstly, the enormous amount of data could not be disentangled completely. In this respect, cautious classification of data within numerous clusters gave the opportunity to the researcher to choose the scope of the research.
In this sense the environmental questions and issues and infrastructural scapes have been excluded from the study.
\\

Then, the situation on the ground has been changing since 2013 and even more rapidly in the last 2 years (2015-2017).
These fluctuating circumstances required a constant adjustment of the scope of data collection and important re-think toward the systematization and categorization of data.
Even though these recent changes have been undertaken to a certain extent in the project, the researcher is aware of the possible gaps and missing updates.
\\

Another pertinent issue was raised around \textbf{the case study bias toward verification} (\href{ref}{\citealt{flyvbjerg_five_2006}}).
While the aim of ANT application was to reduce the influences of the researcher's preconceived notions, the researcher encountered difficulties in this sense within data analyses.
Namely, thorough data triangulation was possible on a limited scale, while several cluster of data from the field needed additional verification so that they were excluded from the results.
In this scope, another important matter revolves around the categorizations of qualitative data proposed within ANT, MAS and MAS-ANT. 
Such systematization of the field data required an additional, multitudinous critical and multidisciplinary review from experts, professionals and practitioners for the further uses.
Above all, identified clusters of contextual data [facts, factors, motives, values, aspects etc.] need an extensive quantitative verification before it can be used in practice.
\\

Finally, the researcher aimed to take a critical stance towards the capacity of the study to summarize and develop generalizations and theoretical abstractions.
There is \textbf{a yawning gap of multifarious verification still to be done for transferring localized categories into abstract clusters for general application}.
As \href{Robinson}{\citealt{robinson_urban_2013}} elaborated, there is not a paved path for moving from the concrete field-work findings and abstractions to an universal analytical framework in urban analysis. A theory/method may work well for a single case, but it does not necessarily apply on other; if it works for several, it may not be the case for many; and if it is applicable for many, very often it does not imply all the cases in all the circumstances.
Therefore, it is important to acknowledge that MAS-ANT has a potential to be applied in different contexts, but to which extent it cannot be stated nor expressed.


\section{Practical Implications}

This thesis aimed to define a method for addressing a concrete situation on the ground through a process of understanding and dealing with current difficulties as they emerge and evolve.
In this sense, visualized interpretations of complex agent-network-process map could be applied for \textbf{practice-based research on the ground}.
Formal and informal groups and international and local organizations as well as activists and citizens could profit from from such exposition of actors and relations for better understanding of how and by whom cities are made and governed.
\\

The style of \textbf{infographics} used for data display is highly open for computerization, which enable continual updates of the results, extensive generalizations and conclusions drawing and uninterrupted input and progressive description of new elements and relations.
Bearing this aim in mind, it is important to emphasize that such an idea has been already introduced in urban studies and even more so in practice-based urban research.
\\

Cities tend to urbanize technologies semi-autonomously with increases in density and networked systems that new technologies have made possible.
There emerges, however, a growing discrepancy between the dynamics of socio-spatial changes and the weakness of technical supervision; not only in the course of human factors, but also in technological sources and solutions (\href{Vauquelin}{\citealt{vauquelin_planification_2010}}).
Thus, this research contribute to responding to the necessity to shift the deterministic concept of how to approach urban research to a more comprehensive, network-oriented vision that considers generating \textbf{ICT-oriented approaches and tools} (\href{Huang}{\citealt{huang_ict-oriented_2012}}).
\\

Bearing this general picture in mind from the very beginning, this research aimed to take a practical course in terms of finding an intermediary between the qualitative data analysis and the data display closely related to modern means of communication.
As regards, the display of results from this study through illustrations create a precondition for the visualization of these data and eventual digitalization of urban development processes through an ICT model.

\subsection{Data visualization}

This exercise of visualizing data through MAS-ANT methodological approach expresses an attempt to depict the complexity of urban actors, forces and artifacts and the dynamics of networks, interdependences and processes to a legible, data-loaded scheme of nodes and links. 
\\

The idea behind the visual interpretations was to create a pattern that can be visualized digitally with HTML5 or Java.
Static illustrations are limited in terms of the amount of data that can be taken into account and represented while keeping track of the legibility of the diagram.
Conversely, dynamic data visualization gives an opportunity to present a lot of data simultaneously, having a large and ever growing background database. 
\\

Accordingly, digitalized version can be easily modified and updated with data, elements, relations, and conclusions, in comparison to the paper-based version.
Furthermore, digitally visualized data are more user-friendly and user-oriented in the sense that users can easily navigate through, decide upon the scale of the diagram, choose what they want to see and users can browse vertically and horizontally across the dataset.

\subsection{Urban Development Model}

The idea of a digital urban development model addresses the current change trends in urban analyses towards an open-ended future concept with an emphasis on inclusive, transparent and flexible procedures (\href{Rode}{\citealt{rode_city_2006}}).
The flow of information and pervasiveness of communication technology constitute a developmental core of modern society (\href{ref}{\citealt{sassen_cities_2012}}).
Furthermore, this technological advancement has influenced the perception and constitution of reality, they have allowed accelerating process of the dispersion of activities, a transfer of products, hypermobility of capital, data distribution, knowledge sharing, and a redefinition of physical space (\href{ref}{\citealt{firmino_pervasive_2008}}). 
\\

Therefore, a further extension from the illustration through data visualization is a multidimensional digitalized model.
%mapping time - take from ANT map text
The model should be dynamically programmed to comprise local patterns [actor-networks, contextual elements and maintenance, transformation and change processes].
The initial database and the algorithms for their entailment, implication and structuralization should be constructed on the principles of the MAS-ANT methodological approach and its graphical and data visualization tactics.
The represented state of urban development processes should be continuously updated with relevant data to mirror current situation in the field (e. g. changes in the regulatory framework and space).
\\

The model should enable users to intervene and propose their interpretations of nodes, links and processes.
A sum of urban development processes should be based on the individual's direct experience and continually congregated from the data input from participants.
Such tendency fosters social inclusion not only in urban practice but also in urban research working also as a knowledge sharing platform on urban development, complexity and dynamics of urban systems and on the local context.
Therefore, an inclusive and dynamic urban development model is actually an articulation of human life in an urban realm which encourages active citizenship.

\section{Future Prospects}

The far-reaching contributions of this research is addressed on theoretical and methodological level. 
\\
In theoretical terms, the question of \textbf{urban development in terms of urban system transitions} is an assumption that should be investigated further. Even more so in terms of the distinction of maintenance, transformation and change processes, how they are assembled and distribute in local contexts.
Besides, urban development as a set of system transitions in ordinary cities, notwithstanding their developed, developing and transitional nature, should also be put to investigation. 
\\

Another issue is possible practical usefulness of \textbf{MAS-ANT schematic interpretation of data and processes} for participatory planning and ground-up interventions.
Its flexibility and iterative nature should be investigated in terms of its capacity to exceed the abstractions of urban planning, the concrete specifics of urban design and the politicization of urban transformations and participatory processes, bringing all of these together as bridges for realms of ideas about urban future.
\\

In this sense it is not only that information act upon technology, but technologies also act upon information.
While technology and information form an integral part of all human activity today, all processes and relations of individual and collective existence are directly shaped by the new technological medium.
What is more, much of the economic, social, political and cultural action shifts into cyberspace (\href{Mitchell}{\citealt{mitchell_city_1996}}), in the form of a legitimate second reality (\href{Baudrillard}{\citealt{baudrillard_l_1983}}) where the interrelations of  all material and non-material objects revised in terms of spaces of relations (\href{Graham}{\citealt{graham_relational_1999}}).
Controversially, it does not make the actual places (urban spaces) redundant, but rather it initiates an active reconstruction of urban places (\href{ref}{\citealt{graham_splintering_2001}}) as social constructs whose meaning depends on particular social contexts and their nodes of intersection (\href{ref}{\citealt{healey_treatment_2004}}).
In digital world, all cities become ordinary, the same as the Internet enable all different, ordinary voices/opinions to be heard.

\section{.. Few More Words}

The conceptual framework explained herein pinpoints the blurred and askew morphology of post-socialist cities in the gamut of ordinary cities around the world which requires dynamic solutions in order to skip the classical, western procedure of urban formation and development. Consequently, this particular context shows the increased need for proper techniques that are spatially and temporally adjusted to current socio-spatial issues. 
\\

The urban development of post-socialist cities is perceived as a dynamic concept, a multi- dimensional and integrated system composed of qualitatively different and semi-autonomous processes, with the inclining tendency to address the economic, social, demographic, political and technological state of an urban environment.  In view of all this, this thesis proposes and an overarching approach to urban development that can encompass all discrepant decision-making forces: future-oriented urban projections (urban planning strategies), in situ transformation forces and potentials (urban transformat andions), and to follow the creative paths of urban dwellers (participatory urban design activities) for imagining paths of new urban futures.
The question of facilitating and localizing urban transitions rests with overlapping urban scenarios from dissonant layers of decision making, tracking cultural identities, and the requirements and needs of all urban actors, and, in general, indicating the contextual processes of maintenance, transformation and change within an urban system.

\newpage
\chapter{List of Acronyms and Abbreviations}

\newpage
\appendix
\chapter{Data collection}

\section{Interviews}
\section{Questionnaires}
\section{Workshops}


\newpage
\chapter*{Bibliography}
%%%%%%%%%%%%%%%%%%%%%%%%%%%%%%%%%%%%%%%%%%%%%%%%%%

\begin{small}
\addcontentsline{toc}{chapter}{Bibliography}
\bibliography{bibliography/Ch6_scientific,bibliography/Ch6_practicebased,bibliography/Ch6_media,bibliography/Ch6_legal,bibliography/Ch1_scientific,bibliography/Ch1_practicebased,bibliography/Ch2_scientific,bibliography/Ch2_practicebased,bibliography/CH2_media,bibliography/Ch2_legal,bibliography/Ch3_scientific,bibliography/Ch3_practicebased,bibliography/Ch7_practicebased,bibliography/Ch7_scientific,bibliography/Ch7_media,bibliography/Ch7_legal,bibliography/Ch4_practicebased,bibliography/Ch4_scientific,bibliography/Ch4_media,bibliography/Ch4_legal,bibliography/Ch5_scientific,bibliography/Ch5_practicebased,bibliography/Ch5_legal,bibliography/Ch5_media,bibliography/Ch8_scientific,bibliography/Ch8_practicebased}
\end{small}

\section{Scientific References}
\section{Legal Documents}
\section{Grey Lit}
\section{Media Sources}

\newpage
\chapter{CV Marija Cvetinovic}




%%%%%%%%%%%%%%%%%%%%%%%%%%%%%%%%%%%%%%%%%%%%%%%%%%


\newpage
\appendix
\noappendicestocpagenum
\addappheadtotoc

\end{document}
