
\documentclass[11pt]{report}
\usepackage[a4paper,margin=2cm, bindingoffset=2cm]{geometry}
\usepackage{appendix}
\usepackage{amsmath}
\usepackage{booktabs}
\usepackage{threeparttable}
\usepackage{natbib}
\usepackage{changes}
\usepackage{color,soul}
\bibliographystyle{chicago}

%to stop orphan lines
\widowpenalty=10000
\clubpenalty=10000
\raggedbottom

%line spacing
\linespread{1.3}

\begin{document}
\begin{titlepage}
\begin{center}
\large
\textsc{EPFL} \\
\ \\
\textsc{CODEV and LASUR} \\
\ \\
\textsc{EDAR}\\
\ \\
\ \\
\ \\
\ \\
\ \\
\ \\
\ \\
\Huge
\textbf{Urban Development Model }
\ \\
\ \\
\large by
\ \\
\ \\
Marija Cvetinovic
\vfill
Thesis for the degree of Doctor of Philosophy \\
\ \\
\ \\
XXX 2017
\end{center}
\end{titlepage}
\begin{titlepage}
\begin{center}
\ \\
\ \\
\ \\
\ \\
\ \\
\ \\
\ \\
\ \\
\ \\
\ \\
\ \\
\ \\
XXXX
\end{center}
\end{titlepage}

%Roman Page Numbering
\pagenumbering{roman}
\chapter*{{Abstract}\markboth{Acknowledgements}{Acknowledgements}}
\addcontentsline{toc}{chapter}{Abstract}
xxxxxxxxxxxxxxxxxxxxxxxxxxxxxxxxxxxxxxxxxxxxxxxxxxxxxxxxxxxxxxxxxxxxxxxxxxxxxxxxxxxxxxxxxxxxxxxxxxxxxxxxxxxxxxxxxxxxxxxxxxxxxxxxxxxxxx
xxxxxxxxxxxxxxxxxxxxxxxxxxxxxxx.\\


\tableofcontents
\listoffigures
\addcontentsline{toc}{chapter}{List of Figures}
\listoftables
\addcontentsline{toc}{chapter}{List of Tabes}

\chapter*{Declaration of Authorship}
\addcontentsline{toc}{chapter}{Declaration Of Authorship}
I, xxxx, declare that the thesis entitled xxxxxxxxxxxxxxxxxxx and the work presented in the thesis are both my own, and have been generated by me as the result of my own original research. I confirm that:
\begin{itemize}
\item this work was done wholly or mainly while in candidature for a research degree at this University;
\item where any part of this thesis has previously been submitted for a degree or any other qualification at this University or any other institution, this has been clearly stated;
\item where I have consulted the published work of others, this is always clearly attributed;
\item where I have quoted from the work of others, the source is always given. With the exception of such quotations, this thesis is entirely my own work;
\item I have acknowledged all main sources of help;
\item where the thesis is based on work done by myself jointly with others, I have made clear exactly what was done by others and what I have contributed myself;
\end{itemize}
\vspace{2cm}
Signed: \dotfill
\vspace{2cm}
\newline
\noindent
Date

%Include Acknowledgements in TOC
\chapter*{Acknowledgements}
\addcontentsline{toc}{chapter}{Acknowledgements}
This work was undertaken with financial support of the 

This thesis would not have been possible without the support of many people. I would like to express my sincere gratitude to:
\begin{itemize}
\item 
\item ...
\end{itemize}



%%%%%%%%%%%%%%%%%%%%%%%%%%%%%%%%%%%%%%%%%%%%%%%%%%

\chapter{Introduction}
%Arabic Page Numbering
\pagenumbering{arabic}

%%%%%%%%%%%%%%%%%%%%%%%%%%%%%%%%%%%%%%%%
Urban development is a widespread archetype for out-of-reach improvement in cities of the Global South. However, in its essence it is more a kind of constant catch up with the West and western urban paradigm than an elaborated form of intrinsic local perception, knowledge and action toward urban transitions.
\\
In this highly competitive international arena, transitional countries experience grave consequences due to the paucity of practical experience within the dominant/ruling western ideology of the urban. They are caught in this new context of relentless rules of market economy, decentralized political and administrative powers, lack of resources, scarcity of general international investment and scant interest for dramatic shifts in all aspects of their social organization and spatial transformations. The blurred and askew morphology of post-socialist cities in transitional countries is therefore the result of continuous pressure from the negative side effects of imitating and lagging behind conventional urbanization models and accelerating globalization patterns imported or imposed by the Global North, or colloquially known \deleted{in Serbia} as the West.
\\
The urban transformation of Serbian cities falls into this cliché of the new post-socialist urban reality, which emerged during the “transition to markets and democracy” (Tsenkova, 2006). The dismantling of the communist system during the late 1980s represented a substantial change in all aspects of social organization, the economic model and the political system. However, Serbia is still identified as a post-socialist melting pot where representative democracy, civil society and market economy principles collide and merge with authoritarianism, vertical decision making and populism practices. In such a situation, concern about the urban has been left out and given over as a battlefield for social needs in practice and technical solutions on paper and an easy prey to the exercise of power and interest. Therefore, the practice of planning and designing Serbian cities has been narrowed down to a mere technical issue most often even without an actual or adequate realization in practice. Not to mention that very few theoretical or general methodological research studies bothered to examine alternative planning modes, techniques and instruments in transition, but continued with the manner of replication from well-known counterparts of the Global North.
\\
Among others, architects have a vital role in not only directing but also framing the path of urban formation and development in post-socialist cities. Even more so, as they are primarily focused on practice and “savoir faire” about making the built environment, while acquisition of land and illegal construction are spatial interventions that have marked post-socialist production of space more than any planning or theoretical activity. Though we have different drivers on the global scale and in developed countries, there is a global trend of resorting to sociological, planning or even IC approach in scientific studies on the urban. 
In order for architectural intervention in space to compete for more relevancy and rigour,  architects all over the world have been gradually grown interest for scientific discourse on built structures, spaces and cities in general.  In the rivalry between spatial and social basis for their interpretations, the fact that the field of architectural research is not yet standardized in terms of methodologies and techniques opens the floor for experiments and innovations \added{reference}. \added{In the circumstances of developmental bouillon or "developmental schizophrenia"(\hl{Vujosevic}) at the local level and an overall urge for architectural research framework internationally, my aim is to elaborate an architectural standpoint when applied on complex post-socialist urban reality in order to establish a methodological approach suitable for architectural scientific discourse on the matter.}
\\
In my striving to contribute to post-socialist architectural research, the far-reaching aim is to capture post-socialist urban dynamics  in order to skip the classical procedure of urban development based on western planning paradigm and provide its practical application on multiple levels of urban decision making. This to be achieved requires supple \added{methodological} approaches which should better correspond to post-socialist socio-spatial patterns on multiple levels (state, city, municipality, community, and neighbourhood) and explain the correlations of various urban elements. Practice-oriented, locally focused and globally tuned  approach to complex urban reality of post-socialist cities envisions embracing the dynamics of urban systems and operationality of architectural performance for circumscribing visual interpretations that enable continuous conclusion drawing and up-to-date introduction of any new element that may appear in the system.

\section{Field of Study}
\deleted{\textbf{research scope}: urban development in cities}
\\
Due to growing social and physical transformations that become ever more intensified as current globalization continues to spread out profit maximization, consumption patterns and information networks (Harvey 2012), the cities have been experiencing a progressive reorganization at spatial and social levels. Even though accelerating urbanisation is a worldwide process, it still assumes different forms and meanings, depending on the prevailing local conditions (Bolay, 2007). These overall circumstances of continuous urban development influence cities to serve as the primary channel linking local realities to global social, political and economic, forces (Yates and Cheng, 2002; Tsenkova, 2006).
\\
Cities are not simply market products and consumption patterns, but locally customized socio-political constructs as well (Marcuse et al. 2008). These external influences form a range of qualitatively different contextual circumstances for positive urban change, i.e. urban development. Most settlements and cities from previous historical periods had reflected upon the various degrees of forethought and conscious design in their layout and function, which is referred to as a fixedly planned development, albeit many had tended to develop organically. Historically speaking, urban planning is a future-orientated, top-down activity and expertise for managing urban development (Allmendinger, 2009; Faludi, 1973). \added{Generally speaking, a range of urban disciplines} \deleted{Backed up by} (urban theory, \deleted{urban} sociology, \deleted{urban} legislation and \deleted{urban} design) aim \deleted{of modern urban planning should be} to decode and harmonize growing urban issues as a side-effect of the current globalization, urbanisation processes and spread of \hl{capitalism} that are mainly affecting cities and production of urban space. 
\\
In practice, these disciplines are embedded in a particular social context or a territorially based system of socio-economic relations, and they react to the shifts in socio-economic and political settings (Tsenkova, 2006b), but have kept privileged relationship toward Western cities, as assumed to be the sources of urban creativity, vitality and innovativeness (\hl{Robinson, 2006:2}). Accordingly, they tend to fail substantially within the range of spatially and economically different environments that have undergone highly dramatic change in political, economic and social terms. \added{For example, urban research and practice in transitional countries in Central and Eastern Europe (CEE) should unfolded to help understand these phenomena in their immediate and wider context, identify patterns of the dynamic reality in these cities and be more consistent with spontaneous, everyday urban \added{transitions} \deleted{development}. Accordingly, a corresponding change in approaching urban development can then be addressed by heterogeneous, iterative and generative process of urban space production in physical and social sense that has surpassed the perception of cities as merely economic, social and cultural venues treating them as complex and dynamic urban systems.In these circumstances it is necessary to apply proper techniques and methodologies for urban research and analyses which encompass complexity and dynamics of cities for the improvement of their living conditions and the facilitation of social interactions in the process of urban development.}
\\
Production of space is also the core concern for architects. As the advancement of architecture relies on the   relevance and reliability of knowledge, gaining knowledge and understanding on management of space and built environment is actually among the concerns of architectural research (\hl{ref}). Hitherto history and theory of architecture have been the main fields of architectural research. However, missing links with the classical scientific discourse has caused a growing concern for what is research appropriate for architectural and design practice as well as for architectural stance in urban studies, especially in terms of methods, approaches, domain and credibility (\hl{ref+Savic 2016b}). Lacking the traditional scope of analysis, architectural research has been a polygon for innovations and experiments.
\\
Architecture is a discipline focused on practice and consequently it urges for parameters, categories and structure for its practice-based analyses. In terms of methods, there has been a significant number of interdisciplinary, transdisciplinary and multidisciplinary endeavours in applied research with an architectural focus in urbanism (ref). What is more, applied fields of research acknowledge the use of methodological hybrids (Datta, 1994, De Lisle 2011). This has open doors for applied social sciences to investigate new methodological opportunities when confronted with complex and multiplex social phenomena (De Lisle 2011). Even more so as methodological and epistemological rigidity leads to ignoring the realities of the practical and cause catastrophic scientific failure of practice-oriented research (Rogers 2008, De Lisle 2011).
\\
This proposal aims to define a method of solving concrete problems through a process of understanding and dealing with current difficulties as they emerge and evolve.
\deleted{Accordingly, a corresponding change in approaching urban development can then be addressed by heterogeneous and iterative approach that has surpassed the perception of cities as merely economic, social and cultural venues treating them as complex and dynamic urban systems. In these circumstances it is necessary to apply proper techniques and methodologies for urban research and analyses which encompass complexity and dynamics of cities for the improvement of their living conditions and the facilitation of social interactions in the process of urban development.
Conversely, cities are dynamic and diverse urban entities that are given to shaping their autonomous and innovative future on the basis of human resources and creative human potential (Knight and Gappert, 1989; Yigitcanlar, 2008).}
The aim of this concept is to transform the general body of knowledge on cities \deleted{and urban planning} into a real-life problem-solving strategy, which is increasingly built using participatory processes that conform to current relationships in modern cities, and by employing ICTs that have redefined human lifestyles, social relations and the concept of space (Castells, 2000; Dijk, 2002).
\deleted{The prosperity of cities depends on how competitive they are on a global economic scale, how flexible they are in terms of adjusting to current trends and needs, and how fertile they are for the development of knowledge and the application of innovation.}

A crucial change in the \added{methodological} paradigm \deleted{of urban planning} can then be circumscribed to the rise of the global concept from static to itinerant and dynamic; where a static world is one in which all change is according to a known law and which does not give rise to uncertainty. When defining the evolution of analyzing and planning an urban phenomenon or process, it is fundamental to state that the existence of a problem of knowledge depends on the future being different from the past, while the possibility of the solution to the problem depends on the future being like the past. Therefore, change in some sense is a condition of the existence of any problem. But the process of formulating changes in terms of unchanging "laws” cannot be actuated into completeness, and gives rise to the rule of permutations and combinations with uncertainty as one of its fundamental facts. (Richard de Neufville, Real Options: Dealing with Uncertainty in Systems Planning and Design)
\deleted{The concept of progress is central to modern society and it is orientated towards a positive vision of the future. In \added{an urban} scope \deleted{of modern urban planning}, this concept corresponds to that of risk, where the control of all future events is calculable and predictable in probabilistic terms. This new concept of urban planning is based on the notion of an open-ended future, which implies that uncertainty must be accepted and managed, authorities and actual urban actors should be ready for new requirements and renewability as conditions change, and professionals are to increase their knowledge of risk and vulnerability in urban environments. In this sense, the planning of discourse relates to a master narrative of modernity, including ideas of rationality, objectivity, scientific evidence, values and possible control through normativity.}
\\
\textbf{case study}: a neighbourhood

\subsection{Background}
\textbf{ordinary city} (Jennifer Robinson)
\\
\added{Nowadays, while about 50 percent of the world population lives in urban environments (United Nations 2008), these major uncertainties of contemporary life, created by the new method of production and management, are acutely symbolized by concerns about the city and urban life (Healey 1997) 
Cities are primary venues, power poles and capacity builders of economic, social and cultural development at stake in modern societies (Castells, 1998).Conversely, cities are dynamic and diverse urban entities that are given to shaping their autonomous and innovative future on the basis of human resources and creative human potential (Knight and Gappert, 1989; Yigitcanlar, 2008).}
\\
\deleted{Most settlements and cities from previous historical periods had reflected upon the various degrees of forethought and conscious design in their layout and function, which is referred to as a fixedly planned development, albeit many had tended to develop organically.} Seeing city development from a historical point of view, it has been apparent that expanding physically has dominated the planning process, treating the essential social nucleus as mere afterthought. Additionally, the rapid development of cities in the 20th century, which had been preceded by technological advances, has considerably changed the morphology of cities. What is therein produced is a city as “a collection of conceptually ill-decoded enigmas” which in the process of planning has lost its substance as urban phenomenon (Yves Pedrazzini, Jean-Claude Bolay: Social Practices and Spatial Changes)
\\
On the contrary, according to one of the leading urban theories of David Harvey and Manuel Castells, urban planning cannot be seen as an autonomous process of spatial development, but rather it is situated in its political economic context and constantly overlaps current economic and social changes  (Taylor, 2006). In other words, urban planning in practice is intrinsically connected to the property market (which in turn involves a particular political ideology) and this tends to maintain current social order (Dear and Scott, 1981; Taylor, 2006), both of which are grounded in the development and expansion of industrial capitalism, neo-liberalism and consumerism (Ellin, 1999; Harvey, 1989). In other words, urban areas are, and have always been, the spatial and symbolic manifestations of broader social forces (Giddens, 1992).
\\
The prosperity of cities depends on how competitive they are on a global economic scale, how flexible they are in terms of adjusting to current trends and needs, and how fertile they are for the development of knowledge and the application of innovation.
\textbf{entry point}: post-socialist city
\\
Urban reality of post-socialist cities is seen as a complex, multifaceted network of causes and effects that evidence  the dynamics of  interaction  and  interconnections  among  people  (urban  actors  and  stakeholders),  objects  (built  environment), territories (space), institutions (regulatory framework), infrastructure and social  aspects (political, economic and cultural circumstances) (Firmino et al., 2008).

\textbf{research challenge}: legitimacy of urban decision making for urban development - planning and participation and design and interests
\\
The  conceptual  framework  explained  herein  examines  the  blurred  and  askew  morphology of  post-socialist  cities (as well as developing and emerging) which  requires  dynamic  solutions  in  order  to  skip  the  classical  procedure  of  urban  formation  and 
development and transform the negative side effects of imitating and lagging behind the western urbanization model and those of the accelerating globalization into a development impetus suited to these environments.
\added{The concept of progress is central to modern society and it is orientated towards a positive vision of the future. In \added{an urban} scope \deleted{of modern urban planning}, this concept corresponds to that of risk, where the control of all future events is calculable and predictable in probabilistic terms. This new concept of urban planning is based on the notion of an open-ended future, which implies that uncertainty must be accepted and managed, authorities and actual urban actors should be ready for new requirements and renewability as conditions change, and professionals are to increase their knowledge of risk and vulnerability in urban environments. In this sense, the planning of discourse relates to a master narrative of modernity, including ideas of rationality, objectivity, scientific evidence, values and possible control through normativity.}
\\
\textbf{urbanity}
Takodje sam se zbula jer mi se ucinilo da moram ovu kombinaciju metoda da ispratim nekim teoretskim okvirom, pa sam uvela pojam urbaniteta, jer mi nivo urbaniteta obuhvata i stanje i promenu. A u celom doktoratu sam mislila da obradim kako onda kombinacijom urbaniteta (opis stanja i agenata promena) i odlucivanja (sortiranje agenata i promena prema slojevima: planiranje, investicione transformacije, participativne aktivnosti) prikazujemo dinamiku urbanog razvoja, koristeci  ANT pa MAS. 
Distinctive category familiar but not exclusive to architectural analysis is “urbanity”. The relationship between the physicality of urban form and the social components of urban life generates the level of urbanity - the quality of continuous harmonization of the variety of structural elements, social factors and vested interests existing in an urban environment (Holanda 2002, Canuto et al. 2012). Moreover, all these urban key elements are assumed to be equal agents in the continuous process of urban development that has been marked by maintenance, transformation and change of the urban system in order to improve its living conditions and facilitate social interactions.

\subsection{Problem statement}
urban decision making

\subsection{Research Objectives}

\textbf{Overall objective:}
grasp the actual urban development process in cities encompassing dynamics and complexity of the change at the local level in a rather transparent way

(identify variables in objectives)

\subsubsection{RO1}

RO1a: identify and sort out all active agents and contextual resources at the neighbourhood level and map their interconnections and networks 

RO1b:identify how socio-spatial patterns of an ordinary city are constituted at the neighbourhood level

RO1c: Define the state of an urban environment in an ordinary city

RO1d: re-formulate urban development to fit the idea of dynamic state of an ordinary city

trace the morphology of decision making at the neighbourhood level

\subsubsection{RO2}

RO2a: elaborate and encode spatial, social and technical differences and specificity of post-socialist context

RO2b: elaborate and encode spatial, social and technical differences and specificity of an ordinary city

spatial, social and technical differences and specificity ARE \textbf{socio-spatial patterns}

RO2c: gain an in-depth understanding of the level of urbanity in an ordinary city

\subsubsection{RO3}

RO3a: proceduralize urban development process to trace the morphology of decision making at the neighbourhood level

RO3b: re-conceptualize urban development process
to trace the constant change of state of an ordinary city

\section{Thesis Aims and Scope}

\textbf{Overall research question:}

investigate socio-spatial patterns of  post-socialist cities in order to reinvent a more inclusive and flexible approach to understanding Urban development dynamics engaging all active agents in an urban context 

\textbf{Central hypothesis:}

An Urban development model, interpreted through MAS-ANT methodological approach, embodies interconnections and networks of all urban key agents initialized by layers of decision making and determines its level of urbanity.
Such relational object/structure is a transparent engine for capturing the dynamics of the process of urban development. 

\subsection{Research Questions}

\subsubsection{RQ1}

RQ1a: What are the conditions for specifying the level of urbanity in an ordinary city?

RQ1b: What are the conditions for specifying the level of urbanity in an ordinary city?

RQ1c: What constitutes spatial and social differences and specificity in an ordinary city?

\subsubsection{RQ2}

RQ2a: Why the morphology of influences among different decision-making levels determines pathways for upgrading the level of urbanity?

RQ2b: Why the morphology of urban decision-making determines pathways for urban change? 

\subsubsection{RQ3}

RQ3a: How to design a dynamic urban development model of an ordinary city in a constant state of change? 

RQ3b: How to frame urban development process in an ordinary city to embody its urban dynamics?

\subsection{Main Concepts}

\textbf{urban development} as a process

\textbf{decision making:}
Politics is open, but decisions become locked in. Governance is how the decisions are taken. (Hudson and Leftwich, 2014)

\textbf{agency}

individuals, organizations, coalitions

top down, bottom up, interest based (going through and across the structure) (Hudson and Leftwich, 2014)

\textbf{urbanity}

\subsection{Adopted Methodology}

\textbf{methodologies} - exploratory, correlational, embody relations in a transparent way, city as a system/organism
\\
Jer ANT dosta koriste sociolozi za analize gradova i urbanog, a sama MAS je vise matematicka-kompjuterska metoda. 

\section{Contribution}

Relate Contribution to Conclusions
Ideja mi je da se naprave vizuelne interpretacije koje lako mogu da se kompjuterizuju (html5) i onda lako menjaju i na osnovu toga stalno izvode zakljucci i uvode i opisuju novi elementi.
A usput bi proistekla i ta nova definicija urbaniteta.

\section{Thesis Structure}

The study is structured in seven chapters. ...

%%%%%%%%%%%%%%%%%%%%%%%%%%%%%%%%%%%%%%%%%%%%%%%%%%

\chapter{Literature Review}

%%%%%%%%%%%%%%%%%%%%%%%%%%%%%%%%%%%%%%%%%%%%%%%%%%

\section{Introduction}

This chapter outlines xxxx. ...

\section{Conceptual Framework}

\subsection{An Urban Development Process in an Ordinary City}

In many scientific studies, interest lies in

ordinary city

constant state of change: dynamics and the structure of complex systems

\subsubsection{Socio-spatial patterns of a post-socialist city}

In such a situation, urban planning was not a priority (Sykola, 1999), and it was not considered effective for managing local urban issues (Maier, 1998; M. Vujošević and Nedović-Budić, 2006). Therefore, planning was narrowed down to just one technical issue and very few theoretical or general methodological research studies bothered to examine alternative planning modes in transition, apart from replications of the approaches taken by neo-liberal or institutional economies (Begović, 1995).

\subsection{The Morphology of Decision Making}

xxxxxxxxxxxxxxxxxxx.

\subsection{The Constitution of Urban Agency}


\subsection{Level of urbanity}

Network of All Active Agents and Contextual Resources

the relation between urban life and urban form creates potential (Marcus)
level of urbanity broadens the opportunity for change (Marcus)

\section{Methodologies for understanding urban development complexity and its dynamics}

A General overview of Methodologies for xxx

\subsection{Actor Network Theory}

\paragraph{general - social sciences}

\paragraph{urban studies}

\subparagraph{urban development}

\subparagraph{decision making}

\subparagraph{urbanity}


\subsection{Multi-Agent System}

xxxxxxx

\section{Theoretical Framework}


%%%%%%%%%%%%%%%%%%%%%%%%%%%%%%%%%%%%%%%%%%%%%%%%%%

\chapter{Methodological Approach}

%%%%%%%%%%%%%%%%%%%%%%%%%%%%%%%%%%%%%%%%%%%%%%%%%%

\subsection{Research Framework}

xxx

\subsection{Context-specific Research Questions}


CoRQ1: How socio-spatial patterns of a post-socialist city influence its level of urbanity?

CoRQ2: What constitutes the urban knowledge and local capacity in a post-socialist city?

CoRQ3: Why an urban development model of a post-socialist city is an iterative procedure of
merging different levels of decision-making through a multi-layered network of connections among all urban key agents?

\section{Research Hypotheses}
\subsection{H1}

H1a: Specificities of socio-spatial patterns in a post-socialist city emerge from the blurred and askew interconnections and 
interrelations  of  different  levels  of  decision  making  (urban  planning  strategies,  tactical  urban  transformations,  and 
participatory urban design operations) and set the conditions for specifying the level of urbanity in a post-socialist city. 

H1b: Level of urbanity indicates an opportunity for change within socio-spatial patterns of an ordinary city which emerge from interconnections and interrelations  of  different  layers  of urban decision  making (top-down urban planning, real estate transformations, bottom-up participatory activities)

interconnections and interrelations  of  different  layers  of  decision  making (top-down urban planning, real estate transformations, bottom-up participatory activities) IS \textbf{Morphology of decision making}: it gives an exhaustive image of a city/urban environment 

\subsection{H2}

H2a:  Urban  development  in  a  post-socialist  city  is  determined  by  upgrading  the  level  of  urbanity.  Breaking  down  the 
morphology of influences among different decision-making levels through mapping a multi-layered network of interactions 
and interconnections among urban key agents (urban actors, built environment, space, regulatory framework, infrastructure, 
social practices) clarifies a dynamic urban development model in a post-socialist city.

H2b: Level of urbanity indicates an opportunity for change within socio-spatial patterns of an ordinary city.
Breaking  down  the morphology of urban
decision-making through mapping level of urbanity clarifies its urban development dynamics.

\subsection{H3}

H3a: A dynamic urban development model set as an iterative procedure of merging different levels of decision-making in a post-socialist  city  reinterprets  a  multi-layered  network  of  interactions  and  interconnections  among  urban  key  agents  by applying the Multi-agent system (MAS) and Actor-network theory (ANT) methodological approach.

H3b: A dynamic urban development model set as an iterative procedure of mapping the level of urbanity reinterprets a  multi-layered  morphology  of urban decision making
in terms of transformation, maintenance, and/or change of the system.

\section{Research Design}

procedure
independent variables
dependent variables
complementary measures

\subsection{Case study}

based on Qualitative methods slides (2.6 and 3.1):
explanatory - capture a process (theory testing)
descriptive - prepared and illustrated (ethnographic, describe the incidence/prevalence of the phenomena)
structure: linear analytic
phenomenological approach - contemporary social phenomenon (urban development) within its real-life context (post-socialist neighbourhood) - case about events (process)


\subsection{Case study selection}
\subsection{Local Context}

post-socialist city and transition
legacy from the past (path dependency, socialism) and prospects for the future (transition)

\section{Methods}

\subsection{Savamala Case study - Data collection} \label{sec:predis}

\subsection{MAS-ANT approach - Data analysis}

\subsubsection{Discourse analysis}
\subsubsection{Structural analysis}

An urban development model is based on:

Measuring the efficiency of urban planning

Testing the legitimacy of urban transformation interests

Recognizing the opportunities of bottom-up urban design initiatives


\subsubsection{Setting a procedure}

\section{Model Building - Data display}

Reporting the findings...

%%%%%%%%%%%%%%%%%%%%%%%%%%%%%%%%%%%%%%%%%%%%%%%%%%

\chapter{Case study}

%%%%%%%%%%%%%%%%%%%%%%%%%%%%%%%%%%%%%%%%%%%%%%%%%%

\section{Background}


\section{Stimulants and deterrents of decision-making tradition in Belgrade}

xxx. ...

\subsection{Post-socialist Urban Planning}

Top-down Management of Urban Conflicts

\subsection{Legitimacy of Interests in a Post-socialist City}

Tactical Urban Transformations 

\section{Dynamism of urban agency in Savamala}

xxx. ...

\subsection{Network of civic engagement}

Participatory Urban Design Operations


%%%%%%%%%%%%%%%%%%%%%%%%%%%%%%%%%%%%%%%%%%%%%%%%%%

\chapter{ANT Data Analysis}

\section{A forward-thinking overview of building an urban development model for Savamala}

xxx...

\section{Urban assemblage map: urban key agents and contextual resources}

\chapter{MAS Model Building}

%%%%%%%%%%%%%%%%%%%%%%%%%%%%%%%%%%%%%%%%%%%%%%%%%%

xxx. ...

\section{Body of urban relations: Urban Development dynamics}

xxx...

%%%%%%%%%%%%%%%%%%%%%%%%%%%%%%%%%%%%%%%%%%%%%%%%%%

\chapter{Conclusions}

%%%%%%%%%%%%%%%%%%%%%%%%%%%%%%%%%%%%%%%%%%%%%%%%%%

\section{Conclusions related to the research framework}

xxxx ...

\subsubsection{[..] to the research objectives}

xxxx

\subsection{[..] to the research questions}

xxxx

\subsection{[..] to the methodological approach}

xxxx

\section{Conclusions related to the theoretical framework}

xxx

\subsection{Urban Development Taxonomy}

xxx

\subsection{An Ordinary City}

xxx

\subsection{A Post-socialist City}

xxx

\subsection{Urbanity}

xxx

\section{Practical Implications}

xxx

\subsection{Urban Development model}

xxxx

\section{Limitations of the research}

xxx

\section{Future Prospects}

xxx


%%%%%%%%%%%%%%%%%%%%%%%%%%%%%%%%%%%%%%%%%%%%%%%%%%

\begin{small}
\addcontentsline{toc}{chapter}{Bibliography}
\bibliography{ThesisBib}
\end{small}

%%%%%%%%%%%%%%%%%%%%%%%%%%%%%%%%%%%%%%%%%%%%%%%%%%


\newpage
\appendix
\noappendicestocpagenum
\addappheadtotoc

\end{document}
