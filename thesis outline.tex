
\documentclass[11pt]{report}
\usepackage[a4paper,margin=2cm, bindingoffset=2cm]{geometry}
\usepackage{appendix}
\usepackage{amsmath}
\usepackage{booktabs}
\usepackage{threeparttable}
\usepackage{natbib}
\usepackage{changes}
\usepackage{color,soul}
\bibliographystyle{chicago}

%to stop orphan lines
\widowpenalty=10000
\clubpenalty=10000
\raggedbottom

%line spacing
\linespread{1.3}

\begin{document}
\begin{titlepage}
\begin{center}
\large
\textsc{EPFL} \\
\ \\
\textsc{CODEV and LASUR} \\
\ \\
\textsc{EDAR}\\
\ \\
\ \\
\ \\
\ \\
\ \\
\ \\
\ \\
\Huge
\textbf{Urban Development Model}
\ \\
\ \\
\large by
\ \\
\ \\
Marija Cvetinovic
\vfill
Thesis for the degree of Doctor of Philosophy \\
\ \\
\ \\
XXX 2017
\end{center}
\end{titlepage}
\begin{titlepage}
\begin{center}
\ \\
\ \\
\ \\
\ \\
\ \\
\ \\
\ \\
\ \\
\ \\
\ \\
\ \\
\ \\
XXXX
\end{center}
\end{titlepage}

%Roman Page Numbering
\pagenumbering{roman}
\chapter*{{Abstract}\markboth{Acknowledgements}{Acknowledgements}}
\addcontentsline{toc}{chapter}{Abstract}
xxxxxxxxxxxxxxxxxxxxxxxxxxxxxxxxxxxxxxxxxxxxxxxxxxxxxxxxxxxxxxxxxxxxxxxxxxxxxxxxxxxxxxxxxxxxxxxxxxxxxxxxxxxxxxxxxxxxxxxxxxxxxxxxxxxxxx
xxxxxxxxxxxxxxxxxxxxxxxxxxxxxxx.\\


\tableofcontents
\listoffigures
\addcontentsline{toc}{chapter}{List of Figures}
\listoftables
\addcontentsline{toc}{chapter}{List of Tabes}

\chapter*{Declaration of Authorship}
\addcontentsline{toc}{chapter}{Declaration Of Authorship}
I, xxxx, declare that the thesis entitled xxxxxxxxxxxxxxxxxxx and the work presented in the thesis are both my own, and have been generated by me as the result of my own original research. I confirm that:
\begin{itemize}
\item this work was done wholly or mainly while in candidature for a research degree at this University;
\item where any part of this thesis has previously been submitted for a degree or any other qualification at this University or any other institution, this has been clearly stated;
\item where I have consulted the published work of others, this is always clearly attributed;
\item where I have quoted from the work of others, the source is always given. With the exception of such quotations, this thesis is entirely my own work;
\item I have acknowledged all main sources of help;
\item where the thesis is based on work done by myself jointly with others, I have made clear exactly what was done by others and what I have contributed myself;
\end{itemize}
\vspace{2cm}
Signed: \dotfill
\vspace{2cm}
\newline
\noindent
Date

%Include Acknowledgements in TOC
\chapter*{Acknowledgements}
\addcontentsline{toc}{chapter}{Acknowledgements}
This work was undertaken with financial support of the 

This thesis would not have been possible without the support of many people. I would like to express my sincere gratitude to:
\begin{itemize}
\item 
\item ...
\end{itemize}



%%%%%%%%%%%%%%%%%%%%%%%%%%%%%%%%%%%%%%%%%%%%%%%%%%

\chapter{Introduction}
%Arabic Page Numbering
\pagenumbering{arabic}

%%%%%%%%%%%%%%%%%%%%%%%%%%%%%%%%%%%%%%%%
Urban development is a widespread archetype for out-of-reach improvement in cities of the Global South. However, in its essence it is more a kind of constant catch up with the West and western urban paradigm than an elaborated form of intrinsic local perception, knowledge and action toward urban transitions.
\\
In this highly competitive international arena, transitional countries experience grave consequences due to the paucity of practical experience within the dominant/ruling western ideology of the urban. They are caught in this new context of relentless rules of market economy, decentralized political and administrative powers, lack of resources, scarcity of general international investment and scant interest for dramatic shifts in all aspects of their social organization and spatial transformations. The blurred and askew morphology of post-socialist cities in transitional countries is therefore the result of continuous pressure from the negative side effects of imitating and lagging behind conventional urbanization models and accelerating globalization patterns imported or imposed by the Global North, or colloquially known as the West.
\\
The urban transformation of Serbian cities falls into this cliché of the new post-socialist urban reality, which emerged during the “transition to markets and democracy” (Tsenkova, 2006). The dismantling of the communist system during the late 1980s represented a substantial change in all aspects of social organization, the economic model and the political system. However, Serbia is still identified as a post-socialist melting pot where representative democracy, civil society and market economy principles collide and merge with authoritarianism, vertical decision making and populism practices. In such a situation, concern about the urban has been left out and given over as a battlefield for social needs in practice and technical solutions on paper and an easy prey to the exercise of power and interest. Therefore, the practice of planning and designing Serbian cities has been narrowed down to a mere technical issue most often even without an actual or adequate realization in practice. Not to mention that very few theoretical or general methodological research studies bothered to examine alternative planning modes, techniques and instruments in transition, but continued with the manner of replication from well-known counterparts of the Global North.
\\
Among others, architects have a vital role in not only directing but also framing the path of urban formation and development in post-socialist cities. Even more so, as they are primarily focused on practice and "savoir faire" about making the built environment, while acquisition of land and illegal construction are spatial interventions that have marked post-socialist production of space more than any planning or theoretical activity. Though we have different drivers on the global scale and in developed countries, there is a global trend of resorting to sociological, planning or even IC approach in scientific studies on the urban. 
In order for architectural intervention in space to compete for more relevancy and rigour,  architects all over the world have been gradually grown interest for scientific discourse on built structures, spaces and cities in general.  In the rivalry between spatial and social basis for their interpretations, the fact that the field of architectural research is not yet standardized in terms of methodologies and techniques opens the floor for experiments and innovations \hl{(ref)}. In the circumstances of developmental bouillon or "developmental schizophrenia"(\hl{Vujosevic}) at the local level and an overall urge for architectural research framework internationally, my aim is to elaborate an architectural standpoint when applied on complex post-socialist urban reality in order to establish a methodological approach suitable for architectural scientific discourse on the matter.
\\
In my striving to contribute to post-socialist architectural research, the far-reaching aim is to capture post-socialist urban dynamics  in order to skip the classical procedure of urban development based on western planning paradigm and provide its practical application on multiple levels of urban decision making. This to be achieved requires supple methodological approaches which should better correspond to post-socialist socio-spatial patterns on multiple levels (state, city, municipality, community, and neighbourhood) and explain the correlations of various urban elements. Practice-oriented, locally focused and globally tuned  approach to complex urban reality of post-socialist cities envisions embracing the dynamics of urban systems and operationality of architectural performance for circumscribing visual interpretations that enable continuous conclusion drawing and up-to-date introduction of any new element that may appear in the system.

\section{Field of Study}
Due to growing social and physical transformations that become ever more intensified as current globalization continues to spread out profit maximization, consumption patterns and information networks (Harvey 2012), the cities have been experiencing a progressive reorganization at spatial and social levels. Even though accelerating urbanisation is a worldwide process, it still assumes different forms and meanings, depending on the prevailing local conditions (Bolay, 2007). These overall circumstances of continuous urban development influence cities to serve as the primary channel linking local realities to global social, political and economic forces (Yates and Cheng, 2002; Tsenkova, 2006).
\\
Cities are not simply market products and consumption patterns, but locally customized socio-political constructs as well (Marcuse et al. 2008). These external influences form a range of qualitatively different contextual circumstances for positive urban change, i.e. urban development. Most settlements and cities from previous historical periods had reflected upon the various degrees of forethought and conscious design in their layout and function, which is referred to as a fixedly planned development, albeit many had tended to develop organically. Generally speaking, a range of urban disciplines (urban planning, theory, sociology, legislation and design) aim to decode and harmonize growing urban issues as a side-effect of the current globalization, urbanisation processes and spread of \hl{capitalism} that are mainly affecting cities and production of urban space and bid for the expertise on managing urban development (Allmendinger, 2009; Faludi, 1973). 
\\
In practice, these disciplines are embedded in a particular social context or a territorially based system of socio-economic relations, and they react to the shifts in socio-economic and political settings (Tsenkova, 2006b), but have kept privileged relationship toward Western cities, as assumed to be the sources of urban creativity, vitality and innovativeness in urban domain (\hl{Robinson, 2006:2}). Accordingly, they tend to fail substantially within the range of spatially and economically different environments that have undergone highly dramatic change in political, economic and social terms. For example, urban research and practice in transitional countries in Central and Eastern Europe (CEE) should unfold to help understand these phenomena in their immediate and wider context, identify patterns of the dynamic reality in these cities and be more consistent with spontaneous, everyday urban transitions. Accordingly, a corresponding change in approaching urban development can then be addressed by heterogeneous, iterative and generative process of urban space production in physical and social sense that has surpassed the perception of cities as merely economic, social and cultural venues treating them as complex and dynamic urban systems. In these circumstances it is necessary to apply proper techniques and methodologies for urban research and analyses which encompass complexity and dynamics of cities for the improvement of their living conditions and the facilitation of social interactions in the process of urban development.
\\
However, each discipline keeps its own track and pace in approaching urban matters. My architectural background has moulded my own research interest towards gaining knowledge and understanding on management of space and built environment. Moreover, production of space is also the core concern for architects. Architecture is a discipline focused on practice and consequently it urges for parameters, categories and structure for its practice-based analyses. Hitherto history and theory of architecture have been the main fields of architectural research. But the ever growing relevance of architecture for transforming the general body of knowledge on cities into a real-life problem-solving strategy that address human lifestyles, social relations and the concept of space \hl{other ref than initial Castells 2000, Dijk 2002 that is addressing ICT}, insists on the advancement of architecture in terms of the relevance and reliability of the knowledge herein produced (\hl{ref}). However, missing links with the classical scientific discourse has caused a growing concern for what is research appropriate for architectural and design practice as well as for architectural stance in urban studies, especially in terms of methodologies, methods, approaches, domain and credibility (\hl{ref+Savic 2016b}). Lacking the traditional scope of analysis, architectural research has been a polygon for innovations and experiments.
\\
In terms of methods, there has been a significant number of interdisciplinary, transdisciplinary and multidisciplinary endeavours in applied research with an architectural focus in urbanism (ref). What is more, applied fields of research acknowledge the use of methodological hybrids (Datta, 1994, De Lisle 2011). This has open doors for applied social sciences to investigate new methodological opportunities when confronted with complex and multiplex social phenomena (De Lisle 2011). Even more so as methodological and epistemological rigidity leads to ignoring the realities of the practical and cause catastrophic scientific failure of practice-oriented research (Rogers 2008, De Lisle 2011).
\\
What I have recognized as a crucial change in the methodological paradigm of an applicable urban research from the architectural standpoint can then be circumscribed to the rise of the global concept from static to iterative and dynamic. Commonly speaking a static world is one in which all transitions are according to a known law and which do not give rise to uncertainty. When defining the evolution of analyzing and simulating an urban phenomenon or process, it is fundamental to state that the existence of a problem depends on the future being different from the past, while the paradigmatic possibility of finding the solution to the problem depends on the future being like the past. Therefore, a transition in some sense is a condition of the existence of any problem. The complex empirical realities of urban transitions collide therein with the powerful and dominant policy of continuous comprehensive production of knowledge. A scientific approach towards formulating the dynamics of urban transitions have to count on uncertainty as one of its fundamental facts and in this way accept and deal with an open-ended future and the limits of human knowledge about it.
\\
Gaining knowledge has come to be a strategic activity rather than a search for truth (Kirby 2013), so that science becomes incapable of controlling society and the rationalized reality appears false and irrelevant (Alfasi et al., 2004). Given these conditions, the growing gap between the formal structure and the dynamics that take place in cities triggers an internal and independent process by which the system tends to spontaneously self-organize (Portugali 2011). Therefore, a city should be conceived as an organism, not a mechanism \hl{Charles Laundry, The Creativity City}. In these terms, the city is interpreted as a living system which is constantly mutating and emitting new elements, a container for processes of coming to be, breaking up and falling out, fragmenting and recomposing. Contemporary cities tend to be concentrations of multiple socio-spatial circuits, diverse cultural hybrids, and sources of economic dynamism - a venue where the past and the present converge upon one another. The city tells a story of one society and its attempts to move towards a positive vision of the future, through complex ranges of processes that flow together to construct a single consistent, coherent, albeit uncertain, interactive and multifaceted time-space system (Graham, 1998). These ceaseless processes are the core of spontaneous, everyday of urban development. Grasping the scope of urban development occurs as a major challenge for modern science about cities.
\\
Within the framework of this research, my intention is not to produce another pattern applicable to certain cities to a certain extent, but rather to apprehend a process that embodies, in a transparent way, the complexity and dynamics of the mentioned relations, to ponder upon means of generating a vibrant and fluid context open to permanent transformations and most importantly also to grasp the idea of an adjusted and balanced model adaptive to changing views and situations on progressive/accelerating urban development. (Portugali Complexity cognition and the city). This to be achieved requires supple approaches which should aim at explaining the correlations of various urban elements and to better correspond to the socio-spatial patterns of the range of urban environments. In such a plenitude of factors, we choose case study as an adequate research method and a neighbourhood as a relevant level of analysis.
\\
Dynamic urban context is a complex phenomenon with a plenitude of data. Case study research method enables close, in-depth and holistic examination of a great deal of data, but requires a bounded environment in order to accurately describe and illustrate such a context and to use it for broader interpretations and demystification of modern cities. Specifying physical limits is not in itself enough for circumscribing the identified complexity of urban transitions, the issue of scale is also at stake. In urban terms, different spatial and social elements are intensified or muted at different levels (global, national, regional, local). In order to acquire active follow-up, interpretation and assessment of urban issues, it is important to define a representative environment, a robust source of prominent urban "processes". Thereupon, we argue for a neighbourhood level of analysis , as an urban micro environment, which may become a paradigm for complexity and dynamics of modern urban context and eventually increases the body of knowledge on cities concerning the methodologies used to deal with urban development and corresponding urban transitions.
\\
\textbf{"The contemporary city is a variegated and multiplex entity - a juxtaposition of contradictions and diversities, the theatre of life itself" (Amin and Graham, 1997).}

\subsection{Background}
At the beginning of the 21st century, the world experienced a progressive reorganization at an economic, political and social level: profit maximization, globalization of urban processes and the devastating history of deindustrialization (Harvey 2012) and dematerialization of the world. Nowadays, while about 50 percent of the world population lives in urban environments (United Nations 2008), the question of techniques and methodologies for urban development research and analyses should undoubtedly address these major shifts in urban life and contemporary cities (Healey 1997). Cities are rather primary venues, power poles and capacity builders of economic, social and cultural development at stake in modern societies (Castells, 1998). Conversely, cities are dynamic and diverse urban entities that are given to shaping their autonomous and innovative future on the basis of human resources and creative human potential (Knight and Gappert, 1989; Yigitcanlar, 2008). The prosperity of cities depends on how competitive they are on a global economic scale, how flexible they are in terms of adjusting to current trends and needs, and how fertile they are for the development of knowledge and the application of innovation. These major uncertainties of contemporary life, created mainly but not exclusively by the current method of production and management, are acutely symbolized by concerns about urban development (Healey 1997).
\\
Urban development is widely accepted even though also contested category usually associated with urbanisation processes in "so called" developing countries. Lots of professionals in urban research and practice use the term, not to mention the great number of people around the world affected by their work. Nevertheless, the notion of the word "development" itself means different things to most of them.
However, traditional and widespread interpretation comply with the western paradigm of development: modernisation and economic growth \hl{ref}. Interpreted in this way, the notion of urban development actually promotes the leading hierarchies and categorization of cities in the world, based on the impact of globalization, new transnational  economic  progress  and  networking  of  cities \hl{ref}. Both of these approaches impose the hierarchical relation among them, as Jennifer Robinson (2006) bluntly puts it, "while some are exemplars and others are imitators". Besides, the chronological paradigm of western urban planning dilutes when it is spatially translated to these qualitatively different environments (Robinson 2002), causing them to lose their substance as an urban phenomenon through the ill-decoded application of western patterns (Bolay 2004). In addition, modern urban thought could be stuck in this rut, inducing negative background effects on the whole gamut of urban activities (Amin et al. 1997) and causing urban conflicts to thrive on the basis of inequitable power relationships, and cultural differences, as they develop from an individual level towards a socio-urban dimension (UN Habitat 2009).
\\
In  this  respect,  Jennifer  Robinson  summarizes  that  categorization  and  differentiation  of  cities, according to Modernity and Development, are a pure product of colonial past. This actually means that in the scope of widely praised universal image of "cityness" as the final goal of the ambition of cities, successful examples of cities are included inside these categories. \underline{World cities} are defined in relation to their regional, national and international influence inside a global economy where the country categorization is transferred into this world city categorization \hl{ref}. Conversely, \underline{global cities} are categorized according to their industrial potential of transnational management and control \hl{ref}. They both focus on the characteristics of cities and potentials in the scope of global economy, its flows and networks, which has proved to be insufficient, exclusive and restrictive for cities in less developed countries, if we keep up with the same terminology at the national level. On the other hand, cities which are outside these categorizations but with a same ambition and vision of "cityness" are regarded as \underline{third World} or \underline{developing cities}. Consequently, there are even more categorizations such as those of western, wealthy, third world, developed and developing cities. However, with all of them together, there is still a vast number of cities which are left out and with barely any possibility to ever fit in any of the categories (Robinson, 2006). 
\\
\textbf{"Ordinary cities also emerge from a post-colonial critique of urban studies and signal a new era for urban studies research characterised by a more cosmopolitan approach to uderstanding cityness and city futures. This can underpin a field of study that encompasses all cities and that distributes the difference amongst cities as diversity rather than as hierarchical categories. It is the ordinary city, then, that comes into view within a postcolonialised urban studies" (Robinson, 2006)}.
\\
This brilliant insight put forward by Jennifer Robinson in her book "Ordinary cities between modernity and development" questions the geopolitics of urban theory and urban development (Fraser 2006). Taken from this standpoint, each and every city is an indicator of what an urbanized society is and what course of urban development it may take. My research scope in this thesis perseveres with post-colonial critique of urban studies and the notion of ordinary cities, introduced by Amin and Graham (1997) and further developed by Robinson (2002). This concept approaches the knowledge of diversity and complexity that exists within the world and "distributes the differences amongst cities as diversity rather than as hierarchical category" Robinson (2002). 
\\
Ordinary cities approach provides unique assemblage of internally different, distinctive and \added{context-based} urban transitions as well as overlapping space-, time- and relation- networks across cities. In other words, it is not only necessary to examine the ways in which countries/cities interface with the global economy, but also social, cultural and historical legacies that each country/city carries into the era of globalization. Within such domain for explanations, this thesis revolves around the interpretation of urban development as an answer to the question "how can cities facilitate urban transitions while also maintaining the culture and values of the community itself?" \hl{(ref article: Does Placemaking Cause Gentrification? It’s Complicated.)} The idea of indicating what encompasses urban development of an ordinary city lead to identifying its internal and external influences \added{that constitute the core of maintenance, transformation or change processes in an urban system of a city}, when treated equally within the global hierarchy of cities (Robinson 2002). 
\\
This approach makes a worldwide, broad, general and mutable process of urban development actually connected to place - making an actual urban setting a vital factor for case specific uncertainties and a polygon for transformation of global aspects to meet local specifications. My aim is to move away from the general theoretical research into an on-site practice-based investigation. Consequently, this research project attempts to show how the real-life focus on Savamala neighbourhood in Belgrade eventually increases the body of knowledge on post-socialist urban environment and the methods used to deal with complex and dynamic urban context. Complying with "ordinary cities" approach, I would like to elaborate that post-socialist cities in transitional countries meet extraordinary difficulties when copying urban models from the West.  The cause is found in the lack of the institutional infrastructure and cultural patterns essential for the functional unity present in western cities (Petrovic 2009). Furthermore, fundamentality and intensity of economic and political change in Balkan post-socialist countries may be a historic exemplary of social transition hard to find in a "typical" capitalist city (Sykora 1994). Its internal environment is in a state of flux, with the rapid adjustment of the physical, economic, social, and political structures of the city itself (Sykola, 1999).
\\
Included in this range of spatially and economically turbulent surroundings, post-socialist cities in transitional countries have undergone highly dramatic change in political, economic and social terms. The mayor consequences of such \deleted{post-socialist} transition introduce, on one hand, the disastrous effects of increasing social polarization (inequity), deinstitutionalization of socio-spatial practices (informality) and unfair wealth redistribution (poverty), and, on the other, the huge socio-cultural base inherited from the socialist period with centralised and authoritarian practices in urban governance. This has had a profound influence on the spatial adaptation and social repositioning of post-socialist cities.
\\
Turbulent social times, such as the disintegration of Yugoslavia’s political system and the introduction of new context of market economy, decentralized administrative powers and a lack of investment and resources are reflected in chaotic urban development pattern. Urban systems of post-socialist cities are highly susceptible to tense on-going transformations, diverse but reciprocal in their nature: economic transformations (transformation of production and consumption in relation to space, income polarization and poverty), political transformations (urban governance, political voluntarism, participation and decentralization), spatial transformations (demographic trend and distribution of functions) and social transformations (social exclusion-inclusion, social activism and informality). In other words, what proceeded after the end of the socialist era is a neoliberal model of urban planning with the supremacy of market-oriented solutions for urban problems (Sager 2011). Conversely, with the huge socio-cultural base inherited from the socialist period, cities in transitional countries have continued to be centres of economic growth with a variety of services, expansion, technological innovation and cultural diversity. While some trends and directions within these transformations are clear and defined, uncertainty dominates decision making and implementation in the turbulent environment of post-socialist cities (Nedovic-Budic 2001). Therefore, the post-socialist period in these cities contains prevailing characteristics of the disintegration of the preceding system rather than a coherent vision of what should follow. 
\\
In practice these conditions ended by having the strategic plan as an advisory long-term urban vision, but leaving the real actions and decision making to political and market forces. Thenceforth, urban development of post-socialist cities most often has exceeded and diluted the common strategic framework defined from top-down: to establish clear links between the process of strategy development, its institutional framework, the hierarchical structure of long-term and short-term objectives of all actors involved, and the real-time changes happening simultaneously in an urban environment. The major characteristics of post-socialist urban development are: a multitude of actors, various economic, social and political interests, social aspects and fragmentation of urban spaces. Consequently, post-socialist cities lack complex, operational logistics \hl{check if appropriate(Repetti et al. 2010)} to link top-down changes to bottom-up interventions in urban systems. There exists an growing discrepancy between the national and global levels, on one side, and city and neighbourhood levels, on the other. 
\\
The  conceptual  framework  explained  herein  pinpoints  the  blurred  and  askew  morphology of  post-socialist  cities which  requires  dynamic  solutions  in  order  to  skip  the  classical, western procedure  of  urban  formation  and development. Consequently, this particular context shows the increased need for proper techniques that are spatially and temporally adjusted to current issues. The far-reaching goal actually is to transform the negative side effects of imitating and lagging behind the western urbanization model and those of the accelerating globalization into a development impetus suited to these environments. Urban development of post-socialist cities is perceived as a dynamic concept, a multi-dimensional integrated system composed of qualitatively different and semi-autonomous processes, with the inclining tendency to improve the economic, social, demographic, political and technological state of an urban environment. In view of all this, we need an overarching theory of urban development that can encompass all discrepant decision making forces: future-oriented urban projections (urban planning strategies), in situ transformation forces and potentials (urban transformations), and follow the creative paths of urban dwellers (participatory urban design activities) for imagining new urban futures. The question of facilitating and localizing urban transitions rests with overlapping urban scenarios from dissonant levels of decision making, tracking cultural identities, requirements and needs of all urban actors, and, in general, indicating contextual processes of maintenance, transformation and change of an urban system. 

\subsection{Problem statement}
The focus of this thesis is urban dynamics of post-socialist cities. The issue is not addressed as a problem to solve, but rather as a moving target for an exploratory observation of the way how cities function and how various urban transitions condition urban development of post-socialist cities.
\\
Post-socialist cities are treated herein as a range of qualitatively distinctive cities that "deal differently with their difference" \hl{ref}. In their incompleteness, plurality and informality, post-socialist cities in transitional countries represent dynamic and diverse arenas of contemporary urban life, experience and theory. Included in this range of spatially and economically developing surroundings, transitional countries in Central and Eastern Europe (CEE) have undergone highly dramatic change in political, economic and social terms. The disintegration of Yugoslavia’s socialist system led to the destabilization of the institutions and the social value system in Serbia. Such confusing political and social circumstances have deprived an average citizen of sufficient information about the possibilities and tools to take an active part in the development of their city. These factors provoked a legal void susceptible to shady deals and questionable public-private partnership (illegality), a lack of strategically proactive urban governance which resulted in tolerance to illegal building practices (informality), and the increasing social polarization (inequity) and poverty in this region {the number of poor people had reached 100 million in CEE by 2001 (Tsenkova 2006a)}. This rather organic path of urban development leads to the classifying of post-socialist cities in transitional countries as unregulated capitalist cities (investment-led) with third world urban development elements (substantial illegal activities and informal markets) (Petrovic 2009). Conditioned by the geographic location of Serbia (CEE), murky circumstances of transition towards \hl{liberal market and ????} are followed by a set of decentralization and democratization protocols for joining EU, availability of European research and civil sector funds as well as the promotion of participation and engagement from the ground up \hl{ref}. Having said that, the lack of successful urban planning models and actions make possible that the rising economy of social exchange and local capacity building could contribute to an improvement of life and functionality of urban structures and systems and effectively address the tensions between top-down and bottom-up urban planning in a post-socialist city.
\\
\hl{specific contextual circumstances- non-human agents}
\\
Conversely, under the hood of scientific neutrality, urban development concept is critically approached, broken down and recomposed as a process of urban transitions, not as an indicator or the final product in urban practice. Urban development of post-socialist cities is seen as a complex, multifaceted network of urban transitions that evidence: 
\begin{enumerate}
\item the level of urbanity - qualitative processes of maintenance, transformation and changing processes of an urban system;
\item legitimacy of different layers of urban decision making top-down urban planning strategies, tactical urban transformations, and bottom-up participatory activities;
\item urban key agents - constitutive elements for the morphology of urban decision making;
\item numerous urban conflicts, social practices and contextual resources resulted from the incompleteness, plurality and informality of post-socialist cities.
\end{enumerate}  

\section{Thesis Aims and Scope}
\textbf{research scope:} grasp the actual urban development process in cities
\\
\textbf{research impact:} the quality of an urban system generates a vibrant and fluid context open to permanent transitions gives rise to potential to originate diverse opportunities for new rounds of exchanges among research, innovation, action and development (Bolay et al. 2011).

\subsection{Research Objectives}
\textbf{Overall objective:} encompassing complexity and dynamics of urban transitions at the local level in a rather transparent way
\\
\deleted{identify variables in objectives}

\subsubsection{RO1}
re-formulate urban development concept in terms of urban transitions to fit the idea of dynamic state of an ordinary city in its full complexity
\begin{itemize}
\item RO1a: identify urban system complexity and sort out active \added{urban key} agents \deleted{and contextual resources} \deleted{and map their interconnections and networks} 
\item RO1b: map how socio-spatial patterns of an ordinary city are constituted in urban networks
\item RO1c: trace the morphology of decision making 
\item RO1d: define dynamic state of a complex urban system in an ordinary city - description of empirical reality of urban networks and processes
\end{itemize}

\subsubsection{RO2}
gain an in-depth understanding of the level of urbanity in an ordinary city as an indicator of urban transitions
\begin{itemize}
\item RO2a: connect the level of urbanity to socio-spatial patterns in an ordinary city
\item RO2b: elaborate and encode socio-spatial patterns (spatial, social and technical differences and specificity) of post-socialist context - local urban conflicts, social practices and contextual resources
\item RO2c: connect urban transition processes to socio-spatial patterns
\item RO2d: contextualize urban transitions in post-socialist cities 
\end{itemize}

\subsubsection{RO3}
conceptualize a methodological hybrid for tracing \added{urban complexity and dynamics}
\begin{itemize}
\item RO3a: specify a neighbourhood level of analysis 
\item RO3b: describe urban complexity - networks - to indicate the morphology of urban decision making
\item RO3c: trace urban processes - the level of urbanity - to indicate dynamic socio-spatial patterns
\item RO3d: proceduralize urban transitions for circumscribing urban development model
\end{itemize}

\subsection{Research Questions}
\textbf{Overall research question:} HOW To investigate socio-spatial patterns of  post-socialist cities in order to reinvent a more inclusive and flexible approach to understanding Urban development dynamics engaging the complexity of an urban context? 

\subsubsection{RQ1}

RQ1a: What are the conditions for specifying the level of urbanity in an ordinary city?

RQ1b: What are the conditions for specifying the level of urbanity in an ordinary city?

RQ1c: What constitutes spatial and social differences and specificity in an ordinary city?

\subsubsection{RQ2}

RQ2a: Why the morphology of influences among different decision-making levels determines pathways for upgrading the level of urbanity?

RQ2b: Why the morphology of urban decision-making determines pathways for urban change? 

\subsubsection{RQ3}

RQ3a: How to design a dynamic urban development model of an ordinary city in a constant state of change? 

RQ3b: How to frame urban development process in an ordinary city to embody its urban dynamics?

\subsection{Main Concepts}
\textbf{urban development} as a process
While the urban system comprises the dynamics of  interaction  and  interconnections among people (urban actors and stakeholders), objects (built environment), territories (space), institutions (regulatory framework), infrastructure and social  aspects (political, economic and cultural circumstances) (Firmino et al., 2008) correlated through the morphology of urban decision making. The issue at stake is to encompasses planning, interests, design and participation in an overarching urban decision making procedure that comprises and reconciles all its different layers: top-down urban planning strategies, tactical urban transformations, and bottom-up participatory activities recognized on site.  
\\
In general, the main research challenge of this thesis is testing the legitimacy of urban decision making for urban development.  
The conceptual framework explained herein examines the urban key agents
and numerous urban conflicts, \added{social practices} and contextual resources
in their incompleteness, plurality and informality
\\
in Savamala, a historical but deteriorating city quarter in Belgrade, where a set of bottom-up urban transformations and participatory spatial interventions intent in this respect to make these environments and swift, investor-based developments
\textbf{decision making:}
Politics is open, but decisions become locked in. Governance is how the decisions are taken. (Hudson and Leftwich, 2014)

\textbf{agency}

individuals, organizations, coalitions

top down, bottom up, interest based (going through and across the structure) (Hudson and Leftwich, 2014)

\textbf{urbanity}
Therefore, urban development could bend the way how each of them address urbanity as its constitutive reality and its ultimate positive goal.
Takodje sam se zbula jer mi se ucinilo da moram ovu kombinaciju metoda da ispratim nekim teoretskim okvirom, pa sam uvela pojam urbaniteta, jer mi nivo urbaniteta obuhvata i stanje i promenu. A u celom doktoratu sam mislila da obradim kako onda kombinacijom urbaniteta (opis stanja i agenata promena) i odlucivanja (sortiranje agenata i promena prema slojevima: planiranje, investicione transformacije, participativne aktivnosti) prikazujemo dinamiku urbanog razvoja, koristeci  ANT pa MAS. 
Distinctive category familiar but not exclusive to architectural analysis is “urbanity”. The relationship between the physicality of urban form and the social components of urban life generates the level of urbanity - the quality of continuous harmonization of the variety of structural elements, social factors and vested interests existing in an urban environment (Holanda 2002, Canuto et al. 2012). Moreover, all these urban key elements are assumed to be equal agents in the continuous process of urban development that has been marked by maintenance, transformation and change of the urban system in order to improve its living conditions and facilitate social interactions.
\added{Due to these circumstances, the urban development of post-socialist cities is perceived as a multi-dimensional integrated system composed of qualitatively different and semi-autonomous processes, with the inclining tendency to improve the economic, social, demographic, political and technological state of an urban environment.}

\subsection{Adopted Methodology}

\textbf{methodologies} - exploratory, correlational
\\
Jer ANT dosta koriste sociolozi za analize gradova i urbanog, a sama MAS je vise matematicka-kompjuterska metoda. 
according to Kuhn's paradigm shift (1962) science about the city is constantly swinging as a pendulum between scientific and hermeneutics/humanities approach - quantitative analysis vs. descriptive study \hl{Portugali Complexity Cognition and the City}

\section{Contribution}

Relate Contribution to Conclusions
Ideja mi je da se naprave vizuelne interpretacije koje lako mogu da se kompjuterizuju (html5) i onda lako menjaju i na osnovu toga stalno izvode zakljucci i uvode i opisuju novi elementi.
A usput bi proistekla i ta nova definicija urbaniteta.
This proposal aims to define a method of solving concrete problems through a process of understanding and dealing with current difficulties as they emerge and evolve.

\section{Thesis Structure}

The study is structured in seven chapters. ...

%%%%%%%%%%%%%%%%%%%%%%%%%%%%%%%%%%%%%%%%%%%%%%%%%%

\chapter{Literature Review}

%%%%%%%%%%%%%%%%%%%%%%%%%%%%%%%%%%%%%%%%%%%%%%%%%%

\section{Introduction}

This chapter outlines xxxx. ...

\section{Conceptual Framework}

\subsection{An Urban Development Process in an Ordinary City}
The concept of progress is central to modern society and it is orientated towards a positive vision of the future. In an urban scope, this concept corresponds to that of risk, where the control of all future events is calculable and predictable in probabilistic terms. This new concept of urban planning is based on the notion of an open-ended future, which implies that uncertainty must be accepted and managed, authorities and actual urban actors should be ready for new requirements and renewability as conditions change, and professionals are to increase their knowledge of risk and vulnerability in urban environments. In this sense, the planning of discourse relates to a master narrative of modernity, including ideas of rationality, objectivity, scientific evidence, values and possible control through normativity.
\\
There are different and numerous interpretations of what is and should be urban development, such as \hl{(Evropski regionalizam 1, World Bank and find other references)}:
\begin{itemize}
\item the course of a culture
\item meeting the needs of human and natural worlds
\item economic growth
\item "right to development" between the developed and developing
\item modernization
\item emphasize raising the living standards by addressing issues of health and safety, inclusion and equity
\end{itemize}  
Robinson (2006) clarifies   her  cosmopolitan  tactics  for  surpassing  hierarchical categorization of cities in the world, which in terms modernity and development of kept them apart. Her main point is to avoid the hegemony of western urban theory, as  well as the growing  strength of a discourse of development, which from 1970s onward has been emphasizing the differences between cities in the west and elsewhere, by: 
\begin{enumerate}
\item Dislocating accounts of ―urban modernity from the big cities of the west which claims to be its 
originator
\item Tracking  and  gather  differences  within  the  world  of  cities  in  order  to  justify  that  people  in different  places  have  invented  new  ways  of  urban  life  and  their  particular  production  and circulation of novelty, innovation and new fashions
\item Enriching  all  cities  with  better  future  perspective  depending  on  their  distinctiveness  and creative potential, without any hierarchical order among ordinary cities; they are all equal as 
sites of the production and circulation of modernity.
\end{enumerate}
In many scientific studies, interest lies in ordinary city constant state of change: dynamics and the structure of complex systems.

\subsubsection{Socio-spatial patterns of a post-socialist city}
In such a situation, urban planning was not a priority (Sykola, 1999), and it was not considered effective for managing local urban issues (Maier, 1998; M. Vujošević and Nedović-Budić, 2006). Therefore, planning was narrowed down to just one technical issue and very few theoretical or general methodological research studies bothered to examine alternative planning modes in transition, apart from replications of the approaches taken by neo-liberal or institutional economies (Begović, 1995).
\\
Thus, the crucial failures of post-socialist urban planning have come about through the lack of consensus on priority goals, action-oriented programs of implementation and coordination of different levels, sectors and areas. 
\\
In transitional countries, the course of merging socialist and neoliberal socio-economic condition, regulatory practices and organizational solutions led to inefficiently operationalized and inconsistently formalized institutional reforms rather known as "growth without development" \hl{Vujosevic}.
The post-socialist urban governance fails substantially through the lack of consensus on priority goals, action-oriented implementation and horizontal and vertical coordination.

\subsection{The Morphology of Decision Making}
On the contrary, according to one of the leading urban theories of David Harvey and Manuel Castells, urban planning cannot be seen as an autonomous process of spatial development, but rather it is situated in its political economic context and constantly overlaps current economic and social changes  (Taylor, 2006). In other words, urban planning in practice is intrinsically connected to the property market (which in turn involves a particular political ideology) and this tends to maintain current social order (Dear and Scott, 1981; Taylor, 2006), both of which are grounded in the development and expansion of industrial capitalism, neo-liberalism and consumerism (Ellin, 1999; Harvey, 1989). In other words, urban areas are, and have always been, the spatial and symbolic manifestations of broader social forces (Giddens, 1992).

\subsection{The Constitution of Urban Agency}


\subsection{Level of urbanity}

Network of All Active Agents and Contextual Resources

the relation between urban life and urban form creates potential (Marcus)
level of urbanity broadens the opportunity for change (Marcus)

\section{Methodologies for understanding urban development complexity and its dynamics}

A General overview of Methodologies for xxx

\subsection{Actor Network Theory}

\paragraph{general - social sciences}

\paragraph{urban studies}

\subparagraph{urban development}

\subparagraph{decision making}

\subparagraph{urbanity}


\subsection{Multi-Agent System}

xxxxxxx

\section{Theoretical Framework}


%%%%%%%%%%%%%%%%%%%%%%%%%%%%%%%%%%%%%%%%%%%%%%%%%%

\chapter{Methodological Approach}

%%%%%%%%%%%%%%%%%%%%%%%%%%%%%%%%%%%%%%%%%%%%%%%%%%

\subsection{Research Framework}

xxx

\subsection{Context-specific Research Questions}


CoRQ1: How socio-spatial patterns of a post-socialist city influence its level of urbanity?

CoRQ2: What constitutes the urban knowledge and local capacity in a post-socialist city?

CoRQ3: Why an urban development model of a post-socialist city is an iterative procedure of
merging different levels of decision-making through a multi-layered network of connections among all urban key agents?

\section{Research Hypotheses}
\textbf{Central hypothesis:} An Urban development model, interpreted through MAS-ANT methodological approach, embodies networks of urban key agents initialized by the morphology of urban decision making and determines its level of urbanity.
Such relational object/structure is a transparent engine for capturing the complexity and dynamics of \hl{urban transitions (urban development processes)}. 
\subsection{H1}

H1a: Specificities of socio-spatial patterns in a post-socialist city emerge from the blurred and askew interconnections and 
interrelations  of  different  levels  of  decision  making  (urban  planning  strategies,  tactical  urban  transformations,  and 
participatory urban design operations) and set the conditions for specifying the level of urbanity in a post-socialist city. 

H1b: Level of urbanity indicates an opportunity for change within socio-spatial patterns of an ordinary city which emerge from interconnections and interrelations  of  different  layers  of urban decision  making (top-down urban planning, real estate transformations, bottom-up participatory activities)

interconnections and interrelations  of  different  layers  of  decision  making (top-down urban planning, real estate transformations, bottom-up participatory activities) IS \textbf{Morphology of decision making}: it gives an exhaustive image of a city/urban environment 

\subsection{H2}

H2a:  Urban  development  in  a  post-socialist  city  is  determined  by  upgrading  the  level  of  urbanity.  Breaking  down  the 
morphology of influences among different decision-making levels through mapping a multi-layered network of interactions 
and interconnections among urban key agents (urban actors, built environment, space, regulatory framework, infrastructure, 
social practices) clarifies a dynamic urban development model in a post-socialist city.

H2b: Level of urbanity indicates an opportunity for change within socio-spatial patterns of an ordinary city.
Breaking  down  the morphology of urban
decision-making through mapping level of urbanity clarifies its urban development dynamics.

\subsection{H3}

H3a: A dynamic urban development model set as an iterative procedure of merging different levels of decision-making in a post-socialist  city  reinterprets  a  multi-layered  network  of  interactions  and  interconnections  among  urban  key  agents  by applying the Multi-agent system (MAS) and Actor-network theory (ANT) methodological approach.

H3b: A dynamic urban development model set as an iterative procedure of mapping the level of urbanity reinterprets a  multi-layered  morphology  of urban decision making
in terms of transformation, maintenance, and/or change of the system.

\section{Research Design}

procedure
independent variables
dependent variables
complementary measures

\subsection{Case study}

based on Qualitative methods slides (2.6 and 3.1):
explanatory - capture a process (theory testing)
descriptive - prepared and illustrated (ethnographic, describe the incidence/prevalence of the phenomena)
structure: linear analytic
phenomenological approach - contemporary social phenomenon (urban development) within its real-life context (post-socialist neighbourhood) - case about events (process)


\subsection{Case study selection}
\subsection{Local Context}

post-socialist city and transition
legacy from the past (path dependency, socialism) and prospects for the future (transition)

\section{Methods}

\subsection{Savamala Case study - Data collection} \label{sec:predis}

\subsection{MAS-ANT approach - Data analysis}

\subsubsection{Discourse analysis}
\subsubsection{Structural analysis}

An urban development model is based on:

Measuring the efficiency of urban planning

Testing the legitimacy of urban transformation interests

Recognizing the opportunities of bottom-up urban design initiatives


\subsubsection{Setting a procedure}

\section{Model Building - Data display}

Reporting the findings...

%%%%%%%%%%%%%%%%%%%%%%%%%%%%%%%%%%%%%%%%%%%%%%%%%%

\chapter{Case study}

%%%%%%%%%%%%%%%%%%%%%%%%%%%%%%%%%%%%%%%%%%%%%%%%%%

\section{Background}
In post-socialist cities, urban planning should link the top-down changes (linked to national and global level) to the bottom-up changes in the urban systems of the city, by emphasizing diversity and reciprocity in the nature of the on-going transformations: economic transformations (transformation of production and consumption in relation to space, income polarization and poverty), political transformations (urban governance, participation and decentralization), spatial transformations (demographic trend and distribution of functions) and social transformations (social inclusion, social activism and informality)
While some trends and directions within these transformations are clear and defined, uncertainty dominates decision-making and implementation in the turbulent environment of post-socialist cities (Nedović-Budić, 2001). The internal environment is in a state of flux, with the rapid adjustment of the physical, economic, social, and political structures of the city itself (Sykola, 1999). This captures the pace of change and the multi-layered nature of transformation, with the focus on the process of change in the city’s economy, society, system of governance and the spaces of production and consumption.
\section{Stimulants and deterrents of decision-making tradition in Belgrade}

xxx. ...

\subsection{Post-socialist Urban Planning}

Top-down Management of Urban Conflicts

\subsection{Legitimacy of Interests in a Post-socialist City}

Tactical Urban Transformations 

\section{Dynamism of urban agency in Savamala}

Moreover, the way cities function shapes the expectations and actions of all the urban actors involved, who also influence the constitution of the city itself. 

\subsection{Network of civic engagement}

Participatory Urban Design Operations


%%%%%%%%%%%%%%%%%%%%%%%%%%%%%%%%%%%%%%%%%%%%%%%%%%

\chapter{ANT Data Analysis}

\section{A forward-thinking overview of building an urban development model for Savamala}

xxx...

\section{Urban assemblage map: urban key agents and contextual resources}

\chapter{MAS Model Building}

%%%%%%%%%%%%%%%%%%%%%%%%%%%%%%%%%%%%%%%%%%%%%%%%%%

xxx. ...

\section{Body of urban relations: Urban Development dynamics}

xxx...

%%%%%%%%%%%%%%%%%%%%%%%%%%%%%%%%%%%%%%%%%%%%%%%%%%

\chapter{Conclusions}

%%%%%%%%%%%%%%%%%%%%%%%%%%%%%%%%%%%%%%%%%%%%%%%%%%

\section{Conclusions related to the research framework}

xxxx ...

\subsubsection{[..] to the research objectives}

xxxx

\subsection{[..] to the research questions}

xxxx

\subsection{[..] to the methodological approach}

xxxx

\section{Conclusions related to the theoretical framework}

xxx

\subsection{Urban Development Taxonomy}

xxx

\subsection{An Ordinary City}

xxx

\subsection{A Post-socialist City}

xxx

\subsection{Urbanity}

xxx

\section{Practical Implications}

xxx

\subsection{Urban Development model}

xxxx

\section{Limitations of the research}

xxx

\section{Future Prospects}

xxx


%%%%%%%%%%%%%%%%%%%%%%%%%%%%%%%%%%%%%%%%%%%%%%%%%%

\begin{small}
\addcontentsline{toc}{chapter}{Bibliography}
\bibliography{ThesisBib}
\end{small}

%%%%%%%%%%%%%%%%%%%%%%%%%%%%%%%%%%%%%%%%%%%%%%%%%%


\newpage
\appendix
\noappendicestocpagenum
\addappheadtotoc

\end{document}
