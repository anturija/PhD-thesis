The conceptual framework explained herein pinpoints the blurred and askew morphology of post-socialist cities which requires dynamic solutions in order to skip the classical, western procedure of urban formation and development. Consequently, this particular context shows the increased need for proper techniques that are spatially and temporally adjusted to current socio-spatial issues. The far-reaching goal, actually, is to transform the negative side effects of imitating and lagging behind the western urbanization model and those of the accelerating globalization into a development impetus suited to these environments.